%Current fault-tolerance approaches rely exclusively on either time or hardware redundancy for recovery. Rollback recovery  exploits time redundancy but can incur significant delay and high energy cost. On the other hand, process replication relies on hardware redundancy and requires a significant increase in resources and  power consumption.

In this paper, we propose Rejuvenating Shadows as a novel power-aware fault tolerance model, which guarantees forward progress, maintains consistent level of resilience, and minimizes implementation complexity and runtime overhead. Empirical experiments demonstrated that the Rejuvenating Shadows model outperforms in-memory checkpointing/restart in both execution time and resource utilization, especially in failure-prone environments.

Leaping induced by failure has proven to be critical in reducing the divergence between a main and its shadow, 
%with respect to workload execution.
%Consequently, the time to recover from subsequent failures is reduced significantly. 
thus reducing the recovery time for subsequent failures. Consequently, the time to recover from a failure increases with failure intervals.  
Based on this observation, a proactive approach is to ``force" leaping when the divergence between a main and its shadow exceeds a specified threshold. 
In our future work, we will further study this approach to determine what behavior triggers forced leaping in order to optimize the average recovery time. 

%we will study forcing the shadDuring experimentation we noticed the problem that recovery time in Rejuvenating Shadows can become substantial when the failure interval is large (Figure~\ref{fig:single_failure}). To deal with this issue, we are studying the idea of ``forced leaping", which borrows the idea from periodic checkpointing and forces a leaping whenever failure has been absent for a long time, in order to reduce the divergence between mains and shadows. Optimal intervals for forced leaping will be explored to balance between runtime overhead and failure recovery overhead. 

%In the future we plan to explore the integration with fault prediction techniques and the viability of dynamic and partial shadowing for platforms where nodes exhibit different ``health" status, e.g., some nodes may be more reliable while others are more likely to fail~\cite{gainaru2012fault}. 
%With this taken into account, we can apply dynamic scheduling of shadows only for mains that are likely to fail, to further reduce the resource requirement. 
%Another future direction is to study complier-assisted program slicing for fault detection. Specifically, slices that are fraction of their mains can run lazily as shadows and provide fault detection capability with reasonable coverage. 



