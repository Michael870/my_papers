In today's large-scale High Performance Computing (HPC) systems, an increasing portion of the computing capacity is wasted due to failures and recoveries. It is expected that exascale machines will decrease the mean time between failures to a few hours. This makes fault tolerance a major challenge for the HPC community. Moreover, the stringent power cap set by the US Department of Energy for exascale computing further complicates this challenge. This work explores novel methodologies to fault tolerance that achieves forward recovery, power-awareness, and scalability. The proposed model, referred to as Rejuvenating Shadows, has been implemented in MPI, and empirically evaluated with multiple benchmark applications that represent a wide range of HPC workloads. The results demonstrate that Rejuvenating Shadows has negligible runtime overhead during failure-free execution, and can achieve significant advantage over checkpointing/restart when failures are prone.


