In today's large-scale High Performance Computing (HPC) systems, an increasing portion of the computing capacity is wasted due to failures and recoveries. It is expected that exascale machines will decrease the mean time between failures to a few hours, making fault tolerance a major challenge. This work explores novel methodologies to fault tolerance that achieve forward recovery, power-awareness, and scalability. The proposed model, referred to as Rejuvenating Shadows, is able to deal with multiple types of failure and maintain consistent level of resilience. An implementation is provided for MPI, and empirically evaluated with various benchmark applications that represent a wide range of HPC workloads. The results demonstrate Rejuvenating Shadows' ability to tolerate high failure rates, and to outperform in-memory checkpointing/restart in both execution time and resource utilization.


