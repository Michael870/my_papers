Today's scientific discoveries and business intelligence are driven by large-scale parallel simulation and data analytics. To meet the increasing computing demands from 
virtually every aspect of the society, HPC is continuously evolving to solve more 
complex and challenging problems. 
Recently, the explosively growing machine learning applications have increased the adoption as well as 
impact of HPC technologies as they also exploit parallelism and hardware acceleration to speed up the processing of 
massive amount of data.


HPC workloads have traditionally been run only on bare-metal hardware to drive maximum 
performance. 
The roadblock to virtualization was due to the concern that the extra hypervisor layer could introduce 
performance overhead. 
However, this has changed with the introduction of increasingly sophisticated 
%hardware support for virtualization and software optimization~\cite{mergen2006virtualization,madukkarumukumana2008resource,bugnion2017hardware}. Performance of 
hardware support for virtualization and software optimization~\cite{mergen2006virtualization}. %Performance of 
%these highly parallel HPC workloads has increased dramatically over the last decade, 
%enabling organizations to embrace the numerous benefits that a virtualization platform can 
%offer~\cite{michael2018overcommit}. 
As a result, we are witnessing a popular trend that institutions transform 
their on-prem bare-metal clusters to virtualized, shared private cloud. For instance, the Johns Hopkins 
University Applied Physics Laboratory has virtualized their 3728-core bare-metal cluster 
to share between Windows and Linux users~\cite{vmware2017josh}. %It is reported that resource utilization 
%has improved by 9.1\% to 29.2\%, and simulation workloads have sped up by an average of 4\%~\cite{vmware2017josh}.

At the same time, public cloud, such as Amazon AWS and Google GCP, is becoming a popular option for 
HPC. Recent studies show that the usage of public cloud has grown more than five-fold among all HPC 
%sites worldwide, from 13\% in 2011 to 74\% in 2018~\cite{hyperion2019}.
sites worldwide~\cite{hyperion2019}.
With virtually unlimited scalability and on-demand resource subscription, public cloud starts to host 
both compute- and data-intensive workloads across various industry verticals. These workloads span the traditional HPC 
applications, like genomics and 
weather prediction, as well as emerging applications, like machine learning and deep learning. 


There is a fruitful body of research on resource management in 
%cloud computing~\cite{singh2016survey,zhan2015cloud,gill2018chopper}. Dynamic resource scheduling and 
cloud computing~\cite{singh2016survey}. Dynamic resource scheduling and 
load balancing are used 
%to maximize system utilization~\cite{adhikari2018heuristic,panwar2015load}. These techniques, however, 
to maximize system utilization. These techniques, however, 
are not straightforward to apply to HPC workloads which are highly sensitive to resource change and interference. 
In fact, resource management has been identified as one of the open 
challenges for HPC cloud~\cite{netto2018hpc}. 
Currently, system admins are often limited to statically and conservatively reserve 
resources based on peak resource requirements to adhere to service level agreement (SLA). For example, Microsoft Azure collaborates with Cray to offer supercomputers to HPC users, 
and Amazon AWS reserves dedicated nodes to provide full-size VMs. 
This essentially offsets 
the elasticity and efficiency promises of the cloud computing business model. 

% To this end, we present \textit{vHERO} as a novel approach to cloud 
% resource allocation for HPC, by \underline{v}irtualizing 
% \underline{H}PC workloads with multi-t\underline{e}nancy and  \underline{r}esource  \underline{o}ver-commitment.
% Based on full virtualization virtual machine (VM), vHERO goes beyond traditional way of 
% statically partitioning resources among tenants and instead applies resource over-commitment to optimize 
% system utilization and maximize throughput. By giving each tenant a virtual cluster that mimics the 
% underlying physical cluster, vHERO delegates the resource management task 
% to the hypervisor to add an extra layer of flexibility as well as to improve efficiency. When all tenants are busy consuming their cycles, 
% vHERO guarantees that each tenant is getting his/her fair share according to pre-defined SLA. When 
% some tenant is not fully using the allocated resources, vHERO takes advantage of the work-conserving 
% property of the hypervisor scheduler to assign the idle resources to other tenant(s) who can benefit 
% from additional resources. Consequently, system admins can ensure quality-of-service (QoS) while maximizing system utilization. 

To this end, we present our work in combining full virtualization with resource over-commitment for  HPC cloud.
Based on virtualization, we go beyond traditional way of 
statically partitioning resources and instead offer each tenant a virtual 
cluster of VMs that mimic the underlying physical cluster. By simultaneously running 
multiple virtual clusters on a physical cluster and delegating resource management  
to the hypervisor, our approach not only supports 
multi-tenancy with security isolation, but also achieves improved flexibility and efficiency. Specifically, when all tenants are busy, 
the hypervisor guarantees that each tenant get a fair share according to pre-defined SLA; when 
some tenant is idle, a work-conserving scheduler in the hypervisor assigns the idle resources to other tenant(s) who can benefit 
from additional resources. Consequently, system admins will be able to ensure quality-of-service (QoS) while maximizing system utilization. In the remainder of this paper, we will use VMware ESXi hypervisor to demonstrate our approach as well as share our experience and best practices. 

The rest of the paper is organized as follows. 
Section II provides background and motivation. Section III explains our approach to virtual infrastructure configuration with resource over-commitment. Section IV presents empirical evaluation results along with experience that we have gained. Section V concludes this work and points out future directions.
