\documentclass{acm_proc_article-sp}
\usepackage{subfigure}
\begin{document}

\title{Shadow Computing}
\subtitle{An Energy-Aware Resiliency Scheme for High Performance Computing}
%
% You need the command \numberofauthors to handle the 'placement
% and alignment' of the authors beneath the title.
%
% For aesthetic reasons, we recommend 'three authors at a time'
% i.e. three 'name/affiliation blocks' be placed beneath the title.
%
% NOTE: You are NOT restricted in how many 'rows' of
% "name/affiliations" may appear. We just ask that you restrict
% the number of 'columns' to three.
%
% Because of the available 'opening page real-estate'
% we ask you to refrain from putting more than six authors
% (two rows with three columns) beneath the article title.
% More than six makes the first-page appear very cluttered indeed.
%
% Use the \alignauthor commands to handle the names
% and affiliations for an 'aesthetic maximum' of six authors.
% Add names, affiliations, addresses for
% the seventh etc. author(s) as the argument for the
% \additionalauthors command.
% These 'additional authors' will be output/set for you
% without further effort on your part as the last section in
% the body of your article BEFORE References or any Appendices.

\numberofauthors{1} %  in this sample file, there are a *total*
% of EIGHT authors. SIX appear on the 'first-page' (for formatting
% reasons) and the remaining two appear in the \additionalauthors section.
%
\author{
% You can go ahead and credit any number of authors here,
% e.g. one 'row of three' or two rows (consisting of one row of three
% and a second row of one, two or three).
%
% The command \alignauthor (no curly braces needed) should
% precede each author name, affiliation/snail-mail address and
% e-mail address. Additionally, tag each line of
% affiliation/address with \affaddr, and tag the
% e-mail address with \email.
%
% 1st. author
\alignauthor
Bryan Mills\raisebox{4pt}{$\ast$}, Taieb Znati\raisebox{4pt}{$\ast$$\dagger$}, Rami Melhem\raisebox{4pt}{$\ast$}\\
       \affaddr{Department of Computer Science\raisebox{4pt}{$\ast$}}\\
       \affaddr{Telecommunications Program\raisebox{4pt}{$\dagger$}}\\
       \affaddr{University of Pittsburgh}\\
       \affaddr{Pittsburgh, PA  15260}\\
       \email{(bmills, znati, melhem)@cs.pitt.edu}
}

\maketitle
\thispagestyle{empty}

%%\begin{IEEEkeywords}
%%shadow computing, fault tolerance, scheduling, resiliency
%%\end{IEEEkeywords}

\begin{abstract}
As the demand for cloud computing continues to increase, cloud service
providers face the daunting challenge to meet the negotiated SLA
agreement, in terms of reliability and timely performance, while
achieving cost-effectiveness. This challenge is increasingly
compounded by the increasing likelihood of failure in large-scale
clouds and the rising cost of energy consumption.  This paper proposes
Shadow Replication, a novel profit-maximization resiliency model,
which seamlessly addresses failure at scale, while minimizing energy
consumption. The basic tenet of the model is to associate a suite of
shadow processes to execute concurrently with the main process, but
initially at a much reduced execution speed, to overcome failures as
they occur. Two computationally-feasible schemes are proposed to
achieve shadow replication. A performance evaluation framework is
developed to analyze these schemes and compare their performance to
traditional replication-based fault tolerance methods, focusing on the
inherent tradeoff between fault tolerance, the specified SLA and
profit maximization. The results show Shadow Replication leads to
significant energy reduction, and is better suited for
compute-intensive execution models, where up to 30\% more profit
increase can be achieved.


%several experimental studies are carried out to assess the performance
%of the different resiliency schemes, with respect to profit
%maximization in different cloud computing environments. 




%The challenge is to derive the
%execution speed, both before and after failure, of a shadow in order
%to ensure adherence to the negotiated SLA, while maximizing profit. To
%this end, we present an optimization model to derive the shadow
%execution speeds, which takes into consideration a computing node
%failure rate and the negotiated SLA. Several computationally-feasible
%methods are then proposed to solve this model. 



%As companies continue to increase their reliance upon cloud computing
%services there will be an increasing demand for reliable and timely
%service. However, as cloud-based systems increase in size and
%complexity it is expected that reliability will degrade, causing both
%delays in service and increases in energy consumption. This will cause
%fault tolerance to be a critical system feature to providing
%applications at large scale. In this work, we propose ``shadow
%replication'', a fault tolerance method that makes use of DVFS to
%provide energy-aware, profit-maximizing system resilience to
%task-based cloud computing services.  We analyze different resilience
%methods and identify the system parameters which are most relevant to
%the tradeoff between fault tolerance and profit, and present results
%which pinpoint the most profitable method.  We also show that in
%certain systems shadow replication can achieve 10-30\% more profit
%than existing fault tolerance methods, i.e. re-execution and
%traditional replication.

%
%
% ``shadow replication'' to
%provide an energy-aware, profit-optimized method that can result in
%upto two-times the profit achieved by existing fault tolerance
%methods. Additionally, we develop analytical models to demonstrate the
%benefits of our approach at the scale expected in future cloud
%computing environments.

\end{abstract}

\section{Introduction}
\label{intro}
Today's scientific discoveries and business intelligence are driven by high-fidelity, 
large-scale simulation and data analytics. To meet the increasing computing demands from 
virtually every aspect of the society, HPC is continuously evolving to solve more 
complex and challenging problems. On the one hand, national labs and research institutes run HPC on 
supercomputers for scientific breakthroughs and national security. On the other hand, enterprises and 
organizations deploy HPC on small to medium sized clusters to process data and extract insights. 
Recently, the explosively growing machine learning applications have increased the adoption as well as 
impact of HPC as they also exploit parallelism and hardware acceleration to speed up the processing of 
massive amount of data.


HPC workloads have traditionally been run only on bare-metal, unvirtualized hardware to drive maximum 
performance. 
The roadblock to virtualization was due to the concern that the extra hypervisor layer could introduce 
performance overhead. 
%The concern was that virtualization could introduce performance overhead due to the extra software 
%layer of hypervisor. 
However, this has started to change with the introduction of increasingly sophisticated 
hardware support for virtualization and software optimization~\cite{madukkarumukumana2008resource,bugnion2017hardware}. Performance of 
these highly parallel HPC workloads has increased dramatically over the last decade, 
enabling organizations to begin to embrace the numerous benefits that a virtualization platform can 
offer~\cite{michael2018overcommit}. As a result, we are witnessing a popular trend that enterprises convert 
their on-prem bare-metal clusters to virtualized, shared private cloud. For instance, the Johns Hopkins 
University Applied Physics Laboratory recently virtualized their 3728-core bare-metal cluster 
to share between Windows and Linux users. The reported improvement in resource utilization 
ranges from 9.1\% to 29.2\%, and simulations speed up by 4\% on average~\cite{vmware2017josh}.

At the same time, public cloud, such as Amazon AWS and Google GCP, is becoming a popular alternative for 
HPC practitioners. Recent studies show that the usage of public cloud has grown more than five-fold among all HPC 
sites worldwide, from 13\% in 2011 to 74\% in 2018~\cite{hyperion2019}.
With virtually unlimited scalability and on-demand resource subscription, public cloud starts to host 
compute- and data-intensive workloads across various industry verticals. These workloads span the traditional HPC 
applications, like genomics and 
weather prediction, as well as emerging applications, like machine learning and deep learning. 

There is a fruitful body of research on resource management in 
Cloud Computing~\cite{singh2016survey,zhan2015cloud,gill2018chopper}. Dynamic resource scheduling and 
load balancing are used 
to maximize system utilization and efficiency~\cite{adhikari2018heuristic,panwar2015load}. These techniques, however, 
are not straightforward to apply to HPC workloads which are highly sensitive to resource change and interference. 
Actually, resource management has been identified as one of the open 
challenges for HPC cloud~\cite{netto2018hpc}. 
Currently, cloud service providers (CSPs) are often limited to statically and conservatively reserve 
resources based on peak resource requirements to respect service level agreements (SLAs). For example, Microsoft Azure 
allocates dedicated supercomputers from Cray, and Amazon AWS offers dedicated nodes for full-size VMs. 
% allocate physical resources
% Despite the 
% numerous benefits promised by Cloud Computing, however, cloud service providers (CSPs) are often limited to statically 
% allocate physical resources to HPC tenants in order to avoid performance interference and enforce 
% service level agreements (SLAs). 
This essentially offsets 
the elasticity and efficiency benefits of the Cloud Computing business model. 

In this paper, we present \textit{virtual throughput clusters (VTC)} as a novel approach for cloud 
resource allocation to efficiently and effectively support 
HPC workloads with multi-tenancy. Based on virtual machine (VM), VTC goes beyond traditional way of 
statically splitting resources among tenants and applies resource over-commitment to optimize 
system utilization and throughput. By giving each tenant a virtual cluster that mimics the 
underlying physical cluster, VTC delegates the resource management task 
to the hypervisor to improve flexibility as well as efficiency. When all tenants are busy consuming their cycles, 
VTC guarantees that each tenant is getting his/her fair share according to pre-defined SLA terms. When 
some tenant is not fully using the allocated resources, VTC takes advantage of the work-conserving 
property of the hypervisor scheduler to assign the idle resources to other tenant(s) who can benefit 
from additional resources. Consequently, CSPs can ensure quality-of-service while maximizing 
system utilization. 

The rest of the paper is organized as follows. 
Section II provides background and motivation. Section III introduces the design of VTC, followed by validation 
and empirical evaluation results in Section IV. Section V concludes this work and points out future directions.

\section{Related Work}
\label{related_work}
%Extreme-scale computing presents some unique challenges to fault tolerance as faults are no longer 
%an exceptional event \cite{ferreira_sc_2011}. 
Rollback and recovery is the dominant mechanism to achieve fault
tolerance in current HPC environments~\cite{Elnozahy:02:Survey}. In the most general form, rollback and recovery 
involves the periodic saving of the current system state, with the anticipation that
in the case of a failure, computation can be restarted from the most recently saved state. % \cite{Elnozahy:02:Survey}. %The identification of an error, before or during a checkpoint,
%requires that the application rollback to the previously completed checkpoint. 
Coordinated checkpointing is a popular approach for
its ease of implementation.
%Specifically, all processes
%coordinate with one another to produce individual states that satisfy the ``happens before"
%communication relationship \cite{chandy_trans_1972}, which is proved to provide a consistent global state.
%Essentially, the algorithm provides a method for all processes involved to stop operation ``at the same
%time" and transfer their system state to a stable storage. 
%The major benefit of coordinated
%checkpointing stems from its simplicity and ease of implementation. 
Its major drawback, however, is the
lack of scalability, as it requires global coordination
~\cite{elnozahy_dsc_2004}.%riesen_sandia_2010}.
%hargrove2006berkeley}.


In uncoordinated checkpointing, processes checkpoint their states independently and postpone creating a 
globally consistent view until the recovery phase. The major advantage is the reduced overhead during fault free operation. However, the scheme requires that
each process maintains multiple checkpoints and message logs, necessary to construct a consistent 
state during recovery. It can also suffer the well-known domino effect 
 \cite{randell_domino_effect}. One hybrid approach, known as communication induced 
checkpointing, aims at reducing coordination overhead \cite{alvisi_ftc_1999}. The approach, however, may 
cause processes to store useless states. To address this 
shortcoming, ``forced checkpoints" have been proposed \cite{helary_rds_1997}. This approach, however,  may lead to unpredictable
checkpointing rates. Although well-explored, uncoordinated checkpointing has not been widely adopted
in HPC environments, due to its dependency on applications \cite{guermouche_2011_ipdps}.


One of the largest overheads in any checkpointing process is the time necessary to write the checkpointing 
to stable storage. Incremental checkpointing attempts
to address this by only writing the changes since previous checkpoint \cite{Agarwal:04:Adaptive}. %,elnozahy_1992_manetho,li_trans_1994}. %This
%can be achieved using dirty-bit page flags \cite{plank_ftcs_1994,elnozahy_1992_manetho}. Hash based incremental checkpointing, on the other
%hand, makes use of hashes to detect changes \cite{nam_ftc_1997,Agarwal:04:Adaptive}. 
Another proposed scheme, known as in-memory checkpointing, minimizes the overhead of disk access~\cite{zheng_2004_ftccharm,6264677}.
%offloads the checkpointing process to a secondary task and only writes incremental checkpoints \cite{li_trans_1994}.
The main concern of these techniques is the increase in
memory requirement to support the simultaneous execution of the checkpointing and the application. It has been suggested that nodes in extreme-scale systems should be configured with fast local storage~\cite{doe_ascr_exascale_2011}. 
%, which
%improves the performance of checkpointing \cite{doe_ascr_exascale_2011}. 
Multi-level checkpointing, which consists of
writing checkpoints to multiple storage targets, can benefit from such a strategy \cite{Moody:10:SCR}. This,
however, may lead to increased failure rates of individual nodes and complicate the checkpoint writing process.
%Furthermore, it may complicate the checkpoint writing process and requires that the system track the
%current location of all process's checkpoints.


Process replication, or state machine replication, has long been used for reliability and availability in distributed and mission critical systems \cite{schneider_1990_tutorial}. %Replication can be used to detect and correct system failures that are otherwise undetectable,
%such as silent data corruption and Byzantine faults \cite{fiala_2012_sdc}. 
This approach is barely used in HPC systems, primarily due to its high cost and low efficiency.
However, upcoming extreme-scale systems are expected to 
%require a more challenging level of fault tolerance to deal with the 
confront a dramatic growth in both the frequency and diversity of faults.
As a result,
replication has recently been proposed as a
viable alternative to checkpointing in HPC \cite{engelmann09case,Cappello:09:Fault}. 
In addition, full and partial
replication have also been studied to augment existing checkpointing techniques, and to  
detect or correct silent data corruption \cite{stearly_2012_partial,elliott_2012_cpr,ferreira_sc_2011,fiala_2012_sdc}. % There are several different implementations of
%replication in the widely used MPI library, each with their different tradeoffs and overheads. The
%overhead can be negligible or up to 70\% depending upon the communication patterns of the
%application \cite{engelmann2011redundant}. %Moreover, replication alone is not enough to guarantee fault tolerance since
%it is possible that all nodes executing a given process could fail simultaneously, thus
%replication is typically paired with some form of checkpointing. 
Our approach differs from classical process replication in that we dynamically configure the execution rates of main and shadow processes, so that less resource/energy is required while reliability is still assured.  


Replication with dynamic execution rate is also explored in Simultaneous and Redundantly Threaded (SRT) processor whereby one leading thread of execution is running ahead of trailing threads \cite{reinhardt2000transient}. However, 
the focus of \cite{reinhardt2000transient} is on transient faults within CPU while we aim at tolerating both permanent and transient faults across all systems components.
This work is closely related to our previous works \cite{mills_2014_icnc,cui_en7085151,cui_2014_closer} where single or loosely-coupled tasks is considered. Instead, in this paper we explore novel ideas of shadow collocation and shadow leaping in order to satisfy the requirements of future extreme-scale HPC systems. 
%our approach is different in that it tunes the execution rates of the leading and trailing threads in a finer grain, in order to achieve a ``parameterized" trade-off between completion time and energy consumption. 
%Further, we take advantage of the idle time during failure recovery and ``leap" the trailing replicas to achieve forward progress%, largely improving performance in terms of both completion time and energy consumption. 
%. This differs from \cite{reinhardt2000transient}, of which the ``leaping" of the trailing replica results in extra overhead.
%To the best of our knowledge,
%Lazy Shadowing is the first attempt to explore a state-machine replication based framework
%that achieves a fine-grained tradeoff between time and hardware redundancy while meeting resilience and
%power requirements.


\section{Model and Data Structure}
\label{shadow_model}

In this section, we provide an overview of the shadow computing
execution model, under different failure scenarios. We also discuss
the mapping of the processes to the computing infrastructure to ensure
fault-tolerance to failure. Finally, we present the basic data
structure that enables efficient communication between a main process
and its associated shadow. The main purpose of this section is to show
the feasibility of the shadow computing model. The details of how the
execution model and its associated data structure are implemented is
outside the scope of this work, and will be the subject of a future
publication.


\subsection{Execution Model} 

Depending on the occurrence of failure during execution, two scenarios
are possible. The first scenario, depicted in Figure
\ref{simple_shadow_no_failure}, takes place when no failure
occurs\footnote{For the purpose of this discussion, only a single
shadow is considered. The discussion can be easily extended to deal
with multiple shadow processes}. In this scenario, the main process
executes at the optimum processor speed, namely the speed necessary to
achieve the desired level of fault-tolerance, minimize energy
consumption and meet the target response time of the supported
application. The figure shows the completion time of the task, in the
absence of failure. During this time, the main process completes the
total amount of work required by the underlying application. However,
the shadow process, executing at a reduced processor speed, only
completes a significantly smaller amount of the original workload.
Because the likelihood of an individual node failure is low, this
scenario is most likely to occur with high frequency, resulting in a
relatively small amount of additional energy consumption to achieve
fault-tolerance. The benefits of this scheme clearly outweigh the
additional energy cost. Furthermore, it is worth noting that the
failure of the shadow process does not impact the behavior of the main
process.

\begin{figure}[hHtb]
\centering
\subfigure[Shadow Computing \-\- Case of no failure]{
\label{simple_shadow_no_failure}
\psfig{figure=diagrams/shadow_simple_no_failure_work.eps,width=3.1in}
}
\subfigure[Shadow Computing \-\- Case of failure]{
\label{simple_shadow_with_failure}
\psfig{figure=diagrams/shadow_simple_with_failure_work.eps,width=3.1in}
}
\end{figure}

The second scenario, depicted in Figure
\ref{simple_shadow_with_failure}, takes place when failure of the main
process occurs. Upon failure detection, the shadow process increases
its processor speed and executes until completion of the task.  The
processor speed at which the shadow executes after failure is derived
so that the shadow computing model guarantees that the task still
completes by the targeted response time,regardless of when the failure
occurs. Furthermore, shadow computing achieves considerable energy
saving by taking advantage of the fact that the likelihood of
individual component failures in exascale computing is small, thereby
making oblivious the need to execute "duplicate work" unless a failure
occurs. It is assumed that shadow processes can detect failures
although the details of this are beyond the scope of this paper.


It is worth noting that the interplay between resiliency and power
management manifests itself in different ways and must be analyzed
carefully. Operating at lower voltage thresholds, for example, reduces
power consumption but has an adverse impact on the resiliency of the
system to handling high error rates in a timely fashion. Our approach
to deriving optimal execution speed for the main process and its
associated shadows, both before and after failure, seeks to avoid
continuous change in voltage and frequency to prevent potential
thermal and mechanical stresses on the electronic chips and
board-level electrical connections.


% LocalWords: mtbf megawatts exascale gigawatts pre varela
\subsection{Process Mapping}

In the shadow computing model the execution of a task spawns the
creation of both a main process and a suite of shadow processes. These
processes must be carefully mapped to the computing nodes of the
exascale computing infrastructure to achieve fault-tolerant execution.
Consequently, the mapping must be done such that the main and shadow
processes are {\it fault-isolated} from each other, meaning that a
fault affecting one process does not affect the other. Fault-isolation
is necessary to minimize the likelihood that both the main and shadow
processes fail at the same time. In clound computing, fault-isolation
uses the virtual computing capabilities of the infrastructure to
assign main processes and shadows in a way such a given shadow process
can only be run along side an unrelated main process.  Figure
\ref{process_allocation_diagram} illustrates a feasible assignment
that satisfies such a constraint, for the case of three main processes
and their associated shadows.

\begin{figure}[hHtb]
\centering
\psfig{figure=diagrams/overview_architecture.eps,width=3.0in}
\caption { Example Process Mapping }
\label{process_allocation_diagram}
\end{figure}

\subsection{Message Passing}

Another important aspect of the shadow computing model is providing
process communication to achieve synchronization and maintain system
consistency.  A communication model to ensure these requirements must
at a minimum support these two properties:
\begin{itemize}
\item
All messages destined for a task must be delivered to both the main
process and all associated shadow processes.
\item
Any message previously sent from a task must not be duplicated by any
of the running process.
\end{itemize}

To satisfy the communication and synchronization requirements the
shadow computing model, the runtime support environment uses a Global
Message Queue, see Figure \ref{process_allocation_diagram}. All
inter-task communication will occur through a virtual message queue,
which is assumed to be resilient to system faults. When a task spawns
the main and shadow processes the queue is notified of all processes
created. For scalability reasons we assume the queue is a {\it
passive-queue}, meaning it only stores messages and waits for them to
be requested as opposed to forwarding messages to processes. This
eliminates the need for the queue to notify processes directly and
instead lets the processes request them when they are ready. This
allows processes running at a higher execution speed to not interfere
with the execution of processes running slower.

When a message arrives at the queue for delivery to a task it will
hold that message until it has been delivered to all associated
processes\footnote{Any implementation of such a system will have to
address the issue of growing queue size but for this discussion it is
assumed we have an infinite queue.}. This is possible because all
associated processes were registered with the queue when created. An
example of message delivery is depicted in Figure
\ref{queue_message_delivery}. While not depicted, messages would also
also be removed from the queue once the task was completed. This
scheme ensures that all executing processes will receive all messages
destined for the task.

\begin{figure}[hHtb]
\centering
\psfig{figure=diagrams/message_queue_deliever.eps,width=3.0in}
\caption { Example Message Delivery }
\label{queue_message_delivery}
\end{figure}

\begin{figure}[hHtb]
\centering
\psfig{figure=diagrams/message_queue_receive.eps,width=3.0in}
\caption { Example Message Receiving }
\label{queue_message_receiving}
\end{figure}

In order to ensure that messages are not duplicated by shadow
processes we propose that all messages be assigned a unique sequence
number per task. When the queue receives a message from a task it will
determine if that message has already been received by the queue for
the task. If the message is a duplicate it will simply ignore the
message, if however it is a new message it will queued for
delivery. We show an example of the message receiving process in
Figure \ref{queue_message_receiving}. This allows shadow processes to execute
slower and not produce duplicate system messages. The other benefit of
this model is that messages will be processed regardless of their
source, therefore the queue doesn't need to be aware of process
failures.


\section{Energy Optimization Model}
\label{model}
As stated previously, the basic idea of shadow computing is to
associate a number of ``shadow processes'' with each main process. The
main responsibility of a shadow process is to take over the
responsibility of a failed main process and bring the computation to a
successful completion.  In this section, we define a framework for
evaluating shadow computing and then use this to derive a model for
representing the expected energy consumed by the system. We then
describe in terms of this model three different methods for applying
shadow computing in a high performance computing environment.

%%model that describes and show how it can be used to derive the speed
%%of execution of the shadow process that minimizes energy
%%consumption. Without loss of generality, in this section, we focus on
%%the main process and its first shadow. The model can be easily
%%extended to deal with multiple shadows.

\subsection{Shadow Computing Framework}
\label{shadow_computing_framework}

We consider a distributed computing environment executing an
application carried out by a large number of collaborative tasks. The
successful execution of the application depends on the successful
completion of all of these tasks. Therefore the failure of a single
process delays the entire application, increasing the need for fault
tolerance. Each task must complete a specified amount of work, $W$, by
a targeted response time, $R$. The amount of work is expressed in
terms of the number of cycles required to complete the task. Each
computing node has a variable speed, $\sigma$, given in cycles per
second and bounded such that $0\leq\sigma\leq\sigma_{max}$. Therefore
the minimum response time for a given task is $R_{min}=\sigma_{max}*W$.

In order to achieve our desired fault tolerance a shadow process
executes in parallel with the main process on a different computing
node. The main process executes at a single execution speed denoted as
$\sigma_m$. In contrast the shadow process executes at two different
speeds, a speed before failure detection, $\sigma_b$, and a speed
after failure detection, $\sigma_a$. This is depicted in Figure
\ref{shadow_overview}.

\begin{figure}[hHtb]
\centering
\psfig{figure=diagrams/shadow_main_diagram.eps,width=3.5in}
\caption { Overview of Shadow Computing }
\label{shadow_overview}
\end{figure}

Based upon this framework we define some specific time points
signaling system events. The time at which the main process completes
a task, $t_c$, is given as $t_c=W/\sigma_m$. The time at which the
shadow process completes as task, $t_r$, is given as $t_r =(W-\sigma_b
t_c)/\sigma_a$ related but not necessarily equal is $t_R$ which is the
time the system reaches the targeted response time for a given
task. Additionally, we define the time point $t_f$ as the time at which
a failure in the main process is detected.

Using this framework we formalize our objective as the following
minimization problem.
%%% need more horizontal spacing here... not sure the latex-fu
\begin{equation}
\label{optimization_problem}
\begin{split}
\text{minimize  }   & E(\sigma_m,t_0,t_f,t_c) + E(\sigma_b,t_0,t_f,t_c) + E(\sigma_a,t_f,t_r) \\
\text{subject to  } & t_c \leq t_R \\
                  & t_r \leq t_R \\
                  & \sigma_m t_c \geq W \\
                  & \sigma_b t_c + \sigma_a (t_R - t_f) \geq W
\end{split}
\end{equation}
Here the function $E(\sigma,t_0,t_1,t_2)$ represents the energy
consumed by a process running at speed $\sigma$ during the time period
$t_0$ and $min(t_1,t_2)$. The first two constraints state that both
the main process and the shadow process must complete by the targeted
response time. The last constraints ensure that the amount of work
done by those processes must be greater than or equal to the amount of
work defined by the task.

It should be noted that it is assumed that node failures and task
properties are unchangeable system properties therefore the system
parameters we can change are the execution speeds of the
processes. Thus the output of this optimization problem is the
execution speeds, $\sigma_m$, $\sigma_b$ and $\sigma_a$. In the
proceeding sections we will present a power and failure model for
individual computing nodes then use these to model the expected energy
of the shadow computing system. Then in Section
\ref{application_to_hpc} we then apply this model to the high
performance computing environment.

\subsection{Power Model}
\label{power_model}

It is well known that by varying the execution speed of the computing
nodes one can reduce their power consumption at least quadratically by
reducing their execution speed linearly. The power consumption of a
computing node executing at speed $\sigma$ is given by the function
$p(\sigma)$, represented by a polynomial of at least second degree,
$p(\sigma)=\sigma^n$ where $n\geq2$. The energy consumed by a
computing node executing at speed $\sigma$ during an interval of
length $T$ is given by $E(\sigma,T)=\int_{t=0}^T
p(\sigma)dt$. Throughout this paper we substitute the energy for a
particular time interval, t, with the derived value $p(\sigma)t$,
because $p(\sigma)$ is treated as a constant with respect to time. We
further assume that the computing node speed is bounded by the
following equation $0\leq\sigma\leq\sigma_{max}$.

\subsection{Failure Model}
\label{failure_model}

The failure can occur at any point during the execution of the main
task and the completed work is unrecoverable. Because the processes
are executing on different computing nodes we assume failures are
independent events. We also assume that only a single failure can
occur during the execution of a task. If the main task fails it is
therefore implied that the shadow will complete without failure. We
can make this assumption because we know the failure of any one node
is a rare event thus the failure of any two specific nodes is very
unlikely. In order to achieve higher resiliency one
would make use of multiple shadow processes and this failure model
will still be valid.

We assume that a probability density function, $f(t)$ ($\int_0^\infty
f(t)dt=1$), exists which expresses the probability of the main task
failing at time $t$. It is worth noting, that the model does not
depend on any specific distribution.

\subsection{Energy Model}
\label{energy_model}

Given the power model and the failure distribution, the expected
energy consumed by a shadow computing task can be derived. We start by
considering the expected energy consumed by the main process and
derive the following equation:
\begin{equation}
\label{energy_for_main}
\int_{t=0}^{t_c}E(\sigma_m,t)f(t)dt + (1-\int_{t=0}^{t_c}f(t)dt)E(\sigma_m,t_c)
\end{equation}
This first term of the equation represents the expected amount of
energy consumed by the main process if a failure occurs, while the
second term represents the expected energy consumed if no failure
occurs.

Similarly, we can calculate the expected energy consumed by the shadow
process, as follows:
\begin{equation}
\label{energy_for_shadow}
\begin{split}
&  \int_{t=0}^{t_c}E(\sigma_b,t)f(t)dt \\
+& \int_{t=0}^{t_c}E(\sigma_a,(t_r-t))f(t)dt \\
+& (1-\int_{t=0}^{t_c}f(t)dt)E(\sigma_b,t_c)
\end{split}
\end{equation}
The first term represents the expected energy consumed by the shadow
executing at $\sigma_b$ up until the main process fails. The middle
term represents the expected energy consumed when the main process
fails and the shadow begins to execute at the speed, $\sigma_a$. The
last term is the expected energy consumed in the event that no failure
occurs and the shadow executes at $\sigma_b$ the entire duration of
the main process.

The total energy consumed by a shadow computing task is the summation
of the energy consumed by the main process and shadow process. To
expand this model to represent multiple shadows we would multiple the
energy consumed by the shadow by the total number of shadow
processes. Given one shadow process we can combine equations
(\ref{energy_for_main}) and (\ref{energy_for_shadow}) to produce the
single model representing the total expected energy consumed.
\begin{equation}
\label{energy_model}
\begin{split}
 & \int_{t=0}^{t_c}(E(\sigma_m,t)+E(\sigma_b,t))f(t)dt \\
+& \int_{t=0}^{t_c}E(\sigma_a,(t_r-t))f(t)dt \\
+& (1-\int_{t=0}^{t_c}f(t)dt)(E(\sigma_m,t_c )+E(\sigma_b,t_c ))
\end{split}
\end{equation}

% LocalWords: hpc


\section{Application to HPC}
\label{application_to_hpc}
One of the primary goals of high performance computing is to achieve
the maximum possible throughput of the system. Thus when we
apply lazy shadowing to this environment we assume that the
execution speed of the main process should be the maximum possible
execution speed, $\sigma_m=\sigma_{max}$. If no failure occurs then
the task will be completed as soon as possible, known as the minimum
response time. If the main process fails it is assumed that the task
has some laxity as to when it will complete. The amount of laxity is
bounded by the task's targeted response time, which is the time at
which the task must be completed regardless of failure. The targeted
response time is typically represented as a laxity factor, $\alpha$,
of the minimum response time. For example if the minimum response time
is 100 seconds and the targeted response time is 125 seconds, the
laxity factor is 1.25.

We propose two different schemes for for applying lazy shadowing to
high performance computing.
\begin{itemize}
\item 
Energy Optimal Replication - Shadow execution speeds are those that
minimize the consumed energy and guarantee completion by the targeted
response time. This requires us to find $\sigma_b$ and $\sigma_a$ that
solve Equation~\ref{optimization_problem}.
\item 
Stretched Replication - Shadow execution is set to a single speed that
guarantees completion by the targeted response time, $\sigma_b =
\sigma_a = W/R$.
%\item 
%Minimum Work Replication - Shadow execution speed before failure,
%$\sigma_b$, is set to the minimum execution speed that enables the
%shadow to still met the targeted response time. We will show that this
%method is typically energy optimal.
\end{itemize}
The remainder of this section discusses the solution to finding
$\sigma_b$ and $\sigma_a$ for energy optimal replication. 
 

The last constraint in Equation~\ref{optimization_problem} requires that
if the main process fails the shadow process must be able to complete
the given work, $W$, by the targeted response time, $R$. 
The effect of this constraint on the execution speeds depends on the failure 
detection time $t_f$. Since $\sigma_b$ would be set to slower to save energy,
the larger $t_f$ is, the less time is left for the shadow to catch up, and thus 
$\sigma_a$ needs to be larger.
To enforce this constraint for all possible values of $t_f$, we transform it to
the following inequality:
\begin{equation}
\label{work_constraint}
t_c*\sigma_b+(R-t_c)*\sigma_{max} \geq W 
\end{equation}
The intuition for this constraint is that in the worst case the shadow
will have to execute at the maximum possible speed after failure to
achieve the targeted response time. This enforces the constraint such
that if the main process fails at the very last time point, $t_c$,
then the shadow process will still be able to complete the work by the
targeted response time. This places a lower bound on the value for
$\sigma_b$. With this modification, we can use the MATLAB Optimization 
Toolbox to solve Equation~\ref{optimization_problem} and derive the optimal
$\sigma_m$, $\sigma_b$, and $\sigma_a$.



\section{Analysis}
\label{analysis}
Using the energy model and constraints described in the previous
 sections we compare the energy consumed by the optimal energy shadow,
 pure replication, stretched replication and delayed re-execution. For
 the optimal energy shadow we rely upon numerical non-linear
 optimization techniques to find the speeds, $\sigma_m$ and
 $\sigma_b$, that minimize the consumed energy. All results presented
 used the Minimize function available in Mathmatica.

\subsection{Replication vs. Re-execution vs. Optimal Energy Shadow}

Despite improvements in the checkpointing and rollback scheme it has
been shown to not scale as the probability of failure increases. Given
this it is has been suggested that pure process replication or process
re-execution should be used to provide fault tolerance. In this
section we will compare the energy consumed by pure replication,
stretched replication, re-execution and optimal energy shadow.

Pure replication is represented in our model by simply letting all the
speeds equal the maximum speed, $\sigma_m = \sigma_b= \sigma_a
= \sigma_{max}$. For stretched replication we let all execution speeds
be equal $\sigma_m=\sigma_b=\sigma_a=W/R$. Process re-execution simply
re-executes a process if the main process fails. Using our energy
model this is equivalent to setting $\sigma_b$ to zero, therefore no
work is done by the shadow until failure. There are several different
ways of selecting the execution speed of the main process but for our
analysis we let it be the slowest possible speed that allows for
re-execution given $\sigma_{max}$. Therefore, for re-execution
$\sigma_m=W/(R-(W⁄\sigma_{max} ))$.

We find that optimal energy shadow computing consistently out performs
or matches the energy consumed by all other schemes while continuing
to deliver the same fault tolerance
level. Figure \ref{energy_savings_opt_replication_rexecution_grid}
demonstrates this by showing the energy consumption as a function of
the rate of failure, $\lambda$. In this analysis we vary the targeted
response, $R$, specifically we vary $R=\alpha*W/\sigma_{max}$ , where
$\alpha≥1$. Without loss of generality we let $\sigma_{max}=1$.

The first observation is that when $\alpha=1.00$ all fault tolerant
schemes consume the same energy. This is expected since all processes
must work at $\sigma_{max}$ to complete by the targeted response time,
$R$. However, as we increase $\alpha$, thus increasing the slack, we
observe that optimal energy shadow saves as much as 58\% of the energy
consumed by pure replication. The other important observation is that
stretched replication uses almost the same amount of energy as that of
the optimal energy shadow. For all values of $\alpha$ optimal energy shadow
and stretched replication converge as the rate of failure
increases. The reason for this convergence is that as the likelihood
of the main process failing increases the optimal shadow’s execution
speeds evenly spreads the work out, thus approaching the same speed
used in replication, $W/R$.

Re-execution can only be achieved when the targeted response time
 allows enough time to re-execute the task. In other words the system
 must have enough slack to re-execute the task. Because we have fixed
 $\sigma_{max}=1$, we can only achieve re-execution when $\alpha \ge 2$. It can
 observed in the last figure that both stretched replication and
 optimal energy shadow consistently outperforms re-execution.


\begin{figure*}[hHtb]
\centering
\psfig{figure=diagrams/energy_opt_replication_rexecution_grid.eps,width=\textwidth}
\caption { Energy comparison between optimal energy, pure replication and re-execution.}
\label{energy_savings_opt_replication_rexecution_grid}
\end{figure*}

\subsection{System Slack}

Assuming a fixed amount of work, W, there are two ways of producing
slack in the system, one is to increase the targeted response time and
the other is to increase $\sigma_{max}$. The more slack in the system
the more energy optimal shadow computing can save however when
compared to replication and re-execution the way slack is generated
has different effects.

Neither pure replication nor stretched replication can benefit from
increasing the maximum speed, $\sigma_{max}$, given a targeted
response time R. This is because replication will always pick a
consistent speed $\sigma_{max}$ or $\sigma=W/R$. Increasing
$\sigma_{max}$ has no impact on the speed of execution in replication
thus it will have no impact on the energy consumed. However,
increasing $\sigma_{max}$ reduces the energy consumption in optimal
energy shadow computing. This is directly related to the constraint
ensuring that in the worst case the shadow will execute at the maximum
speed possible after failure to achieve the targeted response time. By
increasing $\sigma_{max}$ it allows the shadow to execute at a slower
speed before failure because of the ability to execute faster after
failure. This has the effect of allowing shadow computing to save
energy as the maximum available speed increases. This can be observed
in Figure 5, each graph represents the effect of increasing
$\sigma_{max}$.

\begin{figure*}[hHtb]
\centering
\psfig{figure=diagrams/energy_opt_replication_rexecution_grid_vary_alpha.eps,width=\textwidth}
\caption { Energy comparison between optimal energy, pure replication and re-execution, vary alpha.}
\label{energy_savings_opt_replication_rexecution_grid}
\end{figure*}

Similar to shadow computing, re-execution also benefits from having
faster execution speeds because it can choose a slower execution speed
for its main process. In all cases optimal energy shadow computing
outperforms replication and re-execution. However, observe that as the
rate of failure increases optimal converges to stretched replication
but as the rate of failure decreases it converges to re-execution.

If the maximum execution speed is increasing but the amount of work
and targeted response time is the same then one should choose optimal
energy shadow. This is because it is the only scheme described that
can take full advantage of slack introduce by increasing the execution
speed. Thus you can save energy by harnessing the fact that your
underling architecture can go faster.


\subsection{Job Size}

For this analysis we wanted to understand the relationship between
failure rates and job size as it relates to energy consumption. To do
this we rewrite lambda as the mean time between failure
(MTFB). Because we are using the exponential probability distribution
we know the mean value is given by $1/\lambda$. We then let this
represent the number of seconds between failure such that the MTFB is
the number of days between failure. Similarly, we scale the size of
the job to be number hours/days. This allows us to observe the energy
savings in a realistic context.

\begin{figure*}[hHtb]
\centering
\psfig{figure=diagrams/energy_opt_replication_rexecution_grid_vary_mtbf.eps,width=\textwidth}
\caption { Energy comparison between optimal energy, pure replication and re-execution, vary alpha.}
\label{energy_savings_opt_replication_rexecution_grid}
\end{figure*}

In Figure 5, we show the energy savings using energy optimal shadow
computing over the energy consumed using pure replication. The first
observation is that we consistently save 20-50\% energy. We show these
results for various values of α; again it can be observed that, as the
slack increases, shadow computing achieves more energy
saving. Conversely, as the size of the job increases, the savings
decrease; this is due to a decrease of the slack in the system.


% LocalWords: mtbf megawatt


\section{Conclusion}
\label{conclusion}

In this paper we have introduced shadow computing, an energy efficient
method to provide fault tolerate execution without the limitations of
checkpointing. We then compared this to other known methods,
replication and re-execution, and concluded that shadow computing is
always more energy efficient. We also observed that the amount of
energy saving is highly dependent upon the rate of failure and the
amount of slack present in the system.

Fully harnessing the potential of shadow computing to deal with
failures brings about several challenging questions that need to be
addressed: How can this concept be used to improve fault detection and
layer coordination, understanding faults and silent errors and
improving situational awareness? What level of synchronization is
required between the main process and its associated shadow processes
to minimize impact on other application processes? What state, if any,
must be saved to ensure “smooth” transition to the primary shadow
process upon failure of the main process?  Future work will be focused
on investigating these questions for different types of failure to
better understand the advantages and limitations of this approach to
achieve high levels of fault-tolerance in extreme scale cloud
computing environments.


\bibliographystyle{abbrv}
%\bibliographystyle{ieee}
\bibliography{ics2013}

\end{document}
