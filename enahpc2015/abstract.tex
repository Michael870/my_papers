Current fault-tolerance frameworks are designed to deal with both
physical and logical fail stop errors and typically rely on the
classic checkpoint-restart approach for recovery. The proposed
solutions differ in their design, the type of faults they manage and
the fault tolerance protocol they use.  The major shortcoming of
checkpoint-restart stems from the potentially prohibitive costs, in
terms of execution time and energy consumption. Since faults are both
voluminous and diverse, the inherent instability of future large-scale
high performance computing infrastructure calls for a wholesale
reconsideration of the fault tolerance problem and the exploration of
radically different approaches that go beyond adapting or optimizing
existing checkpointing and rollback techniques. To this end, we
propose Shadow Computing, a novel energy-aware computational model to
support scalable fault-tolerance in future large-scale
high-performance computing infrastructure. We present three techniques
for applying shadow computing to high performance computing: energy
optimal replication, stretched replication, and minimum work
replication. All of the proposed schemes provide the same
fault-tolerance as pure replication while providing significant energy
savings.
