Using the energy model and solutions described in the previous
sections we now compare the energy consumed by optimal energy shadow,
stretched replication and minimum work replication. Without loss of
generality, we assume $\sigma_{max}$ is normalized such that
$\sigma_{max}=1$. Execution speeds will be presented as factors of
that maximum speed. In addition to comparing energy savings we also
consider the energy models sensitivity to input variables.

\subsection{Energy Savings}

The first and obvious analysis is to compare the expected energy
consumption of shadow computing with pure replication. We represent
``Energy Savings'' as a percentage of the amount of energy saved when
using shadow computing versus pure replication. This analysis makes use
of the energy model described in Section \ref{model} to calculate the
total expected energy of a given task, which is assumed to have a main
process and one shadow process. To calculate the energy consumed by
pure replication we let $\sigma_m = \sigma_b = \sigma_a
=\sigma_{max}$. When evaluating our methods we let the execution
speeds be determined by the equations derived in Section
\ref{application_to_hpc}.

\subsubsection{Optimal Shadow}

In figure \ref{energy_savings_pure_vs_optimal_w_12} we show the energy
savings for a 12 hour task for various targeted response time factors,
$\alpha$. Recall that $\alpha$ is the laxity we allow our tasks if
there is failure in the main process. Note that if we let $\alpha=1.0$
this is equivalent to pure replication and shadow computing provides
no energy savings. This comparison looks at the savings as a function
of the mean time between failure, MTBF, of individual computing
nodes. We vary MTBF from $1/2$ day to 25 days, this is short given
todays component failure rates, however as the MTBF increases the
amount of energy savings continue asymptotically to the values presented
at 25 days.

\begin{figure}[hHtb]
\centering
\psfig{figure=diagrams/energy_savings_pure_vs_optimal_w_12.eps,width=3.5in}
\caption { Energy savings for optimal shadow vs pure replication for a 12 hour task. }
\label{energy_savings_pure_vs_optimal_w_12}
\end{figure}

The first observation is that if we allow the task to take twice as
long upon failure, $\alpha=2.0$, we can approach the maximum possible
savings of 50\%. Even if we have very little laxity, $\alpha=1.25$,
optimal shadow still saves upto 21\% compared to pure replication.
The maximum of 50\% savings is because if the shadow process is given
twice as much time to finish then optimal replication does little or
no work until a failure occurs. The shadow will then execute at the
maximum speed to achieve the targeted response time. If the likelihood
of failure is high, low MTBF, then the optimal shadow will do more
work prior to failure therefore expending more energy. You can see
this effect on $\sigma_b$ in Table
\ref{table_of_sigma_b}. Observing at the bottom right corner you
will notice that the shadow does very little if the likelihood of
failure is low and we have enough time to ``re-execute'' the entire
task. As we increase the failure rate or decrease laxity the optimal
energy shadow increases the amount of work done prior to failure.

\begin{table}[ht]
\centering
\begin{tabular}{c|r r r r} % centered columns (5 columns)
$\alpha$ / MTFB & 1.25 & 1.50 & 1.75 & 2.00 \\
\hline 
0.5  & 0.80 & 0.67 & 0.57 & 0.500 \\ 
1.0  & 0.80 & 0.67 & 0.57 & 0.500 \\
5.0  & 0.78 & 0.65 & 0.55 & 0.484 \\
10.0 & 0.75 & 0.56 & 0.47 & 0.407 \\
25.0 & 0.75 & 0.50 & 0.30 & 0.251 \\
1yr  & 0.75 & 0.50 & 0.25 & 0.024 \\
3yr  & 0.75 & 0.50 & 0.25 & 0.012 \\
5yr  & 0.75 & 0.50 & 0.25 & 0.005 \\ [1ex]      % [1ex] adds vertical space
\hline 
\end{tabular}
\caption{Optimal energy values of $\sigma_b$, $\sigma_{max}=1.0$.}
\label{table_of_sigma_b}
\end{table}

\subsubsection{Stretched Replication}

Recall that stretched replication sets the shadow's to a constant
execution speed which is the minimum speed necessary to complete the
job by the targeted response time, $\sigma_b = \sigma_a = W/R$. Figure
\ref{energy_savings_replication_stretched_optimal_compare_w_12}
compares this method to the savings achieved by using the energy
optimal solution. Despite being naive it still provides significant
energy savings, between 17\%-37\% depending on the system laxity,
represented by $\alpha$.

\begin{figure}[hHtb]
\centering
\psfig{figure=diagrams/energy_savings_pure_vs_stretched_vs_opt_w_12hrs.eps,width=3.2in}
\caption { Energy savings for stretched and optimal replication. W=12 hours }
\label{energy_savings_replication_stretched_optimal_compare_w_12}
\end{figure}

The important observation is that the energy savings of stretched
replication nearly matches energy optimal replication. When MTBF is
low, stretched replication matches the optimal performance. This is
because the optimal solution drives the shadow to complete as much
work as possible because failure is very likely to occur, thus
producing the same execution speeds as stretched replication. As MTBF
increases the optimal will begin to perform better than stretched
replication, this is because the optimal solution will reduce the
amount of work done before failure. This ``gap'' in energy savings
reaches a stable distance as MTBF increases. Also, note that the
decrease in $\alpha$ and the ``gap'' are correlated, demonstrated in Figure
\ref{stacked_bar_streteched_replication_gap}.  The ``gap'' shrinks
from 13\% at $\alpha=2.0$ to less than 1\% at $\alpha=1.10$.

\begin{figure}[hHtb]
\centering
\psfig{figure=diagrams/stacked_bar_stretched_vs_opt_w_12hrs_mtbf_10.eps,width=2.8in}
\caption { Energy savings ``gap'' between stretched and optimal. W=12 hours, MTBF=10days }
\label{stacked_bar_streteched_replication_gap}
\end{figure}

\subsubsection{Minimum Work Replication}

As described in detail in Section \ref{minium_work_replication}
minimum-work replication is motivated by the assumption that the task
size will typically be much less than the MTBF of individual
nodes. When this is the case the optimal energy replication ``hits''
the work constraint because the likelihood of failure is low. Even
when the task size is significantly larger than the individual nodes
MTBF optimal replication is only slightly better than minimum work
replication, see Figure
\ref{stacked_bar_min_work_replication_gap}.

\begin{figure}[hHtb]
\centering
\psfig{figure=diagrams/stacked_bar_minimum_vs_opt_w_48hrs_mtbf_half_day.eps,width=2.8in}
\caption { Energy savings ``gap'' between minimum-work and optimal replication. W=48 hours, MTBF=0.5days }
\label{stacked_bar_min_work_replication_gap}
\end{figure}

\subsection{Sensitivity to Alpha}

The amount of energy savings achieved by either stretched or optimal
shadow is highly dependent upon the amount of laxity the system has in
the event of failure, represented by $\alpha$. To see this effect we
fix the MTBF and then observe the energy savings as a function of
$\alpha$, see Figure
\ref{energy_savings_optimal_vs_pure_vary_alpha}. The major observation
here is that increasing $\alpha$ has largely a linear effect on the
energy savings up until $\alpha=2.0$ at which point the gain minimally
increases to the maximum of 50\% savings. While the job size does have
an effect on actual energy savings, the trend is the same.

\begin{figure}[hHtb]
\centering
\psfig{figure=diagrams/energy_savings_pure_vs_opt_mtbf_1day.eps,width=3.2in}
\caption { Energy savings for optimal shadow vs pure replication as a function of $\alpha$. MTBF=1day}
\label{energy_savings_optimal_vs_pure_vary_alpha}
\end{figure}

\subsection{Sensitivity to Task Size}
\label{job_size_vs_energy_savings}

The task size has little effect on the trends observed; however, it
does change the individual values and inflection points. Due to space
reasons we omitted additional figures that demonstrate this however
you can observe the behavior in Figure
\ref{energy_savings_optimal_vs_pure_vary_alpha}, which shows that the effect
task size has on the energy savings is minimal as we vary
$\alpha$. Intuitively, there should be some impact because the longer a
task runs the more likely it will be interrupted by a component
failure. In Figure
\ref{energy_savings_pure_vs_optimal_MTBF_10} we look at this in more
detail by considering energy savings for a fixed MTBF and vary the
task size. As expected when task size approaches the MTBF the energy
savings also decreases, due to the optimal replication adjusting to
account for the increased likelihood of task interruption by
increasing the speed of the shadow before failure. Once the task size
exceeds the MTBF, then the energy savings begin to increase again. If
the main process is going to fail, it will do so more quickly and
therefore will consume less energy. Our current model only assumes one
node failure which is why the energy savings continues to increase
beyond the MTBF, we discuss this as a potential future work in our
conclusion.

%% removed for space reasons
%%\begin{figure*}[hHtb]
%%\centering
%%\psfig{figure=diagrams/energy_savings_pure_vs_opt_high_gear.eps,width=\textwidth}
%%\caption { Energy savings for optimal shadow vs pure replication. }
%%\label{energy_savings_pure_vs_optimal_4up}
%%\end{figure*}

\begin{figure}[hH]
\centering
\psfig{figure=diagrams/energy_savings_pure_vs_optimal_mtbf_10.eps,width=3.2in}
\caption { Energy savings for optimal shadow vs pure replication. MTBF = 10days}
\label{energy_savings_pure_vs_optimal_MTBF_10}
\end{figure}

% LocalWords: mtbf megawatt
