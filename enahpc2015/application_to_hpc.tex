One of the primary goals of high performance computing is to achieve
the maximum possible throughput of the system. Thus when we
apply shadow computing to this environment we assume that the
execution speed of the main process should be the maximum possible
execution speed, $\sigma_m=\sigma_{max}$. If no failure occurs then
the task will be completed as soon as possible, known as the minimum
response time. If the main process fails it is assumed that the task
has some laxity as to when it will complete. The amount of laxity is
bounded by the task's targeted response time, which is the time at
which the task must be completed regardless of failure. The targeted
response time is typically represented as a laxity factor, $\alpha$,
of the minimum response time. For example if the minimum response time
is 100 seconds and the targeted response time is 125 seconds, the
laxity factor is 1.25.

We propose three different schemes for for applying shadow computing to
high performance computing.
\begin{itemize}
\item 
Energy Optimal Replication - Shadow execution speeds are those that
minimize the consumed energy and guarantee completion by the targeted
response time. This requires us to find $\sigma_b$ and $\sigma_a$ that
minimize equation \ref{energy_model}.
\item 
Stretched Replication - Shadow execution is set to a single speed that
guarantees completion by the targeted response time, $\sigma_b =
\sigma_a = W/R$.
\item 
Minimum Work Replication - Shadow execution speed before failure,
$\sigma_b$, is set to the minimum execution speed that enables the
shadow to still met the targeted response time. We will show that this
method is typically energy optimal.
\end{itemize}
The remainder of this section presents a solution to finding
$\sigma_b$ and $\sigma_a$ for energy optimal replication. We start by
specifying our failure and power model then derive a closed form
solution that produces these execution speeds.

\subsection{Failure Probability}

In our energy model we assume failure is described using any
probability density function. In the remaining sections we use the
exponential probability density function because it is widely accepted
as a model representative of independent node failures, therefore
$f(t)=e^{-\lambda t}$. This distribution also has the benefit of being
differentiable allowing us to produce a closed form equation.

\subsection{Energy Function}

In the remainder of this paper we need to select values for our power
model. In section \ref{power_model} we defined the energy function,
$E(\sigma, T)$, as the integral of the power function over the
interval T. In the remaining sections we assume that the power
function is defined as the squared value of the speed.
\begin{equation}
P(\sigma)=\sigma^2
\end{equation}
Thus the energy function is defined as the following:
\begin{equation}
E(\sigma,T)=\int_{t=0}^T \sigma^2 dt = \sigma^2t
\end{equation}

\subsection{Optimal Execution Speeds}
\label{finding_execution_speeds}

Once we have a known failure and power model we can begin to solve the
optimization problem. We first make the observation that the speed of
the shadow after failure, $\sigma_a$, is dependent upon the the shadow
speed before failure, $\sigma_b$, and the time of failure, $t_f$. It
can trivially be shown that to conserve the most energy one would let
$\sigma_a$ be the slowest possible speed to finish by the targeted
response time, R. Therefore $\sigma_a$ is no longer constant with
respect to the time of failure. From this observation the following
value of $\sigma_a$ can be derived.
\begin{equation}
\label{optimal_sigma_a}
\sigma_a=(W-\sigma_b*t_f)/(R-t_f)
\end{equation}
Substituting this value into the energy model allows us to reduce the
number of variables in our objective function, specifically this
reduces the output of our optimization to one variable, $\sigma_b$. 

The one constraint we have not considered in our optimization is that
if the main process fails the shadow process must be able to complete
the given work, $W$, by the targeted response time, $R$. This is known
as the ``work constraint'' and is represented by the following
inequality.
\begin{equation}
\label{work_constraint}
t_c*\sigma_b+(R-t_c)*\sigma_{max} \geq W 
\end{equation}
The intuition for this constraint is that in the worst case the shadow
will have to execute at the maximum possible speed after failure to
achieve the targeted response time. This enforces the constraint such
that if the main process fails at the very last time point, $t_c$,
then the shadow process will still be able to complete the work by the
targeted response time. This places a lower bound on the value for
$\sigma_b$ and as we will later show typically determines the value of
$\sigma_b$.

\subsection{Optimal Shadow Computing - Solution}
\label{closed_form}

In the preceding sections we have specified components of our general
energy model found in Equation \ref{energy_model}. Our optimization
problem is now well defined as finding a value for $\sigma_b$ that
minimizes the following function.
\begin{equation}
\label{energy_model_sub_sigma_a}
\begin{split}
 & \int_{t=0}^{t_c}(\sigma_{max}^2t+\sigma_b^2t) e^{-\lambda t}dt \\
+& \int_{t=0}^{t_c}{[(W-\sigma_b*t_f)/(R-t_f)]}^2 (R-t) e^{-\lambda t}dt \\
+& (1-\int_{t=0}^{t_c} e^{-\lambda t} dt)(\sigma_{max}^2 t_c + \sigma_b^2 t_c)
\end{split}
\end{equation}
Also note that $t_c=W\sigma_{max}$ because the amount of work, $W$, is
given and main process execution speed is defined as
$\sigma_m=\sigma_{max}$. After solving equation
\ref{energy_model_sub_sigma_a} we can then take the derivative of the
result with respect to $\sigma_b$ and solve for $\sigma_b$.
\begin{equation}
\label{sigma_b_optimal_without_constraint}
\sigma_b =\frac{
\begin{split}
(&W  (-E^{R \lambda} + E^{R \lambda + W \lambda} -
  E^{W \lambda} R \lambda Ei[R \lambda] + \\
 &  E^{W \lambda} R \lambda Ei[(R - W) \lambda]))
\end{split}
}
{
\begin{split}
(&E^{R \lambda} R - E^{R \lambda + W \lambda} r + 
 E^{R \lambda} W + \\
 &E^{W \lambda} R^2 \lambda Ei[R \lambda] - \\
 &E^{W \lambda} R^2 \lambda Ei[(R - W) \lambda])
\end{split}
}
\end{equation}
In this equation $Ei[x]$ represents the exponential integral
function. 

Equation \ref{sigma_b_optimal_without_constraint} gives us the energy
optimal execution speed for the shadow before a failure is detected,
$\sigma_b$. However this execution speed does not guarantee that the
shadow will be able to complete the work by the targeted response
time, $R$. To ensure that the shadow completes enough work we impose
the ``work constraint'' found in Equation \ref{work_constraint}. By
solving the ``work constraint'' for $\sigma_b$ we can use this as a
lower bound; thus producing the final energy optimal execution speed
for the shadow before failure, $\tilde{\sigma_b}$.

\begin{equation}
\label{closed_form_sigma_b}
\tilde{\sigma_b}=Max[\sigma_b, (2 W \sigma_{max} - R \sigma_{max}^2)/W]
\end{equation}

Our final solution now outputs the shadows execution speed before
failure, $\sigma_b$, given the following inputs:
\begin{itemize}
\item
W - The amount of work necessary to complete the task.
\item
R - The targeted response time for the task.
\item
$\sigma_{max}$ - The maximum possible execution speed.
\item
$\lambda$ - Parameter of our probability density function used to model failure.
\end{itemize}

\subsection{Minimum Work Replication}
\label{minium_work_replication}

Most of the time the task size will be much smaller than the mean time
between failure, MTBF, of individual computing nodes. Given this one
can observe that the optimal solution will typically be bounded by the
``work constraint''. Therefore we propose that instead of calculating
energy optimal execution speed one can achieve near optimal
performance by letting the execution speed be derived by the
constraint directly. Minimum work replication can be thought of as a
simplification of the optimal solution derived in the previous
section. This simplification is only feasible because we have fixed
the execution speed of the main process.

\begin{equation}
\label{minimum_work_sigma_b}
\hat{\sigma_b}=(2 W \sigma_{max} - R \sigma_{max}^2)/W
\end{equation}

% LocalWords: mtbf ei dt
