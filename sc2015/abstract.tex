With the concerted efforts from researchers in hardware, software, algorithm, and data management, HPC is moving towards its next generation--extreme-scale computing. Featuring a computing capacity of  quintillion ($10^{18}$) FLOPS, extreme-scale computing will be a significant achievement in computer science and engineering. %, promising new discoveries in computation-intensive research areas, such as astronomy, biology, national security, etc. 
As we approach the new era of computing, however, several daunting scalability challenges need to be conquered. Delivering extreme-scale performance will require a billion-core computing infrastructure, along with a dramatic increase in the number of memory modules, communications devices and storage components. At such a large scale, failure would become a norm instead of exception, driving the system to significantly lower efficiency with unprecedented amount of power consumption. %The frequency and diversity of failures,  as well as the challenge of power, call for rethinking of the fault tolerance problem. 

To tackle this challenge, we propose a proactive, power-aware resilience framework, referred to as Lazy Shadowing, as an efficient and scalable approach to achieve high-levels of resilience, through forward progress, in extreme-scale, failure-prone computing environments. 
Lazy Shadowing associates with each process a ``shadow" process that executes at a reduced rate, and overlaps the recovery time after each failure with the time needed to roll forward the shadows to a consistent state.
Compared to existing fault tolerance methods, our approach can achieve 20\% energy saving with reduced solution time at scale.
