%this algorithm removed shadowed set rejuvenation


In this section, we introduced the basic concept of shadowing and discussed several separate techniques that enable Lazy Shadowing to execute tightly coupled parallel applications efficiently and reliably. In the following, we give an algorithm, as shown in Algorithm 1, that combines all the techniques to illustrate how Lazy Shadowing can be applied to HPC. The algorithm assumes that the total workload of size $W$ has been given, and the number of cores for executing main processes, $M$, and for executing shadow processes, $S$ have been determined based on some objective. %In addition, the forced shadow leaping interval, $T$, should be pre-determined according to the application's message logging characteristics.


\begin{algorithm}[t]
  \SetKwInOut{Input}{input}
  \SetKwInOut{Output}{output}
  \caption{Lazy Shadowing}
  \Input{$W, M, S$}
  \Output{Application execution status}
  \BlankLine
  split $W$ into $M$ tasks\; \nllabel{line:split} 
  cluster $M$ tasks into $S$ shadowed sets\; \nllabel{line:cluster} 
  %$T_l \leftarrow T+t_{now}$\;%, $C_{vul} \gets 0$
  start M(i) and S(i) for each task T(i)\;
    \While{execution not done}
    {
        \If{failure detected in shadowed set SS(j)} %\nllabel{line:if_start_1} 
        {
            \nllabel{line:if_start_1} 
            \eIf{SS(j) is vulnerable}
            {
                notify ``Application failure"\;
                terminate all mains and shadows\;
                repair all failures\;
                \emph{Restart execution}\; %\Comment{re-execution}
            }
            {
                mark SS(j) as vulnerable\;
                %\State $C_{vul} \gets C_{vul} + 1$
                %\If{$C_{vul} == V$}
                %    \State perform shadowed set rejuvenation   
                %\EndIf 
                \If{failure happened to a main M(k)} 
                {
                    promote S(k) to M(k)\;
                    perform shadow leaping\; %failure induced shadow leaping
                    %$T_l \leftarrow T+t_{now}$\;
                }
            }
        }  
        \nllabel{line:if_end_1}
        %\If{$t_{now} \ge T_l$} %
        %{
        %    \nllabel{line:if_start_2} 
        %    perform shadow leaping\;
        %    $T_l \leftarrow T+t_{now}$\;
        %    \nllabel{line:if_end_2}
        %} % 
        %\nllabel{line:if_end_2}
    }
    output ``Application completes"\;
  \label{al:ls}
\end{algorithm}



The total workload is split into $M$ parallel tasks, as shown in line~\ref{line:split}, so that they can be simultaneously executed with $M$
main processes on the $M$ dedicated cores. Afterwards, to utilize the extra $S$ cores for shadowing of 
all the $M$ main processes, in line~\ref{line:cluster} the $M$ tasks is clustered into $S$ shadowed sets, each of which 
has $\alpha=M/S$ cores for $\alpha$ mains and 1 core for all the shadows of the mains in this shadowed set. Then 
all the processes start executing their tasks simultaneously. %, with the forced shadow leaping point 
($T_l$) set to T time from now. % and the counter for vulnerable shadowed sets cleared. 
During the execution,
the system runs a failure monitor that triggers corresponding actions when a failure is detected, as shown from line~\ref{line:if_start_1} to~\ref{line:if_end_1}. A failure may trigger different actions, depending on its type and order. 
If the failure happens in a vulnerable shadowed set (e.g., SS(j)), this would result in an application failure since 
all main process in SS(j) have lost their shadow processes. As a result, the system would terminate all running 
main and shadow processes, repair existing failures by either rebooting or replacing, and restart the 
execution from the very beginning. %(this assumes that checkpointing is not used). 
On the other hand, failure in a non-vulnerable shadowed set
can be tolerated by Lazy Shadowing and does not translate to an application failure. In this case, 
the shadowed set in question (SS(j)) is marked as vulnerable. When a failure happens in a non-vulnerable shadowed set, there are two possible cases that either 
a main fails or a shadow fails. These two cases need to be taken care of separately, because a shadow 
failure does not impact the normal execution of the mains and thus can be ignored, while a main failure
would interrupt the application's execution as the other mains are forced to stop and wait. When a main process (e.g., M(k)) fails, S(k) is promoted to a main process to recover the execution and all the other shadows in the shadowed set are terminated as a result.
At the same time, a shadow leaping is performed to align the states of all the other shadows with those of their associated mains to make forward progress. %, and the forced shadow leaping point is reset. Along with the failure monitor is an alarm for the forced shadow jumping. 
%If the application runs continuously for T time without a main process failure, the alarm will trigger
%a forced shadow leaping to prevent message queue overflow, as shown in line~\ref{line:if_start_2} to~\ref{line:if_end_2}. 
This process continues until the application is successfully completed.





%\linesnumbered


