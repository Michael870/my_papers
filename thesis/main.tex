%\documentclass[dvipdfm,final]{pittetd}%   If you want to use dvipdfm. The file is to be normally processed (LaTeX), and 
%                                   then the program dvipdfm applied to it. This will create the PDF file with bookmarks
%                                   and links. It will also try to convert any PS graphics included.
\documentclass[pdftex,final]{pittetd}   % If you want to use PDFLaTeX instead. This will create the PDF file directly.
%                                   Processing time can be longer. No PS graphics conversion will take place 
%                                   automatically.
%
%Other options: ma, ms, for Master's. 
%11pt, 10pt, font size (12pt is default). 
%final, makes pittetd's warnings (about things that might go against the Format Guidelines) 
%into error messages. 
%Option 'sectionletters' numbers the chapters with Roman numerals (I, II, etc.), sections with 
%letters (A, B), subsections with numbers (1, 2), and subsubsections with lowercase letters (a, b). 
%The four levels of the enumerate environment receive the same treatment. Within the
%text, however, cross references (\ref} produce `the whole thing,' something like I.A.1 
%instead of only 1.
%
\usewithpatch{graphicx}%            Better \usewithpatch than \usepackage because it makes pittetd look for any 
%                                   available patch for the package. 
\usepackage{afterpage}
\usepackage{listings}
\usepackage[linesnumbered]{algorithm2e}
\usepackage{tabularx}
\usepackage{subfigure}
\usepackage{amsmath,amsthm}%        But you can't use \usewithpatch for several packages as in this line. The search for
%                                   patches has to be then forced through:
\patch{amsmatch}
\patch{amsthm}
%
%\patch{pittdiss}                   If you started writing your thesis with the pittdiss class, this patch makes 
%                                   pittetd interpret pittdiss commands.
\newtheorem{theorem}{Theorem}
\newtheorem{corollary}{Corollary}[theorem]
\renewcommand\thesubfigure{(\alph{subfigure})}
%\patch{pitthesis}                  Analogous for the pitthesis class.
\title[Adaptive and Power-aware Fault Tolerance for Future Extreme-scale Computing]{Adaptive and Power-aware Fault Tolerance for Future Extreme-scale Computing}% The optional argument is the %                                   version of the title that will appear in Acrobat Reader's Document Info dialog box.
\author{Xiaolong Cui}
\degree{B.E. in Computer Science, Xi'an Jiaotong University, 2012 \\ M.S. in Computer Science, University of Pittsburgh, 2017}
%\date{July 20th 1967}%             This date is the date of the thesis defense. Default is \today
\year{2017}                        %pittetd will use the current year unless otherwise indicated. So this command is not
%                                   necessary.
\keywords{Extreme-scale computing, Forward recovery, Reliability, Fault tolerance, Shadow Replication, Leaping}% This list appears in the field 'Keywords' of Acrobat Reader's Document Info
%                                   dialog box, and also, optionally, after the abstract.
\subject{Adaptive and Power-aware Fault Tolerance for Future Extreme-scale Computing}%              This fills in the 'Subject' field in Acrobat Reader's Document Info dialog box.
\school{School of Computing and Information}%    The name of the school will be preceeded by 'the' unless otherwise specified, as in:
%\school[certain]{Department of Computer Science}
%
%\chapterfloats%                    Un-comment this to get figures and tables numbered within chapters.
\begin{document}
\maketitle
%
% For the committee membership page, you have to provide the names and affiliations of the members. The first one will 
% be treated by pittetd as the committee chair (thesis/dissertation advisor).
\committeemember{Dr. Taieb Znati, Departmental of Computer Science, University of Pittsburgh}
\coadvisor{Co-advisor: Dr. Rami Melhem, Department of Computer Science, University of Pittsburgh}%         This is used if there are two advisors.
\committeemember{Dr. Rami Melhem, Department of Computer Science, University of Pittsburgh}%         This is used if there are two advisors.
\committeemember{Dr. John Lange, Department of Computer Science, University of Pittsburgh}
\committeemember{Dr. Esteban Meneses, School of Computing, Costa Rica Institute of Technology} 
% etc., as many as needed. For master's theses, the committee may be omitted, naming only the advisor.
\
% etc., as many as needed. For master's theses, the committee may be omitted, naming only the advisor.
%\school{Department of Computer Science}
\makecommittee
\copyrightpage        %             Uncomment this to get a copyright page.
\begin{abstract}
Two major trends in large-scale computing are the rapid growth in HPC with in particular an international exascale initiative, and the dramatic expansion of Cloud infrastructures accompanied by the Big Data passion. To satisfy the continuous demands for increasing computing capacity, future extreme-scale systems will embrace a multi-fold increase in the number of computing, storage, and communication components, in order to support an unprecedented level of parallelism. Despite the capacity and economies benefits, making the upward transformation to extreme-scale poses numerous scientific and technological challenges, two of which are the power consumption and fault tolerance. With the increase in system scale, failure would become a norm rather than an exception, driving the system to significantly lower efficiency with unforeseen power consumption.

This thesis aims at simultaneously addressing the above two challenges by introducing a novel fault-tolerant computational model, referred to as \textit{Leaping Shadows}. Based on Shadow Replication, Leaping Shadows associates with each main process a suite of coordinated shadow processes, which execute in parallel but at differential rates, to deal with failures and meet the QoS requirements of the underlying application under strict power/energy constraints. In failure-prone extreme-scale computing environments, this new model addresses the limitations of the basic Shadow Replication model, and achieves adaptive and power-aware fault tolerance that is more time and energy efficient than existing techniques. 

In this thesis, we first present an analytical model based optimization framework that demonstrates Shadow Replication's adaptivity and flexibility in achieving multi-dimensional QoS requirements. Then, we introduce Leaping Shadows as a novel power-aware fault tolerance model, which tolerates multiple types of failures, guarantees forward progress, and maintains consistent level of resilience. Lastly, the details of a Leaping Shadows implementation in MPI is discussed, along with extensive performance evaluation that includes comparison to checkpoint/restart. Collectively, these efforts advocate an adaptive and power-aware fault tolerance alternative for future extreme-scale computing.


\end{abstract}
% If you say \begin{abstract}[Keywords:] instead of the simple \begin{abstract}, a list of the keywords is appended.
% The list comes from the \keywords command above.
% The starred version \begin{abstract*} typesets the word `ABSTRACT' on the top of the page
\tableofcontents
\listoftables                     % Pittetd will complain if you tell it to create a list of tables when there are no
%                                   tables (as in this sample file). Uncomment this command if you have tables.
\listoffigures                    % Obvious analogous for figures.
\preface
\vfill
\begin{center}
\textit{For my beloved families.} 
\end{center}
\vfill
% This is the text of the preface, with acknowledgments, dedication, etc. It is optional, and you create, as shown, by 
% just saying \preface and starting the preface's actual text. Note that 'foreword' is no longer acceptable as title
% for this preliminary.
%
%Conventions, such as notation (nomenclature) and abbreviations, don't receive their own preliminary page. They can be included as an appendix, or as part of the introduction.
\chapter{INTRODUCTION}
\label{chapter:intro}
Today's scientific discoveries and business intelligence are driven by high-fidelity, 
large-scale simulation and data analytics. To meet the increasing computing demands from 
virtually every aspect of the society, HPC is continuously evolving to solve more 
complex and challenging problems. On the one hand, national labs and research institutes run HPC on 
supercomputers for scientific breakthroughs and national security. On the other hand, enterprises and 
organizations deploy HPC on small to medium sized clusters to process data and extract insights. 
Recently, the explosively growing machine learning applications have increased the adoption as well as 
impact of HPC as they also exploit parallelism and hardware acceleration to speed up the processing of 
massive amount of data.


HPC workloads have traditionally been run only on bare-metal, unvirtualized hardware to drive maximum 
performance. 
The roadblock to virtualization was due to the concern that the extra hypervisor layer could introduce 
performance overhead. 
%The concern was that virtualization could introduce performance overhead due to the extra software 
%layer of hypervisor. 
However, this has started to change with the introduction of increasingly sophisticated 
hardware support for virtualization and software optimization~\cite{madukkarumukumana2008resource,bugnion2017hardware}. Performance of 
these highly parallel HPC workloads has increased dramatically over the last decade, 
enabling organizations to begin to embrace the numerous benefits that a virtualization platform can 
offer~\cite{michael2018overcommit}. As a result, we are witnessing a popular trend that enterprises convert 
their on-prem bare-metal clusters to virtualized, shared private cloud. For instance, the Johns Hopkins 
University Applied Physics Laboratory recently virtualized their 3728-core bare-metal cluster 
to share between Windows and Linux users. The reported improvement in resource utilization 
ranges from 9.1\% to 29.2\%, and simulations speed up by 4\% on average~\cite{vmware2017josh}.

At the same time, public cloud, such as Amazon AWS and Google GCP, is becoming a popular alternative for 
HPC practitioners. Recent studies show that the usage of public cloud has grown more than five-fold among all HPC 
sites worldwide, from 13\% in 2011 to 74\% in 2018~\cite{hyperion2019}.
With virtually unlimited scalability and on-demand resource subscription, public cloud starts to host 
compute- and data-intensive workloads across various industry verticals. These workloads span the traditional HPC 
applications, like genomics and 
weather prediction, as well as emerging applications, like machine learning and deep learning. 

There is a fruitful body of research on resource management in 
Cloud Computing~\cite{singh2016survey,zhan2015cloud,gill2018chopper}. Dynamic resource scheduling and 
load balancing are used 
to maximize system utilization and efficiency~\cite{adhikari2018heuristic,panwar2015load}. These techniques, however, 
are not straightforward to apply to HPC workloads which are highly sensitive to resource change and interference. 
Actually, resource management has been identified as one of the open 
challenges for HPC cloud~\cite{netto2018hpc}. 
Currently, cloud service providers (CSPs) are often limited to statically and conservatively reserve 
resources based on peak resource requirements to respect service level agreements (SLAs). For example, Microsoft Azure 
allocates dedicated supercomputers from Cray, and Amazon AWS offers dedicated nodes for full-size VMs. 
% allocate physical resources
% Despite the 
% numerous benefits promised by Cloud Computing, however, cloud service providers (CSPs) are often limited to statically 
% allocate physical resources to HPC tenants in order to avoid performance interference and enforce 
% service level agreements (SLAs). 
This essentially offsets 
the elasticity and efficiency benefits of the Cloud Computing business model. 

In this paper, we present \textit{virtual throughput clusters (VTC)} as a novel approach for cloud 
resource allocation to efficiently and effectively support 
HPC workloads with multi-tenancy. Based on virtual machine (VM), VTC goes beyond traditional way of 
statically splitting resources among tenants and applies resource over-commitment to optimize 
system utilization and throughput. By giving each tenant a virtual cluster that mimics the 
underlying physical cluster, VTC delegates the resource management task 
to the hypervisor to improve flexibility as well as efficiency. When all tenants are busy consuming their cycles, 
VTC guarantees that each tenant is getting his/her fair share according to pre-defined SLA terms. When 
some tenant is not fully using the allocated resources, VTC takes advantage of the work-conserving 
property of the hypervisor scheduler to assign the idle resources to other tenant(s) who can benefit 
from additional resources. Consequently, CSPs can ensure quality-of-service while maximizing 
system utilization. 

The rest of the paper is organized as follows. 
Section II provides background and motivation. Section III introduces the design of VTC, followed by validation 
and empirical evaluation results in Section IV. Section V concludes this work and points out future directions.

\chapter{BACKGROUND}
\label{chapter:background}
\section{DRAM Structure and Organization}
DRAM-based main memory system is logically organized as a hierarchy of channels, ranks and banks, as illustrated by Figure \ref{fig:dram_org}.
Bank is the smallest structure to be accessed in parallel with each other, which is termed as bank-level parallelism \cite{ISCA08:blp, MICRO09:blp}.
And, rank is formed by clustering multiple, usually eight
\footnote{For illustration purpose, we assume the memory chips are x8, i.e., 8 data I/O pins. The overall structure keeps the same for x4 and x16, except the number of chips in a rank.}
, banks which operate in lockstep, i.e., all banks in a rank respond to a single command. 
Lastly, one channel is composed of an on-chip memory controller and several ranks that share the same narrow command/address and wide data bus.

\begin{figure}
 \centering
  \begin{subfigure}{.34\textwidth}
    \centering
    	\includegraphics[width=\linewidth]{figures/dram_org.pdf}\\
    \caption{Logical hierarchy}
    \label{fig:dram_org}
  \end{subfigure}
%
  \begin{subfigure}{.39\textwidth}
    \centering
    	\includegraphics[width=\linewidth]{figures/dram_rank.pdf}\\
    \caption{Rank organization}
    \label{fig:dram_rank}
  \end{subfigure}
  \vspace{-0.45in}
  \caption{DRAM high-level structure.}
  \label{fig:dram}
\end{figure}

Physically, a DRAM rank is composed of multiple chips, inside which eight banks are deployed as cell arrays.
The logical bank, as shown in Figure \ref{fig:dram_org}, is physically made up of the same numbered bank from all chips.
For instance, {\tt bank 0} of a rank contains {\tt bank 0} 
\footnote{Without specific comment in the rest of the thesis, {\tt bank} refers to a logical bank, which is across chips in a rank; for banks residing in a chip, we would specifically call as \ul{chip bank} to differentiate. The same rule goes with {\tt row}.}
residing in all chips in the rank. 
Likewise, a DRAM row is dispersed across chips, as shown in Figure \ref{fig:dram_rank}. 
In normal accesses to a rank, each chip provides 8 bits at a time simultaneously, which together satisfy the total data bus width of 64-bit.
In addition, to amortize memory access overhead on processor side and also to bridge the giant gap between DRAM core frequency (about 200 MHz) and bus frequency (over 1000 MHz), n-bit prefetch and burst access is supported \cite{ISCA11:agms, ISCA14:half_dram}. n is 8 for commodity DDR3, which translates into a granularity of 64B (64b$\times$8), the popular cache block size. 

\begin{figure}
 \centering
  \begin{subfigure}{.36\textwidth}
    \centering
    	\includegraphics[width=\linewidth]{figures/dram_array.pdf}\\
    \caption{Array}
    \label{fig:dram_array}
  \end{subfigure}
%
  \begin{subfigure}{.16\textwidth}
    \centering
    	\includegraphics[width=\linewidth]{figures/dram_cell.pdf}\\
    \caption{Cell}
    \label{fig:dram_cell}
  \end{subfigure}
  %
  \begin{subfigure}{.16\textwidth}
    \centering
    	\includegraphics[width=\linewidth]{figures/dram_rc.pdf}\\
    \caption{Circuit}
    \label{fig:dram_rc}
  \end{subfigure}
  \vspace{-0.45in}
  \caption{DRAM detailed organization. (a) is the high-level structure of DRAM array, (b) shows cell structure and (c) illustrates the equivalent circuit where $R_c$ is contact resistance and $R_{BL}$ is the bitline resistance.}
  \label{fig:dram_bank}
  \vspace{-0.45in}
\end{figure}

In more detailed level, DRAM cells are packed into 2D arrays, as Figure \ref{fig:dram_cell} shows, where each cell can be uniquely located by vertically bitline and horizontally wordline.
Each cell consists a capacitor to store electrical charge, and one access transistor to control the connection to wordline.
Upon receiving a row address, DRAM fetches the target row into the row buffer, which contains thousands of sense amplifiers to detect the voltage change on bitline.


\section{DRAM Operations and Timing Constraints}
DRAM supports three types of accesses --- read, write, and refresh. An on-chip memory controller (MC) is responsible to receive requests from processors and decompose them into a series of commands such as {\tt ACT}, {\tt RD}, {\tt WR} and {\tt REF}, etc.
The commands are then sent to DRAM modules sequentially following the predefined timing constraints in DDRx standard. 
We briefly summarize the involved commands and timing constraints as follow: 
%A more comprehensive discussion can be found in \cite{Bruce:Jacob}.

\textbf{READ:} as illustrated in Figure \ref{fig:dram_read}, read access starts with an \underline{ACTIVATE} (ACT) command to bring the required row into the sense amplifiers; then, a \underline{READ} (RD) command is issued to fetch data from the row buffer. The interval between  ACT and RD is constrained by {\tt tRCD}. DRAM read is destructive, and hence the charge in the storage capacitors needs to be restored. The restore operation is performed concurrently with RD, and a row cannot be closed until restoring completes, which is determined by {\tt tRAS-tRCD}. Once the row is closed, a \underline{PRECHARGE} (PRE) can be issued to prepare for a new row access. PRE is constrained by timing {\tt tRP}. The time for the whole read process is {\tt tRC=tRAS+tRP}.

\begin{figure}
 \centering
  \begin{subfigure}{.7\textwidth}
    \centering
    	\includegraphics[width=\linewidth]{figures/timing_read.pdf}\\
    \caption{Read operation}
    \label{fig:dram_read}
  \end{subfigure}

\vspace{-0.25in}
  \begin{subfigure}{.7\textwidth}
    \centering
    	\includegraphics[width=\linewidth]{figures/timing_write.pdf}\\
    \caption{Write operation}
    \label{fig:dram_write}
  \end{subfigure}
  \vspace{-0.4in}
  \caption{Commands and timing constraints involved in DRAM accesses. (Timing values are from  \cite{JEDEC:ddr3})}
  \label{fig:operation}
    \vspace{-0.45in}
\end{figure}

\textbf{WRITE:} write works similarly to read, with ACT as the first command to be performed. After {\tt tRCD} has been elapsed, a \underline{WRITE} (WR) is issued to overwrite the content in the row buffer, and then update (restore) the value back into the DRAM cells. Before issuing PRE, the new data overwritten in the sense amps must be safely restored into the target bank, taking {\tt tWR} time. 
To summarize, both RD and WR commands involve the restoring operation
\footnote{Whereas restoring after write is represented by {\tt tWR}, that after read is included in {\tt tRAS}. For ease of presentation, we discuss with a focus on {\tt tWR} and always adjust {\tt tRAS} accordingly.}
, and hence a change in restore time shall affect both DRAM read and write accesses. 

\textbf{Refresh:}
%DRAM needs to be refreshed periodically to prevent data loss. 
refresh commands are issued by memory controller typically every 7.8$\mu$s to refresh a bin, which is composed of multiple rows.
Upon receiving {\tt REF}, DRAM device refresh the designated rows tracked by the internal counter.
%According to JEDEC \cite{JEDEC:ddr3}, 8192 all-bank auto-refresh ({\tt REF}) commands are sent to all DRAM devices in a rank within one retention time interval ({\tt Tret}), also called as one refresh window ({\tt tREFW}) \cite{TC15:refresh, ISCA13:ddr4, HPCA14:parallelrefresh}, typically 64ms for DDR3/4. 
%The gap between two {\tt REF} commands is termed as refresh interval ({\tt tREFI}), whose typical value is 7.8$\mu$s, i.e. 64ms/8192.
%If a DRAM device has more than 8192 rows, rows are grouped into 8192 {\bf refresh bins}. One {\tt REF} command is used to refresh multiple rows in a bin. An internal counter in each DRAM device tracks the designated rows to be refreshed upon receiving {\tt REF}. 
The refresh operation takes {\tt tRFC} to complete, which proportionally depends on the number of rows in the bin.
Whereas typically the whole memory rank is refreshed every 64ms, the vast majority cells can hold data for a much longer time \cite{ISCA12:raidr, ISCA15:reflex}.

\section{DRAM Technology Scaling}
\subsection{Scaling Issues}
With continuously increasing demands on DRAM density and capacity, the cell dimensions keep scaling downward.
%Memory technology scaling drives increasing density and capacity by decreasing cell dimensions.
Past decades saw DRAM's rapid development of 4x density every 3 years \cite{BOOK:cod}. 
Along scaling path from over 100nm to nowadays 2x nm, DRAM also experiences the drop of supply voltage \cite{HPCA16:twr}, more severe signal noise \cite{ISQED08:offset, ISCA13:ddr4} and shorter retention time \cite{ISCA13:archshield, PATENT15:twr}.
However, for reliable operations in DRAM, cell capacitor must be sufficiently large to hold sufficient charge, access transistor is required to be large enough to exert effective control \cite{ISCA09:pcm}, resistance should not be too large to obstruct cell charging process, and sub-threshold leakage should be small to safely hold data for a long time.

The intertwining requirements make the scaling jeopardy. For instance, smaller technology nodes provides smaller contacts of transistor and capacitor, and also narrower bitlines, both of which result in increased resistance (shown in Figure \ref{fig:dram_rc}), which lengthens the restoring time, and further the overall access latency. The growing number of slow and leaky cells has a large impact on system performance. 
There are three general strategies to address this challenge:
\begin{itemize}
\itemsep -1pt
\item 
The first choice is to keep conventional hard timing constraints for DRAM, 
which makes it challenging to handle slow and leaky cells.  Cells that fall
outside of guardbands could be filtered (not used).
% when they are outside guardbands.  
With scaling, however, this approach can incur worse chip yield
and higher manufacturing cost. 
%One choice is not to expose these cells by filtering out chips with weak cells above a threshold. This results in low chip yield and high manufacturing cost. 
Because the DRAM industry operates in an environment of exceedingly tight profit
margins, reducing chip yield for commodity devices is unlikely to be preferred.
%Given that DRAM industry is known for its tight profit margin, reducing chip yield may not be preferred. 

\item 
A second choice is to expose weak cells, falling outside guardbands, and integrate strong yet complex error correction schemes, e.g., \textit{ArchShield} \cite{ISCA13:archshield}. Due to the large number of cells that violate conventional timing constraints such as {\tt tRCD}, {\tt tWR}, significant space and performance overheads are expected.

\item
A third choice is to relax timing constraints \cite{MEM14:twr, DATE15:twr}. This approach is compelling because it can easily maintain high chip yield at extreme technology sizes. 
%Studies suggested relaxed timing constraints in future DRAM chips \cite{MEM14:twr, DATE15:twr}, which helps to keep high yield to make the technology viable. 
However, relaxing timing, without careful management, can cause 
large performance penalties.  
\end{itemize} 
%
Because the third choice is compatible with the need for high chip density 
and yield, we adopt it in this thesis.  We relax restore timing and strive to mitigate associated 
performance degradation. 
Our design principle is also applicable to the second strategy if exposed errors 
can be well managed. We leave this possibility to future work.

\subsection{Related Work on Restoring}
While write recovery time ({\tt tWR}) keeps at 15ns across all generations from DDR to DDR4 \cite{JEDEC:ddr, JEDEC:ddr2, JEDEC:ddr3, JEDEC:ddr4}, it has to be lengthened in deep sub-micron technology nodes, which was first recently discussed by \citeN{MEM14:twr}.
As the first academic work on {\tt tWR} issues in further scaling DRAM, our paper \cite{DATE15:twr} studied the variation behaviors and proposed to utilize chunk remapping to lower restoration durations.
Afterwards, patents on {\tt tWR} were granted: \citeN{PATENT14:twr} raised the idea to adjust timings with respect to temperature, and \citeN{PATENT15:twr} claimed that {\tt tWR} can be increased from 15ns to 60ns, and then raised the idea of exploring backward compatibility.

Whereas the {\tt tWR} scaling issue has been identified in industrials, little academic research have been performed. Restoration has been an silent issue util recently; people started to utilize the reserved timing margins \cite{DATE14:margin, HPCA15:al-dram}, with restoring being included. Besides, later work \cite{ISCA15:mcr} took use of charge variation to relax some timing constraints. However, none of these work targets at future DRAM technologies.

%\section{Related Work}

\chapter{Shadow Replication}
\label{chapter:shadowing}
It is without doubt that our understanding of how to build reliable systems out of unreliable components has led the development of robust and fairly reliable large-scale software and networking systems. The inherent instability of extreme-scale distributed systems of the future in terms of the envisioned high-rate and diversity of faults, however, calls for a reconsideration of the fault tolerance problem as a whole. % and the exploration of radically different approaches that go beyond adapting or optimizing well known and proven techniques.

Shadow Replication is a computational model that goes beyond adapting or optimizing well known and proven techniques, and explores radically different methodologies to fault tolerance~\cite{mills_2014_icnc,mills_2014_pdp,mills2014power}. % that scale to extreme-scale computing infrastructures. 
The proposed solutions differ in the type of faults they manage, their design, and the fault tolerance protocols they use. It is not just a scale up of  ``point" solutions, but an exploration of innovative and scalable fault tolerance frameworks. 
When integrated, it will lead to efficient solutions for a ``tunable" resiliency that takes into consideration the nature and requirements of the application.

The basic tenet of Shadow Replication is to associate with each main process a suite of “shadows” whose size depends on the 
``criticality" of the application and its performance requirements. Each shadow process is an exact replica of the original 
main process, 
and a consistency protocol assures that each shadow process stays consistent with its associated main process.  
Shadow Replication achieves power awareness under QoS requirements by dynamically adjusting the execution rates in response to failures. 
To save power, the shadows initially execute at a lower rate than the main process.
If the main process completes the task successfully, the associated shadows will be terminated immediately. If the main process fails, however, one of the shadow processes will be promoted to a new main process, and possibly increases its execution rate to mitigate delay.

Since the individual failure rate is extremely low, in most instances the main process will not encounter any failure and complete the task. 
Consequently, the additional energy consumed by the
slower shadow is significantly lower than that of a full-speed replica of the
main, resulting in a lot of energy savings. Furthermore, a failure of a shadow process has no bearing on the behavior of its associated main process. These two properties make it possible for Shadow Replication to enable fault tolerance at a much lower cost.

\section{Execution Model}
%Assuming the fail-stop fault model, where a processor stops execution once a fault occurs and failure can be detected by other processors~\cite{gartner_faults_1999,cristian_comm_1991}, 
Assuming there is a Reliability Availability and Serviceability (RAS) system for fault detection~\cite{ferreira_sc_2011}, 
the Shadow Replication fault-tolerance model is formally defined as follows:
\begin{itemize}
	\item A main process, $P_m(W,\text{ }\sigma_m)$, whose responsibility is to executes a task of $W$ workload at an execution rate of $\sigma_m$;
	\item A suite of shadow processes, $P_{s}(W,\text{ }\sigma_b^s, \text{ }\sigma_a^s)$ ($1 \le s \le \cal S)$, where $\cal S$ is the size of the suite. The shadows execute on separate computing nodes. Each shadow process is associated with two execution rates. All shadows start execution simultaneously with the main process at rate $\sigma_b^s$ ($1 \le s \le \cal S$). Upon failure of the main process, all shadows switch their executions to $\sigma_a^s$, with one shadow being designated as the new main process. This process continues until completion of the task.
\end{itemize}

%Assuming a single process failure, Figure \ref{fig:sc_overview} illustrates the behavior of Shadow Replication with one shadow per task. 
To illustrate the behavior of Shadow Replication, we limit the number of shadows to a single process and consider the scenarios depicted in Figure \ref{fig:sc_overview}, assuming a single process failure. 
%If the main process does not fail, it will complete the task ahead of its shadow at $t_c^m$. At this time, we terminate the shadow immediately to save energy. If the shadow fails before the main completes, the failure has no impact on the progress of the main. If the main fails, however, the shadow switches to a higher speed and completes the task at time $t_c^s$. 
Figure \ref{fig:sc_no_fail} represents the case where neither the main nor the shadow fails. The main process, executing
at a higher rate, completes the task at time $t_c^m$. At this time, the shadow process, progressing at a lower rate, stops execution immediately to save energy. Figure \ref{fig:sc_shadow_fail} represents the case where the shadow experiences a failure. This failure, however, has no impact on the progress of the main process, which still completes the task at $t_c^m$. Figure \ref{fig:sc_main_fail} depicts the case where the main process fails while the shadow is in progress. After detecting the failure of the main process, the shadow begins execution at a higher rate, completing the task at time $t_c^s$. 
%Given that the failure rate of an individual node is much lower than
%the aggregate system failure, it is very likely that the main process
%will always complete its execution successfully, thereby achieving fault tolerance at a significantly reduced cost of energy consumed by the shadow. %saving a lot of energy for its associated shadow processes. 

\afterpage{
\begin{figure}[!h]
	\begin{center}
		\subfigure[No Failure]
		{
			\label{fig:sc_no_fail}
			\includegraphics[width=0.6\textwidth]{Figures/exe_dy_1}
		}
		\subfigure[Shadow Process Failure]
		{
			\label{fig:sc_shadow_fail}
			\includegraphics[width=0.6\textwidth]{Figures/exe_dy_2}
		}
		\subfigure[Main Process Failure]
		{
			\label{fig:sc_main_fail}
			\includegraphics[width=0.6\textwidth]{Figures/exe_dy_3}
		}
	\end{center}
	\caption{Shadow Replication for a single task using a pair of main and shadow.}
	\label{fig:sc_overview}
\end{figure}
\clearpage
}



\section{Adaptivity}
\label{sec:shadowing_adaptivity}
In HPC and Cloud systems, performance, resilience, and power consumption are more often than not conflicting objectives. For example, achieving fault tolerance comes with a cost of redundant resources, which unavoidably lead to additional power and energy consumption. 
On the other hand, it has been shown that lowering supply voltages, a commonly used technique to conserve power, increases the probability of transient faults~\cite{chandra2008defect,zhao2008reliability}, and introduces non-trivial performance degradation~\cite{wang2013impact}.

Shadow Replication is a pioneering work in exploring the trade-offs among failure-free operation, the imposed power constraints, and the time to solution of the supported application. 
Internally, Shadow Replication has a set of parameters,    lying on two dimensions, that collectively determine its behavior along with costs. By configuring the shadow suite size based on fault tolerance needs, and dynamically adjusting the main and shadow processes' execution rates, Shadow Replication is able to guarantee a specific performance with certain fault tolerance capability and under a bounded power budget, thereby achieving adaptivity to the desired balance among the three conflicting objectives. 

The size of shadow suite directly reflects the amount of redundancy needed. The more shadows in each suite, the more hardware resources are required to execute the replicas in parallel. Furthermore, the additional hardware resources place a higher demand on the power supply. Therefore, it is desirable to use as few shadows as possible, under the premise that the resilience requirements are met. It is well known that one can use $f+1$ replicas to tolerate $f$ crash failures, and use $2f+1$ replicas to correct $f$ silent failures. To deal with one crash failure, a single shadow that runs as a slower replica of its associated main would be sufficient to maintain acceptable response time. Like shown in Figure~\ref{fig:sc_overview}, two replicas guarantee that at least one can complete the task, if at most one crash failure could occur. Similarly, two shadows are sufficient to guard a main process against a silent failure. Specifically, by using a simple voting mechanism and comparing the outputs of three replicas, one can detect as well as correct a silent failure~\cite{fiala_2012_sdc}. 

In addition to shadow suite size, another dimension of control consists of the execution rates of both the main and shadow processes, before and after failure. 
A closer look at the model reveals that Shadow
Replication is a generalization of traditional fault tolerance
techniques, namely re-execution and process replication. If it allows for flexible completion time, 
Shadow Replication converges to re-execution as
the shadow remains idle during the execution of the main process and
only starts execution upon failure. If the target response time is
stringent, however, Shadow Replication converges to process replication,
as the shadow must execute simultaneously with the main at the maximum
rate. Assuming dual modular redundancy to deal with a crash failure, this adaptability of Shadow Replication is further illustrated in Figure~\ref{fig:three_lines}. 
It is not difficult to imagine that by exploring the combination of execution rates, Shadow Replication covers a ``spectrum" of fault tolerance strategies, including both re-execution and process replication. 
The flexibility of the Shadow Replication model provides the
basis for the design of a fault tolerance strategy that strikes a balance between task completion time and energy saving. 

\begin{figure}[t]
	\begin{center}
		\includegraphics[width=0.8\textwidth]{Figures/three_lines}
	\end{center}
	\caption{Illustration of Shadow Replication's ability to converge to either re-execution or traditional process replication.}
	\label{fig:three_lines}
\end{figure}

\section{Execution Rate Control}

The Shadow Replication model relies on the fact that power savings can be achieved by reducing process execution rate. Furthermore, Shadow Replication needs to dynamically adjust the process execution rates to maintain satisfactory response time while saving power and energy. So far, two techniques have been explored to achieve the desired process execution rate. The first technique directly relates to a hardware feature, while the second one can be easily done via process mapping on any hardware platform.

DVFS is the first technique that our lab studied to reduce the execution rate of a process. While running each main and shadow process on a separate CPU, DVFS allows to reduce the CPU frequency by lowering the supply voltage. In the case of a failure, DVFS also allows to dynamically increase the CPU frequency to speed up a process. It is well known that one can reduce the dynamic CPU power consumption at least quadratically by reducing the frequency linearly, thereby saving power and energy simultaneously. %Its effectiveness, however, may be markedly limited by the granularity of voltage control, the range of frequencies available, and the negative effects on reliability~\cite{Eyerman:2011:FDU:1952998.1952999,zhao2008reliability,zhao2011generalized}.

An alternative approach to DVFS is to collocate multiple processes on a single CPU, while keeping each CPU running at the maximum frequency~\cite{cui_2016_scalcom}. The desired process execution rate can be achieved by controlling the \textit{collocation ratio}, which is defined as the number of processes that time-share a CPU. An example of collocation is depicted in Figure~\ref{fig:sc_mapping},  with a collocation ratio of 3 for shadow processes. A \textit{shadowed set} refers to a set of mains and their collocated shadows. The advantages and disadvantages of collocation will be further discussed in later chapters. 

\begin{figure}[t]
	\begin{center}
		\includegraphics[width=0.8\textwidth]{Figures/sc_mapping.pdf}
	\end{center}
	\caption{An example of collocation. Nine mains and their associated shadows are grouped into three logical shadowed sets. By collocating every three shadows on a core, twelve cores are required.}
	\label{fig:sc_mapping}
\end{figure}

In terms of power and energy saving, DVFS has a different effect from collocation. 
When applying collocation, it is straightforward that fewer hardware resources are required to support the same number of processes than using DVFS. For example, only 12 cores are required to simultaneously execute 18 processes, as shown in Figure~\ref{fig:sc_mapping}. This is a 33.3\% saving in hardware resources compared to DVFS. As a result, collocation brings in reduction in power and energy consumption proportionally to the reduction in hardware resources. On the other hand, although DVFS requires the same amount of hardware as traditional process replication, it saves CPU dynamic power on each process that executes at a reduced rate, thus achieving energy savings. This thesis does not carry out a quantitative comparison between these two techniques, because the exact power and energy consumption largely depend on the specific architecture, such as the ratio between CPU dynamic power and static power, and the relationship between power reduction and frequency reduction due to DVFS.

\section{Summary}
Based on DVFS, Mills studied the Shadow Replication computational model and associated one shadow process with each main process to tolerate fail-stop failures in HPC systems. For HPC throughput consideration, his work assumes that the main process always executes at the maximum rate. Even with this restriction, Mills successfully demonstrates that Shadow Replication can achieve resilience more efficiently than both checkpoint/restart and process replication when power is limited~\cite{mills_2014_icnc,mills_2014_pdp,mills2014power}.

Although DVFS is widely available in today's processors, its effectiveness, however, may be markedly limited by the granularity of voltage control, the range of frequencies available, and the negative effects on reliability~\cite{Eyerman:2011:FDU:1952998.1952999,zhao2008reliability,zhao2011generalized}. At the same time, there are several issues with the Shadow Replication model that have been later identified and that question the applicability of Shadow Replication to tolerating high rate and diverse types of failures in extreme-scale computing environments. 
In the following chapters, this thesis presents novel work that not only addresses the limitations of the basic Shadow Replication model to achieve better adaptivity, sustainability, and scalability, but also verifies the model with prototype implementation and performance evaluation.
 





\chapter{Reward-based optimal Shadow Replication}
\label{chapter:reward}
Cloud Computing has emerged as an attractive platform for increasingly
diverse compute- and data-intensive applications, as it allows for
low-entry costs, on demand resource provisioning and allocation, and
reduced cost of maintaining internal IT
infrastructure~\cite{tchana_cits_2012}. %Cloud computing will continue
%to grow and attract attention from commercial and public market
%segments. Recent studies predict annual growth rate of 17.7 percent by
%2016, making cloud computing the fastest growing segment in the
%software industry.
As the demand for cloud computing
accelerates, cloud service providers (CSPs) will be faced with the
need to expand their underlying infrastructure to ensure the expected
levels of performance and cost-effectiveness, resulting
in a multi-fold increase in the number of computing, storage and
communication components in their datacenters. The direct implication of large datacenters is increased management complexity and propensity to
failure.

Failure to deliver
the service as specified subjects the CSP to a loss of revenue. In addition, CSPs face rising energy costs of their large-scale
datacenters. This raises the question of how
fault tolerance might impact power consumption and ultimately the
expected profit of the service providers. In this work, we address the above challenge 
by studying the application of Shadow Replication for QoS in cloud computing.

%\section{Cloud workload characteristics}
%
%Cloud computing workload ranges from business applications and
%intelligence, to analytics and social networks mining and log
%analysis, to scientific applications in various fields of sciences and
%discovery. These applications exhibit different behaviors, in term of
%computation requirements and data access patterns. While some
%applications are compute-intensive, others involve the processing of
%increasingly large amounts of data. The scope and scale of these
%applications are such that an instance of a job running one of these
%applications requires the sequential execution of multiple computing
%phases; each phase consists of thousands, if not millions, of tasks
%scheduled to execute in parallel and involves the processing of a very
%large amount of data~\cite{lin2010data,Ferdman:2012:CCS:2150976.2150982}. This
%model is directly reflective of the \emph{MapReduce} computational
%model, which is predominately used in
%Cloud Computing \cite{mrbs}.  An instance of this model, is depicted in Figure \ref{fig:system_model}.
%
%
%\begin{figure}[!h]
%	\begin{center}
%		\includegraphics[width=0.6\columnwidth]{figures/system_model_1.pdf}
%	\end{center}
%	\caption{Cloud computing execution model with 2 phases.}
%	\label{fig:system_model}
%\end{figure} 

%Each task is mapped to one compute core and executes at a speed, $\sigma$. The partition of the job among tasks is
%such that each task processes a similar
%workload, $W$. Consequently, baring failures, tasks are expected to
%complete at about the same time. Therefore, the minimal response time
%of each task, when no failure occurs, is
%$t_{min}~=~\frac{W}{\sigma_{max}}$, where $\sigma_{max}$ is the maximum speed. This is also the minimal response
%time of the entire phase. 
%
%As the number of tasks increases, however, the likelihood of a task
%failure during an execution of a given phase increases
%accordingly. This underscores the importance of an energy-efficient
%fault-tolerance model to mitigate the impact of a failing task on the
%overall delay of the execution phase. Lazy Shadowing is a perfect match for the needs. We can easily apply 
%Lazy Shadowing by issuing one main and shadow pair for each task, and the execution can be performed phase 
%by phase, just as previously described. 

\section{Optimization framework}

In this section, we describe a profit-based optimization framework for
Shadow Replication in cloud computing. Using this
framework we compute profit-optimized execution speeds by
optimizing the following objective function:


\begin{equation}
\label{optimization_problem}
%\setlength{\abovedisplayskip}{14pt}
\begin{alignedat}{2}
\max_{\sigma_m,\sigma_b,\sigma_a}     & E[profit] \\
s.t.                                 & 0 \leq \sigma_m \leq \sigma_{max} \\
                                     & 0 \leq \sigma_b \leq \sigma_{m} \\
                                     & 0 \leq \sigma_a \leq \sigma_{max} 
\end{alignedat}
\end{equation}
We assume that processor
speeds are continuous and use nonlinear optimization techniques
to solve the above optimization problem. 

In order to earn profit, service providers must either increase
income or decrease expenditure. We take both factors into
consideration for the purpose of maximizing profit while meeting
customer's requirements. In our model, we set the expected profit to be
expected income minus expected expense. To model the expected income and expected expense, we built models for job reward, process failure distribution, power consumption, and energy cost. Please refer to~\cite{cui_2014_closer} for more details.

\begin{equation}
E[\text{profit}]=E[\text{income}]-E[\text{expense}]
\end{equation}



%\subsection{Reward Model}
%\label{sla_reward_model}
%
%As depicted in Figure \ref{fig:reward}, customers expect that their
%job deployed on cloud finishes by a mean response time $t_{R_1}$.  As a
%return, the provider earns a certain amount of reward, denoted by R,
%for satisfying customer's requirements. However, if the job cannot be
%completed by the expected response time, the provider loses a fraction of $R$
%proportional to the delay incurred. For large delay, the profit loss may translate into a penalty that the CSP must pay to the customer. In this model, the maximum penalty $P$ is paid if the
%delay reaches or exceeds $t_{R_2}$. The four
%parameters, $R$, $P$, $t_{R_1}$ and
%$t_{R_2}$, completely define the reward model.
%
%There are two facts that the service provider must take into account
%when negotiating the terms of the SLA. The first is the response time
%of the main process assuming no failure (Figure
%\ref{fig:sc_no_fail} and Figure \ref{fig:sc_shadow_fail}). This
%results in the following completion time:
%\begin{equation}
%t_c^m=W/\sigma_m
%\label{eq:tcm}
%\end{equation}
%
%If the main process fails (shown in Figure \ref{fig:sc_main_fail}), the
%task completion time by shadow process is the time of the failure,
%$t_f$, plus the time necessary to complete the remaining work.
%
%\begin{equation}
%t_c^s=t_f+\frac{W-t_f \times \sigma_b}{\sigma_a}
%\label{eq:tcs}
%\end{equation}
%
%
%\begin{figure}[t!]	
%	\begin{center}
%		\includegraphics[width=0.6\columnwidth]{figures/reward.pdf}
%	\end{center}
%	\caption{A reward function}
%	\label{fig:reward}
%\end{figure}
%
%
%\subsection{Failure Model}
%
%We assume that two probability density functions, $f_m(t_f)$ and
%$f_s(t_f)$, exist which express the probabilities of the main and shadow
%process failing at time $t_f$ separately. The model does not assume a
%specific distribution. However, in the remainder of this paper we use
%an exponential probability density function, $f_m(t_f)=f_s(t_f)=\lambda
%e^{-\lambda t_f}$, of which the mean time between failure (MTBF) is $\frac{1}{\lambda}$.
%
%\subsection{Power and Energy Models}
%Dynamic voltage and frequency scaling
%(DVFS) has
%been widely exploited as a technique to reduce CPU dynamic power~\cite{flautner_2002_APS,pillai_2001_sosp}. It
%is well known that one can reduce the dynamic CPU power consumption at
%least quadratically by reducing the execution speed linearly. The
%dynamic CPU power consumption of a computing node executing at speed
%$\sigma$ is given by the function $p_d(\sigma)=\sigma^n$ where $n \ge
%2$.
%%% removed burd_1995_systems citation
%
%In addition to the dynamic power, CPU leakage and other components
%(memory, disk, network etc.) all contribute to static power
%consumption, which is independent of the CPU speed. We
%define static power as a fixed fraction of the node power consumed
%when executing at maximum speed, referred to as $\rho$. Hence node
%power consumption is expressed as
%$p(\sigma)=\rho \times \sigma_{max}^n + (1-\rho)\times \sigma^n$. When the execution speed is zero
%the machine is in a sleep state, powered off or not assigned as a
%resource; therefore it will not be consuming any power, static or
%dynamic.  Throughout this thesis we assume that dynamic power is cubic
%in relation to
%speed~\cite{rusu_2003_ecs,zhai_2004_dac}, therefore the
%overall system power when executing at speed $\sigma$ is defined as:
%
%\begin{equation}
%p(\sigma) = \begin{cases} \rho \sigma_{max}^3 + (1-\rho) \sigma^3 & \mbox{if } \sigma > 0 \\ 
%                          0 & \mbox{if } \sigma = 0 \end{cases}
%\label{eq:power_model}
%\end{equation}
%%% I removed chen_2012_srds from speed citation to save space
%
%Using the power model given by \ref{eq:power_model}, the
%energy consumed by a process executing at speed $\sigma$ during an
%interval $T$ is given by
%\begin{equation}
%E(\sigma,T) = p(\sigma) \times T
%\end{equation}
%
%Corresponding to \ref{fig:sc_overview}, there are three
%failure cases to consider: main and shadow both succeed, shadow fails
%and main fails. As described earlier, the case of both the main and
%shadow failing is very rare and will be ignored. The expected
%energy consumption for a single task is then the weighted average of
%the expected energy consumption in the three cases.
%
%First consider the case where no failure occurs and the main process
%successfully completes the task at time $t_c^m$, corresponding to
%\ref{fig:sc_no_fail}.
%\begin{equation}
%\begin{split}
%E_1 = &  ( 1-\int_0^{t_c^m}f_m(t)dt) \times (1 - \int_0^{t_c^m} f_s(t)dt) \times \\
%      &  (  E(\sigma_m,t_c^m) + E(\sigma_b,t_c^m))
%\label{eq:energy_no_failure}
%\end{split}
%\end{equation}
%The first line is the probability of fault-free execution of the main
%process and shadow process. Then we multiple this probability by the
%energy consumed by the main and the shadow process during this fault
%free execution, ending at $t_c^m$.
%
%Next, consider the case where the shadow process fails at some point
%before the main process successfully completes the task, corresponding to
%\ref{fig:sc_shadow_fail}.
%\begin{equation}
%\begin{split}
%E_2 = & (1-\int_0^{t_c^m}f_m(t)dt) \times \\
%      & \int_0^{t_c^m}(E(\sigma_m,t_c^m)+E(\sigma_b,t)) \times f_s(t)dt
%\label{eq:energy_shadow_fail}
%\end{split}
%\end{equation}
%The first factor is the probability that the main process does not
%fail, and the probability of shadow fails is included in the second factor which also contains the energy consumption since it depends on the shadow failure time. Energy consumption comes from the main process until the completion of the task,
%and the shadow process before its failure.
%
%The one remaining case to consider is when the main process fails and
%the shadow process must continue to process until the task completes,
%corresponding to Figure \ref{fig:sc_main_fail}.
%\begin{equation}
%\begin{split}
%E_3 = & (1-\int_0^{t_c^m}f_s(t)dt) \times \int_0^{t_c^m}(E(\sigma_m,t)+\\
%      & E(\sigma_b,t)+E(\sigma_a,t_c^s-t))f_m(t)dt
%\label{eq:energy_main_fail}
%\end{split}
%\end{equation}
%Similarly, the first factor expresses the probability that the shadow process does
%not fail. In this case, the shadow process executes from the beginning to
%$t_c^s$ when it completes the task. However, under our ``at most one
%failure'' assumption, the period during which shadow process may fail
%ends at $t_c^m$, since the only reason why shadow process is still in
%execution after $t_c^m$ is that main process has already failed. There
%are three parts of energy consumption, including that of main process
%before main's failure, that of shadow process before main's failure,
%and that of shadow process after main's failure, all of which depend
%on the failure occurrence time. 
%
%The three equations above describe the expected energy consumption by a
%pair of main and shadow processes for completing a task under
%different situations. However, under our system model it might be the
%case that those processes that finish early will wait idly and
%consume static power if failure delays one task. If it is the case
%that processes must wait for all tasks to complete, then this energy
%needs to be accounted for in our model. The probability of this is the probability that at least one main process fails,
%referred to as the system level failure probability.
%\begin{equation}
%P_f=1-(1-\int_0^{t_c^m}f_m(t)dt)^N
%\label{eq:prob_of_one_main_failure}
%\end{equation}
%Hence, we have the fourth equation corresponding to the energy consumed while waiting in idle. 
%\begin{equation}
%  \begin{split}
%  E_4 = & ( 1-\int_0^{t_c^m}f_m(t)dt) \times (1 - \int_0^{t_c^m} f_s(t)dt) \times \\
%        & 2 P_f \times E(0,t_c^j-t_c^m) + \int_0^{t_c^m}f_s(t)dt \times \\
%        & (1-\int_0^{t_c^m}f_m(t)dt) \times P_f \times E(0,t_c^j-t_c^m) 
%  \end{split}
%\end{equation}
%Corresponding to the first case, neither main process nor shadow
%process fails, but both of them have to wait in idle from task
%completion time $t_c^m$ to the last task's completion (by a shadow
%process) with probability $P_f$. Under the second case, only the main
%process has to wait if some other task is delayed since its shadow
%process has already failed. These two aspects are accounted in the
%first and last two lines in $E_4$ separately.  We use the expected
%shadow completion time $t_c^j$ as an approximation of the latest task
%completion time which is also the job completion time.
%
%By summing these four parts and then multiplying it by $N$ we will have
%the expected energy consumed by Shadow Replication for completing a
%job of $N$ tasks.
%\begin{equation}
%E[\text{energy}]=N \times (E_1 + E_2 + E_3 + E_4)
%\label{eq:total_energy}
%\end{equation}
%
%\subsection{Income and Expense Models}
%The income is the reward paid by customer for the cloud computing
%services that they utilize. It depends on the reward function $r(t)$,
%depicted in \ref{fig:reward}, and the actual job completion
%time. Therefore, the income should be either $r(t_c^m)$, if all main
%processes can complete without failure, or $r^*(t_c^s)$ otherwise. It
%is worth noting that the reward in case of failure should be
%calculated based on the last completed task, which we approximate by
%calculating the expected time of completion allowing us to derive the
%expected reward, i.e. $r^*(t_c^s)=\frac{\int_0^{t_c^m}r(t_c^s) \times
%f_m(t)dt}{\int_0^{t_c^m}f_m(t)dt}$. Therefore the income is estimated
%by the following equation.
%\begin{equation}
%E[\text{income}]= (1-P_f) \times r(t_c^m) + P_f \times r^*(t_c^s)
%\end{equation}
%
%The first part is the reward earned by the main process times the
%probability that all main processes would complete tasks without
%failure. If at least one main process fails, that task would have to
%be completed by a shadow process. As a result, the second part is the
%reward earned by shadow process times the system level failure probability.
%
%If $C$ is the charge expressed as dollars per unit of energy consumption
%(e.g. kilowatt hour), then the expected expenditure would be $C$ times
%the expected energy consumption for all $N$ tasks:
%\begin{equation}
%E[\text{expense}] = C \times E[\text{energy}]
%\label{eq:expense}
%\end{equation}


\section{Profit-aware stretched replication} 
Unlike traditional replication
Shadow Replication is dependent upon failure detection, enabling the
replica to increase its execution speed upon failure and maintain the
targeted response time thus maximizing profit. While this is the case
in many computing environments, there are cases where failure
detection may not be possible. To address this limitation, we propose
profit-aware stretched replication, whereby both the main process and
the shadow execute independently at stretched speeds to meet the
expected response time, without the need for failure
detection. In profit-aware stretched replication both the main and
shadow execute at speed $\sigma_r$, found by optimizing the profit
model.  
%For both traditional replication and stretched replication,
%the task completion time is independent of failure and can be directly
%calculated as:
%\begin{equation}
%t_c=\frac{W}{\sigma_{max}} \text{ or } t_c=\frac{W}{\sigma_r}
%\end{equation}
%
%
%Since all tasks will have the same completion time, the job completion
%time would also be $t_c$. Further, the expected income, which depends
%on negotiated reward function and job completion time, is independent
%of failure:
%\begin{equation}
%E[income]=r(t_c)
%\end{equation}
%
%Since both traditional replication and profit-aware stretched
%replication are special cases of our Shadow Replication paradigm where
%$\sigma_m=\sigma_b=\sigma_a=\sigma_{max}$ or
%$\sigma_m=\sigma_b=\sigma_a=\sigma_r$ respectively, we can easily derive the
%expected energy consumption using \ref{eq:total_energy} with $E_4$
%fixed at 0 and then compute the expected expense using \ref{eq:expense}.
%
%\section{Re-execution}
%
%\noindent 
%Contrary to replication, re-execution initially assigns a single
%process for the execution of a task. If the original task fails, the
%process is re-executed. In the cloud computing execution framework
%this is equivalent to a checkpoint/restart, the checkpoint is
%implicitly taken at the end of each phase and because the tasks are
%loosely coupled they can restart independently. 
%
%Based on the one failure assumption, two cases must be considered to
%calculate the task completion time. If no failure occurs, the task
%completion time is:
%\begin{equation}
%t_c=\frac{W}{\sigma_{max}}
%\end{equation}
%In case of failure, however, the completion time is equal to the sum
%of the time elapsed until failure and the time needed for
%re-execution. Again, we use the expected value
%$t_f^*=\frac{\int_0^{t_c}t \times f_m(t)dt}{\int_0^{t_c}f_m(t)dt}$ to
%approximate the time that successfully completed processes have to
%spend waiting for the last one.
%
%Similar to Shadow Replication, the income for re-execution is the
%weighted average of the two cases:
%\begin{equation}
%E[\text{income}]=(1-P_f) \times r(t_c) + P_f \times r(t_c+t_f^{*})
%\end{equation}
%
%For one task, if no failure occurs then the expected energy consumption can be
%calculated as
%\begin{equation}
%E_5=(1 - \int_0^{t_c} f_m(t)dt) \times (E(\sigma_{max},t_c)+ P_f \times E(0,t_f^{*}))
%\label{eq:energy_first_task}
%\end{equation}
%
%If failure occurs, however, the expected energy consumption can be calculated
%as
%\begin{equation}
%E_6=\int_0^{t_c}(E(\sigma_{max},t) + E(\sigma_{max},t_c)) \times f_m(t) dt
%\label{eq:energy_rexecution_task}
%\end{equation}
%Therefore, the expected energy consumption by re-execution for
%completing a job of $N$ tasks is
%\begin{equation}
%E[energy]=N \times (E_5 + E_6)
%\end{equation}

\section{Performance evaluation}

This section evaluates the expected profit of each of the fault tolerance
methods discussed above under different system environment. We have identified 5
important parameters which affect the expected profit:
\begin{itemize}
\item Static power ratio $\rho$, which determines the portion of power that is unaffected by the execution speed.
\item SLA - The amount of reward, penalty and the required response times.
\item $N$ - The total number of tasks.
\item MTBF - The reliability of an individual node.
\item Workload - The size, $W$, of each individual task.
\end{itemize}


Without loss of generality, we normalize $\sigma_{max}$ to be 1, so
that all the speeds can be expressed as a fraction of maximum
speed. Accordingly, the task workload $W$ is also adjusted such that
it is equal to the amount of time (in hours) required for a single
task. %, preserving the ratios expressed in
%\ref{eq:tcm} and \ref{eq:tcs}. 
The price of
energy is assumed to be 1 unit. We assume that $R$ in our reward model
is linearly proportional to the number of tasks $N$ and the maximal
reward for one task is 3 units, so the total reward for a job is $3
\times N$ units.  However, for the analysis we look
at the average of expenditure and income on each task by dividing the
total expenditure and income by $N$. In our basic configuration we
assume that the static power ratio is 0.5, the task size is 1 hour, the node MTBF 5 is
years, the number of tasks is $100000$, and the response time thresholds for
maximal and minimal rewards are 1.3 hours and 2.6 hours
respectively. Since the maximal power consumption is 1 unit, the
energy needed for the task with one process at maximal speed is also 1
unit. 


With various architectures and organizations, servers deployed at
different data centers will have different characteristics in terms of
power consumption. The static power ratio is used to abstract the
amount of static power consumed versus dynamic power.  

\begin{figure}[!h]	
	\begin{center}
		\subfigure[Profit for different static power ratio. MTBF=5 years, N=100000, W=1 hour, $t_{R_1}$=1.3 hours, $t_{R_2}$=2.6 hours.]
		{
			\label{fig:rho}
			\includegraphics[width=0.4\columnwidth]{figures/rho_profit.pdf}
		}
		\subfigure[Profit for different response time threshold. $\rho$=0.5, MTBF=5 years, N=100000, W=1 hour.]
		{
			\label{fig:t}
			\includegraphics[width=0.4\columnwidth]{figures/t_profit.pdf}
		}
		\subfigure[Profit for different task size over MTBF. $\rho$=0.5, N=100000, $t_{R_1}$=1.3 hours, $t_{R_2}$=2.6 hours.]
		{
			\label{fig:n}
			\includegraphics[width=0.4\columnwidth]{figures/n_profit.pdf}
		}
		\subfigure[Profit for different task size over MTBF. $\rho$=0.5, N=100000, $t_{R_1}$=1.3 hours, $t_{R_2}$=2.6 hours.]
		{
			\label{fig:mtbf}
			\includegraphics[width=0.4\columnwidth]{figures/mtbf_profit.pdf}
		}
	\end{center}

\end{figure}


The potential profit gains achievable by using profit-aware
replication techniques decreases as static power increases, as is shown
in Figure~\ref{fig:rho}. The reason is that our profit-aware
techniques rely upon the fact that one can reduce energy costs by
adjusting the execution speeds. Modern systems have a static power between 40\%-70\% and
it is reasonable to suspect that this will continue to be the case. Within
this target range of static power, Shadow Replication can achieve, on
average, 19.3\% more profit than traditional replication, 8.9\% more
than profit-aware stretched replication, and 28.8\% more than re-execution.

Figure~\ref{fig:t} shows the effect that targeted response time has upon
the profitability of each fault tolerance method. We vary the first threshold $t_{R_1}$ from the minimal response
time $t_{min}$ to $1.9t_{min}$, and set the second threshold $t_{R_2}$
to be always $2t_{R_1}$. Compared to traditional replication, all the other methods increase their profit as the targeted
response time increases, this is expected because each of the other
techniques can make use of increased laxity in time to increase
profit. Re-execution is the most sensitive to the target response
time since it fully relies upon time redundancy, showing that it should only be used when the targeted response time is \emph{not} stringent. 
Again, Shadow Replication always achieves more profit than traditional
replication and profit-aware stretched replication, and the profit
gains are 52.8\% and 39.0\% on average. 

Figure \ref{fig:n} confirms that for small number of tasks
re-execution is more profitable than replication. However, re-execution is not scalable
as its profit decreases rapidly after N reaches 10000. At the same time, traditional
replication and profit-aware stretched replication are not
affected by the number of tasks because neither are affected by the
system level failure rate. On average, Shadow Replication achieves 43.5\%, 59.3\%, and 18.4\%
more profits than profit-aware stretched replication, traditional replication and re-execution, respectively. 

The ratio between task size and node MTBF represents the tasks
vulnerability to failure, specifically it is an approximation of the
probability that failure occurs during the execution of the task. In our
analysis we found that increasing task size will have the same effect
as reducing node MTBF. Therefore, we analyze these together using the
vulnerability to failure, allowing us to analyze a wider range of
system parameters. As expected
re-execution is desired when the vulnerability to failure is
low. As always, Shadow Replication can adjust its execution strategy to maximize the profits, as shown in Figure~\ref{fig:mtbf}.

Lastly, we evaluate 
the profit of each resilience technique using three different
benchmark applications representing a wide range of
application~\cite{mrbs}: Business Intelligence, Bioinformatics and
Recommendation System. Please refer to~\cite{cui_2014_closer} for more details.
%Lastly, we evaluate
%the expected profit of each resilience technique using three different
%benchmark applications representing a wide range of
%application~\cite{mrbs}: Business Intelligence, Bioinformatics and
%Recommendation System. 
%Using the results of the experiments reported in \cite{mrbs}, we
%derived that the time required to process data for above application types are 3.3 MB/s, 6.6 MB/s, and 13.2 MB/s, respectively. 
%
%\begin{figure}[!h]
%	
%	\begin{center}
%	\includegraphics[width=\columnwidth]{figures/application_comparison.png}
%	\end{center}
%	\caption{Application comparison. $\rho$=0.5, N=$500000$, $t_{R_1}$=$1.3t_{min}$, $t_{R_2}$=$2.6t_{min}$.}
%	\label{fig:app_compare}
%\end{figure}
%
%
%In Figure \ref{fig:app_compare} we compare the expected
%profit for each application using each of the 4 resilience techniques. 
%We consider two data sizes expected in future
%cloud computing environments, 500TB and 2PB. The figure shows that
%for business intelligence applications, Shadow Replication achieves significantly larger profits for both data sizes. This
%is because business intelligence applications tend to be IO intensive
%resulting in longer running tasks. Whereas recommendation systems tend
%to require little data IO resulting in shorter running tasks making
%re-execution as good as Shadow Replication. Bioinformatics tends to be in between
%these two applications resulting in shadow computing performing better
%when processing large datasets (2 PB) but not outstanding on smaller
%datasets (500 TB). The take away from this evaluation is that for the
%shown system parameters if phase execution is short, then re-execution
%performs as well as Shadow Replication. Alternatively, if a phase is long (20 minutes or
%greater), then Shadow Replication can be as much as 47.9\% more
%profitable than re-execution. The previous sensitivity analysis can be
%used to extrapolate expected profit for different system parameters.

\section{Summary}

In this work we focus on the objective of satisfying SLA in Cloud Computing and demonstrate that Shadow Replication is capable of achieving multi-dimensional QoS goals. 
To assess the performance of the Shadow Replication, an analytical framework is developed and 
an extensive performance evaluation study is carried out. 
In this study, system properties that affect the
profitability of fault tolerance methods, namely failure rate,
targeted response time and static power, are identified. The failure rate is
affected by the number of tasks and vulnerability of the task
to failure. The targeted response time represents the 
clients' desired job completion time.  
Our performance evaluation shows that in all cases, Shadow Replication outperforms
existing fault tolerance methods. Furthermore, shadow
replication will converge to traditional replication when target response time is stringent, and to re-execution when target response time is relaxed or failure is unlikely.


   






\chapter{Leaping Shadows}
\label{chapter:scale}
In today's large-scale HPC systems, an increasing portion of the computing capacity is wasted due to failures and recoveries. It is expected
that future exascale machines,  featuring a computing capability of exaFLOPS, will decrease the mean time between
failures to a few hours, making fault tolerance extremely challenging. 
Enabling Shadow Replication for resiliency in extreme-scale computing 
brings about a number of challenges and design decisions, including the applicability of this concept to a large number of 
tasks executing in parallel, the effective way to control shadows' execution rates, and the runtime mechanisms and 
communications support to ensure efficient coordination between a 
main and its shadow.


Taking into consideration the main characteristics of compute-intensive and highly-scalable applications, we design a novel fault tolerance model of \textit{Leaping Shadows}. Leaping Shadows is a Shadow Replication based model that associates a suite of shadows
to each main process. To achieve fault tolerance, shadows
execute simultaneously with the mains, but on different
nodes. Furthermore, to save power, shadows execute at a
lower rate than their associated mains. When a main fails,
the corresponding shadow increases its execution rate to speed up recovery.
This chapter focuses on tolerating crash failures under the fail-stop fault model, 
whereby a failed process halts and its internal state and
memory content are irretrievably lost. As a consequence, each main process is replicated with one shadow.  A study of tolerating silent failures will be discussed in Chapter~\ref{chapter:sdc}. 

To achieve higher efficiency and better scalability, however, we adopt radically different methodologies in the design of Leaping Shadows from the Shadow Replication model.
In the original Shadow Replication, shadows are designed to
substitute for their associated mains when failure occurs. The tight coupling and ensuing fate sharing between a main and its shadow increase the implementation complexity and reduce the efficiency of
the system to deal with failures.
In this new model, we deviate from the original design, and use each shadow as a ``rescuer", whose role is to restore the associated main
to its exact state before failure.

Instead of using DVFS to achieve the desired execution rates, this work applies collocation to shadowing for the first time, in order to simultaneously save power and computing resources. In addition, two issues, \textit{divergence} and \textit{vulnerability}, have been identified that limit Shadow Replication's effectiveness in large-scale, failure-prone computing environments. This chapter discusses innovative techniques of {\it leaping} and {\it  rejuvenation} as the provided solutions, respectively. %, to achieve higher efficiency and better scalability.

\section{Shadow Collocation}
\label{sec:shadow_collocation}
In HPC, throughput consideration requires that the rate of the main, $\sigma_m$, and the rate of the shadow after failure, 
$\sigma_s^a$, be set to the maximum. 
The initial execution rate of the shadow, $\sigma_s^b$, however, can be derived by balancing the trade-offs between delay and energy.
For a delay-tolerant, energy-stringent application, $\sigma_s^b$ is set to 0, and the shadow starts executing only upon failure of the main process. 
For a delay-stringent, energy-tolerant application, the shadow starts executing at $\sigma_s^b=\sigma_m$ to guarantee the completion of the task at the specified time $t_m$, regardless of when the failure occurs.  
In addition, a broad spectrum of delay and energy trade-offs in between can be explored either empirically or by using optimization frameworks for delay and energy tolerant applications.


To control the shadows' execution rates, DVFS can be applied while each shadow resides on one processor exclusively. 
The effectiveness of DVFS, however, may be markedly 
limited by the granularity of voltage control, the number of frequencies available, and the negative effects on 
reliability~\cite{Eyerman:2011:FDU:1952998.1952999,Keller:EECS-2015-257,chandra2008defect,zhao2008reliability}. 
An alternative is to collocate shadows.
We use the term processor to represent the resource allocation unit (e.g., a CPU core, a multi-core CPU, or a cluster node), so that our discussion is agnostic to the granularity of the hardware platform. While each main process occupies a processor,  we collocate multiple shadows on each processor and use time sharing to achieve the desired execution rates.
%Compared to DVFS, this approach simultaneously reduces the number of processors required and the corresponding power consumption. 


To execute an application of $M$ parallel tasks, $N=M+S$ processors are required, where $M$ is a multiple of $S$. Each main is allocated one processor (referred to as \textit{main processor}), while $\alpha=M/S$ (referred to as \textit{collocation ratio}) shadows are collocated on a processor (referred to as \textit{shadow processor}). 
The $N$ processors are grouped into $S$ sets, each of which we call a \textit{shadowed set}. Each shadowed set contains $\alpha$ main processors and 1 shadow processor.
This has been illustrated in Figure~\ref{fig:sc_mapping}.  

Contrary to traditional process replication, shadow collocation reduces
the number of processors required to achieve fault-tolerance,
thereby reducing power and energy consumption. Furthermore, the collocation ratio can be adapted to reflect the
propensity of the system to failure. This flexibility, however,
comes at the increased cost of memory requirement at the
shared processor. It is to be noted that this limitation is not
intrinsic to Leaping Shadows, as in-memory checkpoint/restart and multilevel checkpoint/restart also require additional memory to store checkpoints.

Under Shadow Replication, collocation has an important ramification with respect to the resilience of the system. Specifically, only
one failure can be tolerated in each shadowed set. If a shadow processor fails, all the shadows in the 
shadowed set will be lost, although this does not interrupt the execution of the mains. 
On the other hand, if a main processor fails, the associated shadow will be promoted to a new main, and all 
the other collocated shadows will be terminated to speed up the new main.
Consequently, a failure, either in main or shadow processor, will result in losing all the shadows in the shadowed set, thereby losing the tolerance to any other failures. After the first failure, a shadowed set becomes \emph{vulnerable}. To address this issue, Leaping Shadows applies rejuvenation, to be discussed in Section~\ref{sec:rejuvenation}, to maintain a persistent level of system resilience.



\section{Leaping}
\label{sec:leaping_shadows}

In the basic form of Shadow Replication, failures can
have a significant impact on the performance. Since a
shadow executes at a lower rate than its associated main,
a computational \textit{divergence} will occur between the pair of processes. As shown in Figure~\ref{fig:divergence}, the larger this divergence, the more time the lagging shadow will need to ``catch up" in the case of a failure, introducing delay to the recovery process. 
This problem deteriorates as dependencies incurred by messages and synchronization barriers would propagate the delay of one task to others. 
At the same time, divergence has another side-effect in message passing systems. Specifically, forwarding messages from mains to shadows in a message passing system will cause undesired message accumulation. Similarly, the message buffer will bear higher pressure as divergence increases. 

\begin{figure}[!t]
	\begin{center}
			\includegraphics[width=\columnwidth]{Figures/divergence}
	\end{center}
	\caption{Illustration of divergence between a main and its shadow.}
	\label{fig:divergence}
\end{figure}
 
Fortunately, the association with the mains provides an unique opportunity for the shadows to benefit from the faster execution of their mains. By copying the state of a main to its shadow, which is similar to the process of storing a checkpoint in a buddy in \cite{zheng_2004_ftccharm}, forward progress is achieved for the shadow with minimized time and energy. This technique, referred to as \textit{leaping}, effectively limits the divergence between main and shadow. 
As a result, there is no need of concern for buffer overflow, and the recovery time after a failure, which depends on the divergence between the failing main 
and its shadow, is also reduced. 


While recovery may lead to delay due to divergence, we opportunistically overlap shadow leaping with failure recovery to avoid extra overhead. 
Assuming a failure occurrence at time $t_f$, Figure~\ref{fig:leap} shows the concept of leaping overlapped with failure recovery. 
Upon failure of a main process, its associated shadow speeds up to minimize the impact of failure recovery on the other tasks' progress, as illustrated in Figure~\ref{fig:jump1}. 
At the same time, as shown in Figure~\ref{fig:jump2}, the remaining main processes are blocked
at the next synchronization point, which is assumed to take
place shortly after $t_f$. 
Leaping opportunistically takes advantage of this idle time to {\it leap forward} the shadows, so that  
all processes, including shadows, can resume execution from a consistent point afterwards. 
Leaping increases the shadow's rate of progress, at a minimal energy cost. Consequently, it reduces significantly the likelihood of a shadow falling excessively behind, thereby ensuring fast recovery while minimizing the total energy consumption. Note that leaping is applicable, whether shadows are collocated or use DVFS. However, in the case of collocation, the leaping for some shadows could not overlap with the recovery. This will be further discussed in the next section when we integrate leaping with rejuvenation.  

\begin{figure}[!h]
	\begin{center}
        \subfigure[Faulty task behavior.]
		{
			\label{fig:jump1}
			\includegraphics[width=0.6\columnwidth]{Figures/exe_leap_1}
		}
		\subfigure[Non-faulty task behavior.]
		{
			\label{fig:jump2}
			\includegraphics[width=0.6\columnwidth]{Figures/exe_leap_2}
		}
	\end{center}
	\caption{Illustration of shadow leaping after a failure.}
	\label{fig:leap}
\end{figure}

The main objective of leaping is to ensure forward
progress in the presence of failure. However, later on we have identified that leaping is useful in a variety of scenarios. Correspondingly, we define multiple types of leaping, each of which applies to a particular scenario. Above mentioned leaping is referred to as \textit{failure-induced leaping}, as it is triggered by a failure. As also mentioned above, message buffer pressure increases with divergence in message passing systems. If failure-induced leaping is not frequent enough, there may be a need to force a leaping to avoid buffer overflow, thus this type of leaping is referred to as \textit{buffer-forced leaping}. Furthermore, in following chapters we will demonstrate that leaping can be used to achieve forward progress, for both shadow and main processes, in another two scenarios, resulting in \textit{rejuvenation-induced leaping} and \textit{voting-induced leaping}. In all scenarios, leaping always takes
place between a main and its associated shadow, and thus
does not require global coordination. The process which
provides the leaping state is referred to as the \textit{leap-provider},
while the process which receives the leaping
state and rolls forward is referred to as the \textit{leap-recipient}.



\section{Rejuvenation}
\label{sec:rejuvenation}
Another shortcoming of Shadow Replication is that failures can impact the resilience of the system. 
Upon failure of a main, the system can only rely on an
``orphan" shadow to complete the task. 
This is even worse when shadows are collocated. As discussed in Section~\ref{sec:shadow_collocation}, each shadowed set can only tolerate a single failure. 
A trivial approach to
address this shortcoming is to associate a ``suite" of shadows
with each main. Such an approach, however, is resource
wasteful and costly in terms of energy consumption. Instead, Leaping Shadows embraces rejuvenation techniques to improve resource efficiency, when a processor on which a failure occurred can be rebooted to start new processes\footnote{Equivalently, a spare processor can be used for this purpose.}. %This assumption also holds for checkpointing/restart to ensure fair comparative analysis. 

The main objective of rejuvenation is to enable the
system to maintain its intended level of resilience, in the
event of multiple failures. The proposed approach is to use the rescuer shadow to \textit{rejuvenate} the failed main.  Specifically, while the shadow is executing at a high speed to reach the state at which the main failed, a new process is launched to replace the failed main. Furthermore, rather than starting the new process from its initial state, \textit{rejuvenation-induced leaping} is invoked to synchronize the new process' state with that of the recovering shadow. Similar to leaping, rejuvenation is applicable, in spite of the underlying execution rate control mechanism. The following discussion focuses on collocation as it represents the most comprehensive scenario. 

%A direct implication of rejuvenation is that none of the shadows collocated with the recovering shadow need to be terminated, but only suspended  until the recovery is complete. In addition to restoring the system to its intended resilience level, rejuvenation also reduces the overall execution time. 

Figure~\ref{fig:rejuvenation} illustrates the failure recovery process with rejuvenation, assuming that a main $M_i$ fails at time $T_0$. 
In order for its shadow $S_i$ to speed up, the shadows collocated with $S_i$ are temporarily suspended. %, so that $S_i$ can increase its execution rate and finish the recovery as soon as possible. 
Meanwhile, the failed processor is rebooted and then a new process is launched for $M_i$. When, at $T_1$, $S_i$ catches up with the state of $M_i$ before its failure, leaping is performed to advance the new process to the current state of $S_i$. 

Because of the failure of $M_i$, the other mains are blocked at the next synchronization point, which is assumed to take place shortly after $T_0$. During the idle time, a leaping is opportunistically performed to transfer state from each living main to its shadow. Therefore, this leaping has minimal overhead as it overlaps with the recovery, as shown in Figure~\ref{fig:non_faulty_diff}. Figure~\ref{fig:non_faulty_same} shows that leaping for the shadows collocated with $S_i$ are delayed until they resume execution when the recovery completes at $T_1$. After the leaping finishes at $T_2$, all mains and shadow resume normal execution, thereby bringing the system back to its original level of fault tolerance.

\afterpage{
\begin{figure}[!t]
	\begin{center}
		\subfigure[Faulty task]
		{
			\label{fig:faulty}
			\includegraphics[width=0.6\columnwidth]{Figures/rs_1}
		}
		\subfigure[Non-faulty tasks in different shadowed sets]
		{
			\label{fig:non_faulty_diff}
			\includegraphics[width=0.6\columnwidth]{Figures/rs_2}
		}
		\subfigure[Non-faulty tasks in the same shadowed set]
		{
			\label{fig:non_faulty_same}
			\includegraphics[width=0.6\columnwidth]{Figures/rs_3}
		}
	\end{center}
	%\vskip -0.2in
	\caption{Recovery and rejuvenation after a main process fails.}
	\label{fig:rejuvenation}
\end{figure}
\clearpage
}

Figure~\ref{fig:rejuvenation} and the above description assume that the time for rebooting is no longer than the recovery time. If the new $M_i$ is not yet ready when $S_i$ catches up at $T_1$, however, two design choices are possible. In the first, $S_i$ can assume the role of a main and continue execution. In the second, $S_i$ can wait until the launching of the new $M_i$ is complete. The first option requires that all non-failed processes update their internal mapping to identify the shadow as a new main and continue to correctly receive messages, if any. This not only complicates the implementation, but also requires expensive global coordination that is detrimental to scalability. We, therefore, chose the second design option.

The above analysis focuses on rejuvenating a failed main process. 
Failure of a shadow can be addressed in a similar manner. Each failure requires a rebooting of the target processor to launch replacing process(es), but the only difference is that the leap-provider and leap-recipient are reversed, i.e., the main process becomes the leap-provider and shadow becomes the leap-recipient. Collocation makes the rejuvenation of shadow process slightly more complicated, since a shadow processor failure will impact all the collocated shadows on that processor. Specifically, 
if a shadow processor fails, all the shadows in a shadowed set are lost. To rejuvenate, the failed processor is rebooted and then a new process is launched to replace each of the failed shadow processes. It is to be noted that all the mains can continue execution while rebooting the processor. When the newly launched shadows become ready, rejuvenation-induced leaping is invoked to synchronize every shadow with its main.




\section{Analytical Models}
\vskip -0.2in
In the following we develop analytical models to quantify the expected performance of Leaping Shadows, as well as prove the bound on performance loss due to failures. 
All the analysis below is under the assumption that there are a total of $N$ processors, and $W$ is the application workload.  
$M$ of the $N$ processors are allocated for main processes, each having a workload of $w=\frac{W}{M}$, and the rest $S$ processors are for the collocated shadow processes. %For process replication,
Note that process replication is a special case of Leaping Shadows where $\alpha=1$, so 
$M=S=\frac{N}{2}$ and $w=\frac{2W}{N}$. 

Section~\ref{sec:rejuvenation} discusses rejuvenation to maintain a persistent level of system resilience. In certain situations, the underlying assumption, that failed processor can be rebooted or standby processors are available, may not hold. If this is the case, the scheme discussed in Section~\ref{sec:rejuvenation} becomes invalid, and one has to take the risk that a task may lose both the main and shadow processes, resulting in the entire application being re-executed. To be conservative and emphasize on the benefits of leaping in our assessment of Leaping Shadows, we consider the general case where rejuvenation is not applicable. 

\subsection{Application Fatal Failure Probability}
\label{sec:anal_app_fail}
An application has to roll back when all replicas of a task have been lost. We call this an \textit{application fatal failure}, which is inevitable even when every process is replicated. 
In order to take into account the overhead of rollback in the calculation of completion time and energy consumption, we first 
study the probability of application fatal failure. In this work we assume that once an application fatal failure occurs, execution always rolls back to the very 
beginning. 

The impact of process replication on application fatal failure has been studied in~\cite{casanova_inria_2012} and 
results are presented in terms of Mean Number of Failures To Interrupt (MNFTI), i.e., the mean number of failures to cause an application fatal failure.
Applying the same methodology, we derive the new MNFTI
under collocated Leaping Shadows, 
as shown in Table~\ref{tbl:mnfti}. As each shadowed set can tolerate one failure, the results are 
for different numbers of shadowed sets ($S$). Table~\ref{tbl:mnfti} reveals that the MNFTI almost doubles when the number of shadowed set increases by a factor of 4. At $2^{20}$ shadowed sets, the application is expected to go through 1816 processor failures before observing an interrupt. 
Note that when processes are not shadowed, every failure would interrupt the application, i.e., MNFTI=1. 


\begin{table}[!h]
	\caption{Application Mean Number of Failures To Interrupt (MNFTI) when Leaping Shadows is used. Results are independent of $\alpha=\frac{M}{S}$. }
	\centering
	\small
	\begin{tabular}{|c  |c|c|c|c|c|}
		\hline
		$S$ &  $2^{2}$ &  $2^{4}$ &  $2^{6}$ & $2^8$ & $2^{10}$ \\ 
		\hline
		MNFTI &  4.7 & 8.1 & 15.2 & 29.4 & 57.7 \\
		\hline\hline
		$S$ & $2^{12}$ & $2^{14}$ &  $2^{16}$  & $2^{18}$ & $2^{20}$ \\
		\hline
		MNFTI & 114.4 & 227.9 & 454.7 & 908.5  & 1816.0 \\
		\hline
	\end{tabular}
	\label{tbl:mnfti}
\end{table}

To further quantify the probability of application fatal failure, we use 
$f(t)$ to denote the failure probability density function of each processor, and then $F(t) = \int_0^tf(\tau)d\tau$ is the probability that a processor fails in the next $t$ time. 
Since each shadowed set can tolerate one failure, 
then the probability that a shadowed set with $\alpha$ main processors and 1 shadow processor does not fail by time $t$ is the probability of no failure plus the probability of one failure, i.e., 

\begin{equation}
	P_g = \Big(1-F(t)\Big)^{\alpha+1} + {{\alpha+1} \choose 1}F(t)\times \Big(1-F(t)\Big)^{\alpha}
\end{equation}
and the probability that an fatal failure occurs to an application using $N$ processors within $t$ time is the complement of the probability that
none of the shadowed sets fails, i.e.,

\begin{equation}
	P_a = 1 - ({P_g})^{S}
\end{equation}
where $S=\frac{N}{\alpha+1}$ is the number of shadowed sets.
The application fatal failure probability can then be calculated by using $t$ equal to the expected completion time of the application, which will be modeled in the next subsection.

\subsection{Expected Completion Time}
\label{sec:anal_time}
There are two types of delay due to failures. If a failure does not lead to an application fatal failure, the delay corresponds to the catching up of the shadow of the failing main (see Figure~\ref{fig:jump1}). Otherwise, a possibly larger (rollback) delay will be introduced by an application fatal failure. In the following we consider both delays step by step. 
First we discuss the case of $k$ failures without application fatal failure. Should a failure occur during the recovery of a previous failure, its recovery would overlap with the ongoing recovery. To study the worst case behavior, we assume failures do not overlap, so that the execution is split into $k+1$ intervals, as illustrated in Figure~\ref{fig:progress}. 
$\Delta_i$ ($1\le i \le k+1$) represents the $i^{th}$ execution interval, and $\tau_i$ ($1\le i \le k$) is the recovery time after $\Delta_i$. 


\begin{figure}[!h]
	\begin{center}
		\includegraphics[width=0.8\columnwidth]{Figures/app_progress}
	\end{center}
	\caption{Application progress with shadow catching up delays.}
	\label{fig:progress}
\end{figure}


The following theorem expresses the completion time, $T_c^k$, as a function of $k$.

\begin{theorem}
Assuming that failures do not overlap and no application fatal failure occurs, then using Leaping Shadows, 
	$$T_c^k = w + (1-\sigma_s^b)\sum_{i=1}^k\Delta_i$$
\end{theorem}
\begin{proof}
Leaping Shadows guarantees that all the shadows reach the same execution point as the mains (See Figure~\ref{fig:leap}) after a previous recovery, so every recovery time is proportional to its previous execution interval. 
That is, $\tau_i = \Delta_i \times (1 - \sigma_s^b)$. 
According to Figure~\ref{fig:progress}, the completion time with $k$ failures is 
	$T_c^k = \sum_{i=1}^{k+1}\Delta_i + \sum_{i=1}^k\tau_i = w + (1-\sigma_s^b)\sum_{i=1}^k\Delta_i$
\end{proof}

Although it may seem that the delay would keep growing with the number of failures, 
it turns out to be well bounded, as a benefit of shadow leaping: 

\begin{corollary}
The delay induced by failures is bounded by $(1-\sigma_s^b)w$.
\end{corollary}
\begin{proof}
From above theorem we can see the delay from $k$ failures is $(1-\sigma_s^b)\sum_{i=1}^k\Delta_i$. It is straightforward that, for any non-negative integer of $k$, we have the equation $\sum_{i=1}^{k+1}\Delta_i= w$. As a result, 
$\sum_{i=1}^{k}\Delta_i = w - \Delta_{k+1} \le w$. Therefore, $(1-\sigma_s^b)\sum_{i=1}^k\Delta_i \le (1-\sigma_s^b)w$.
\end{proof}

Typically, the number of failures to be encountered is stochastic. Given a failure distribution, however, we can calculate the probability for a specific value of $k$. We assume that failures do not occur during recovery, so the failure probability of a processor during the execution can be calculated as $P_c = F(w)$. Then the probability that there are $k$ failures among the $N$ processors is 
\begin{equation}
\begin{split}
P_s^{k}= & \dbinom{N}{k}{P_c}^k(1-P_c)^{N-k} \\
\end{split}
\end{equation}

The following theorem expresses the expected completion time, $T_{total}$, considering all possible number of failures. 

\begin{theorem}
Assuming that failures do not overlap, then using Leaping Shadows,
$T_{total} = T_{c} / (1 - P_a)$, where $T_{c} = \sum_{i} T_{c}^{i} \cdot P_s^{i}$.
\end{theorem}
\begin{proof}
Without application fatal failure, the completion time considering all possible values of $k$ can be averaged as $T_{c} = \sum_{i} T_{c}^{i} \cdot P_s^{i}$. If an application fatal failure occurs, however, the application needs to roll back to the beginning. With the probability of rollback calculated as $P_a$ in Section~\ref{sec:anal_app_fail}, the total expected completion time is $T_{total} = T_{c} / (1 - P_a)$.
\end{proof}

Process replication is a special case of Leaping Shadows where the collocation ratio for shadows is 1, so we can apply the above theorem to derive the expected completion time for process replication, when it uses the same amount of processors:

\begin{corollary}
The expected completion time for process replication is $$T_{total} = 2W/N / (1 - P_a)$$.
\end{corollary}
\begin{proof}
Using process replication, half of the available processors are dedicated to shadows so that the workload assigned to each task is significantly increased, i.e., $w=2W/N$. Different from cases where $\alpha \ge 2$, failures do not incur any delay except for application fatal failures. 
As a result, without application fatal failure the completion time under process replication is constant regardless of the number of failures, i.e., $T_c=T_c^k=w=2W/N$. Finally, the expected completion time considering the possibility of rollback is $T_{total} = T_c / (1 - P_a) = 2W/N / (1 - P_a)$.
\end{proof}

\subsection{Expected Energy Consumption}
\label{sec:anal_energy}
Power consumption consists of two parts, dynamic power, $p_d$, which exists only when a processor is executing, and static power, $p_s$, which is constant as long as the machine is on. This can be modeled as $p = p_d + p_s$. Note that in addition to CPU leakage, other components, such as memory and disk, also contribute to static power. 

For process replication, all processors are running all the time until the application is complete. Therefore, the expected energy consumption, $En$, is proportional to the expected execution time $T_{total}$: 
\begin{equation}
En = N \times p \times T_{total}
\label{eq:exp_energy1}
\end{equation} 

Even using the same amount of processors, Leaping Shadows can save power and energy, since main processors are idle during the recovery time after each failure, and the shadows can achieve forward progress through failure-induced leaping. During normal execution, all the processors consume static power as well as dynamic power. During recovery time, however, the main processors are idle and consume only static power, while the shadow processors first perform leaping and then become idle. Altogether, the expected energy consumption for Leaping Shadows can be modeled as 
\begin{equation}
En = N \times p_s \times T_{total} + N \times p_d \times w + S \times p_{l} \times T_l.
\label{eq:exp_energy2}
\end{equation}
with $p_{l}$ denoting the dynamic power consumption of each processor during leaping and $T_l$ the expected total time spent on leaping. 

\begin{theorem}
If no subsequent failure happens before the recovery of the previous failure, then using Leaping Shadows, the upper bound on expected energy consumption is
$(2N * p_s + N * p_d + S * p_{l})*w$.
\end{theorem}
\begin{proof}
From Corollary 1.1 we know that the delay is at most $(1-\sigma_s^b)w \le w$, so $T_{total} \le 2w$. Also, since the leaping time overlaps with the recovery time (delay), $T_l \le (1-\sigma_s^b)w \le w$. Therefore, $En = N * p_s * T_{total} + N * p_d * w + S * p_{l} * T_l \le N * p_s * (2w) + N * p_d * w + S * p_{l} * w = (2N * p_s + N * p_d + S * p_{l})*w$.
\end{proof}










\section{Evaluation}
Careful analysis of the models above leads us to identify several important factors that determine the performance. These factors can be classified into three categories, i.e., system, application, and algorithm. The system category includes static power ratio $\rho$ ($\rho=p_s/p$), total number of processors $N$, and MTBF of each processor; the application category is mainly the total workload, $W$; and collocation ratio $\alpha$ in the algorithm category determines the number of main processors and shadow processors ($N=M+S$ and $\alpha=M/S$). In this section, we evaluate each performance metric of Leaping Shadows,  with the influence of each of the factors considered. %First we calculate the application failure probability for various scenarios. Then we proceed to study the expected energy consumption and completion time under different application failure probabilities. 


\subsection{Comparison to Checkpoint/restart and Process Replication}
\label{eval_comparison}
We compare with both process replication and checkpoint/restart, assuming the same number of processors to use. 
The completion time with checkpoint/restart is calculated with Daly's model~\cite{daly_fgcs_2006} assuming 10 minutes for both checkpointing and restart. The energy consumption is then derived with Equation~\ref{eq:exp_energy1}. It is important to point out that we always assume the same number of processors, so that process replication and Leaping Shadows do not use extra processors for the replicas. 

It is clear from THEOREM 1 that the total recovery delay $\sum_{i=1}^k\tau_i$ is determined by the execution time $\sum_{i=1}^k\Delta_i$, independent of the distribution of failures. 
Therefore, our models are generic with no assumption about failure probability distribution, and the expectation of the total delay from all failures is the same as if failures are uniformly distributed~\cite{daly_fgcs_2006}. Specifically, $\Delta_i = w/(k+1)$, and $T_c^k = w + w*(1-\sigma_s^b)*\frac{k}{k+1}$. Further, we assume that each shadow gets a fair share of its processor's execution rate so that $\sigma_s^b = \frac{1}{\alpha}$. 
To calculate Equation~\ref{eq:exp_energy2}, we assume that the dynamic power during leaping is twice of that during normal execution, i.e., $p_{l}=2*p_d$, and the time for leaping is half of the recovery time, i.e., $T_l=0.5*(T_{total} - w)$. 

The first study uses $N=1$ million processors, effectively simulating future extreme-scale computing environments, and assumes that $W=1$ million hours, and static power ratio $\rho=0.5$. 
Our results show that at extreme-scale, the expected completion time and energy consumption of checkpoint/restart are orders of magnitude larger than those of Leaping Shadows and process replication. Therefore, we choose not to plot a separate graph for checkpoint/restart. 

Figure~\ref{fig:t32} reveals that the most time efficient choice largely depends on MTBF. 
When MTBF is high, Leaping Shadows requires less time as more processors are used for main processes and less workload is assigned to each process. As MTBF decreases, process replication outperforms Leaping Shadows as a result of the increased likelihood of rollback for Leaping Shadows. 
In terms of energy consumption, Leaping Shadows has much more advantage over process replication. For MTBF from 2 to 25 years, Leaping Shadows with $\alpha=5$ can achieve 9.6-17.1\% energy saving, while the saving increases to 13.1- 23.3\% for $\alpha=10$. The only exception is when MTBF is extremely low (1 year), Leaping Shadows with $\alpha=10$ consumes more energy because of extended execution time.

\begin{figure}[!h]
	\begin{center}

		\subfigure[Expected completion time]
		{
			\label{fig:t32}
			\includegraphics[width=0.7\columnwidth]{Figures/gen_time.pdf}
		} 
		\subfigure[Expected energy consumption]
		{
			\label{fig:e32}
			\includegraphics[width=0.7\columnwidth]{Figures/gen_energy.pdf}
		} 
		\caption{Comparison of time and energy for different processor level MTBF. $W=10^6$ hours, $N=10^6$, $\rho=0.5$.}
	\end{center}
	\label{fig:com3}
\end{figure}




\subsection{Impact of Processor Count}
\label{eval_processor}
The system scale, measured in number of processors, has a direct impact on the failure rate seen by the application. To study its impact, we vary $N$ from 10,000 to 1,000,000 with $W$ scaled proportionally, i.e., $W=N$. When MTBF is 5 years, the results are shown in Figure~\ref{fig:n5}. Please note that the time and energy for checkpoint/restart when $N=1,000,000$ are beyond the scope of the figures, so we mark their values on top of their columns. When completion time is considered, Figure~\ref{fig:nt5} clearly shows that each of the three fault tolerance alternatives has its own advantage. Specifically, checkpoint/restart is the best choice for small systems at the scale of 10,000 processors, Leaping Shadows outperforms others for systems with 100,000 processors, while process replication has slight advantage over Leaping Shadows for larger systems. On the other hand, Leaping Shadows wins for all system sizes when energy consumption is the objective. 

When MTBF is changed to 25 years, the performance of checkpoint/restart improves a lot, due to the reduced frequency of checkpointing and decreased need of restarting. However, compelled to rollback every time there is a failure, checkpoint/restart is still orders of magnitude worse in time and energy than that of the other two approaches. Leaping Shadows benefits much more than process replication from the increased MTBF. As a result, Leaping Shadows is able to achieve shorter completion time than process replication when $N$ reaches 1,000,000.

\begin{figure}[!h]
	\begin{center}
		\subfigure[Expected completion time]
		{
			\label{fig:nt5}
			\includegraphics[width=0.7\columnwidth]{Figures/tnt5}
		} 
		\subfigure[Expected energy consumption]
		{
			\label{fig:ne5}
			\includegraphics[width=0.7\columnwidth]{Figures/tne5}
		} 
	\end{center}
	\caption{Comparison of time and energy for different number of processors. $W=N$, MTBF=5 years, $\rho=0.5$.}
	\label{fig:n5}
\end{figure}


\subsection{Impact of Workload}
\label{eval_workload}
To a large extent, workload determines the time exposed to failures. With other factors being the same, an application with a larger workload is likely to encounter more failures during its execution. Hence, it is intuitive that workload would impact the performance comparison. 
Fixing $N$ at 1,000,000, we increase $W$ from 1,000,000 hours to 12,000,000 hours. Figure~\ref{fig:w25} assumes a MTBF of 25 years and shows both the time and energy. Checkpoint/restart has the worst performance in all cases. In terms of completion time, process replication is more efficient when workload reaches 6,000,000 hours. Considering energy consumption, however, Leaping Shadows is able to achieve the most savings in all cases. When MTBF of 5 years is used, the difference is that process replication consumes less energy than Leaping Shadows when $W$ reaches 6,000,000.
%\vspace{0.5in}

\begin{figure}[!h]
	\begin{center}
		\subfigure[Expected completion time]
		{
			\label{fig:wt25}
            \includegraphics[width=0.7\columnwidth]{Figures/twt25}
			%\includegraphics[width=0.7\columnwidth]{Figures/tne5}
		} 
		\subfigure[Expected energy consumption]
		{
			\label{fig:we25}
			\includegraphics[width=0.7\columnwidth]{Figures/twe25}
            %\includegraphics[width=0.7\columnwidth]{Figures/tne5}
		} 
	\end{center}
	\caption{Comparison of time and energy for different workloads. $N=10^6$, MTBF=25 years, $\rho=0.5$.}
	\label{fig:w25}
    \vspace{-0.5in}
\end{figure}

\subsection{Impact of Static Power Ratio}
\label{eval_static_power}
With various architectures and organizations, servers
vary in terms of
power consumption. The static power ratio $\rho$ is used to abstract the
amount of static power consumed versus dynamic power. 
%$\rho$ does not impact the completion time, but power and energy consumption.
Considering modern systems, we vary $\rho$ from 0.3 to 0.7 and study its effect
on the expected energy consumption. The results for Leaping Shadows with $\alpha=5$ are normalized to that of process replication and shown in 
Figure~\ref{fig:power_ratio}. The results for other values of $\alpha$ have similar behavior and thus are not shown. Leaping Shadows achieves
more energy saving when static power ratio is low, since it saves dynamic 
power but not static power. When static power ratio is low ($\rho=0.3$), Leaping Shadows
is able to save 20\%-24\% energy for the MTBF of 5 to 25 years. The saving decreases to 5\%-11\% when $\rho$ reaches 0.7. 

\begin{figure}[!h]
	\begin{center}
		\includegraphics[width=0.7\columnwidth]{Figures/ts_power_5}
	\end{center}
	%\vskip -0.22in 
	\caption{Impact of static power ratio on energy consumption. $W=10^6$ hours, $N=10^6$, $\alpha$=5.}
	\label{fig:power_ratio}
\end{figure}


\subsection{Adding Collocation Overhead}
\label{eval_collocation}
%The last study is conducted to capture the impact on the 
%performance of Leaping Shadows brought by 
%collocation overhead. We re-model the speed of shadows as $\sigma_s^b=\frac{1}{\alpha^{1.5}}$ to simulate the 
%effect of memory thrashing and context switch. 

Leaping Shadows increases memory requirement\footnote{Note that this problem is not intrinsic to Leaping Shadows, as in-memory checkpoint/restart also requires extra memory.} when multiple shadows are collocated. Moreover, this may have an impact on the execution rate of the shadows due to cache contention and context switch. 
To capture this effect,  
we re-model the rate of shadows as $\sigma_s^b=\frac{1}{\alpha^{1.5}}$.
Figure~\ref{fig:comp_vary_fail_speed} shows the impact of collocation overhead on expected energy consumption for Leaping Shadows with $\alpha=5$, with all the values normalized to that of process replication. %The results for other values of $\alpha$ have similar behavior and thus are not shown. 
As expected, energy consumption is penalized because
of slowing down of the shadows. It is surprising, however, that the impact is quite small, with the largest difference being 4.4\%. The reason is that failure-induced leaping can take advantage of the recovery time after each failure and achieve forward progress for shadow processes that fall behind. 
The results for other values of $\alpha$ have similar behavior. 
When $\alpha=10$, the largest difference further decreases to 2.5\%. 

%As expected, both completion time and energy consumption are penalized because
%of slowing down of the shadows. It is surprising, however, that Leaping Shadows with $\alpha=3$ is impacted by the 
%most, while when $\alpha=9$, which means collocating more shadows on each shadow processor, is slightly influenced. After careful analysis,
%we realize that the reason is $\alpha=3$ had the largest values for completion time and energy consumption. Even the percentage of increase after adding the penalty is the smallest, the absolute increase is still the largest. 
%When $\alpha=9$, Leaping Shadows can still achieve 15\%-20\% energy saving with less than 7\% increase in completion time. 

\begin{figure}[!h]
	\begin{center}
		\includegraphics[width=0.7\columnwidth]{Figures/tne5}
        %\includegraphics[width=0.7\columnwidth]{Figures/collocation.pdf}
	\end{center}
	\caption{Impact of collocation overhead on energy consumption. $W=10^6$ hours, $N=10^6$, $\rho$=0.5, $\alpha$=5.}
	\label{fig:comp_vary_fail_speed}
\end{figure}






\section{Summary}

As the scale and complexity of HPC systems continue to increase, both the failure rate and power consumption are expected to increase dramatically, making it extremely challenging to deliver the designed performance. Existing fault tolerance methods rely on either time or hardware redundancy. Neither of them appeals to the next generation of supercomputing, as the first approach may incur significant delay while the second one constantly wastes over 50\% of the system resources.

In this work, we present a comprehensive discussion of the techniques that enable Leaping Shadows to achieve scalable resilience in future extreme-scale computing systems. In addition, we develop a series of analytical models to assess its performance in terms of reliability, completion time, and energy consumption. 
Through comparison with traditional process replication and checkpoint/restart, we identify the scenarios where each of the alternatives should be chosen for best performance.












\chapter{Tolerance of Silent data corruption}
\label{chapter:sdc}
Modern scientific discoveries and business intelligence rely heavily on large-scale simulation and data analytics. The next generation of parallel applications will require massive computing capacity to support the execution of predictive models and analysis of massive quantities of data, with significantly higher resolution and fidelity than what is possible within existing computing infrastructures. In order to deliver the desired performance of emerging  applications, future HPC and Cloud computing infrastructure is expected to continue to grow in both scale and complexity, resulting in an urgent need for efficient and scalable fault tolerance solutions to all kind of causes and symptoms. 

A significant body of work targets at improving the scalability of checkpoint/restart, while also lots of efforts are devoted to making process replication based approaches a viable alternative. Most of the existing work assumes a fail-stop fault model, such that failures are detectable by monitoring hardware or network. Silent data corruption (SDC) is yet a different class of failures. Different from crash failures, SDC may remain undetected and corrupt the intermediate or final results. It materializes as bit flips in storage (both volatile and non-volatile memory) or even within processing cores. In modern computers, a single bit flip in memory can be detected with CRC and corrected with error correction code (ECC). Double bit flips, however, is beyond the hardware fault tolerance capability in most systems. Meanwhile, even single bit flips in the processor core remain undetected as only caches feature ECC while register files or even ALUs typically do not~\cite{fiala_2012_sdc}.



In large-scale production systems, SDC has become a major concern for the user, which can not only cause data loss, but also has serious impact on the integrity of job outputs, jeopardizing scientific research or business decisions. 
SDC is one of the most critical problems in cloud data processing~\cite{wang2015understanding}. 
As the capacity of and memory and disks grows, the likelihood of SDC caused by hardware failures also increases. For example, a hard drive failure resulted in Facebook temporarily losing over 10\% of photos published on their social network~\cite{wang2015understanding}. Similarly, SDC due to software bugs also becomes prevalent as the scale of cloud systems keeps expanding. For example, Amazon Simple Storage Service (S3) once experienced a data corruption problem caused by a load balancer bug~\cite{balding2009question}. In HPC systems, SDC is also attracting more attention. It is reported that, due to the high density of DIMMs, Cray XT5 at Oak Ridge National Lab’s encounters double bit flips on a daily basis~\cite{geist2011monster}. Although previous studies  have shown that disk errors and DRAM errors in large-scale production systems are happening often enough to require attention, little research has been done to answer to this challenge~\cite{fiala_2012_sdc}.

This chapter applies the Leaping Shadows model, discussed in Chapter~\ref{chapter:scale}, to deal with SDC in large-scale computing systems. In order to detect and correct SDC, we come up with a new scheme that equips Leaping Shadows with triple modular redundancy, similar to \cite{fiala_2012_sdc}. This new scheme inherits the adaptivity and power-awareness from Leaping Shadows, and optimizes a combination of process execution rates to balance the trade-offs among time, hardware, and power. In the following sections, we first describe a parallel programming paradigm typically used in compute- and data-intensive applications. Next, we
present the new Leaping Shadows scheme with its execution model. Then we build analytical models to study its performance and power requirements, and form an optimization framework, which derive the optimal execution parameters. Lastly, comparative analysis and evaluation results are given.  

%HPC and Cloud are two ecosystems that are designed for different applications and with disparate design principles. However, Big data technologies, such as Hadoop, clustered storage, and data visualization, are now merging with traditional HPC technologies. On the one hand, an increasing portion of HPC workloads is becoming data intensive. On the other hand, Big data applications are requiring more and more computing power. As the boundaries between Cloud and HPC continue to blur, it is clear that there is an urgent demand for a systematic computational model that adapts to the computing platform and accommodates the underlying workloads. 

%Recognizing this challenge, we study Leaping Shadows as a systematic computational model, that achieves power-aware fault tolerance for both compute- and data-intensive applications. In chapter~\ref{chapter:scale}, we discuss the application of the Leaping Shadows model to compute-intensive workloads in HPC, in which communication and synchronization may be frequent. Analytical models and evaluation demonstrate that Leaping Shadows can achieve higher performance and significant energy savings in comparison to existing approaches in most cases. This chapter focuses on data-intensive applications in the Cloud.

\section{Parallel Programming Paradigm}

Parallel computing frameworks, such as MPI, MapReduce, and Pregel, have been widely adopted for large-scale data analytics and simulation. To efficiently handle the sheer amount of data, and to utilize a cluster of nodes and/or multiple processors within a node, these frameworks typically arrange a job into hundreds or thousands of tasks scheduled to execute in parallel.
There are two widely adopted approaches to parallelism.
In task-parallelism, we partition the various tasks carried out in solving the problem among the processors. In data-parallelism, we partition the data used in solving the problem among the processors, and each processor carries out more or less similar operations on its part of the data. Despite the parallelism approach, the parallel processing paradigm is abstracted by the Bulk Synchronous Parallel (BSP) model~\cite{skillicorn1997questions}. 
%, each processing one partition of the whole dataset\footnote{For iterative and graph algorithms, multiple stages of such parallel execution may be needed.}. This paradigm is depicted in Figure~\ref{fig:bsp}.



According to the BSP model, there is a set of processors which may follow different threads of computation, with each processor equipped with fast local memory and interconnected by a communication network. A BSP computation proceeds in a series of global supersteps, which consists of three components:
\begin{itemize}
	\item Computation: every participating processor may perform local computations, i.e., each process can only make use of values stored in the local fast memory of the processor. 
    \item Communication: The processes exchange data between themselves to facilitate remote data storage capabilities.
	\item Barrier synchronization: When a process reaches this point (the barrier), it waits until all other processes have reached the same barrier.
\end{itemize}
%This is depicted in Figure~\ref{fig:bsp}. 
Pregel is directly inspired by the BSP model. 
In MPI programs, whose main body usually consists of a loop, each iteration could be a BSP superstep. This is also true for iterative MapReduce applications, in which each superstep includes a map or a reduce phase. 


%\begin{figure}[!h]
%  \begin{center}
%      \includegraphics[width=0.8\columnwidth]{Figures/bsp}
%  \end{center}
%  \caption{Bulk synchronous parallel processing paradigm.}
%  \label{fig:bsp}
%\end{figure}

In today's parallel computing frameworks, re-execution on top of a heartbeat protocol is mainly used to provide fault tolerance and to deal with staggers. This approach works fine for system scales such that failures are rare and applications that are delay-tolerant. If failures are frequent, however, large delay can be incurred since one faulty task or stagger may delay the whole job execution. In addition, re-execution only handles crash failures. This is not acceptable for applications with strict response time requirements or applications that are vulnerable to SDC. In contrast, Leaping Shadows, if applied, will enable these frameworks to trade-off among multiple objectives, while respecting any hard or soft deadline and handling both crash failures and SDC. 

\section{Leaping Shadows Execution Model}
%\section{Dealing with silent data corruption}
Different from crash failures which crash a processor, silent data corruption allows a faulty processor to continue to completion but may silently generate incorrect results. To deal with $f$ failures, $(2f+1)$ replicas are needed, and voting is required periodically to detect failure. For example, when at most 1 silent data corruption could occur, 3 replicas are sufficient to detect and tolerate the failure. In the following, we will focus on tolerating one silent data corruption per task. 

Compared to traditional replication techniques, Leaping Shadows can deal with silent data corruption with higher efficiency and less resource requirement. For each task, Leaping Shadows associates two shadows with each main process. One shadow is designated as the primary shadow, and its duty is to compare with its associated main at a voting point to detect SDC. Based on the BSP model, each barrier synchronization is naturally a voting point. In order to detect failure as soon as possible,
the primary shadow executes at the same rate as its associated main. In addition to a primary shadow, a secondary shadow is needed to correct a SDC, once occurred, through voting. 
To save energy, the secondary shadow executes at a potentially lower rate than the other two replicas, and dynamically speeds up if failure is detected. 

To take advantage of the forward progress of the fast replicas, Leaping Shadows performs a leaping at every voting point to leap forward the secondary shadow when possible. Specifically, when the main and the primary shadow both arrive at a voting point and they reach  agreement, the results and execution state are copied to the secondary shadow to achieve forward progress. This is illustrated in Figure~\ref{fig:silent_model}. Since this leaping is triggered by a voting, it is referred to as \textit{voting-induced leaping}. If the main or the primary shadow experiences a SDC, the failure will be detected at the next voting point. At this time, the secondary shadow speeds up to reach the specific voting point, and participates in the voting to detect which replica fails. Then a leaping from one correct replica to the failed replica can rejuvenate the failed one, after which all replicas resume normal execution. Note that the secondary shadow is only useful when one of the other two replicas fails. If the secondary shadow fails, it will automatically get rejuvenated by leaping. 

\begin{figure}[!h]
  \begin{center}
      \includegraphics[width=0.8\columnwidth]{Figures/silent_model}
  \end{center}
  \caption{Tolerating SDC by applying Leaping Shadows with triple modular redundancy.}
  \label{fig:silent_model}
\end{figure}


\section{Analytical Models and Optimization Framework}
To assess the efficiency of Leaping Shadows to deal with SDC, following subsections develop analytical models for the expected response time and energy consumption under a failure distribution. Then using these analytical models, an optimization problem is formulated to minimize the expected energy consumption of Leaping Shadows under strict deadline constraint. This not only demonstrates Leaping Shadows' adaptivity to the desired trade-off, but also provides a framework with which we can perform comparative performance analysis to state-of-the-art approaches.

\subsection{Notations}
Let $W$ denote the total workload to process a task. Let $N$ denote the number of BSP supersteps, which is also the number of voting points. Then each voting interval has a workload of $w=\frac{W}{N}$. Let $\sigma_{max}$ denote the maximum execution rate. $R_{min}=\frac{W}{\sigma_{max}}$ is the minimal response time. Let $\overline{R}=(1+\alpha)R_{min}$ $(0\leq \alpha \leq 1)$, where $\alpha$ is called laxity factor, denote the target response time considering fault tolerance overhead. Let $\lambda$ denote the failure rate, and $f(t)$ denote the failure density function. %Let $d(t)$ denote the failure detection time at a voting point. 
Let $E(\sigma, [t_1, t_2])$ denote the energy consumption of a replica when executing at rate $\sigma$ for an interval from $t_1$ to $t_2$. For each leaping, let $T_l$ denote the time overhead, and $E_l$ denote the energy cost.

To deal with silent data corruption, the Leaping Shadows model associates two shadows with the main process for each task. According to the execution model described in above section, there are three execution rates:
\begin{itemize}
	\item $\sigma_m$, the execution rate of the main and the primary shadow 
    \item $\sigma_b$, the execution rate of the secondary shadow before a SDC is detected 
    \item $\sigma_a$, the execution rate of the secondary shadow after a SDC is detected 
\end{itemize}

\subsection{Response Time}
Assuming there is at most one silent data corruption per task, two scenarios need to be considered, i.e., a SDC occurs to the main process or the primary shadow, or neither of them fails. The failure of the secondary shadow has no impact on the response time, thus is ignored in the following analysis. 

In the first scenario, where no failure occurs, the execution time is determined by the main process, as $t_r^m=\frac{W}{\sigma_m}$. Considering the time for leaping, the total response time is $t_{rl}^m=\frac{W}{\sigma_m} + N \times T_l$.

In the second scenario, 
where a fast replica fails, the delay is the time for the secondary shadow to catch up with respect to a voting interval. The time for the main process to complete a voting interval is $t_v = \frac{w}{\sigma_m}$. The time required by the secondary shadow to complete the remaining work in the current interval is $t_d = \frac{w - t_v\times \sigma_b}{\sigma_a}$. The execution time is $t_r^s = t_r^m + t_d$. Considering the time for leaping, the total response time is $t_{rl}^s=t_r^s + N \times T_l$.


\subsection{Power and Energy Consumption}

Dynamic voltage and frequency scaling
(DVFS) is assumed in this work to achieve the desired execution rate of a process. It
is well known that one can reduce the dynamic processor power consumption at
least quadratically by reducing the frequency linearly. The
dynamic processor power consumption executing at rate
$\sigma$ is given by the function $p_d(\sigma)=\sigma^n$ where $n \ge
2$. Throughout this section, we assume that dynamic power is cubic in relation to the processor frequency.

In addition to the dynamic power, processor leakage and other components
(memory, disk, network etc.) all contribute to static power
consumption, which is independent of the processor frequency. In this section, we
define static power as a fixed fraction of the total power consumed
when executing at maximum rate, referred to as $\rho$. Hence, a processor's
power consumption is expressed as
$p(\sigma)=\rho \times \sigma_{max}^3 + (1-\rho)\times \sigma^3$.

Next we calculate the expected energy consumption for a single task under Leaping Shadows. 
Corresponding to the two scenarios in the above response time analysis, the energy consumption also falls into two cases. 
If neither of the main process and primary shadow fails, the energy consumption of the three replicas weighted by its probability is
\begin{equation}
%\begin{split}
E_1 =  (1 - \int_{0}^{t_r^m} f(t)dt)^2  \times 
       \{2E(\sigma_m, [0, t_r^m])+E(\sigma_b, [0, t_r^m])\}
%E_1 = & (1 - \int_{0}^{t_r^m} f_m(t)dt)^2  \times \\
%      & \{2E(\sigma_m, [0, t_r^m])+E(\sigma_b, [0, t_r^m])\}
%\end{split}
\end{equation}
The first factor is the probability that neither of them fails, as $\int_{0}^{t_r^m} f(t)dt$ is the probability that a replica encounters a SDC during task execution. The second factor models the energy of the three replicas from task start to end, with two replicas execute at rate $\sigma_m$ and one replica executes at $\sigma_b$.

If the main or the primary shadow fails, the energy consumption weighted by its probability is 
\begin{equation}
\begin{split}
E_2 = & 2 \times (1 - \int_{0}^{t_r^m} f(t)dt) \times \int_{0}^{t_r^m} f(t)dt \times 
\\  & \{2E(\sigma_m, [0, t_r^m])+E(\sigma_b, [0, t_r^m])  + 2E(0, [t_r^m, t_r^s])+E(\sigma_a, [t_r^m, t_r^s]\} 
%E_2 = & 2(1 - \int_{0}^{t_r^m} f_m(t)dt) \times \int_{0}^{t_r^m} f_m(t)dt \times 
%\\  & \{2E(\sigma_m, [0, t_r^m])+E(\sigma_b, [0, t_r^m]) \\ & + 2E(0, [t_r^m, t_r^s])+E(\sigma_a, [t_r^m, t_r^s]\} 
\end{split}
\end{equation}
The first line calculates the probability that one fast replica fails while the other successfully completes. In addition to the energy in the first scenario, also accounted is the energy consumed during the secondary shadow catches up. This energy corresponds to the main process and the primary shadow idly waiting and the secondary shadow speeding up to reach the next voting point. 



All in all, the total energy consumption is the sum of the above two, plus the energy cost of leaping, i.e., $E_{total}=E_1 + E_2 + N\times E_l$. 

\subsection{Optimization}
\label{sec:silent_opt}
When applying Leaping Shadows to deal with SDC, an optimization problem formulation is also needed to derive the optimal execution rates for both the main and shadow processes. With the generic optimization framework introduced in Chapter~\ref{chapter:shadowing}, we define the optimization objective as minimizing energy under response time constraint:

\begin{equation}
\begin{alignedat}{2}
\min_{\sigma_m,\sigma_b,\sigma_a} \qquad  & E_{total} (W,N,\overline{R},\rho,\lambda,\sigma_{max}, T_l, E_l)  \\
s.t.  \qquad          & 0 \leq \sigma_m \leq \sigma_{max} \\
                      & 0 \leq \sigma_b \leq \sigma_m\\
                      & \sigma_b \leq \sigma_a \leq \sigma_{max} \\
                      & t_{rl}^s \leq \overline{R}
\end{alignedat}
\end{equation}

The first constraint says the execution rate of the main process and primary shadow should observe the physical processor limit. The second constraint indicates that the initial rate of the secondary shadow should not exceed that of the main process and primary shadow. The third constraint ensures that the secondary shadow could speed up after detecting a failure. The last constraint guarantees that the deadline is met even in the case of failure. Same as before, non-linear optimization techniques can be used to solve the above problem, and the output will be the three optimal execution rates.

\section{Evaluation}
Using the optimization framework developed above, this section evaluates the performance of Leaping Shadows by comparing with traditional process replication using triple modular redundancy~\cite{fiala_2012_sdc}, under various environments and different application requirements. Without a differential treatment of the replicas, process replication requires all three replicas execute at the same rate before and after a failure. In the comparison, however, we also optimize this single rate for process replication in a way similar to Section~\ref{sec:silent_opt}\footnote{Process replication is a special case of Leaping Shadows where $\sigma_m = \sigma_b = \sigma_a$.}, in order to demonstrate the benefits of the unique design in Leaping Shadows. 

After careful analysis of the analytical models, we identify the important parameters of static power ratio, laxity in deadline, workload, number of voting interval, and leaping cost, and study the impact of each parameter. With the understanding that failure rate of each processor will remain more or less the same in the near future, We fix MTBF to be 5 years for realistic consideration. %Also, because workload does not change the energy consumption as in above subsection, we do not show its study and instead fix it to be 100 hours per processor. 
The comparison results are shown in Figure~\ref{fig:silent_eval}.

\afterpage{
\begin{figure}[!t]
	\begin{center}
		\subfigure[Impact of static power ratio. $\alpha=50\%$, $N=10$, $W=100$ hours.]
		{
			\label{fig:silent_power}
			\includegraphics[width=0.45\textwidth]{Figures/silent_power}
		}
		\subfigure[Impact of laxity. $\rho=0.3$, $N=10$, $W=100$ hours.]
		{
			\label{fig:silent_laxity}
			\includegraphics[width=0.45\textwidth]{Figures/silent_laxity}
		}
        \vskip 0.5in
        \subfigure[Impact of voting interval. $\alpha=50\%$, $\rho=0.5$, $W=100$ hours.]
		{
			\label{fig:silent_interval_1}
			\includegraphics[width=0.45\textwidth]{Figures/silent_interval_1}
		}
        \subfigure[Impact of voting interval. $\alpha=50\%$, $\rho=0.3$, $W=100$ hours.]
		{
			\label{fig:silent_interval_2}
			\includegraphics[width=0.45\textwidth]{Figures/silent_interval_2}
		}
        \vskip 0.5in
        \subfigure[Impact of task workload. $\alpha=25\%$, $\rho=0.5$, $N=10$.]
		{
			\label{fig:silent_workload}
			\includegraphics[width=0.45\textwidth]{Figures/silent_workload}
		}
        \subfigure[Impact of leaping cost. $\alpha=50\%$, $\rho=0.5$, $N=10$, $W=100$ hours.]
		{
			\label{fig:silent_leap}
			\includegraphics[width=0.45\textwidth]{Figures/silent_leap}
		}
	\end{center}
	\caption{Comparison between Leaping Shadows and process replication for energy consumption under silent data corruption. MTBF=5 years.}
	\label{fig:silent_eval}
\end{figure}

\clearpage
}

Figure~\ref{fig:silent_power} reveals the energy consumption of the two compared approaches at different static power ratios, when laxity is 50\% and number of voting intervals is 10. When static power ratio is less than 0.8, both approaches reduces the execution rates to minimize energy. At 0 static power ratio, Leaping Shadows saves 25.6\% energy compared to process replication. The saving decreases as static power ratio increases, and finally Leaping Shadows converges to process replication when static power ratio reaches 1. Modern computers has a static power ratio between 40\% and 70\%. Within this target range of static power ratio, Leaping Shadows achieves 12.5\% energy savings with respect to process replication.
%Note that energy saving is less than that of crash failure tolerance because both fast and slow replicas have to process a partition from the same end in order to detect and tolerate silent data corruption.

With 10 voting intervals and 0.3 as the static power ratio, the impact of laxity is illustrated in Figure~\ref{fig:silent_laxity}. When there is no laxity, Leaping Shadows is forced to execute all three replicas at the maximum rate, which is essentially process replication. As laxity increases, both compared approaches have more room to slow down, thereby reducing energy. By coupling two fast replicas with one slow one and uses leaping to achieve forward progress for the slow replica, Leaping Shadows always saves 13.5\%-16.9\% energy compared to process replication when there is laxity.

One interesting questions is how the number of voting intervals changes the picture. Without considering the overhead of leaping, intuition tells us that the more voting intervals, the less effect a failure can have on the total execution time of Leaping Shadows, and thus the better performance. This is mostly true, according to Figure~\ref{fig:silent_interval_1}. However, the figure also shows that after the number of voting intervals reaches 5, its impact becomes negligible. We also study this behavior with static power ratio changed to 0.3, as shown in Figure~\ref{fig:silent_interval_2}. 
As a result, both of the two compared approaches are able to reduce energy consumption by further slowing down. 
Although the difference between them decreases at a small number of voting intervals, Leaping Shadows keeps increasing its energy savings with the number of voting intervals, up to 20.

The next parameter studied is the total workload, which determines the failure-free execution time, and thus the propensity of the task to failures. For Figure~\ref{fig:silent_workload}, we vary the task workload from 1 hour to 1000 hours. Compared to the 5 years' MTBF, all the workloads considered are relatively small, thus the failure of a given task is very unlikely. Therefore, there is only slight change in the energy consumption. 

All the above experiments ignore the overhead of leaping. Assuming each time leaping consumes 1 unit of energy, the last experiment studies the impact of leaping overhead, which is defined as a fraction of minimum response time. Figure~\ref{fig:silent_leap} reveals that when the overhead is below 30\%, it has negligible impact. At 40\% overhead, Leaping Shadows is forced to slightly increase its rates, and thus incurs a slightly higher energy consumption. When overhead is 50\%, it essentially offsets the laxity, which is 50\%, and leaves no room for Leaping Shadows to slow down. As a result, all replicas need to execute at the maximum rate and end up with consuming more energy than process replication, which does not perform leaping at all.

\section{Summary}
Scientific research and engineering development are increasingly relying on computational modeling, simulation, and data analytics to augment theoretical analysis. Behind the wheel, data has become the ultimate driving force that yields insights and propels innovation. With a torrent of data generated every second from distributed sensors, social media, software logs and so on, it is critical to analyze and visualize the data in a timely manner,  at massively parallel scale, and with fault tolerance capabilities. 

Leaping Shadows is a novel fault-tolerant computational model that unifies HPC and Big Data analytics. 
The flexibility within the model allows it to embrace different optimization techniques in accordance with the underlying workloads, whether compute-intensive or data-intensive. 
By designing the execution model of Leaping Shadows in accordance with the generic Bulk Synchronous Parallel model, Leaping Shadows can be applied to both HPC and Cloud environments, to deal with different types of failures, or multiple types of failures at the same time. 

While previous chapters discuss on tolerance of crash failures, this chapter extends the Leaping Shadows model and studies the tolerance of silent data corruption. 
By exploring the interplay between performance, fault-tolerance, and energy consumption, Leaping Shadows is predicted to save a significant amount of energy (up to 64.9\%) compared to existing fault tolerance approaches, while respecting strict response time requirements. In the future, we plan to implement this model and perform intensive empirical evaluation to verify the accuracy of the analytical models.     











\chapter{\texorpdfstring{\MakeLowercase{rs}MPI}{rsMPI}: an implementation in MPI}
\label{chapter:implementation}
%Implementation issues have been briefly discussed in \cite{cui_2016_scalcom}. 
This section presents the details of our full-feature implementation of rsMPI, which is an MPI library for Rejuvenating Shadows. 
Similar to rMPI and RedMPI~\cite{ferreira_sc_2011,fiala_2012_sdc}, rsMPI is implemented as a separate layer between MPI and user application. It uses the standard MPI profiling interface to intercept every MPI call and enforces Rejuvenating Shadows logic. In this way, we not only can take advantage of existing MPI performance optimization that numerous researches have spent years on, but also achieve portability across all MPI implementations that conform to the MPI specification.
When used, rsMPI transparently spawns the shadow processes during the initialization phase, manages the coordination between main and shadow processes, and guarantees order and consistency for messages and non-deterministic MPI events.
%Once completed, users should be able to link to the library without any change to existing codes. 

\subsection{MPI rank}
A rsMPI world has 3 types of identities: main process, shadow process, and coordinator process that coordinates between main and shadow. A static mapping between srMPI rank and application-visible MPI rank is maintained so that each process can retrieve its identity. For example, if the user specifies $N$ processes to run, rsMPI will translate it into $2N + K$ processes, %where $K$ is the number of shadowed sets. Then the first $N$ ranks will be the mains, the next $N$ ranks be the shadows, and the last $K$ ranks be the coordinators. 
with the first $N$ ranks being the mains, the next $N$ ranks being the shadows, and the last $K$ ranks being the coordinators. 
We also statically group the processes into shadowed sets according to a user configuration file. Figure~\ref{fig:logical_org} shows an example rsMPI world with 16 application-visible processes grouped into 4 shadowed sets. Using the MPI profiling interface, we added wrapper for MPI\_Comm\_rank() and MPI\_Comm\_size(), so that each process (main or shadow) gets its correct execution path.

\begin{figure}[!t]
  \begin{center}
      \includegraphics[width=\columnwidth]{figures/logical_org}
  \end{center}
  \caption{Logical organization of a MPI world for Rejuvenating Shadows. $N=16$, $K=4$.}
  \label{fig:logical_org}
\end{figure}

\subsection{Execution rate control}
While the mains always execute at maximum rate for HPC's throughput consideration, the shadows can be configured to execute slower by collocation \cite{cui_2016_scalcom}. 
When using rsMPI, the user has the flexibility to specify the desired collocation ratio (the number of shadows to collocate on each processing unit) through a configuration file that we provide. Accordingly, rsMPI will generate an MPI rankfile and provide it to the MPI runtime to control the process mapping. Note that rsMPI always maps the main and shadow processes of the same task onto different nodes. This is preferred as a fault on a node will not affect both of them. 
%In addition, rsMPI will automatically translate the number of processes (MPI ranks) specified for the application into the number of processes needed by rsMPI. For example, if the user specifies $N$ processes, rsMPI will translate it into $2N + K$ processes, where $K$ is the number of shadowed sets. Therefore, rsMPI will spawn $N$ main processes, $N$ shadow processes, and $K$ shadow coordinator processes for $K$ shadowed sets during MPI initialization. The logical organization is depicted in Figure~\ref{fig:logical_org}. 
To minimize resource usage, each coordinator is collocated with all the shadows in the shadowed set. Since each coordinator simply waits for incoming control messages (discussed below) and does minimal work, it has negligible impact on the execution rate of the collocated shadows. 


\subsection{Coordination between mains and shadows}
Each shadowed set has a coordinator process dedicated to coordination between the main and shadow processes in the shadowed set. 
Coordinators do not execute any application code, but just wait for srMPI defined control messages, and then carry out some 
coordination work accordingly. There are three types of control messages, i.e., termination, failure, and leaping. Correspondingly, each coordinator is responsible for three types of work:
\begin{itemize}
  \item when a main finishes, it notifies its coordinator, which then forces the associated shadow to terminate.
  \item when a main fails, its associated shadow needs to catch up, so the coordinator will temporarily suspend the other collocated shadows, and resume their execution once the recovery is done.
  \item when a main or shadow initiates a leaping, either because of failure or buffer overflow, the coordinator triggers leaping at the associated shadow or main.
\end{itemize}
%Main and shadow processes can communicate with their coordinator via control messages defined by rsMPI. 
% Firstly, when a main process finishes, it notifies its coordinator, which then forces the associated shadow process to terminate. Secondly, when a main process fails, its associated shadow process needs to catch up, so the coordinator will temporarily suspend the other collocated shadows, and resume their execution once the recovery is done. Lastly, when a main or shadow initiates a leaping, either because of failure or buffer overflow, the coordinator triggers leaping at the associated shadow or main.  
%the RAS system will notify the corresponding shadow coordinator, which then promotes the associated shadow process to a new main process and kills the collocated shadows.
To separate control messages from data messages, rsMPI uses a dedicated MPI communicator, created during MPI initialization, for the control messages. In addition, to ensure fast response and minimize the number of messages, coordinators also use OS signals to communicate with their collocated shadows. %There are three types of coordination in srMPI.

\subsection{Message passing and consistency}
We wrapped around every MPI communication function and implemented the consistency protocol described in Section~\ref{sec:shadow}. For sending functions, such as MPI\_Send() and MPI\_Isend(), rsMPI requires the main to duplicate the sending while the shadow does no work. For receiving functions, such as MPI\_Recv() and MPI\_Irecv(), both the main and the shadow does one receiving from the main process at the sending side. Internally, collective communication in rsMPI uses point-to-point communication in a binomial tree topology, which demonstrates excellent scalability.

We assume that only MPI operations can introduce non-determinism, and the SYNC message shown in Figure~\ref{fig:cons_protocol} is introduced to enforce determinism. MPI\_ANY\_SOURCE may result in different message receiving orders between a main and its shadow. To deal with this, we serialize MPI\_ANY\_SOURCE message receiving by having the main first do the receiving and then use a SYNC message to forward the message source information to its shadow, which then issues a receiving with the specific source. Other operations, such as MPI\_Wtime() and MPI\_Probe(), can be dealt with in a similar manner by forwarding the result from a main to its shadow.


\subsection{Leaping}
Checkpointing/restart requires each process to save its execution state, which can be used later to retrieve the computation. Leaping is similar to Checkpointing/restart in saving a process' state, but the state is always transferred between a pair of main and shadow. 
To reduce the size of data involved in saving a process' state, we choose to implement leaping in the same way as application-level checkpointing. rsMPI provides a routine for users to register any data as process state. Application developer could use domain knowledge to identify only necessary state data, or use compiler techniques to automate this~\cite{5160999}. 

rsMPI provides the following API for process state registration:

void leap\_register\_state(void *addr, int count, MPI\_Datatype dt);

For each piece of data to be registered, three parameters are needed: a pointer to the address of the data, the number of data items, and the datatype. Internally, rsMPI uses a linked list to keep track of all registered data. %After each call of ``leap\_register\_state()", rsMPI will add a node to its internal linked list to record the three parameters. 
During leaping, the linked list is traversed to retrieve all registered data as the process state.

Coordination of leaping is easier than coordination of checkpointing, since leaping is always between a pair of main and shadow. To synchronize the leaping between a main and a shadow, the coordinator in the corresponding shadowed set is involved. For example, when a main detects failure of another main and initiates a leaping, it will send a control message to its coordinator, which then uses a signal to notify the associated shadow to participate in the leaping. 

Different from Checkpointing where the process state is saved to storage, leaping directly transfers process state between a main and its shadow. 
Since MPI provides natural support for message passing between processes, 
rsMPI uses MPI messages to transfer process state. Although multiple pieces of data can be registered as a process' state, only a single message is needed to transfer the process state, as MPI supports derived datatypes. To prevent the messages carrying process state from mixing with application messages, rsMPI uses a separate communicator for transferring process state. With the synchronization of leaping by coordinator and the fast transfer of process state via MPI messages, the overhead of leaping is minimized. 

A challenge in leaping lies in the need for maintaining state consistency across leaping. To make sure a pair of main and shadow stay consistent after a leaping, not only user-defined states should be transferred correctly, but also lower level states, such as program counter and message buffer, need to be updated correspondingly. Specifically, the lagging process needs to satisfy two requirements. 
Firstly, after leaping the lagging process should discard all obsolete messages before resuming normal execution. Secondly, the lagging process should resume execution at the same point as the target process. We discuss our solutions below, under the assumption that the application's main body consists of a loop, which is true in most cases. 
%Firstly, after updating its state, the lagging process should resume execution at the same point as the target process. Secondly, the lagging process should discard all obsolete message before resuming normal execution. To address these issues, first we assume that the application's main body consists of a loop, which is true in most cases. 

To satisfy the second requirement, we restrict leaping to always occur at certain possible points, and uses internal counter to make sure that both lagging and target processes start leaping from the same point. For example, when a main process triggers a leaping and asks coordinator to notify its associated shadow, the coordinator will trigger the shadow's specific signal handler. The signal handler does not carry out leaping, but sets a flag for leaping and receives a counter value that indicates the leaping point from its main process. Then, the shadow will check the flag and compare the counter value at every possible leaping point. Only when both the flag is set and counter value matches will the shadow start leaping. In this way, it is guaranteed that after leaping the main and shadow will resume execution from the same point. To balance the trade-off between the implementation overhead and the flexibility of checking for when to perform leaping, we choose MPI receive operations as the possible leaping points. 

There is no straightforward solution for the first problem, as the message buffer is maintained by MPI runtime and not visible to rsMPI. Alternatively, rsMPI borrows the idea of message logging to correctly discard all obsolete messages. During normal execution, both the main and shadow record the meta data (i.e., MPI source, tag, and communicator) for all received messages in the receiving order. During leaping, the meta data at the main is transferred to the shadow, so that the shadow knows about the messages that have been received by its main but not by itself. Then the shadow combines MPI probe and MPI receive operations to remove the messages from MPI runtime buffer in correct order. 

%Alternatively, we choose to remove obsolete message from message buffer by having the process execute all the skipped MPI communication routines after it finishes leaping. To achieve this, we require the user to define a function for the MPI communication functions used in the application's main body loop. The function should have two parameters to specify the starting and ending index for skipped iterations. In addition, the user needs to register the function with rsMPI with the following library call:

%void leap\_register\_func(void (*func)(int, int));

%To discard all obsolete messages after leaping, the process that updates its process state will call the registered function, for which the two parameters will be automatically specified by rsMPI. Essentially, it executes all the  MPI communication functions from the skipped iterations and consumes all the useless messages.  

%\subsection{Failure injection and detection}
%As one main goal of this work is to achieve fault tolerance, an integrated fault injector is required to evaluate the effectiveness and efficiency of rsMPI to tolerate failures during execution. To produce failures in a manner similar to naturally occruing process failures, our failure injector is designed to be distributed and co-exist with all rsMPI processes. Failure is injected by sending a specific signal to the target process.

%Failure detection is beyond the scope of srMPI, and we assume the underlying hardware platform has a RAS system that provides this functionality. In our prototype, we emulate a RAS system with a signal handler installed at every main and shadow process. The signal handler catches failure signal from failure injector, and uses a rsMPI defined failure message via a dedicated communicator to notify all other processes of its failure. Similar to ULFM, processes in srMPI can detect failure only when it does an MPI receive operation. When starting a srMPI receive, srMPI checks for failure messages before it does the actual MPI receive operation.

%\subsection{Double in-memory checkpointing}
%We also implemented checkpointing to compare with srMPI in the presence of failures. To be optimistic, we chose double in-memory checkpointing that is much more scalable then disk-based checkpointing~\cite{zheng2004ftc}. Same as leaping in srMPI, our implementation provides an API for process state registration. This API requires the same parameters as leap\_register\_state(void *addr, int count, MPI\_Datatype dt), but internally, it needs to allocate extra memory in order to store the state of a ``buddy" process. Another provided API is checkpoint(), which can be used to insert a checkpoint in the application code. For fairness, our implementation also uses MPI messages to transfer state between buddies.  

\chapter{Conclusion}
\label{chapter:summary}
As our reliance on IT continues to increase, the complexity and urgency of the problems our society will face in the future will increase much faster than are our abilities to understand and deal with them. Future IT systems are likely to exhibit a level of interconnected complexity that makes it prone to failure and exceptional behaviors. The high risk of relying on IT systems that are failure-prone calls for new approaches to enhance their performance and resiliency to failure.

HPC and Cloud are two ecosystems that are designed for different applications and with disparate design principles. However, Big data technologies, such as Hadoop, clustered storage, and data visualization, are now merging with traditional HPC technologies. 
On the one hand, an increasing portion of HPC workloads is becoming data intensive.
On the other hand, Big data applications are requiring more and more computing power. 
As the boundaries between Cloud and HPC continue to blur, it is clear that there is an urgent demand for a systematic computational model that adapts to the computing platform and accommodates the underlying workloads. 

This thesis presents Leaping Shadows as a novel fault-tolerant computational model that unifies HPC and Big Data analytics and scales to future extreme-scale computing systems. The flexibility in the model allows it to embrace different execution strategies in accordance with the underlying workloads, whether it is compute-intensive or data-intensive. Leaping Shadows takes advantage of the unique design in the Shadow Replication model that original main processes are associated with ``lazy" shadows. The differential and dynamic execution rates control enables Leaping Shadows to achieve fault tolerance with power awareness, as well as adaptivity to trade-offs among performance, resilience, and energy costs.  Furthermore, by incorporating creative optimization techniques, Leaping Shadows is able to maintain a consistent level of resilience across high rate and diverse types of failures, with improved performance and reduced resource requirements. 

This thesis systematically studies the viability of Leaping Shadows to enhance system resilience in emerging extreme-scale, failure-prone computing environments. As a first step, customized execution dynamics are designed to deal with different types of failures. Then, analytical models and optimization frameworks are built to derive the optimal process execution rates, while at the same time multiple mechanisms are explored to effectively achieve the desired execution rates. To further verify Leaping Shadows and to validate the analytical models, a prototype implementation is provided in the context of HPC. Empirical evaluation with various benchmark applications confirms that Leaping Shadows is able to outperform state-of-the art fault tolerance approaches. 

The study of the Leaping Shadows model in this thesis is not meant to be complete. The flexibility and diversity in the model point to many future directions. In current design of Leaping Shadows, each main is associated with the same number of shadows. This is ignorant of the variance in the underlying hardware reliability and above application criticality. 
Previous studies have shown that failure rates both vary across systems and vary from node to node within the same system~\cite{schroeder2007,di2014lessons}. According to \cite{di2014lessons}, 19\% of the nodes account for 92\% of the machine check errors on Blue Waters. %The reason for the non-uniform distribution of failure is complicated and may attribute to the manufacture process, heterogeneous architecture, environment factors (e.g. temperature, voltage supply), and/or workloads. 
At the same time, within a system processes may have different criticality. %One possible reason is that the execution model assigns different roles to different processes. 
For example, in the master-slave execution model the master process is a single point of failure, making the failure of the master process much more severe than that of a slave process. %Another possibility is when the user has the option to specify the QoS. For example, in Cloud Computing, users may choose different level of QoS in terms of Service Level Agreement with different amount of payment. 
In fact, \textit{heterogeneous shadowing} techniques can be explored to dynamically harness all available resources to achieve the highest level of QoS. 
%Firstly, in a system where CPU cores have different propensity to failures, mapping from processes to physical cores will largely impact the successful execution of each process. Secondly, Lazy Shadowing allows different tasks to use different number of shadows. 
Within the resource limitation, more shadows would be allocated for more critical mains or mains that are more likely to fail. 

Failure-induced leaping has proven to be critical in reducing the divergence between a main and its shadow, thus reducing the recovery time for subsequent failures. Consequently, the time to recover from a failure increases with failure intervals. Based on this observation, a proactive approach is to ``force" leaping when the divergence between a main and its shadow exceeds a specified threshold. This is analogous to checkpoint/restart in that checkpoints are periodically taken to minimize the cost of lost work due to a failure. Thus, it is worth studying this approach to determine what behavior triggers forced leaping in order to optimize the average recovery time.

Another future direction is topology-aware process mapping~\cite{von2012topology}. Process mapping is of vital importance in Leaping Shadows, since it not only determines the failure isolation degree, but also impacts communication performance. For the main and shadow(s) of the same task, we would like to place them far away so that they are unlikely to be victims of a single failure. On the other hand, placing mains and shadows close to each other tends to minimize message forwarding cost, especially under the receiver-forwarding protocol. Therefore, a process mapping algorithm needs to be developed to balance the trade-offs, while considering the interconnect topology. 

In the extreme cases where hardware resources are quite limited, there may be no redundant hardware to support the execution of the shadows. If this is the case, one might still apply Leaping Shadows with every main collocated with a shadow, which is associated with a different main. Furthermore, to prevent shadows from taking too much resources and extensively slowing down the mains, shadows may only be allowed to ``steal" CPU cycles when mains are blocked. It is expected that Leaping Shadows with such collocation should be able to achieve fault tolerance with comparable performance under the given limitation on resources. However, it remains a question whether there is an efficient mechanism to precisely control the CPU sharing while ensuring performance isolation. Also, process mapping is an important problem to study.

Lastly, the idea of approximate computing can be applied to shadows. Instead of having shadows as exact replicas of the mains, one can assign a reduced version of the computation to the shadows or let them process a portion of the input data. Many workloads today, such as HPC simulation and large-scale machine learning, often have the flexibility in tuning the fidelity of their results, such as the granularity of a simulation or the precision of convergence.
Energy and performance gains may be achieved, when relaxing the precision constraints in the case of a failure.
















%%
%\chapter{Introduction}%             And when you type '\chapter', pittetd understands that this is the body of the 
%%                                   document. 
%We begin by saying that we do not really have much to say, but for the sake of clarity we divide our topic in chapters.
%
%\chapter{Second chapter}
%The topics treated in this chapter can be somewhat obscure. For humanitarian considerations, the chapter will be subdivided.
%\section{FIRST SECTION}%            Remember to capitalize the sections (otherwise, the bookmark will be lowercase)
%The topic treated here, given its complexity, merits an additional subdivision.
%\subsection{First subsection of the section}
%This is well-known topic, and we shall discuss it no more.
%\subsection{Second subsection of the section}
%This is a very complicated topic and we shall discuss it in our next paper.
%\chapter{Conclusions}
%This is the second chapter of the present dissertation. It is more interesting than the first one, for it is the last one.
%
%\appendix                          After this command, chapters will be formatted as appendices. For example:
%\chapter{Raw data}
%
\bibliographystyle{abbrv}
\safebibliography{main}       %   \safebibliography is used the same way as \bibliography, but gives pittetd
%                                   a greater chance to succeed in formatting the bibliography when non-standard
%                                   BibTeX styles are used.
%\begin{singlespace}
%\bibliographystyle{plainnat}
%{
%\footnotesize
%\bibliography{main}
%}
%\end{singlespace}
\end{document}
