%Extreme-scale computing presents some unique challenges to fault tolerance as faults are no longer 
%an exceptional event \cite{ferreira_sc_2011}. 
Rollback recovery is the dominant mechanism to achieve fault
tolerance in current HPC environments~\cite{Elnozahy:02:Survey}. In the most general form, rollback recovery 
involves the periodic saving of the execution state (checkpoint), with the anticipation that
in the case of a failure, computation can be restarted from a previously saved checkpoint. % \cite{Elnozahy:02:Survey}. %The identification of an error, before or during a checkpoint,
%requires that the application rollback to the previously completed checkpoint. 
Coordinated checkpointing is a popular approach for
its ease of implementation.
Specifically, all processes
coordinate with one another to produce individual states that satisfy the ``happens before"
communication relationship \cite{chandy_trans_1972}, which is proved to provide a consistent global state.
%Essentially, the algorithm provides a method for all processes involved to stop operation ``at the same
%time" and transfer their system state to a stable storage. 
The major benefit of coordinated checkpointing stems from its simplicity and ease of implementation. 
Its major drawback, however, is the lack of scalability, as it requires global coordination
~\cite{elnozahy_dsc_2004,riesen_sandia_2010}.
%hargrove2006berkeley}.


In uncoordinated checkpointing, processes checkpoint their states independently and postpone creating a 
globally consistent view until the recovery phase. The major advantage is the reduced overhead during fault free operation. However, the scheme requires that
each process maintains multiple checkpoints %and message logs, necessary to construct a consistent 
%state during recovery. It 
and can also suffer the well-known domino effect 
 \cite{randell_domino_effect,alvisi_ftc_1999,helary_rds_1997}. One hybrid approach, known as communication induced 
checkpointing, aims at reducing coordination overhead \cite{alvisi_ftc_1999}. The approach, however, may 
cause processes to store useless states. To address this 
shortcoming, ``forced checkpoints" have been proposed \cite{helary_rds_1997}. This approach, however,  may lead to unpredictable
checkpointing rates. 
Although well-explored, uncoordinated checkpointing has not been widely adopted
in HPC environments for its complexities. 
%its dependency on applications \cite{guermouche_2011_ipdps}.


One of the largest overheads in any checkpointing process is the time necessary to write the checkpointing 
to stable storage. Incremental checkpointing attempts
to address this by only writing the changes since previous checkpoint \cite{Agarwal:04:Adaptive,elnozahy_1992_manetho,li_trans_1994}. 
This can be achieved using dirty-bit page flags \cite{plank_ftcs_1994,elnozahy_1992_manetho}. Hash based incremental checkpointing, on the other hand, makes use of hashes to detect changes \cite{nam_ftc_1997,Agarwal:04:Adaptive}. 
Another proposed scheme, known as in-memory checkpointing, minimizes the overhead of disk access~\cite{zheng_2004_ftccharm,6264677}.
%offloads the checkpointing process to a secondary task and only writes incremental checkpoints \cite{li_trans_1994}.
The main concern of these techniques is the increase in
memory requirement to support the simultaneous execution of the checkpointing and the application. 
It has been suggested that nodes in extreme-scale systems should be configured with fast local storage~\cite{doe_ascr_exascale_2011}. 
%, which
%improves the performance of checkpointing \cite{doe_ascr_exascale_2011}. 
Multi-level checkpointing , which consists of
writing checkpoints to multiple storage targets, 
can benefit from such a strategy \cite{Moody:10:SCR}. This,
however, may lead to increased failure rates of individual nodes and complicate the checkpoint writing process.
%Furthermore, it may complicate the checkpoint writing process and requires that the system track the
%current location of all process's checkpoints.


Process replication, or state machine replication, has long been used for reliability and availability in distributed and mission critical systems \cite{schneider_1990_tutorial}. %Replication can be used to detect and correct system failures that are otherwise undetectable,
%such as silent data corruption and Byzantine faults \cite{fiala_2012_sdc}. 
%This approach is barely used in HPC systems, primarily due to its low efficiency.
%However, upcoming extreme-scale systems are expected to 
%require a more challenging level of fault tolerance to deal with the 
%confront a dramatic growth in both the frequency and diversity of faults.
%As a result,
Although it is initially rejected in HPC communities, 
replication has recently been proposed to address the deficiencies of checkpointing for upcoming extreme-scale systems \cite{Cappello:09:Fault,engelmann2011redundant}. 
Full and partial
process replication have also been studied to augment existing checkpointing techniques, and to  
detect and correct silent data corruption \cite{stearly_2012_partial,elliott_2012_cpr,ferreira_sc_2011,fiala_2012_sdc}. % There are several different implementations of
%replication in the widely used MPI library, each with their different tradeoffs and overheads. The
%overhead can be negligible or up to 70\% depending upon the communication patterns of the
%application \cite{engelmann2011redundant}. %Moreover, replication alone is not enough to guarantee fault tolerance since
%it is possible that all nodes executing a given process could fail simultaneously, thus
%replication is typically paired with some form of checkpointing. 
Our approach is largely different from classical process replication in that we dynamically configure the execution rates of main and shadow processes, so that less resource/energy is required while reliability is still assured.  


Replication with dynamic execution rate is also explored in Simultaneous and Redundantly Threaded (SRT) processor whereby one leading thread is running ahead of trailing threads \cite{reinhardt2000transient}. However, 
the focus of \cite{reinhardt2000transient} is on transient faults within CPU while we aim at tolerating both permanent and transient faults across all system components.
%Also, this work differs from \cite{mills_2014_icnc,cui_en7085151,cui_2014_closer}, where shadowing with DVFS is studied for single or loosely-coupled tasks. %Instead, in this paper we explore novel ideas of shadow collocation and shadow leaping in order to satisfy the requirements of future extreme-scale HPC systems. 
%our approach is different in that it tunes the execution rates of the leading and trailing threads in a finer grain, in order to achieve a ``parameterized" trade-off between completion time and energy consumption. 
%Further, we take advantage of the idle time during failure recovery and ``leap" the trailing replicas to achieve forward progress%, largely improving performance in terms of both completion time and energy consumption. 
%. This differs from \cite{reinhardt2000transient}, of which the ``leaping" of the trailing replica results in extra overhead.
%To the best of our knowledge,
%Lazy Shadowing is the first attempt to explore a state-machine replication based framework
%that achieves a fine-grained tradeoff between time and hardware redundancy while meeting resilience and
%power requirements.