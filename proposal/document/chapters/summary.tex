\begin{table}[ht]
\vspace{-0.3in}
\caption{Timeline of Proposed Work.}
\vspace{-0.15in}
\centering
\scalebox{0.85}
{
\begin{tabular}{|l|l|l|}
\hline
\textbf{Date} & \textbf{Content} & \textbf{Deliverable results} \\
\hline
Sep.-   & Explore restoring in approximate computing & Pin-based framework for \\
 & in Section \ref{work:approx} of Chapter \ref{chapter:future_study}& restoring approximation \\
\hline
Mar - May  & Integrate restoring with information leakage & Experimental data of memory\\% through access pattern &  Paper submission \\
 & in Section \ref{work:security} of Chapter \ref{chapter:future_study}& performance and security\\
\hline
Jun - Sep & Study restoring in Hybrid Memory Cube (HMC) & Modified simulator of temperature\\
 & in Section \ref{work:stacked} of Chapter \ref{chapter:future_study}&  effect of restoring in HMC \\
\hline
Jul - Oct  & Thesis writing & Thesis ready for defense \\
\hline
Oct - Dec & Thesis revising & Completed thesis \\
\hline
\end{tabular}
}
\label{tab:timeline}
\end{table}





Current fault tolerance approaches rely exclusively on either time or hardware redundancy to hide failures from being seen by users. 
Rollback recovery, which exploits time redundancy, requires full or partial re-execution when failure occurs.  
Such an approach
can incur a significant delay, % subjecting cloud service providers to SLA violations,
and high power costs due to extended execution time.
On the other hand, Process Replication relies on hardware redundancy and executes multiple
instances of the same task in parallel to guarantee completion with minimal delay. 
This solution, however, requires a significant increase in hardware resources and increases the power consumption proportionally. 
