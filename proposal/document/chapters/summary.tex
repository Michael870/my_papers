Current fault tolerance approaches rely exclusively on either time or hardware redundancy to hide failures from being seen by users. 
Rollback recovery, which exploits time redundancy, requires full or partial re-execution when failure occurs.  
Such an approach
can incur a significant delay, % subjecting cloud service providers to SLA violations,
and high power costs due to extended execution time.
On the other hand, Process Replication relies on hardware redundancy and executes multiple
instances of the same task in parallel to guarantee completion with minimal delay. 
This solution, however, requires a significant increase in hardware resources and increases the power consumption proportionally. 

The system scale needed to solve the complex and urgent problems that our society will be facing is going beyond the scope of existing 
fault tolerance approaches. This thesis builds on top of the shadow replication fault-tolerant computational model, and 
devises novel techniques to simultaneously address the power and resilience challenges for future extreme-scale systems while guaranteeing system efficiency and application QoS. 
A combination of modeling, simulation, and experiments will be used to test the viability of the proposed ideas. The proposed work will be completed following the timeline shown in 
Table~\ref{tab:timeline}.


\begin{table}[ht]
\vspace{-0.3in}
\caption{Timeline of Proposed Work.}
\vspace{-0.15in}
\centering
\scalebox{0.85}
{
\begin{tabularx}{\textwidth}{|c|X|X|}
\toprule
\textbf{Date} & \textbf{Content} & \textbf{Deliverable results} \\
\midrule
Sep. - Nov.  & Finish the implementation of lsMPI with shadow leaping & A functional MPI library for Lazy Shadowing \\
\hline
December & Measure the performance of lsMPI w/ and w/o failures & A paper for publication \\
\hline
January & Analysis of failure repository & Failure distribution across nodes within a system \\
\hline
Feb. - Mar.  & Study the problem of process mapping and partial shadowing & Analytical models and optimization problem formulation \\
\hline
April  & Build a trace based simulator to evaluate process mapping and partial shadowing & A paper for publication \\ 
\hline
May - Aug. & Thesis writing & Thesis ready for defense \\
\hline
Sep. - Oct. & Thesis revising & Completed thesis \\ 
\bottomrule
\end{tabularx}
}
\label{tab:timeline}
\end{table}





