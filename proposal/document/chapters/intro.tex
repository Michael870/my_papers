As our reliance on IT continues to increase, the complexity and urgency of the problems our society will face in the future drives us to build more powerful and accessible computer systems. Among the different types of computer systems, High Performance Computing (HPC) and Cloud Computing systems are the two most powerful ones. For both of them, the compute power attributes to the massive amount of parallelism, which is supported by the massive amount of CPU cores, memory modules, communication devices, storage components, etc. 

Nowadays, two types of systems are widely used to enable us to solving 



\section{Problem Statement}

\section{Research Overview}

\subsection{Achieve Fast Rows via Reorganization (Completed)}

\subsection{Refresh-aware Partial Restore (Completed)}

\subsection{Explore Restoring in Extended Scenarios (Future)}


\section{Contributions}
This thesis makes the following contributions:

\begin{itemize}
\item We perform pioneering study on DRAM restoring in deep sub-micron scaling. We built models to simulate restoring behaviors and then generate DRAM devices to faithfully repeat the manufacturing process  and perform architectural-level studies.
\item Targeting at restoring issues, we propose schemes from different perspectives. On device and architectural levels, we apply chunk remapping and chip clustering techniques to achieve fast memory access; on system level, we maximizing performance improvement by allocating hot pages of the running workloads to fast regions.
\item Going further, we integrate restoring variation characteristics with approximate computing to strike a good balance among performance, energy and accuracy. We then explore restoring issues in extended scenarios including information leakage and 3D-stacked memory.  
\end{itemize}


\section{OUTLINE}
\label{outline}
The rest of this proposal is organized as follow:  
Chapter \ref{chapter:background} introduces the DRAM structures, operations and scaling issues.
In Chapter \ref{chapter:twr_reorganize}, we build models to study restoring effects, and then propose a series of techniques to shorten restoring timing values.
In Chapter \ref{chapter:partialrestore}, we explore the correlation between restoring and refresh, and seek the opportunities to early terminate restore operations.
Further restoring explorations are discussed in Chapter \ref{chapter:future_study}.
Chapter \ref{chapter:timeline} and \ref{chapter:summary} lists the timeline and concludes the proposal, respectively.








