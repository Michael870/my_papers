As our reliance on IT continues to increase, the complexity and urgency of the problems our society will face 
in the future drives us to build more powerful and accessible computer systems. Among the different types of 
computer systems, High Performance Computing (HPC) and Cloud Computing systems are the two most powerful ones. 
For both of them, the compute power attributes to the massive amount of parallelism, which is supported by 
the massive amount of CPU cores, memory modules, communication devices, storage components, etc. 

Since CPU frequency flatten out in early 2000s, parallelism has been the standard way to achieve performance increase. 
In HPC, Terascale performance was achieved with fewer than 10,000 heavyweight single-core processors in the late 90’s. 
A decade later, petascale performance required about ten times processors that each has multiple cores. Nowadays, a race
is underway to build the world's first exascale machine to accelerate scientific discoveries and breakthroughs. It is 
projected that an exascale machine will achieve billion-way parallelism by using one million sockets each supporting 
1,000 cores. Similar trend is happening in Cloud Computing. 
As the demand for Cloud Computing accelerates, cloud service providers  
will be faced with the need to expand their underlying infrastructure to ensure the expected levels of performance, reliability and cost-effectiveness. 
%Large IT companies are all building large-scale data centers 
%to exploit the power and econies of scale. 
For example, Microsoft, Google, Facebook, and Rackspace have hundreds of thousands 
of web servers in their dedicated data centers to support their business. 

Unfortunately, several challenging issues come with the increase in system scale. As today's HPC and Cloud Computing systems grow to 
meet tomorrow's compute power demand, the behavior of the systems will be increasingly difficult to specify, predict and manage. 
This upward trend, in terms of scale and complexity, has a direct negative effect on the overall system reliability. 
Even with the expected improvement in the reliability of future computing technology, the rate of system level failures will 
dramatically increase with the number of components, possibly by several orders of magnitude. 
At the same time, the rapid 
growing power consumption, as a result of the increase in system components, is another major concern. 
%It is reported that 
%the power required to run the machines as well as cool them has become the largest cost factor in a large-scale system's operating 
%expenses.  
At future extreme-scale, failure would become a norm rather than an exception, 
driving the system to significantly lower efficiency with unprecedented amount of power consumption. 

\section{Problem Statement}

The system scale needed to address our future computing needs will come at the cost of increasing complexity, unpredictability, 
and operating expenses. 
Regardless of the reliability of individual component, the system level reliability will continue to decrease as the number of 
components increases. Concerned with the energy costs, the U.S Department of Energy has set 20 megawatts as the power upperbound 
for future exascale systems, challenging the research community to provide a 1000x improvement in performance with only a 
10x increase in power consumption. 
Achieving high resilience to failures under strict power constraints is a daunting and critical challenge that must be addressed 
to enable future extreme-scale systems. In the following, we further explore the dimensions of these challenges and their impact 
on the design and performance of future extreme-scale systems.

\subsection{Resilience challenge}

It is projected that the Mean Time Between Failures (MTBF) of future extreme-scale systems will be at the order of hours, meaning 
that several failures will occur every day. Without an efficient fault tolerance mechenism, the applications running on the 
systems will be continuously interrupted, requiring the execution to be restarted every time there is a failure. 

Today, two major approaches exist for fault tolerance. The first approach is rollback recovery, of which coordinated checkpointing/
restart is the most widely used technique in HPC to tame scale. The second approach is state machine replication (or process 
replication) that are used in Cloud Computing and mission critical systems. Many studies have predicted the performance of future 
extreme-scale systems with these two approaches. Unfortunately, for both of the two approaches, more than half of  the system's 
efficiency will lost due to frequent failures.  
 
\subsection{Power challenge}
There has been a steady rise in power consumption in large-scale distributed systems, due to the continuous growth in system components. 
Figure 1 shows a steady rise in system power consumption to 1-3MW in 2008 but then a sharp increase to 10-20MW in the following years. The trend shows that system power consumption could surpass 50MW by 2016.


\section{Research Overview}

\subsection{Achieve Fast Rows via Reorganization (Completed)}

\subsection{Refresh-aware Partial Restore (Completed)}

\subsection{Explore Restoring in Extended Scenarios (Future)}


\section{Contributions}
This thesis makes the following contributions:

\begin{itemize}
\item We perform pioneering study on DRAM restoring in deep sub-micron scaling. We built models to simulate restoring behaviors and then generate DRAM devices to faithfully repeat the manufacturing process  and perform architectural-level studies.
\item Targeting at restoring issues, we propose schemes from different perspectives. On device and architectural levels, we apply chunk remapping and chip clustering techniques to achieve fast memory access; on system level, we maximizing performance improvement by allocating hot pages of the running workloads to fast regions.
\item Going further, we integrate restoring variation characteristics with approximate computing to strike a good balance among performance, energy and accuracy. We then explore restoring issues in extended scenarios including information leakage and 3D-stacked memory.  
\end{itemize}


\section{OUTLINE}
\label{outline}
The rest of this proposal is organized as follow:  
Chapter \ref{chapter:background} introduces the DRAM structures, operations and scaling issues.
In Chapter \ref{chapter:twr_reorganize}, we build models to study restoring effects, and then propose a series of techniques to shorten restoring timing values.
In Chapter \ref{chapter:partialrestore}, we explore the correlation between restoring and refresh, and seek the opportunities to early terminate restore operations.
Further restoring explorations are discussed in Chapter \ref{chapter:future_study}.
Chapter \ref{chapter:timeline} and \ref{chapter:summary} lists the timeline and concludes the proposal, respectively.








