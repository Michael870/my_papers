\begin{table}[ht]
\vspace{-0.3in}
\caption{Timeline of Proposed Work.}
\vspace{-0.15in}
\centering
\scalebox{0.85}
{
\begin{tabular}{|l|l|l|}
\hline
\textbf{Date} & \textbf{Content} & \textbf{Deliverable results} \\
\hline
Jan - Feb  & Explore restoring in approximate computing & Pin-based framework for \\
 & in Section \ref{work:approx} of Chapter \ref{chapter:future_study}& restoring approximation \\
\hline
Mar - May  & Integrate restoring with information leakage & Experimental data of memory\\% through access pattern &  Paper submission \\
 & in Section \ref{work:security} of Chapter \ref{chapter:future_study}& performance and security\\
\hline
Jun - Sep & Study restoring in Hybrid Memory Cube (HMC) & Modified simulator of temperature\\
 & in Section \ref{work:stacked} of Chapter \ref{chapter:future_study}&  effect of restoring in HMC \\
\hline
Jul - Oct  & Thesis writing & Thesis ready for defense \\
\hline
Oct - Dec & Thesis revising & Completed thesis \\
\hline
\end{tabular}
}
\label{tab:timeline}
\end{table}

The proposed works will be undertaken as shown in the Table \ref{tab:timeline}.
I will start the effort with task (1) to develop Pin-based framework for approximate computing of restoring. This task involves program annotation, chip generation, QoS evaluation, and conventional performance simulation, etc.
While multiple complicated subtasks are there, this task has been partially finished, and will not take much time to complete.
Afterwards, I'll move to task (2) to study information leakage in restoring scenario, and this task is partially on basis of the previous approximation work.
With the completion of task (2), the overall goal of exploring DRAM restoring in application level have been reached, and then I'll start the study restoring's temperature effect in HMC, i.e., task (3).
The general infrastructure can be borrowed from my previous HMC work \cite{ICCD15:dlb}. This task might be performed concurrently with other jobs, and thus might take more time to finish.
At the end of task (3), the holistic exploration of DRAM restoring is considered finished, and thus I'll summarize all the tasks into my final thesis.