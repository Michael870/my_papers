\documentclass[driverfallback=dvipdfmx,final]{pittetd}

\usewithpatch{graphicx}
\usepackage{amsmath,amsthm}
\usepackage{multicol}
\usepackage{tabularx}
\usepackage{fixltx2e}
\usepackage{url}
\usepackage{color}
\usepackage{wrapfigure}
\usepackage{amssymb}

%\usepackage{subcaption}
\usepackage[labelformat=simple,font=footnotesize]{subcaption}
\usepackage[font=small,labelfont=bf]{caption}

\usepackage[square, authoryear]{natbib}
% ADD THE FOLLOWING COUPLE LINES INTO YOUR PREAMBLE
\let\OLDthebibliography\thebibliography
\renewcommand\thebibliography[1]{
  \OLDthebibliography{#1}
  \setlength{\parskip}{0pt}
   \setlength{\bibsep}{0ex} %% vertical spacing between references
  \setlength{\itemsep}{7pt plus 0.3ex} %%
}

\renewcommand\thesubfigure{(\alph{subfigure})}
\usepackage{soul}
\usepackage{footnote}
\usepackage{pifont}
\usepackage{chngcntr}
\counterwithin{figure}{chapter}
\counterwithin{table}{chapter}
%\usepackage[nottoc]{tocbibind}

%\usepackage{tikz}
%\usetikzlibrary{fit,positioning}

\let\citeN\citet
\let\cite\citep

\patch{amsmatch}
\patch{amsthm}
\title[Adaptive and Power-aware Fault Tolerance for Future Extreme-scale Computing]
{Adaptive and Power-aware Fault Tolerance for Future Extreme-scale Computing}
\author{Xiaolong Cui}
\degree{Bachelor of Engineering \\ Xi'an Jiaotong University \\2012}
%\date{July 20th 1967} % This date is the date of the thesis defense. Default is \today
%\year{1967}   % pittetd will use the current year unless otherwise indicated. So this command is not necessary.
\keywords{\LaTeX, pittetd, theses, format}   % default, don't change
\subject{Entity/Event-Level Sentiment Detection and Inference} %This fills in the 'Subject' field in Acrobat Reader's Document Info dialog box.
\school{Kenneth P. Dietrich School of \\ Arts and Sciences} %The name of the school will be preceeded by 'the' unless otherwise specified, as in:
%\school[certain]{department}
%
%\chapterfloats%                    Un-comment this to get figures and tables numbered within chapters.
\begin{document}
\maketitle
%
% For the committee membership page, you have to provide the names and affiliations of the members. The first one will 
% be treated by pittetd as the committee chair (thesis/dissertation advisor).
\committeemember{Dr. Taieb Znati, Department of Computer Science, with joint appointment in Telecommunication Program, University of Pittsburgh}
\coadvisor{Dr. Rami Melhem, Department of Computer Science, University of Pittsburgh}%         This is used if there are two advisors.
\committeemember{Dr. Rami Melhem, Department of Computer Science, University of Pittsburgh}%         This is used if there are two advisors.
\committeemember{Dr. John Lange, Department of Computer Science, University of Pittsburgh}
\committeemember{Dr. Esteban Meneses, School of Computing, Costa Rica Institute of Technology} 
% etc., as many as needed. For master's theses, the committee may be omitted, naming only the advisor.
\school{Computer Science Department}
\makecommittee
\copyrightpage                     %Uncomment this to get a copyright page.
\begin{abstract}
As the demand for computing power continue to increase, both HPC community and Cloud service provides are building larger computing platforms to take advantage of the power and economies of scale. On the HPC side, several of the most powerful countries are competing for developing the next generation supercomputer--exascale computing machines to accelerate scientific discoveries, big data analytics, etc. On the Cloud side, large IT companies are all expanding large-scale datecenters, for both private usage and public services. However, aside from the benefits, several daunting challenges will appear when it comes to extreme-scale.

This thesis aims at simultaneously solving two major challenges, i.e., power consumption and fault tolerance, for future extreme-scale computing systems. We come up with a novel computational model, referred to as Lazy Shadowing, as a power-aware and scalable approach to achieve high-levels of resilience, through forward progress, in extreme-scale, failure-prone computing environments. Two approaches have been proposed to realize this idea. Accordingly, precise mathemetical models and optimization framework have been developed to quantify and optimize the improvement in system efficiency and energy savings, respectively. 

In this work, I propose to continue the research in three aspects. Firstly, I propose to develop a MPI-based prototype to valdate the correctness of Lazy Shadowing in real environment. Using the prototype, I will run benchmarks and real applications to measure its performance compared to state-of-the-art approaches. Then, I propose to study the problem of mapping main and shadow processes to physical cores, with the consideration of hardware, architecture, environment, etc. Last but not least, I propose to further explore the potential of Lazy Shadowing and improve its efficiency. Based on the specific system configuration, application characteristics, and QoS requirement, I will study the viability of partial shadowing in three dimensions, i.e., time, space, and workload.     

\end{abstract}
% If you say \begin{abstract}[Keywords:] instead of the simple \begin{abstract}, a list of the keywords is appended.
% The list comes from the \keywords command above.
% The starred version \begin{abstract*} typesets the word `ABSTRACT' on the top of the page


\tableofcontents
%\listoftables                      %Pittetd will complain if you tell it to create a list of tables when there are no
%                                   tables (as in this sample file). Uncomment this command if you have tables.
%\listoffigures                     %Obvious analogous for figures.
%\preface
% This is the text of the preface, with acknowledgments, dedication, etc. It is optional, and you create, as shown, by 
% just saying \preface and starting the preface's actual text. Note that 'foreword' is no longer acceptable as title
% for this preliminary.
%
%Conventions, such as notation (nomenclature) and abbreviations, don't receive their own preliminary page. They can be included as an appendix, or as part of the introduction.
%
\chapter{INTRODUCTION}
\label{chapter:intro}
Today's scientific discoveries and business intelligence are driven by high-fidelity, 
large-scale simulation and data analytics. To meet the increasing computing demands from 
virtually every aspect of the society, HPC is continuously evolving to solve more 
complex and challenging problems. On the one hand, national labs and research institutes run HPC on 
supercomputers for scientific breakthroughs and national security. On the other hand, enterprises and 
organizations deploy HPC on small to medium sized clusters to process data and extract insights. 
Recently, the explosively growing machine learning applications have increased the adoption as well as 
impact of HPC as they also exploit parallelism and hardware acceleration to speed up the processing of 
massive amount of data.


HPC workloads have traditionally been run only on bare-metal, unvirtualized hardware to drive maximum 
performance. 
The roadblock to virtualization was due to the concern that the extra hypervisor layer could introduce 
performance overhead. 
%The concern was that virtualization could introduce performance overhead due to the extra software 
%layer of hypervisor. 
However, this has started to change with the introduction of increasingly sophisticated 
hardware support for virtualization and software optimization~\cite{madukkarumukumana2008resource,bugnion2017hardware}. Performance of 
these highly parallel HPC workloads has increased dramatically over the last decade, 
enabling organizations to begin to embrace the numerous benefits that a virtualization platform can 
offer~\cite{michael2018overcommit}. As a result, we are witnessing a popular trend that enterprises convert 
their on-prem bare-metal clusters to virtualized, shared private cloud. For instance, the Johns Hopkins 
University Applied Physics Laboratory recently virtualized their 3728-core bare-metal cluster 
to share between Windows and Linux users. The reported improvement in resource utilization 
ranges from 9.1\% to 29.2\%, and simulations speed up by 4\% on average~\cite{vmware2017josh}.

At the same time, public cloud, such as Amazon AWS and Google GCP, is becoming a popular alternative for 
HPC practitioners. Recent studies show that the usage of public cloud has grown more than five-fold among all HPC 
sites worldwide, from 13\% in 2011 to 74\% in 2018~\cite{hyperion2019}.
With virtually unlimited scalability and on-demand resource subscription, public cloud starts to host 
compute- and data-intensive workloads across various industry verticals. These workloads span the traditional HPC 
applications, like genomics and 
weather prediction, as well as emerging applications, like machine learning and deep learning. 

There is a fruitful body of research on resource management in 
Cloud Computing~\cite{singh2016survey,zhan2015cloud,gill2018chopper}. Dynamic resource scheduling and 
load balancing are used 
to maximize system utilization and efficiency~\cite{adhikari2018heuristic,panwar2015load}. These techniques, however, 
are not straightforward to apply to HPC workloads which are highly sensitive to resource change and interference. 
Actually, resource management has been identified as one of the open 
challenges for HPC cloud~\cite{netto2018hpc}. 
Currently, cloud service providers (CSPs) are often limited to statically and conservatively reserve 
resources based on peak resource requirements to respect service level agreements (SLAs). For example, Microsoft Azure 
allocates dedicated supercomputers from Cray, and Amazon AWS offers dedicated nodes for full-size VMs. 
% allocate physical resources
% Despite the 
% numerous benefits promised by Cloud Computing, however, cloud service providers (CSPs) are often limited to statically 
% allocate physical resources to HPC tenants in order to avoid performance interference and enforce 
% service level agreements (SLAs). 
This essentially offsets 
the elasticity and efficiency benefits of the Cloud Computing business model. 

In this paper, we present \textit{virtual throughput clusters (VTC)} as a novel approach for cloud 
resource allocation to efficiently and effectively support 
HPC workloads with multi-tenancy. Based on virtual machine (VM), VTC goes beyond traditional way of 
statically splitting resources among tenants and applies resource over-commitment to optimize 
system utilization and throughput. By giving each tenant a virtual cluster that mimics the 
underlying physical cluster, VTC delegates the resource management task 
to the hypervisor to improve flexibility as well as efficiency. When all tenants are busy consuming their cycles, 
VTC guarantees that each tenant is getting his/her fair share according to pre-defined SLA terms. When 
some tenant is not fully using the allocated resources, VTC takes advantage of the work-conserving 
property of the hypervisor scheduler to assign the idle resources to other tenant(s) who can benefit 
from additional resources. Consequently, CSPs can ensure quality-of-service while maximizing 
system utilization. 

The rest of the paper is organized as follows. 
Section II provides background and motivation. Section III introduces the design of VTC, followed by validation 
and empirical evaluation results in Section IV. Section V concludes this work and points out future directions.

\chapter{BACKGROUND}
\label{chapter:background}
\section{DRAM Structure and Organization}
DRAM-based main memory system is logically organized as a hierarchy of channels, ranks and banks, as illustrated by Figure \ref{fig:dram_org}.
Bank is the smallest structure to be accessed in parallel with each other, which is termed as bank-level parallelism \cite{ISCA08:blp, MICRO09:blp}.
And, rank is formed by clustering multiple, usually eight
\footnote{For illustration purpose, we assume the memory chips are x8, i.e., 8 data I/O pins. The overall structure keeps the same for x4 and x16, except the number of chips in a rank.}
, banks which operate in lockstep, i.e., all banks in a rank respond to a single command. 
Lastly, one channel is composed of an on-chip memory controller and several ranks that share the same narrow command/address and wide data bus.

\begin{figure}
 \centering
  \begin{subfigure}{.34\textwidth}
    \centering
    	\includegraphics[width=\linewidth]{figures/dram_org.pdf}\\
    \caption{Logical hierarchy}
    \label{fig:dram_org}
  \end{subfigure}
%
  \begin{subfigure}{.39\textwidth}
    \centering
    	\includegraphics[width=\linewidth]{figures/dram_rank.pdf}\\
    \caption{Rank organization}
    \label{fig:dram_rank}
  \end{subfigure}
  \vspace{-0.45in}
  \caption{DRAM high-level structure.}
  \label{fig:dram}
\end{figure}

Physically, a DRAM rank is composed of multiple chips, inside which eight banks are deployed as cell arrays.
The logical bank, as shown in Figure \ref{fig:dram_org}, is physically made up of the same numbered bank from all chips.
For instance, {\tt bank 0} of a rank contains {\tt bank 0} 
\footnote{Without specific comment in the rest of the thesis, {\tt bank} refers to a logical bank, which is across chips in a rank; for banks residing in a chip, we would specifically call as \ul{chip bank} to differentiate. The same rule goes with {\tt row}.}
residing in all chips in the rank. 
Likewise, a DRAM row is dispersed across chips, as shown in Figure \ref{fig:dram_rank}. 
In normal accesses to a rank, each chip provides 8 bits at a time simultaneously, which together satisfy the total data bus width of 64-bit.
In addition, to amortize memory access overhead on processor side and also to bridge the giant gap between DRAM core frequency (about 200 MHz) and bus frequency (over 1000 MHz), n-bit prefetch and burst access is supported \cite{ISCA11:agms, ISCA14:half_dram}. n is 8 for commodity DDR3, which translates into a granularity of 64B (64b$\times$8), the popular cache block size. 

\begin{figure}
 \centering
  \begin{subfigure}{.36\textwidth}
    \centering
    	\includegraphics[width=\linewidth]{figures/dram_array.pdf}\\
    \caption{Array}
    \label{fig:dram_array}
  \end{subfigure}
%
  \begin{subfigure}{.16\textwidth}
    \centering
    	\includegraphics[width=\linewidth]{figures/dram_cell.pdf}\\
    \caption{Cell}
    \label{fig:dram_cell}
  \end{subfigure}
  %
  \begin{subfigure}{.16\textwidth}
    \centering
    	\includegraphics[width=\linewidth]{figures/dram_rc.pdf}\\
    \caption{Circuit}
    \label{fig:dram_rc}
  \end{subfigure}
  \vspace{-0.45in}
  \caption{DRAM detailed organization. (a) is the high-level structure of DRAM array, (b) shows cell structure and (c) illustrates the equivalent circuit where $R_c$ is contact resistance and $R_{BL}$ is the bitline resistance.}
  \label{fig:dram_bank}
  \vspace{-0.45in}
\end{figure}

In more detailed level, DRAM cells are packed into 2D arrays, as Figure \ref{fig:dram_cell} shows, where each cell can be uniquely located by vertically bitline and horizontally wordline.
Each cell consists a capacitor to store electrical charge, and one access transistor to control the connection to wordline.
Upon receiving a row address, DRAM fetches the target row into the row buffer, which contains thousands of sense amplifiers to detect the voltage change on bitline.


\section{DRAM Operations and Timing Constraints}
DRAM supports three types of accesses --- read, write, and refresh. An on-chip memory controller (MC) is responsible to receive requests from processors and decompose them into a series of commands such as {\tt ACT}, {\tt RD}, {\tt WR} and {\tt REF}, etc.
The commands are then sent to DRAM modules sequentially following the predefined timing constraints in DDRx standard. 
We briefly summarize the involved commands and timing constraints as follow: 
%A more comprehensive discussion can be found in \cite{Bruce:Jacob}.

\textbf{READ:} as illustrated in Figure \ref{fig:dram_read}, read access starts with an \underline{ACTIVATE} (ACT) command to bring the required row into the sense amplifiers; then, a \underline{READ} (RD) command is issued to fetch data from the row buffer. The interval between  ACT and RD is constrained by {\tt tRCD}. DRAM read is destructive, and hence the charge in the storage capacitors needs to be restored. The restore operation is performed concurrently with RD, and a row cannot be closed until restoring completes, which is determined by {\tt tRAS-tRCD}. Once the row is closed, a \underline{PRECHARGE} (PRE) can be issued to prepare for a new row access. PRE is constrained by timing {\tt tRP}. The time for the whole read process is {\tt tRC=tRAS+tRP}.

\begin{figure}
 \centering
  \begin{subfigure}{.7\textwidth}
    \centering
    	\includegraphics[width=\linewidth]{figures/timing_read.pdf}\\
    \caption{Read operation}
    \label{fig:dram_read}
  \end{subfigure}

\vspace{-0.25in}
  \begin{subfigure}{.7\textwidth}
    \centering
    	\includegraphics[width=\linewidth]{figures/timing_write.pdf}\\
    \caption{Write operation}
    \label{fig:dram_write}
  \end{subfigure}
  \vspace{-0.4in}
  \caption{Commands and timing constraints involved in DRAM accesses. (Timing values are from  \cite{JEDEC:ddr3})}
  \label{fig:operation}
    \vspace{-0.45in}
\end{figure}

\textbf{WRITE:} write works similarly to read, with ACT as the first command to be performed. After {\tt tRCD} has been elapsed, a \underline{WRITE} (WR) is issued to overwrite the content in the row buffer, and then update (restore) the value back into the DRAM cells. Before issuing PRE, the new data overwritten in the sense amps must be safely restored into the target bank, taking {\tt tWR} time. 
To summarize, both RD and WR commands involve the restoring operation
\footnote{Whereas restoring after write is represented by {\tt tWR}, that after read is included in {\tt tRAS}. For ease of presentation, we discuss with a focus on {\tt tWR} and always adjust {\tt tRAS} accordingly.}
, and hence a change in restore time shall affect both DRAM read and write accesses. 

\textbf{Refresh:}
%DRAM needs to be refreshed periodically to prevent data loss. 
refresh commands are issued by memory controller typically every 7.8$\mu$s to refresh a bin, which is composed of multiple rows.
Upon receiving {\tt REF}, DRAM device refresh the designated rows tracked by the internal counter.
%According to JEDEC \cite{JEDEC:ddr3}, 8192 all-bank auto-refresh ({\tt REF}) commands are sent to all DRAM devices in a rank within one retention time interval ({\tt Tret}), also called as one refresh window ({\tt tREFW}) \cite{TC15:refresh, ISCA13:ddr4, HPCA14:parallelrefresh}, typically 64ms for DDR3/4. 
%The gap between two {\tt REF} commands is termed as refresh interval ({\tt tREFI}), whose typical value is 7.8$\mu$s, i.e. 64ms/8192.
%If a DRAM device has more than 8192 rows, rows are grouped into 8192 {\bf refresh bins}. One {\tt REF} command is used to refresh multiple rows in a bin. An internal counter in each DRAM device tracks the designated rows to be refreshed upon receiving {\tt REF}. 
The refresh operation takes {\tt tRFC} to complete, which proportionally depends on the number of rows in the bin.
Whereas typically the whole memory rank is refreshed every 64ms, the vast majority cells can hold data for a much longer time \cite{ISCA12:raidr, ISCA15:reflex}.

\section{DRAM Technology Scaling}
\subsection{Scaling Issues}
With continuously increasing demands on DRAM density and capacity, the cell dimensions keep scaling downward.
%Memory technology scaling drives increasing density and capacity by decreasing cell dimensions.
Past decades saw DRAM's rapid development of 4x density every 3 years \cite{BOOK:cod}. 
Along scaling path from over 100nm to nowadays 2x nm, DRAM also experiences the drop of supply voltage \cite{HPCA16:twr}, more severe signal noise \cite{ISQED08:offset, ISCA13:ddr4} and shorter retention time \cite{ISCA13:archshield, PATENT15:twr}.
However, for reliable operations in DRAM, cell capacitor must be sufficiently large to hold sufficient charge, access transistor is required to be large enough to exert effective control \cite{ISCA09:pcm}, resistance should not be too large to obstruct cell charging process, and sub-threshold leakage should be small to safely hold data for a long time.

The intertwining requirements make the scaling jeopardy. For instance, smaller technology nodes provides smaller contacts of transistor and capacitor, and also narrower bitlines, both of which result in increased resistance (shown in Figure \ref{fig:dram_rc}), which lengthens the restoring time, and further the overall access latency. The growing number of slow and leaky cells has a large impact on system performance. 
There are three general strategies to address this challenge:
\begin{itemize}
\itemsep -1pt
\item 
The first choice is to keep conventional hard timing constraints for DRAM, 
which makes it challenging to handle slow and leaky cells.  Cells that fall
outside of guardbands could be filtered (not used).
% when they are outside guardbands.  
With scaling, however, this approach can incur worse chip yield
and higher manufacturing cost. 
%One choice is not to expose these cells by filtering out chips with weak cells above a threshold. This results in low chip yield and high manufacturing cost. 
Because the DRAM industry operates in an environment of exceedingly tight profit
margins, reducing chip yield for commodity devices is unlikely to be preferred.
%Given that DRAM industry is known for its tight profit margin, reducing chip yield may not be preferred. 

\item 
A second choice is to expose weak cells, falling outside guardbands, and integrate strong yet complex error correction schemes, e.g., \textit{ArchShield} \cite{ISCA13:archshield}. Due to the large number of cells that violate conventional timing constraints such as {\tt tRCD}, {\tt tWR}, significant space and performance overheads are expected.

\item
A third choice is to relax timing constraints \cite{MEM14:twr, DATE15:twr}. This approach is compelling because it can easily maintain high chip yield at extreme technology sizes. 
%Studies suggested relaxed timing constraints in future DRAM chips \cite{MEM14:twr, DATE15:twr}, which helps to keep high yield to make the technology viable. 
However, relaxing timing, without careful management, can cause 
large performance penalties.  
\end{itemize} 
%
Because the third choice is compatible with the need for high chip density 
and yield, we adopt it in this thesis.  We relax restore timing and strive to mitigate associated 
performance degradation. 
Our design principle is also applicable to the second strategy if exposed errors 
can be well managed. We leave this possibility to future work.

\subsection{Related Work on Restoring}
While write recovery time ({\tt tWR}) keeps at 15ns across all generations from DDR to DDR4 \cite{JEDEC:ddr, JEDEC:ddr2, JEDEC:ddr3, JEDEC:ddr4}, it has to be lengthened in deep sub-micron technology nodes, which was first recently discussed by \citeN{MEM14:twr}.
As the first academic work on {\tt tWR} issues in further scaling DRAM, our paper \cite{DATE15:twr} studied the variation behaviors and proposed to utilize chunk remapping to lower restoration durations.
Afterwards, patents on {\tt tWR} were granted: \citeN{PATENT14:twr} raised the idea to adjust timings with respect to temperature, and \citeN{PATENT15:twr} claimed that {\tt tWR} can be increased from 15ns to 60ns, and then raised the idea of exploring backward compatibility.

Whereas the {\tt tWR} scaling issue has been identified in industrials, little academic research have been performed. Restoration has been an silent issue util recently; people started to utilize the reserved timing margins \cite{DATE14:margin, HPCA15:al-dram}, with restoring being included. Besides, later work \cite{ISCA15:mcr} took use of charge variation to relax some timing constraints. However, none of these work targets at future DRAM technologies.

%\section{Related Work}

\chapter{Achieve Fast Restoring via Reorganization}
\label{chapter:twr_reorganize}
\section{Modeling Restore Effects}
Modeling and simulation are required to perform the studies on further scaling DRAM. On device level, the model needs to capture the critical components including transistor, capacitor, sense amplifier, and other peripheral circuits; and the model should also cover the primary parameters and dimensions, such as transistor length/width, capacitance and voltage, etc. Following the principles, we built SPICE modeling on basis of a Rambus tool \cite{MICRO10:rambus}, and simulated data write operation. %The simulation is repeatedly performed on 20nm and 14nm technology nodes.


Further, to involve process variation effects, the models should be inherently statistical following certain distributions.
Using the aforementioned cell model, we generate 100K samples and curve fit using log-normal distribution. Similar to recent PV studies \cite{ISCA12:raidr,HPCA14:mosaic}, we include bulk distribution to depict the normal variation that dominates the majority of cells, and tail distribution to depict random manufacturing defects
{\footnote{Note that not all cells following the tail distribution are treated as defects. The worst ones are covered by conventional redundant repairs \cite{HPCA14:mosaic}.}}.
Table \ref{tab:tech} summarizes the parameters for bulk and tail distributions after curve fitting with our cell samples. 

\begin{savenotes}
\begin{table}[htbp]
\vspace{-0.2in}
\caption{Modeling Parameters}
\vspace{-0.2in}
\centering
\scalebox{0.85}
{
\begin{tabular}{l|llll|lcc}
\hline
tech node&	$\mu_{bulk}$&	$\sigma_{bulk}$&	$\mu_{tail}$&	$\sigma_{tail}$&	$\phi$&	random weight\\
\hline\hline
%20nm&	2.031&	0.21&	3.081&	0.063&		0.3	&0.5\\
14nm \footnote{Whereas we modeled both 20nm and 14nm nodes, here we only show the case of 14nm because of space limitation. Data and results on 20nm can be found in \cite{DATE15:twr}.} &	2.048&	0.247&	3.283&	0.0735&	0.3	&0.5\\
\hline\hline
\end{tabular}
}
\label{tab:tech}
\end{table}
\end{savenotes}

To obtain the chip maps, we use the VARIUS tool \cite{SM08:varius} to involve both within-die (WID) and die-to-die (D2D) process variations. 
Similar to prior PV studies \cite{ICICDT:weight,HPCA14:mosaic}, we assume the same share of systematic and random components, and choose $\phi=0.3$ meaning that the correlation range equals to 30\% of the chip's side length, as shown in Table \ref{tab:tech}. 
With the constructed models and collected parameters, then we can move forward to generate chips, and then form ranks and DIMMs using the pool of chips. Next, architectural explorations can be conducted on the collected memory system. 
%To get close to real manufacturing process, it is necessary to guarantee the quantities of chips, ranks and DIMMs large enough to be swept over in simulations.

\section{Proposed Designs}
In this section, we elaborate the proposed designs. First, we discuss the post-fabrication schemes in terms of both coarse chip-level and fine chunk-level restoring management; then, we extend the schemes to assembly phase to deliberately form ranks by clustering compatible chips; and finally, the schemes are integrated with OS-level page allocation to maximum performance gains.

\subsection{Chip-specific Restoring Control}
Conventionally, a single set of timing constraints is applied to the whole memory system, which totally ignores the existing variations and thus a too conservative setting.
A simple enhancement can be made by exposing chip variations, i.e., setting different {\tt tWR}s for different chips. 
%For this purpose, a post-fabrication test process is performed by the manufacturer to determine the {\tt tWR} of each chip while a DIMM is then constructed using chips with the same or similar {\tt tWR}s
%\footnote {Differing from later rank formation, here we consider the chip as a whole, which is unaware of the internal distribution.}
%. Each DIMM derives its {\tt tWR} from the chip-row 
%that has the worst {\tt tWR} of the entire DIMM (as shown in Figure \ref{fig:chip_default}), or the worst one after adopting a small number of spares to rescue those slowest chip-rows. 
As illustrated by Figure \ref{fig:chip_default}, the chip-specific {\tt tWR} design helps to improve chip yield rate as otherwise a chip with {\tt tWR}=24ns would be discarded if {\tt tWR} is set as 23ns or less in the standard.  While technically all fabricated chips can now be treated as good ones, those with very large {\tt tWR} (e.g., twice as large as the expected {\tt tWR}) should still be marked as failed chips as DIMMs constructed from them tend to have very low performance. 

\begin{figure}
 \centering
  \begin{subfigure}{.32\textwidth}
    \centering
    	\includegraphics[width=\linewidth]{figures/default.pdf}\\
    \caption{Chip-specific}
    \label{fig:chip_default}
  \end{subfigure}
%
  \begin{subfigure}{.32\textwidth}
    \centering
    	\includegraphics[width=\linewidth]{figures/chunk_unsort.pdf}\\
    \caption{Chunk-specific}
    \label{fig:chunk_unsort}
  \end{subfigure}
  %
  \begin{subfigure}{.32\textwidth}
    \centering
    	\includegraphics[width=\linewidth]{figures/chunk_sort.pdf}\\
    \caption{Chunk-specific w/ remap}
    \label{fig:chunk_sort}
  \end{subfigure}
  \vspace{-0.45in}
\caption{Comparison of different schemes: (a) The chip-specific {\tt tWR}; (b) The chunk-specific {\tt tWR}; (c) The chunk-specific {\tt tWR} with chunk remapping. For illustration purpose, each rank consists of two chips while each chip contains two four-row banks. One {\underline {\bf DIMM-row}} (i.e., the row exposed to the OS) consists of two {\underline {\bf chip-row}} segments --- the number in each chip-row indicates its corresponding {\tt tWR}, i.e., the {\tt tWR} of the weakest cell.}
\label{fig:schemes}
  \vspace{-0.45in}
\end{figure}

\subsection{Chunk-specific Restoring Control}
Even though {\tt tWR} exhibits a wide range of variations when scaling in deep sub-micron regime, only a small number of cells need long recovery time.  Setting a DIMM's {\tt tWR} based on the chip-row that has the worst {\tt tWR} is still too pessimistic.
We therefore propose to partition each memory bank into a number of smaller chunks and set the chunk level {\tt tWR} based on the worst chip-row within the chunk. 
The chunk level {\tt tWR} is then exposed to the memory controller to aid cheduling.

In Figure \ref{fig:schemes}(b), one chunk consists of two rows. Since the first chunk has 23ns and 18ns {\tt tWR}s for its two chip-rows, its chunk {\tt tWR} is set to 23ns.
By take advantage of these fast chunks, a chunk-{\tt tWR}-aware memory controller can speed up memory accesses that fall into the fast chunks
\footnote{For discussion purpose, a {\em chip-chunk} is referred to as one chunk within one chip; a {\em DIMM-chunk} is referred to as the set of same-index chip-chunks from different chips of the DIMM. For example, the 2nd DIMM-chunk consists of the 2nd chip-chunk from each chip.}.


\subsection{Chunk-specific with Remapping}

The previous design can only form a DIMM-chunk from the same-index chip-chunks, which can be optimized to further reduce {\tt tWR} values. This is because the chip-chunks that are of the same index may exhibit significant {\tt tWR} difference. It would be beneficial to form a chunk using chip-chunks that are of the same or similar {\tt tWR}s. 

For the example in Figure \ref{fig:chunk_sort}, if we form the first DIMM-chunk using the 4th chip-chunk from chip 0 and the 1st chip-chunk from chip 1, the {\tt tWR} of this chunk can be as low as 18ns. Constructing a number of such fast chunks helps to speed up the average row access time of the given DIMM.

The chunk remapping is done in two steps: (1) after detecting the {\tt tWR} for each chip-chunk, we compute the averaged {\tt tWR} for each chip-bank, and sort chip-banks independently on each chip. A {\bf DIMM-bank} consists of chip-banks that are of the same index on the sorted list; (2) For chip-chunks within each chip-bank, we sort them again such that each {\bf DIMM-chunk} consists of chip-chunks that are of the same index on the sorted list. 

While only one access is allowed to access one bank at any time, the multiple banks in a DIMM can be accessed simultaneously. To maintain the same bank level parallelism, we treat the chip-chunks from one bank as a group in chunk remapping. In Figure \ref{fig:schemes}(c), 
DIMM-chunk 0 and 1 belong the DIMM-bank 0. Since DIMM-chunk 0 is constructed using chip-chunk 3 on chip 0, DIMM-chunk 1 needs to use chunks from the same group, i.e., chip-chunk 2 on chip 0. In this way, simultaneously accessing two different DIMM-banks will never compete for the same chip-bank on any chip. 

\subsection{Restore Time aware Rank Construction}
\label{sec:match_twr}

A DIMM rank is composed of multiple chips, which work in lockstep fashion. The access speed of one logical row is determined by its worst chip-row. While chunk-remapping does not have to form a DIMM-row using the chip rows that of the same physical index, it may still be ineffective when one of the chips that form a rank contains many slow rows. A bad chip would lead to a slow rank no matter how the chunks are remapped.

\begin{figure}[ht]
\centering
\includegraphics[width=0.8\textwidth]{figures/device_match.pdf}
\vspace{-0.15in}
\caption{Rank construction consists of three steps --- (1) chip sorting and seed chip selection;
(2) distributing chips to bins; (3) constructing DRAM ranks using chips from each bin.}
\label{fig:device_match}
  \vspace{-0.25in}
\end{figure}

We further propose to construct DRAM ranks using compatible chips, rather than random chip selection in the baseline design. 
Given $N$ DRAM chips, our goal is to construct a better rank set (and each rank contains $R$ chips). The rows in each chip are divided into $K$ chunks and we use $M$ bins to assist rank construction.

We first compute the average chip level {\tt tWR}, which uses the chunk level {\tt tWR} values of each chip. The latter can be collected during post-fabrication testing.
We sort the chips based on their average {\tt tWR} values, and choose $M$ seed chips, i.e., the chips on the sorted chip list whose indice can be divided by $\lfloor N/M\rfloor$. The seed chips are distributed to $M$ bins.

We then place the rest of chips into $M$ bins based on their similarity to the seed chip of each bin. The chunk level {\tt tWR} values of each chip are treated as a $K$-item vector. The similarity of two chips is calculated using the Hamming distance of the two $K$-item vectors. The candidate chip is placed in the bin whose seed chip has the smallest Hamming distance, i.e., the highest similarity, to the candidate chip.

Once a bin reaches its size limit, i.e., $n\times R$  where $n = \lfloor{N/M/R}\rfloor$, and $n\times R \leq N/M$, it can no longer accept new chips. In the algorithm, an extra bin $Bin_{M+1}$ is used to hold the leftover chips.
When filling chips to each bin, we construct a rank if a bin has $R$ chips (the seed chip is used to form a rank in the last batch). 

Since the algorithm needs to scan each chip and compute its similarity with all seed chips, the time complexity is $O(N\times M \times K)$. Here $M$ and $K$ are constant. $M$ is usually small ($M << N$) while $K$ can be relatively large, e.g., $K$=1024. Therefore, the time complexity is linear to the number of candidate chips. This is a light weight rank construction scheme, compared that in \cite{DAC15:radar}. Our experiments show that the two algorithms achieve similar rank level {\tt tWR} results. 
%. As a comparison, 
%the recently proposed rank construction scheme \cite{DAC15:radar} needs to sort the candidate chips continuously, which results in time complexity up to $O(N^3)$. 

\subsection{Restore Time aware Page Allocation}
\label{subsec:page_alloc}
%The translation of virtual to physical address is supported in hardware by Memory Management Unit (MMU), and the virtual-physical mapping is determined by operating system (OS). 
Traditional page allocation is restore time oblivious as all physical pages have the same access latency. However, when a set of fast DRAM chunks are constructed and exposed to the memory controller, %it is beneficial to exploit the access latency difference to speed up program execution. 
the memory system can be more effective if fast chunks are assigned to service performance-critical pages. In this paper, the page criticality is estimated using its access frequency \cite{ISCA13:charm,TC01:alloc}. Studies have shown that it is usually a small subset of pages, referred to as hot pages, that are frequently accessed in modern applications \cite{ISCA09:hot_page,ICS11:hot_page,TODAES13:hot_page}. %We adopt the offline profiling approach as in \cite{ISCA13:charm} to identify hot pages.

%\begin{figure}
 %\centering
 %   	\includegraphics[width=0.8\linewidth]{figures/TODAES_data/{page.dat}.pdf}\\
  %\vspace{-0.45in}
  %\caption{The page access distributions in SPEC CPU2006.}
  %\label{fig:page}
  %\vspace{-0.1in}
%\end{figure}


%Figure \ref{fig:page} studies the page access distribution of a set of SPEC CPU2006 applications. The figure shows that different applications have very different access behaviors: for some workloads, e.g., $459.Gem$ and $470.lbm$, accesses are evenly distributed such that the number of accumulative requests grows linearly with the number of touched pages; for some other applications, e.g., $429.mcf$ and $403.gcc$, most memory accesses come from a small subset of hot pages. The hot pages are the ones to be allocated in fast DRAM chunks. Further experimental analysis shows that the majority workloads touch less than 1/8 of the total memory space, while some benchmarks (i.e., $459.lbm$ and $429.mcf$) use up all available space.

In this paper, our goal is to illustrate that a restore time aware page allocator can take advantage of the latency difference of the DRAM chunks. For this purpose, we adopt a simple strategy that profiles program execution offline \cite{ISCA13:charm} and statically allocates hot pages to fast chunks. 
In the case if profiles are not accurate, we may need to design and enable more flexible strategies, e.g., such as the detailed behavioral synthesis \cite{TODAES11:partition} 
and data migration and compression \cite{TODAES08:bankmem}. We leave this as our future work. 

\section{Architectural Enhancements}
%In this section, we'll present the architectural implementations and also the overheads.

%\subsection{Implementation}
In order to exploit restore time variations at either chip or chunk levels, a post-fabrication testing needs to be performed to detect restore time at fine-granularity. Given that cell restore time is thermal dependent --- study showed that it becomes worse at low temperature \cite{MEM14:twr}, the manufacturer needs to record the worse timing constraints under chip's allowed working conditions. The values are organized as a table (with each entry in the table recording affected timing constraints {\tt tWR}/{\tt tRAS} of its corresponding DIMM chunk) and saved in non-volatile storage in the DIMM \cite{MICRO13:rowclone}. The memory controller loads this table at boot time and schedules memory accesses accordingly.% to maximize bandwidth and avoid conflicts. 
%As an example, two READ operations cannot be scheduled back to back to a DIMM bank if the later one accesses a fast chunk and shall compete with the preceding READ for using the data bus. 

\begin{figure}[ht] % !=overrides latex; h=here; t=top; b=bottom; p=special page for floating objects
\begin{center}
%\vspace{-0.1in}
\includegraphics[width=0.7\textwidth]{figures/TODAES_data/design.pdf}
\vspace{-0.2in}
\caption{The on-DIMM architectural enhancement.}
\label{fig:design}
\end{center}
\vspace{-0.45in}
\end{figure}

To enable chunk re-organization, we need one extra chunk remapping table as shown in Figure \ref{fig:design}. Similar as HP's MC-DIMM \cite{SC09:mcdimm} and Mini-rank \cite{MICRO08:minirank}, our design integrates a bridge chip on-DIMM, which remaps the physical address sent from the memory controller to device row addresses in each chip. For the chunk remapping table, each entry maps the corresponding DIMM-chunk to the chip-chunk at each chip.  
%Given the following Table \ref{tab:art}, when the bridge chip receives a request asking for data in the 10th DIMM-chunk, it translates the requests to asking for segment data from the 1220$^{th}$ chunk from chip 0, the 124$^{th}$ chunk from chip 1, etc. 

%\begin{table}[htbp]
%\centering
%\caption{Remap table}
%\begin{tabular}{cccccccc}
%\hline
%DIMM\_chunk &chip0\_chunk &chip1\_chunk &... &chip7\_chunk \\
%\hline 
%...	 	&...	&... 	&...		&...\\
%10	 	&1220		&124 	&...		&256\\
%...	 	&...	&... 	&...		&...\\
%\hline 
%\end{tabular}
%\vspace{-0.2in}
%\label{tab:art}
%\end{table}

\section{Experimental Methodology}
\subsection{Configuration}
To evaluate the effectiveness of our designs, we compared them to traditional repair solutions using an in-house chip-multiprocessor system simulator. We modeled a quad-core system with the parameters shown in Table \ref{tab:configuration}. 

We used VARIUS to generate 90 chips, and then form ranks in different fashion discussed in Section \ref{sec:match_twr}. The memory system to be simulated is composed of two ranks. We constructed five rank pairs and tested the proposed designs with all pairs.
The DRAM timing constraints follow Hynix DDR3 SDRAM data sheet \cite{hynix:ddr3}. 
%For the schemes exploiting chunk level timing constraints, we added two CPU cycles to access the timing table.
%For the schemes performing chunk-remapping, we added one extra DRAM cycle to access the mapping table.

\begin{table}[htbp]
\vspace{-0.2in}
\centering
\caption{System Configuration}
\vspace{-0.15in}
\scalebox{0.85}
{
\begin{tabular}{l|l}
\hline\hline
Processor				&four 3.2Ghz cores; four-issue; 128 ROB size\\
\hline
				&L1(private): 64KB, 4-way, 3 cycles\\
Cache			&L2(shared): 2MB, 6-way, 32 cycles\\
				&64B cacheline\\
\hline
Memory				&Bus frequency: 1066 MHz\\
Controller 	& 128-entry queue; close page\\
\hline
			&1channel, 2ranks/channel, 8banks/rank, \\
DRAM				&16K rows/bank, 8KB/row,\\
				&1066 MHz, tCK=0.935ns, width: x8\\
\hline\hline

\end{tabular}
}
\label{tab:configuration}
\vspace{-0.2in}
\end{table}

\subsection{Workloads}
We used SPEC CPU2006 and simulated 1 billion instructions after skipping the warm-up phase of each benchmark. 
%The Read and Write MPKI (memory accesses per kilo instructions) for each workload is profiled to indicate the memory intensiveness. 
Based on MPKI, the applications are classified into three categories (Spec-High/Spec-Med/Spec-Low).%, as shown in Table \ref{tab:bench}.
The workloads are running in rate mode, where all cores execute the same task.

We performed timing simulation until all cores finish the execution, and averaged the execution time of all the four cores. We constructed five rank pairs, i.e., DIMMs. One simulation run used one DIMM while the reported results are the average of the runs using different DIMMs.

\section{Results}
We evaluated the following schemes:

--- \texttt{Baseline}. The baseline sets {\tt tWR} to 15ns, following existing specification. It is the ideal baseline due to scaling. The results of other schemes are normalized to the baseline for comparison.

--- \texttt{Relax-}$x$. Given that scaling in deep sub-micron regime leads to worse timing, this scheme relaxes time constraints to achieve x\% yield. We relaxed {\tt tWR} and set {\tt tRAS}/{\tt tRC} accordingly. We tested x=85 and x=100, respectively.

--- \texttt{Spare-}$x$. One commonly adopted post-fabrication repair approach is to integrate sparing rows/columns \cite{BOOK:jacob} to mitigate performance and yield loss. 
%It is implemented by using a laser programmable link to disconnect the abnormal rows/columns and connect the spare ones \cite{Bruce:Jacob}. 
In our experiments, we set the spare density as high as 16 spare rows per 512-row block, which resides in the aggressive spectrum \cite{BOOK:fault,SSC96:spare}. Given that spares may be reserved for high-priority repairs, such as defects and retention failures, we testsed x=0, 2, 8, 16 spares out of each 512-row block, respectively. 

--- \texttt{ECC}. ECC is implemented by placing one extra ECC chip to correct errors in data chips. Though ECC is conventionally used to correct soft errors, it can be potentially used to tolerate weak cells. Exploiting ECC chips to rescue slow rows sacrifices soft error resilience and hurts reliability \cite{DFT05:ecc}.  

--- {\tt Chunk-}$x$. This scheme implements the chunk-specific restore time control, with each bank being divided into $x$ chunks. Each DIMM chunk has its own timing constraints, which are exposed to the variation-aware memory controller.

--- \texttt{ChunkSort-}$x$. This scheme implements the chunk-specific restore time control with chunk remapping, with each bank being divided into $x$ chunks. Similar as {\tt Chunk-}$x$, the timing constraints of each chunk are exposed to the memory controller.

--- \texttt{ChunkBin-}$x$. This schemes is similar as {\tt Chunk-}$x$. The difference is that it constructs ranks using the proposed bin-based matching scheme.

--- \texttt{ChunkSortBin-}$x$.  This schemes is similar as {\tt ChunkSort-}$x$. The difference is that it constructs ranks using the proposed bin-based matching scheme.

We compared these schemes on memory access latency and system performance, and studied their sensitivity to different system configurations. 

\subsection{Execution Time}

\begin{figure}[ht]
\centering
\includegraphics[width=0.7\textwidth]{figures/TODAES_data/14nm/{stat_RAND_main-cycles.perc.dat}.pdf}
\vspace{-0.2in}
\caption{The execution time comparison of different schemes under random page allocation policy. Representative applications and the geometric means for highly memory-intensive (Spec-High) and all applications (Spec-All) are presented here.}
\label{fig:tech_time}
  \vspace{-0.25in}
\end{figure}

From Figure \ref{fig:tech_time}, we observed that 
(1) DRAM scaling has a large impact on restore time. To maintain a high yield rate, the timing constraints have to be vastly relaxed from 16 cycles to over 40 cycles, which significantly hurts performance. On average, {\tt Relax-100} and {\tt Relax-85} prolong the execution time by 37.0\% and 34.9\%, respectively. Highly memory-intensive applications tend to have large degradation (i.e., over 40\%). 
(2) Adding spare rows helps to mitigate performance losses: {\tt Spare-8} is 31.9\% worse than the ideal. 
(3) {\tt ECC} works only slightly better than {\tt Spare-8}. This is because SEC-DED ECC can only correct one bit in each 64-bit block. Since there  might be multiple cells violating timing constraints within such a 64-bit block, ECC lacks the ability to effectively adapt restore time variations.
(4) {\tt Chunk-4k} is less than 1\% better than {\tt ECC} as it exposes chunk-level restore time variations. There are a small number of chunks that have smaller tWRs than the single tWR in {\tt ECC}. Due to random page allocation policy, the exposed fast chunks cannot be fully exploited, and thus the performance improvement is pretty limited.
(5) {\tt ChunkSort-4k} works better than {\tt Chunk-4k} because it helps to construct more fast chunks. On average,
{\tt ChunkSort-4k} helps to reduce the performance loss from 37\% in {\tt Relax-100} to 26.5\%, and 4\% better than {\tt Chunk-4k} for {\tt Spec-High}.

In addition, restore time aware rank construction helps to reduce tWR ---  {\tt ChunkBin-4k} is 3\% better than  {\tt Chunk-4k} while  {\tt ChunkSortBin-4k} is 4.8\% better than  {\tt ChunkSort-4k}. 
Interestingly, {\tt ChunkBin-4k} and {\tt ChunkSort-4k} achieve comparable performance improvements over the baseline. While both schemes require post-chip-fabrication testing to extract chunk level tWR values, the former needs rank clustering, which imposes extra step and cost during fabrication; the latter needs to embed mapping table and thus introduces extra runtime overhead.  
{\tt ChunkSortBin-4k} achieves the best performance while it incurs both extra fabrication cost and runtime overhead. 

\subsection{Page Allocation Effects}

Figure \ref{fig:tech_time_14nm} compare the results using random and restore-time-aware page allocation schemes.
From the figure, by making better use of fast chunks, restore time aware page allocation speeds up the execution of all chunk based schemes, e.g., for {\tt ChunkSortBin-4k}, restore time aware allocation achieves 15\% improvement over random allocation.
Restore time aware allocation is very effective for most benchmark programs --- on average, {\tt ChunkSortBin-4k} is only 2\% worse than the ideal {\tt Baseline}. 

\begin{figure}
\centering

\begin{minipage}[b]{0.42\linewidth}
\centering
\includegraphics[width=\linewidth]{figures/TODAES_data/14nm/{RAND_small.dat}.pdf}\\
\subcaption{With random page allocation}%\label{fig:20nm_main}
\end{minipage}%
\begin{minipage}[b]{0.42\linewidth}
\centering
\includegraphics[width=\linewidth]{figures/TODAES_data/14nm/{PROF_small.dat}.pdf}\\
\subcaption{With restore time aware page allocation}%\label{fig:14nm_main}
\end{minipage}%
\vspace{-0.4in}
\caption{The execution time comparison of different schemes at 14nm technology node.}
\label{fig:tech_time_14nm}
\vspace{-0.45in}
\end{figure}


In the figure, $470.lbm$ achieves small improvement because it evenly accesses a large number of memory pages and lacks very hot pages.  Given that a small number of chunks have shorter than 15ns tWR values, it is not surprising to find that some benchmark programs, e.g., $403.gcc$ and $400.per$, have their hot pages fit in these fast chunks and thus outperform {\tt Baseline}.

Also in the figure, we observed that the effectiveness of restore time aware rank construction is diminishing after adopting restore time aware allocation. For example, on average, {\tt ChunkSort-4k} and {\tt ChunkSortBin-4k} have less than 1\% difference when using restore time aware allocation. 
Nevertheless, those benchmarks with large footprint and relatively uniform access pattern, e.g., $470.lbm$, can still achieve distinct benefits.

\subsection{Further Studies}
The effectiveness of conventional Sparing technique strongly depends on the sparing levels being used; the proposed chunk-based schemes depends on the chunk granularity. Hence, we conducted sensitivity studies on these two key parameters. 
And the experimental results show that diminishing returns using more spares because of increasingly more slow cells.
As expected, the study varying number of chunks shows that higher improvement can be achieved with increasing storage and latency overheads.

\section{Conclusion}
In this work, we studied DRAM scaling effects on restore time, and showed that future DRAM chips need relaxed timing constraints to maintain high yield and to keep the manufacturing cost low. Existing approaches are ineffective to address the performance losses. 
We proposed schemes to expose restore time variations at chunk level and devised architectural enhancements to enable find-grained variation-aware scheduling. We then proposed restore time aware rank construction and page aware page allocation schemes to make better use of fast chunks. The experimental results show that our schemes achieve as high as 25\% average performance improvement over traditional solutions.



\chapter{Shorten Restoring using Refresh}
\label{chapter:partialrestore}
%\section{Background on Refresh}
%DRAM needs to be refreshed periodically to prevent data loss. According to JEDEC \cite{JEDEC:ddr3}, 8192 all-bank auto-refresh ({\tt REF}) commands are sent to all DRAM devices in a rank within one retention time interval ({\tt Tret}), also called as one refresh window ({\tt tREFW}) \cite{TC15:refresh, ISCA13:ddr4, HPCA14:parallelrefresh}, typically 64ms for DDR3/4. The gap between two {\tt REF} commands is termed as refresh interval ({\tt tREFI}), whose typical value is 7.8$\mu$s, i.e. 64ms/8192.If a DRAM device has more than 8192 rows, rows are grouped into 8192 {\bf refresh bins}. One {\tt REF} command is used to refresh multiple rows in a bin. An internal counter in each DRAM device tracks the designated rows to be refreshed upon receiving {\tt REF}. The refresh operation takes {\tt tRFC} to complete, which proportionally depends on the number of rows in the bin.

%The refresh rate of one bin is determined by the leakiest cell in the bin. Liu~{\em et~al.} \cite{ISCA12:raidr} reported that fewer than 1000 cells require a refresh window shorter than 256ms in a 32GB DRAM main memory. Given that the majority of rows have retention time longer than 64ms, it is beneficial to enable multi-rate refresh, i.e., different bins are refreshed at different rates. The weakest cell in one bin determines the refresh rate of the bin. For discussion purpose, a DRAM cell/row/bin that is refreshed at 256ms is referred to as a \underline{256ms-cell}/\underline{row}/\underline{bin}, respectively.

%We adopt the {\em flexible auto-refresh} mechanism from \cite{ISCA15:reflex} to support multi-rate refresh, i.e., 8192 refresh commands are sent every 64ms --- one for each bin. If a bin needs to be refreshed every 256ms,  {\em flexible auto-refresh} sends four {\tt REF} commands in 256ms to this bin. However, only one is a real refresh while the other three are dummy ones that only increment the refresh counter. 

%While the bin counter is maintained in the memory controller and incremented sequentially, the actual row addresses (responding to each bin-refresh command) are generated internally inside DDR3/4 devices \cite{JEDEC:ddr3,JEDEC:ddr4}. This mapping may be non-linear because of vendor's full flexibility to implement the refresh.Recent studies \cite{ISCA15:reflex, ISCA14:disturbance} assume this mapping can be made known to the memory controller. We make the same assumption in this paper.
%We assume that the memory controller knows the mapping between bin address and row address, the same as that in \cite{ISCA15:reflex}, and similar to \cite{ISCA14:disturbance}.

\section{Motivation of Partial Restore}
Due to intrinsic leakage, the voltage level of a DRAM cell reduces monotonically after a full restore. The solid curve in Figure \ref{fig:partial} illustrates the voltage decay of an untouched cell (i.e., not accessed) within one refresh window, commonly 64ms. Stored data is safe as long as the voltage remains above V$_{min}$ (0.73V$_{dd}$
\footnote{Value is calculated on basis of charge sharing expression \cite{BOOK13:sharing} and offset noise \cite{JSSC02:sense}, details can be found in \cite{HPCA16:twr}.}
 here)  before the next refresh. If a read or write access occurs, the post-access restore operation fully charges the cell, as shown in the figure. However, the full charge is often unnecessary if the access is {\em close in time} to the next refresh, which will fully restore the cell anyway.

\begin{figure}[htbp]
\begin{center}
%\vspace{-0.2in}
\centering
\includegraphics[width=3.2in]{figures/HPCA16/rt_partial.pdf}
\vspace{-0.2in}
\caption{DRAM cell voltage is fully restored by either refresh commands or memory accesses.  (V$_{full}$ indicates fully charged; V$_{min}$ is the minimal voltage to avoid data loss).}
\label{fig:partial}
\vspace{-0.3in}
\end{center}
\end{figure}

Based on this observation, we propose that post-access restore {\em partially charges} a cell's voltage to the level that the cell would have if the cell had been untouched in one refresh window. The restore operation is terminated when this target voltage level is reached. 

The cell charging curve starts with a deep slope and flattens when approaching V$_{full}$ \cite{HPCA15:aldram, PATENT11:dram}, as demonstrated in SPICE modeling. Hence, reducing target voltage can drastically shorten restore time. For example, SPICE modeling indicates that restoring a cell's charge to 0.89 V$_{dd}$ rather than 0.975 V$_{dd}$ (fully charged) reduces {\tt tWR} from 25 to 15 DRAM cycles, i.e., a 40\% reduction.

We next describe two schemes, {\tt RT-next} and {\tt RT-select}, to enable partial restore.  These schemes are applied by the memory controller.

\section{Proposed Designs}
\subsection{RT-next: Refresh-aware Truncation}

\begin{savenotes}
\begin{table*}[htbp]
\vspace{-0.2in}
\begin{center}
\caption{Adjusted restore timing values in {\tt RT-next}  (using the SPICE model)}
\vspace{-0.15in}
\label{rt_next_timing}
\scalebox{0.75}
{
\begin{tabular}{|c||c|c|c|c|*3c|}
\hline\hline
sub-window& \multicolumn{3}{|c|}{Distance to next refresh} & Target restore     &{\tt tRAS}	  &{\tt tWR}	  &{\tt tRCD}\\
            & 64ms-row & 128ms-row & 256 ms-row              & voltage (V$_{dd}$) &\multicolumn{3}{c|}{(DRAM cycles) \footnote{Timing values are gotten from SPICE modeling, more details can be found in \cite{HPCA16:twr}.}} \\ \hline \hline
1st         & [64ms, 48ms) & [128ms, 96ms) & [256ms, 192ms)  & 0.975		&42		&25	   	&15 \footnote{The studies focus on the relationship between restore and retention. Consequently, unrelated timing values, such as {\tt tRCD}, are unchanged.}  \\ \hline
2nd         & [48ms, 32ms) & [96ms, 64ms) & [192ms, 128ms)   & 0.92          	&27      &18    	&15  \\ \hline
3rd         & [32ms, 16ms) & [64ms, 32ms) & [128ms, 64ms)    & 0.86          	&21      &14    	&15  \\ \hline
4th         & [16ms, 0ms) & [32ms, 0ms) & [64ms, 0ms)        & 0.80          	&18      &11     &15  \\ \hline \hline
\multicolumn{4}{|c|}{No Truncation}            &0.975		&42		&25    	&15  \\ \hline \hline
%refresh margin &0.73          	&15      &9     &15  \\ \hline \hline
\end{tabular}
}
\end{center}
\vspace{-0.1in}
\end{table*}
\end{savenotes}

{\tt RT-next} truncates a long restore operation according to the time distance to its next refresh. %The sooner the next refresh is, the less charge the cells in the row need, and the earlier the restore operation can be terminated.  
%{\tt RT-next} works as follows. 
The refresh window is partitioned into multiple sub-windows, each of which provides a set of timing parameter values.
In the following, we use four sub-windows to discuss our proposed schemes --- Table \ref{rt_next_timing} lists the adjusted timing values for the device that we model in this paper. The smaller the timing values are, the larger opportunity the truncation has. While distinguishing more sub-windows provides finer-grained control and the potential to exploit more truncation benefits, it complicates the control and has diminishing benefits as shown in our experiments. 

As illustrated by Figure~\ref{fig:rtnext}, when servicing a read or write access, {\tt RT-next}  
calculates the time distance to the next refresh command and determine the sub-window that the access falls in. It then truncates its restore operation using the adjusted timing parameters, e.g., the right most columns in Table \ref{rt_next_timing}.

\begin{figure}[htbp]
\begin{center}
%\vspace{-0.1in}
\centering
\includegraphics[width=3.2in]{figures/HPCA16/rt_next.pdf}
\vspace{-0.2in}
\caption{{\tt RT-next} safely truncates restore operation. Exponential curve has been verified from SPICE modeling, and linear line shows the conservative restoring goals.}
\label{fig:rtnext}
\vspace{-0.45in}
\end{center}
\end{figure}

%The memory controller also needs to consider the page policy (open or close).  A restore is truncated by a {\tt PRE} command from the memory controller. 
%For a close-page policy, every access can potentially truncate restore early. 
%For an open-page policy, truncating restore of a preceding access may not beneficial if its following access is a row buffer hit. 
%We evaluate both policies in the experiments.

%Figure~\ref{fig:rtnext} illustrates the working mechanism of {\tt RT-next}, where reads {\tt 'a'}, {\tt 'b'}, and {\tt 'c'} are serviced in the first, the second, and the fourth sub-windows, respectively.  Generally, {\tt RT-next} is a conservative approach to restore the cell charge.
%The figure shows how {\tt RT-next} is conservative in two ways:



%\footnotetext{\hi{Note that the negative slope is the conservatively approximated restore curve; the voltage decay curve, not shown in the figure, is always below the restore line.}}

%{\tt RT-next} is compatible with baseline auto refresh policy \cite{ISCA15:refresh, JEDEC:ddr3} that sends out {\tt REF} commands sequentially for all refresh bins. Given a DRAM row being accessed, the memory controller checks which bin the last {\tt REF} command is for and then determine how far the {\tt REF} is to be sent for the bin that the being accessed row resides.

\vspace{0.1in} 
{\underline{\bf {\tt RT-next} in multi-rate refresh.}}
\footnote{Retention time is modeled following \cite{EDL09:ret,ISCA12:raidr,ISCA13:archshield}, and leaky cells are randomly distributed based on prior works \cite{ISCA12:raidr, ISCA15:reflex, ICCD14:proactive}}
Applying {\tt RT-next} in a multi-rate refresh environment works similarly to the case that has only one rate. To calculate the distance between a memory access and the next refresh to its bin, 
{\tt RT-next} uses the same formula except adding the extra refresh rounds for low rate, i.e., 128/256ms, bins.
%replacing the refresh rate value with the individual refresh rate attached to each bin. 
Here we assume the underlying multi-rate refresh scheme has profiled and tagged each bin with its best refresh rate, e.g., 64ms, 128ms, or 256ms. 

As shown in Figure \ref{fig:rtnext_m}, it simplifies the timing control in memory controller by restoring a cell's post-access voltage according to the linear line between V$_{full}$ and V$_{min}$ (rather than the exponential decay curve). Given a 64ms-row and a 256ms-row, accesses falling in the same corresponding sub-window can use the same timing values in Table \ref{rt_next_timing}. 

\begin{figure}[htbp]
\begin{center}
\vspace{-0.1in}
\centering
\includegraphics[width=3.2in]{figures/HPCA16/rt_next_m.pdf}
\vspace{-0.2in}
\caption{Restoring voltage according to linear line simplifies timing control in multi-rate refresh --- a 64ms-row and a 256ms-row share the same timing values in each correspond sub-window.}

\label{fig:rtnext_m}
\vspace{-0.45in}
\end{center}
\end{figure}

%In particular, although two different bins (e.g., 64ms and 256ms bins) have different sub-interval durations, they use the same adjusted timings (see Table \ref{rt_next_timing}). The same timings are used because, as shown in Figure \ref{fig:rtnext_m}, restore is conservatively performed using the approximated linear curve. 
%The cell loses about the same portion of charge within each sub-interval.
%The cell necessitates the same portion of charge within each sub-interval.
%\hi{Such interval division minimizes the required number of timing sets and thus greatly simply the design of memory controller}


\subsection{RT-select: Proactive Refresh Rate Upgrade}
Refresh and restore are two correlated operations that determine the charge in a cell. Less frequently refreshed bins can be exploited to further shorten the post-access restore time. 
We next present {\tt RT-select}, a scheme that upgrades refresh rate for more truncation opportunities. 

Figure \ref{fig:rtselective} illustrates the benefit of refreshing a 256ms-row (in multi-rate refresh) at 128ms rate.  Given that this row is a 256ms-row, its voltage level decreases to V$_{min}$ after 256ms. As shown in Figure \ref{fig:rtselective}(a), the refresh commands sent at +64ms, +128ms, and +192ms marks are dummy ones. The access ``Rd'' appears in the 2nd sub-window; it has a distance within [192ms, 128ms) to the next refresh command. According to {\tt RT-next}, the restore can be truncated after reaching 0.92V$_{dd}$ (using the 256ms-row column in Table \ref{rt_next_timing}).

Now, suppose the dummy refresh at +128ms is converted to a real refresh, i.e.,  the row is ``upgraded'' to a 128ms-row.  With this earlier refresh, 
%However, if the {\em dummy refresh at 128ms is converted to a real refresh (i.e., the 
%row is upgraded to a 128ms-row), then 
the access ''Rd'' is at most 64ms away from the next refresh. 
Using the 128ms-row column in the timing adjustment table, {\tt RT-next} 
can truncate the restore after it reaches 0.86V$_{dd}$, achieving significant timing 
reduction for the restore operation (Figure \ref{fig:rtselective}(b)).

%{\underline{if we convert the dummy refresh}}
%{\underline{command at 128ms mark to a real refresh, i.e., this row is}} \\
%{\underline{upgraded to a 128ms-row}}, access ``R$_a$'' is at most 64ms away from its next refresh. Using the 128ms-row column in the timing adjustment table, {\tt RT-next} can truncate the restore after it reaches 0.86V$_{dd}$, achieving significant timing reduction in restore operations.

\begin{figure*}[htbp]
\begin{center}
%\vspace{-0.1in}
\centering
\includegraphics[width=6.4in]{figures/HPCA16/rt_selective.pdf}
\vspace{-0.2in}
\caption{The voltage target can be reduced if a strong row is refreshed at higher rate.}
\label{fig:rtselective}
\vspace{-0.3in}
\end{center}
\end{figure*}

Refreshing a 256ms-row at 128ms rate exposes more truncation benefits, as shown in 
Figure \ref{fig:rtselective}(c). For access "Rd'', it is sufficient to restore the voltage to 0.80V$_{dd}$ rather than 0.86V$_{dd}$ in above discussion. 
This is because a 256ms-row, even if being refreshed at 128ms rate, leaks slower than a real 128ms-row. We can adjust the timing values by following the solid thick line in \ref{fig:rtselective}(c), rather than the dashed thick line from a real 128ms-row, as shown in \ref{fig:rtselective}(b).
In particular, %as summarized in Table \ref{rt_selective_timing}, 
a row access, even if it is 128ms away from the next refresh to the row, just needs to restore the row to 0.86V$_{dd}$, rather than V$_{full}$ (=0.975V$_{dd}$) for a real 128ms-row. 

%\begin{table}[htbp]
%\vspace{-0.2in}
%\begin{center}
%\caption{Adjusted restore timing values in {\tt RT-select}}
%\vspace{-0.2in}
%\label{rt_selective_timing}
%\scalebox{0.7}
%{
%\begin{tabular}{|c||c|c|*3c|}
%\hline\hline
%Upgrade       &  Distance to Next  & Target restore      &{\tt tRAS}	  &{\tt tWR}	  &{\tt tRCD}\\
%              &  refresh           &  voltage (V$_{dd}$) &\multicolumn{3}{c|}{(DRAM cycles)} \\ \hline \hline
%256ms-$>$128ms  & [128ms, 64ms)      & 0.86 		&21		   &14	   	&15  \\ \cline{2-6}
%              & [64ms, 0ms)        & 0.80     &18      &11    	&15  \\ \hline \hline
%256ms-$>$64ms   & [64ms, 0ms)        & 0.80 		&18		   &11	   	&15  \\ \hline \hline
%128ms-$>$64ms   & [64ms, 32ms)        & 0.86 		&21		   &14	   	&15  \\ \cline{2-6} 
%			 & [32ms, 0ms)        & 0.80 		&18		   &11	   	&15  \\ \hline \hline 
%\end{tabular}
%}
%\end{center}
%\vspace{-0.1in}
%\end{table}

{\bf RT-select scheme.} 
While upgrading refresh rate reduces restore time, it generates more real refresh commands, which not only prolongs memory unavailable period but also consumes more refresh energy. 
Previous work shows that refresh may consume over 20\% of the total memory energy for a 32Gb DRAM device \cite{TC15:refresh, ISCA12:raidr}. Blindly upgrading the refresh rate of all rows is thus not desirable.

{\tt RT-select} upgrades the refresh rate of {\em selected bins}, those were touched, for {\em one refresh window}. It works as follows. 
A 3-bit rate flag is attached to each refresh bin to support multi-rate refresh. 
When there is a need to upgrade, e.g., from 256ms to 128ms rate, {\tt RT-select} updates the rate flag as shown in section~3.5, which converts the dummy refresh at +128ms in Figure~\ref{fig:rtselective}. 
A real refresh command rolls the rate back to its original rate, i.e., {\tt RT-select} only upgrades the touched bin for one refresh window, which incurs modest refreshing overhead to the system.

\section{Architectural Enhancements}

To enable {\tt RT-next} and {\tt RT-select}, we enhance the memory controller, as shown in Figure~\ref{fig:arch}. RT adds a truncation controller, to adjust the timing for read, write, and refresh accesses. This control is similar to past schemes that change timings \cite{HPCA13:tldram, ISCA13:charm, HPCA14:nuat}.  The memory controller has a register that records the next bin to be refreshed, referred to as {\tt Bin$_c$}, which rolls over every 64ms. 
It can also infer the mapping from row address to refresh bin, the same as that in \cite{ISCA15:reflex, ISCA14:disturbance}. 

To support multi-rate refresh, the memory controller keeps a small table that uses 3 bits to record the refresh rate of each refresh bin, similar to that in \cite{ISCA12:raidr, ISCA15:reflex}. As shown in Table \ref{tab:flag}, a 64ms-/128ms-/256ms- bin is set as `000'/`01A'/`1BC', respectively. Here `A' and `BC' are initialized to ones and decrement every 64ms.
While the refresh bin counter increments every in 7.8$\mu$s(=64ms/8192), a real {\tt REF} command is sent to refresh the corresponding bin only if its bin flag is `000', `010', or `100'. 
`A' and `BC' are changed back to `1' and `11' afterwards, respectively.

\begin{figure}[htbp]
\begin{center}
%\vspace{-0.1in}
\centering
\includegraphics[width=3.25in]{figures/HPCA16/rt_arch.pdf}
\vspace{-0.1in}
\caption{The RT architecture (the shaded boxes are added components).}
\label{fig:arch}
\vspace{-0.45in}
\end{center}
\end{figure}

When upgrading the refresh rate of a refresh bin, we update the rate flag according to the last column in Table \ref{tab:flag}. For example, when upgrading a 128ms-bin to 64ms rate, we set the rate flag as `010', which triggers the refresh in the next 64ms duration and roll back to `011' afterwards. This effectively upgrades for one round. Upgrading 256ms-row to 128ms rate sets the flag as `1BC$\oplus$0B0', which always sets the middle bit to zero to ensure that the refresh distance is never beyond 128ms, and thus the sub-window can only be {\tt 3rd} and {\tt 4th} referring to Table \ref{rt_next_timing}. In general, the distance calculation in {\tt RT-select} is adjusted by adding further refresh rounds indicated by the two least significant bits (LSB) of rate flag.

 \begin{table}[htbp]
 \vspace{-0.25in}
\caption{Refresh rate adjustment table}
\vspace{-0.3in}
\begin{center}
\scalebox{0.8}
{
\begin{tabular}{l|l|l}
\hline \hline 
{Profiled refresh rate}	&{Rate flag}	&{Flag after rate upgrade}\\
\hline\hline
64ms		&$000$	& n/a\\
\hline
128ms		&$01A$	&128$\rightarrow$64ms: 010\\
\hline
256ms		&$1BC$	&256$\rightarrow$128ms: 1BC$\oplus$0B0\\
			&		&256$\rightarrow$64ms: 100\\
\hline\hline
\end{tabular}
}
\end{center}
\label{tab:flag}
%\vspace{-0.2in}
\end{table}

To enable multi-rate refresh, the %3KB (=3bits*8192) 
rate table is accessed before each refresh to determine if a real or dummy command should be sent. To enable {\tt RT-select}, the rate table is also 
accessed before each memory access to decide the refresh distance, and then to complete the upgrade after the access.
%accessed before each memory access to decide if it is necessary to upgrade the rate. 
The extra energy is minimal, as shown in Section~\ref{subsec:energy}.

\section{Experimental Methodology}\label{sec:experiment}
\subsection{Configuration}\label{subsec:setting}
To evaluate the effectiveness of our proposed designs, we performed the simulation using the memory system simulator USIMM \cite{SIMU:usimm}, which simulates DRAM system with detailed timing constraints. 
USIMM was modified to conduct a detailed study of refresh and restore operations.

We simulated a quad-core system with settings listed in Table \ref{tab:configuration}, similar to those in \cite{HPCA13:refresh_pausing, HPCA14:nuat}. 
The DRAM timing constraints follow Micron DDR3 SDRAM data sheet \cite{SIMU:datasheet}. By default, DRAM devices are refreshed with 8K {\tt REF} within 64ms, and {\tt tRFC} is 208 DRAM cycles, which translates into a {\tt tREFI} of 7.8 $\mu$s (i.e., 6240 DRAM cycles). As  \cite{HPCA13:refresh_pausing}, the baseline adopts closed page policy, which works better in multicore systems \cite{PACT12:close_page}. 

\begin{table}[htbp]
\vspace{-0.2in}
\caption{System Configuration}
\vspace{-0.3in}
\begin{center}
\scalebox{0.8}
{
\begin{tabular}{l|l}
\hline\hline
Processor			&four 3.2Ghz cores; 128 ROB size\\
				&Fetch width: 4, Retire width: 2, Pipeline depth: 10\\
\hline
				&Bus frequency: 800 MHz\\
 				&Write queue capacity: 64\\
Memory			&Write queue watermarks: 40/20\\	
Controller			&Address mapping: rw:cl:rk:bk:ch:offset\\
				&Page management policy: close-page with FRFCFS\\

\hline
				&2channels, 1rank/channel, 8banks/rank, \\
DRAM			&64K rows/bank, 8KB/row, 64B cache line\\
				&tCK=1.25ns, width: x8\\
%DRAM			%&tCAS(CL): 13.75ns, tRCD: 13.75ns, tRC: 48.75ns\\
				%&tCWD: 6.25ns (5 cycles), tBURST: 5.0ns (4 cycles)\\
				%&tRAS: 35ns, tRP: 13.75ns, tFAW: 24 cycles,\\
				%&tRRD: 5 cycles, tRFC: 208nCK, tREFI: 7.8$\mu$s\\
\hline\hline

\end{tabular}
}
\end{center}
\label{tab:configuration}
\vspace{-0.4in}
\end{table}

\subsection{Workloads}
For evaluation, we use workloads from the Memory Scheduling Championship \cite{SIMU:msc}, which covers a wide variety of benchmarks, including five commercial applications \textit{comm1} to \textit{comm5}, nine benchmarks from PARSEC suite and two benchmarks each from the SPEC suite and the Biobench suite. Among them, \textit{MT-fluid} and \textit{MT-canneal} are two multithreaded workloads.
As \cite{HPCA13:refresh_pausing}, the benchmarks are executed in rate mode, and the time to finish the last benchmark is computed as the execution time.


\section{Results}\label{sec:results}

%\subsection{Schemes to Study}
We evaluated our proposed RT schemes on system performance, memory access latency and energy, and then studied their sensitivities to different configurations.
To study different aspects of RT, we analyze different set of schemes in each figure.

\subsection{Impact on Performance}

Figure \ref{fig:time} compares the execution time of different schemes.
The results are normalized to \texttt{Baseline}. 
In the figure, {\tt Gmean} is the geometric mean of the results of all workloads.

\begin{figure*}[htbp] 
\centering
\centering
\includegraphics[width=6.5in]{{figures/HPCA16/DATA/0903.dat/4Gb/stat_4Gb_Cycles.perc.main.dat}.pdf}
\vspace{-0.45in}
\caption{Performance comparison of different schemes. {\tt Baseline} and {\tt ConvTm} adopts projected and conventional timings, respectively; {\tt NoRefresh} assumes no refresh activity in {\tt Baseline}; {\tt RT-next-XX} truncates restore based on its distance to next fresh, and {\tt f64} and {\tt var} supports fixed 64ms refresh rate and multiple rates, respectively; {\tt RT-all-upYY} aggressively upgrades refresh bins with rates lower than {\tt YY} to {\tt YY}; finally, {\tt RT-sel-upYY} are optimized schemes to balance the restore benefits and refresh overhead.}
\label{fig:time}
\vspace{-0.45in}
\end{figure*}

On average, \texttt{RT-next-f64} achieves 10\% improvement over {\tt Baseline} by truncating restore time. {\tt RT-next-var} identifies more truncation opportunities in multi-rate refresh DRAM modules and achieves better, i.e., 15\%, improvement. 
While {\tt RT-all-up128} truncates more restore time through refresh rate upgrade, it introduces extra refresh overhead and thus is slightly worse than {\tt RT-next-var}.
\texttt{RT-sel-up128} achieves 2.4\% improvement over \texttt{RT-next-var}
by balancing refresh operations and restore benefits. 
The performance gap between upgrading all rows and selective upgrading is even larger when we aggressively upgrade refresh rate to 64ms. \texttt{RT-sel-up64} achieves the best performance --- it is 19.5\% speedup over {\tt Baseline}, or 4.5\% better than \texttt{RT-next-var}.
The performance trend across the schemes demonstrates that our restoring schemes achieves a good balance between refresh and restore.

%Generally, memory access intensive workloads such as \textit{com1}, \textit{libq} and \textit{mumm} benefit most from the reduced restore timing.
%Particularly, \textit{MT-f} obtains the largest performance improvement because of the parallel access patterns and relatively tight gaps between accesses, which greatly enlarges the effect of shortened {\tt RAS} and {\tt WR}.

\subsection{Energy Consumption}\label{subsec:energy}
Figure \ref{fig:energy} compares the energy consumption of different schemes. 
We reported the energy consumption breakdown ---  background ({\tt bg}), activate/precharge ({\tt act/pre}), read/write ({\tt rd/wr}) and refresh ({\tt ref}). We summarized the results according to benchmark suites, where results are averaged over workloads within each suite.
We used the Micron power equations \cite{TOOL:power}, and the parameters from vendor data sheets \cite{SIMU:datasheet} with scaling. 

To enable truncation in multi-rate refresh DRAM modules, we need to query the refresh rate for each access. The refresh rates for 8192 bins are organized as 3KB direct mapped cache with 8B line size. We used CACTI5.3 \cite{URL:cacti} to model the cache with 32nm technology --- it requires 0.22ns access time, occupies 0.02mm$^2$ area, consumes 1.47mW standby leakage power, and spends 3.33pJ energy per access. The extra energy is trivial (less than 0.5\%) and is reported together with {\tt bg}. 

\begin{figure}[htbp] 
%\vspace{-0.1in}
\begin{center}
\centering
\includegraphics[width=0.7\textwidth]{{figures/HPCA16/DATA/0903.dat/4Gb/stat_4Gb_energy.all.dat.cat}.pdf}
\vspace{-0.1in}
\caption{Comparison of memory system energy.}
\label{fig:energy}
\end{center}
\vspace{-0.4in}
\end{figure}

From the figure, we observed that the device refresh energy for 4Gb chips is small. 
Due to increased refresh operations, {\tt RT-all-up128/-up64} consume much more refresh energy than {\tt RT-all-up128/-up64}, respectively. \texttt{RT-sel-up64} saves 17\% energy compared to {\tt Baseline}, and consumes slightly lower energy than \texttt{NoRerefresh} due to decreased execution time. And, as expected, {\tt RT-sel-} refresh schemes is more energy efficient than {\tt RT-all-} refresh peers.


\subsection{Comparison against the State-of-the-art}
Figure \ref{fig:state} compares RT with three related schemes in the literature. 
\begin{itemize}
\itemsep -1pt
\item
{\tt Archshield+} implements a scheme that treats all the cells with long restore latency as failures and adopts Archshield \cite{ISCA13:archshield} to rescue them. 
\item
{\tt MCR} is the recently proposed scheme that trade DRAM capacity for better timing parameters \cite{ISCA15:mcr}. {\tt $2x$ MCR} and {\tt $4x$ MCR} are the two options that reduce DRAM capacity to 50\% and 25\% of the original, respectively.
\item
{\tt ChunkRemap} implements the scheme that differentiates chunk level restore difference and constructs fast logic chunks through chunk remapping \cite{DATE15:twr}. 

\end{itemize}

\begin{figure}[htbp] 
\begin{center}
\centering
\includegraphics[width=3.4in]{{figures/HPCA16/DATA/0903.dat/4Gb/stat_4Gb_Cycles.perc.statart.dat}.pdf}
\vspace{-0.2in}
\caption{Comparison against the state-of-the-art.}
\label{fig:state}
\vspace{-0.45in}
\end{center}
\end{figure}

The figure shows that {\tt Archshield+} and {\tt ChunkRemap} are approaching {\tt ConvTm} while {\tt RT-sel-up64} is 5.2\% better than {\tt ConvTm}, exploiting more benefits from reduced restore time.
{\tt $4x$ MCR} outperforms {\tt RT-sel-up64} by a modest percentage, and {\tt RT-sel-up64} works better than {\tt $2x$ MCR}.

{\tt MCR} shares similarity with {\tt RT-select}, i.e., we share the observation that a line that is refreshed more frequently can be restored to a storage level lower than V$_{full}$. {\tt MCR} exploits this with significant DRAM capacity reduction while {\tt RT-select} takes a light weight design that upgrades used bins only for one refresh window, and leaves all the other bins being refreshed at original rates. 

%From the figure, {\tt $4x$ MCR} outperforms {\tt RT-sel-up64} by a modest percentage. 
%This is because {\tt MCR} improves the baseline by reducing not only restore time but also sensing time while {\tt RT-sel-up64} focuses only restore time. 
%{\tt RT-sel-up64} works better than {\tt $2x$ MCR} because it upgrades the refresh rate of a bin for one refresh window at a time, which significantly reduces refresh overhead (as shown with the difference from {\tt RT-all-up64}). 



%By focusing on restoring and respecting the DRAM auto-refresh, we only improve tRAS and tWR in this paper, whereas {\t MCR} reduces both tRCD and tRFC besides tRAS \footnote{Although {\tt MCR} \cite{ISCA15:mcr} failed to take write restoring (tWR) into account in the paper, for a fair comparison, we embed tWR reductions into MCR schemes.}
%. 
%Nevertheless, the figure shows that our proposed {\tt RT-sel-up64} outperforms {\tt MCR-2} and approaches {\tt MCR-4} (the gap is within 3\%).
%The performance difference is pretty narrow because: (1) our RT schemes take use of the multi-rate refresh to do restoring, which is capable to achieve restoring reduction and refresh improvement simultaneously comparing to {\tt Baseline}; to the contrary, {\tt MCR} ignored the retention time variation to use a fixed 64ms refresh rate; (2) RT schemes combines distance-aware restore and refresh rate upgrade, which makes the best timing values outperform those can be achieved by {\tt MCR}.


\begin{table}[htbp]
\vspace{-0.2in}
\caption{Comparing EDP between RT and MCR (lower is better).}
\vspace{-0.3in}
\begin{center}
\scalebox{0.8}
{
\begin{tabular}{*5c}
\hline\hline
Cases		& {\tt ConvTm} & {\tt RT-sel-up64}	& {\tt $2x$ MCR}	&{\tt$4x$ MCR-4}\\
\hline
\texttt{Same Chip}	& 1.0$\times$ &0.715$\times$		&0.753$\times$		&0.713$\times$\\
\texttt{Same Capacity}	& 1.0$\times$  &0.715$\times$		&0.918$\times$		&1.068$\times$\\
\hline\hline

\end{tabular}
}
\end{center}
\label{tab:edp}
\vspace{-0.45in}
\end{table}

Given that {\tt MCR} improves performance at a significant capacity reduction.
We next comparing the energy-delay-product (EDP) --- ``Same Chip'' is optimistic assumption as {\tt $4x$ MCR} has only 25\% available capacity, which is likely to have more page faults in practice. ``Same capacity'' enlarges the raw chip in {\tt MCR} by two/four times, which introduces more background power. {\tt RT-sel-up64} shows good potential as its EDP closely matches that of {\tt MCR} under ``same chip'' setting, and is much better under ``same capacity'' setting.

 
%Note that performance improvement of {\tt MCR} is resulted from a serious capacity effectiveness loss. For {\tt MCR-4}, four rows are formed as a group and only one row can function at a time, and thus 75\% capacity becomes unavailable. And thus, devices should be enlarged by four times to avoid frequent page fault. Further, we conservatively assume that in larger chips, all powers except {\tt bg} stay the same as 4Gb-chip.

%From Table \ref{tab:edp}, we can see that for the scenario of same overall capacity using 4Gb-chip, which may introduce substantial page fault overhead, {\tt RT-sel-up64} achieves a better EDP than {\tt MCR-2}, and vey close to {\tt MCR-4}. The explanation is that despite that {\tt MCR-4} has a better performance, its refresh energy is significantly higher than that of {\tt RT-sel-up64} and thus the EDP gets almost the same. However, for same effective capacity, 8Gb-/16Gb-chip is used for {\tt MCR-2} and {\tt MCR-2}, {\tt RT-sel-up64} shows a much lower EDP.

\subsection{Further Studies}
\label{SEC:ideal_comp}
To further evaluate the effectiveness of RT, we compare against several \textit{ideal} schemes, including refresh-free scheme, conventional timings and best interval timings, etc. The results show that RT schemes defeat all of those schemes and the gap to the most ideal scheme of \textit{best interval without refresh} is within 3\%.

In addition, we evaluate the performance sensitivity by varying configurations including chip density, refresh granularity, refresh sub-window and page management policy. The results positively demonstrate the robustness of the achieved performance.

\section{Conclusion}\label{sec:conclusion}
In this paper, we studied the restoring issues in further scaling DRAM, identified partial restore opportunity and proposed two restore truncation (RT) schemes to exploit the opportunities with different tradeoffs. Our experimental results showed that, on average, RT improves performance by 19.5\% and reduces  energy consumption by 17\%.


\chapter{Further Explorations of Restoring}
\label{chapter:future_study}
This chapter will briefly discuss further explorations of restoring, as future work. We'll extend the restoring topic to approximate computing, memory information leakage and 3D stacked memory.

\section{Combine Restoring with Approximate Computing} \label{work:approx}
\subsection{Introduction on Approximate Computing}
Energy and power are increasing concerns in nowadays computer systems, ranging from mobile devices to data centers; and much energy is spent on guaranteeing correctness \cite{PLDI11:enerj}.
Nevertheless, many modern applications have intrinsic tolerance to inaccuracy \cite{ISCA10:relax, PLDI11:enerj}. For instance, lots of problems have no perfect answer in domains like machine learning, computer vision and sensor data analysis, and hence the adopted solutions rely on heuristic approach; and, large-scale data analytics cares more about aggregate trends rather than the correctness of individual data elements. Apparently, these applications provide good opportunities to explore energy-accuracy tradeoff.

Approximate computing necessitates the collaboration of different layers spanning circuits, architectures and algorithms. On program level, we should annotate the approximate data, which is non-critical and able to tolerate inexactness; on instruction level, the system should distinguish approximate and precise instructions and then take use of different strategies to execute; the fundamental energy savings rely on hardware techniques, including voltage reduction, floating point rounding and refresh rate reduction, etc.

To quantify the energy-accuracy tradeoff, we need to measure the output quality of approximate execution and should ensure that the quality loss is sustainable.
The qualify-of-service (QoS) metrics are per-application specific, and are measured by comparing the final outputs of approximate execution against those of precise one.
For instance, if the outputs are images, then the QoS can be evaluated as the average difference of per-RGB value \cite{PLDI11:enerj} between the executions.

\subsection{Related Work}
Prior works on approximate computing performed explorations from both hardware \cite{DATE06:cmos, DATE10:proc, ISCA10:relax, ASPLOS11:flikker, ASPLOS12:disciplined, MICRO12:neural, MICRO13:appro, MICRO14:appro, MICRO15:doppelganger} and software \cite{PLDI10:green, PLDI11:enerj, ASPLOS11:knob, OOPSLA15:topaz}. Among them, there is significant research on storing approximate data more efficiently, which is also the focus of our tentative exploration.
Flikker \cite{ASPLOS11:flikker} refreshes approximate data at lower rates to save DRAM refresh energy, which is recently extended by \cite{MEM14:sparkk, CASES15:appro}.
\citeN{ASPLOS12:disciplined} proposed to apply dual voltage to SRAM array to balance energy and accuracy; Drowsy caches \cite{ISCA02:drowsy} reduces the supply voltage to save power; to improve PCM's lifetime and performance, \cite{MICRO13:appro} proposed to reduce write precision and reuse failed cells.

\subsection{General Ideas}
According to the observations of our prior studies \cite{DATE15:twr}, whereas the worst-case {\tt tWR} of the whole memory suffers from significant increase, a large portion of cells have {\tt tWR} within the specification range.
Such variation provides opportunity for applying approximate computing to achieve performance and performance improvements: {\tt tWR} can be aggressively reduced to low value(s) as long as we can guarantee the correctness of precise data, and control the error impact of approximate data.

Program annotation can be performed by inserting assembly code, which will be identified by Pintool. Pintool simulates the real instruction execution, and when stepping into annotated approximate region, it injects errors using the underlying DRAM bit mask.
As prior arts \cite{PLDI11:enerj, MICRO14:appro}, we will control the annotation to let the program run to the end, and compare the approximate results against precise ones to report QoS. Simultaneously, Pintool outputs memory accesses, which are then fed into conventional memory simulator to collect performance and energy values. The most challenging part lies on the tradeoff between QoS and performance achievement: fine-grained and aggressive {\tt tWR} control benefits memory performance, but is likely to introduce too many errors. We will take some measurements to constraint the errors, candidate solutions can be dedicated allocation, location correction or memory bit remapping, etc.

%\begin{itemize}
%\itemsep 0pt
%\item Fixed tWR, e.g., 15, and then do mapping or correction
%\item Random mapping, apply full tWR for precise data
%\end{itemize}

\section{Study Security Issues of Restoring Variation} \label{work:security}
%\subsection{Introduction on Memory Security Issues}
Modern computing systems suffer from increasing concerns on privacy, security and trust issues, with timing channel attacks as a representative example.
And recently, the concern on timing channel attack has moved from shared caches \cite{BSDCan05:cache_fun, ISCA07:cache_channel, HASP13:cache} and on-chip networks \cite{NOCS12:noc_channel, ISCA13:surfnoc} to shared main memory \cite{CCS13:oram, HPCA14:channel, MICRO15:fs}.
Memory access pattern can leak a significant amount of sensitive information through statistical inference \cite{CCS13:oram}.

As a remedy, ORAM \cite{CCS13:oram} was proposed to conceal a client's access pattern to remote storage by continuously shuffling and re-encrypting data as they are accessed.
\citeN{HPCA14:channel} proposed temporal partitioning (TP) to isolate thread accesses to hide access pattern. As an improvement, Fixed Service (FS) policies were studied by \cite{MICRO15:fs} to reshape memory access behaviors without much performance degradation.

Compared to the simple case of a single set of timings for the whole memory system, restoring variations in further scaling DRAM are likely to leak more information. For instance, various memory access speeds to different memory regions may expose the footprint to malicious users. And things can be much worse with the adoption of NUMA-aware page allocation and approximate computing. The former easily leak the frequently accessed data, and the latter correlates data to its location origin \cite{ISCA15:prob}. 

In addition, simply borrowing the schemes in \cite{HPCA14:channel, MICRO15:fs} would introduce higher overhead because of the much longer worst-case restoring timings.
As a result, it is necessary to integrate information leakage, restoring issues, page allocation and approximate computing to come out a workable solution with acceptable performance loss and safety guarantee.


%\subsection{Related Work}

%\subsection{Motivations and Proposed Ideas}

\section{Explore Restoring in 3D Stacked DRAM} \label{work:stacked}
Recent advances in die stacking techniques enables efficient integration of logic and memory dies in a single package, with a concrete example of Hybrid Memory Cube \cite{HMC:spec2}.
HMC is especially promising for its innovative architecture that stacks multiple memory dies atop of the bottom logic die, and adopts packetized serial link interface to transfer data and requests \cite{ICCD15:dlb}.
With the superior high bandwidth, low latency and packet-based interface, lots of work have proposed to move computation units inside the logic die \cite{JMicro:ndp, ISCA15:pim}.
However, thermal management is a big issue in stacked memories \cite{DAC06:3dmodel, WONDP14:thermal}, and the deployment of bottom computation logics like simple cores \cite{ISCA15:pim} and even GPU \cite{HPDC14:pim_gpu} worsens the issue. Besides, temperature variations exist among vertical dies \cite{ICCD13:temp}. It is known that DRAM is sensitive to temperature changes, including refresh \cite{HPCA15:al-dram, ISCA13:ddr4} and restoring time \cite{PATENT14:twr, MEM14:twr}.
Therefore, it is worthwhile to explore restoring time in stacked memories, and utilize the temperature characteristics to dig more opportunities to boost performance. 

%\subsection{Introduction on Stacked DRAM}

%\subsection{Related Work}

%\subsection{Motivations and Proposed Ideas}

\chapter{TIMELINE OF PROPOSED WORK}
\label{chapter:timeline}
\begin{table}[ht]
\vspace{-0.3in}
\caption{Timeline of Proposed Work.}
\vspace{-0.15in}
\centering
\scalebox{0.85}
{
\begin{tabular}{|l|l|l|}
\hline
\textbf{Date} & \textbf{Content} & \textbf{Deliverable results} \\
\hline
Jan - Feb  & Explore restoring in approximate computing & Pin-based framework for \\
 & in Section \ref{work:approx} of Chapter \ref{chapter:future_study}& restoring approximation \\
\hline
Mar - May  & Integrate restoring with information leakage & Experimental data of memory\\% through access pattern &  Paper submission \\
 & in Section \ref{work:security} of Chapter \ref{chapter:future_study}& performance and security\\
\hline
Jun - Sep & Study restoring in Hybrid Memory Cube (HMC) & Modified simulator of temperature\\
 & in Section \ref{work:stacked} of Chapter \ref{chapter:future_study}&  effect of restoring in HMC \\
\hline
Jul - Oct  & Thesis writing & Thesis ready for defense \\
\hline
Oct - Dec & Thesis revising & Completed thesis \\
\hline
\end{tabular}
}
\label{tab:timeline}
\end{table}

The proposed works will be undertaken as shown in the Table \ref{tab:timeline}.
I will start the effort with task (1) to develop Pin-based framework for approximate computing of restoring. This task involves program annotation, chip generation, QoS evaluation, and conventional performance simulation, etc.
While multiple complicated subtasks are there, this task has been partially finished, and will not take much time to complete.
Afterwards, I'll move to task (2) to study information leakage in restoring scenario, and this task is partially on basis of the previous approximation work.
With the completion of task (2), the overall goal of exploring DRAM restoring in application level have been reached, and then I'll start the study restoring's temperature effect in HMC, i.e., task (3).
The general infrastructure can be borrowed from my previous HMC work \cite{ICCD15:dlb}. This task might be performed concurrently with other jobs, and thus might take more time to finish.
At the end of task (3), the holistic exploration of DRAM restoring is considered finished, and thus I'll summarize all the tasks into my final thesis.

\chapter{SUMMARY}
\label{chapter:summary}
As our reliance on IT continues to increase, the complexity and urgency of the problems our society will face in the future will increase much faster than are our abilities to understand and deal with them. Future IT systems are likely to exhibit a level of interconnected complexity that makes it prone to failure and exceptional behaviors. The high risk of relying on IT systems that are failure-prone calls for new approaches to enhance their performance and resiliency to failure.

HPC and Cloud are two ecosystems that are designed for different applications and with disparate design principles. However, Big data technologies, such as Hadoop, clustered storage, and data visualization, are now merging with traditional HPC technologies. 
On the one hand, an increasing portion of HPC workloads is becoming data intensive.
On the other hand, Big data applications are requiring more and more computing power. 
As the boundaries between Cloud and HPC continue to blur, it is clear that there is an urgent demand for a systematic computational model that adapts to the computing platform and accommodates the underlying workloads. 

This thesis presents Leaping Shadows as a novel fault-tolerant computational model that unifies HPC and Big Data analytics and scales to future extreme-scale computing systems. The flexibility in the model allows it to embrace different execution strategies in accordance with the underlying workloads, whether it is compute-intensive or data-intensive. Leaping Shadows takes advantage of the unique design in the Shadow Replication model that original main processes are associated with ``lazy" shadows. The differential and dynamic execution rates control enables Leaping Shadows to achieve fault tolerance with power awareness, as well as adaptivity to trade-offs among performance, resilience, and energy costs.  Furthermore, by incorporating creative optimization techniques, Leaping Shadows is able to maintain a consistent level of resilience across high rate and diverse types of failures, with improved performance and reduced resource requirements. 

This thesis systematically studies the viability of Leaping Shadows to enhance system resilience in emerging extreme-scale, failure-prone computing environments. As a first step, customized execution dynamics are designed to deal with different types of failures. Then, analytical models and optimization frameworks are built to derive the optimal process execution rates, while at the same time multiple mechanisms are explored to effectively achieve the desired execution rates. To further verify Leaping Shadows and to validate the analytical models, a prototype implementation is provided in the context of HPC. Empirical evaluation with various benchmark applications confirms that Leaping Shadows is able to outperform state-of-the art fault tolerance approaches. 

The study of the Leaping Shadows model in this thesis is not meant to be complete. The flexibility and diversity in the model point to many future directions. In current design of Leaping Shadows, each main is associated with the same number of shadows. This is ignorant of the variance in the underlying hardware reliability and above application criticality. 
Previous studies have shown that failure rates both vary across systems and vary from node to node within the same system~\cite{schroeder2007,di2014lessons}. According to \cite{di2014lessons}, 19\% of the nodes account for 92\% of the machine check errors on Blue Waters. %The reason for the non-uniform distribution of failure is complicated and may attribute to the manufacture process, heterogeneous architecture, environment factors (e.g. temperature, voltage supply), and/or workloads. 
At the same time, within a system processes may have different criticality. %One possible reason is that the execution model assigns different roles to different processes. 
For example, in the master-slave execution model the master process is a single point of failure, making the failure of the master process much more severe than that of a slave process. %Another possibility is when the user has the option to specify the QoS. For example, in Cloud Computing, users may choose different level of QoS in terms of Service Level Agreement with different amount of payment. 
In fact, \textit{heterogeneous shadowing} techniques can be explored to dynamically harness all available resources to achieve the highest level of QoS. 
%Firstly, in a system where CPU cores have different propensity to failures, mapping from processes to physical cores will largely impact the successful execution of each process. Secondly, Lazy Shadowing allows different tasks to use different number of shadows. 
Within the resource limitation, more shadows would be allocated for more critical mains or mains that are more likely to fail. 

Failure-induced leaping has proven to be critical in reducing the divergence between a main and its shadow, thus reducing the recovery time for subsequent failures. Consequently, the time to recover from a failure increases with failure intervals. Based on this observation, a proactive approach is to ``force" leaping when the divergence between a main and its shadow exceeds a specified threshold. This is analogous to checkpoint/restart in that checkpoints are periodically taken to minimize the cost of lost work due to a failure. Thus, it is worth studying this approach to determine what behavior triggers forced leaping in order to optimize the average recovery time.

Another future direction is topology-aware process mapping~\cite{von2012topology}. Process mapping is of vital importance in Leaping Shadows, since it not only determines the failure isolation degree, but also impacts communication performance. For the main and shadow(s) of the same task, we would like to place them far away so that they are unlikely to be victims of a single failure. On the other hand, placing mains and shadows close to each other tends to minimize message forwarding cost, especially under the receiver-forwarding protocol. Therefore, a process mapping algorithm needs to be developed to balance the trade-offs, while considering the interconnect topology. 

In the extreme cases where hardware resources are quite limited, there may be no redundant hardware to support the execution of the shadows. If this is the case, one might still apply Leaping Shadows with every main collocated with a shadow, which is associated with a different main. Furthermore, to prevent shadows from taking too much resources and extensively slowing down the mains, shadows may only be allowed to ``steal" CPU cycles when mains are blocked. It is expected that Leaping Shadows with such collocation should be able to achieve fault tolerance with comparable performance under the given limitation on resources. However, it remains a question whether there is an efficient mechanism to precisely control the CPU sharing while ensuring performance isolation. Also, process mapping is an important problem to study.

Lastly, the idea of approximate computing can be applied to shadows. Instead of having shadows as exact replicas of the mains, one can assign a reduced version of the computation to the shadows or let them process a portion of the input data. Many workloads today, such as HPC simulation and large-scale machine learning, often have the flexibility in tuning the fidelity of their results, such as the granularity of a simulation or the precision of convergence.
Energy and performance gains may be achieved, when relaxing the precision constraints in the case of a failure.














\appendix                      
%After this command, chapters will be formatted as appendices. For example
%\chapter{Possible Appendix}
%\label{appendix:effectEventInferenceRules}
%\input{appendixEffectEventInference}

%\chapter*{BIB}
\begin{singlespace}
%\nocite{*}
%\renewcommand{\bibsection}{} %remove automatically generated title, like "REFERENCE" or "BIBLIGROPHY"
\bibliographystyle{plainnat}
{
\footnotesize
\bibliography{references/refresh,references/twr,references/scaling,references/jedec,references/dram,references/approx_security,references/stacked,references/page_os,references/hpca16}
}
\end{singlespace}
%\bibliography{gfbf,sentiment,agreementStudy,effectEventCorpus,effectEventInference,generalEventCorpus,generalEventInference,myWork}          %\safebibliography is used the same way as \bibliography, but gives pittetd
%                                   a greater chance to succeed in formatting the bibliography when nonstandard
%                                   BibTeX styles are used.
\end{document}
