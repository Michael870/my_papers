With concerted efforts from researchers in hardware, software, algorithm, and data management, HPC is moving towards extreme-scale, featuring a computing capability of quintillion ($10^{18}$) FLOPS. 
As we approach the new era of computing, however, several daunting scalability challenges remain to be conquered. Delivering extreme-scale performance will require a computing platform that supports billion-way parallelism, necessitating a dramatic increase in the number of computing, storage, and networking components. At such a large scale, failure would become a norm rather than an exception, driving the system to significantly lower efficiency with unprecedented amount of power consumption. %The frequency and diversity of failures,  as well as the challenge of power, call for rethinking of the fault tolerance problem. 

To tackle these challenges, we propose an adaptive and power-aware algorithm, referred to as Lazy Shadowing, as an efficient and scalable approach to achieve high-levels of resilience, through forward progress, in extreme-scale, failure-prone computing environments. 
Lazy Shadowing associates with each process a ``shadow" (process) that executes at a reduced rate, and opportunistically rolls forward each shadow to catch up with its leading process during failure recovery.
%overlaps the recovery time after each failure with the time needed to roll forward the shadows to a consistent state.
Compared to existing fault tolerance methods, our approach can achieve 20\% energy saving with potential reduction in solution time at scale.
