Energy efficiency is an important design constraint for real-time systems. In large systems. energy costs account for a large portion of the operation expenses. In small systems, such as embeded real-time systems, energy efficiency is extremely important because of the limited battery life. Besides, higher energy consumption has an adverse effect on the environment that attracts more and more attention. 
A popular technique for energy management is Dynamic Voltage and Frequency Scaling (DVFS). With DVFS, CPU frequency can be reduced by decreasing the voltage supply, which also reduces the power consumption. 

At the same time, real-time systems are often deployed for mission-critical tasks, where reliability is a must. Due to electromagnetic interference or cosmic radiation, however, computer systems are susceptible to failures. Therefore, being able to tolerate failures is of extreme importance for real-time critical tasks.  
With the emergence of multi-core platform, using task replication to improve reliability and tolerate failures becomes a viable solution. Recent study shows that using only two replicas of each task will dramatically improve reliability~\cite{casanova2012combining}.


Energy efficiency and fault tolerance, however, are two conflicting requirements. Several studies show that DVFS leads to higher failure rate, thereby reducing the reliability of the system~\cite{1382539,6604518}. On the other hand, running multiple replicas of each task results in higher energy consumption. Therefore, it is challenging to achieve both energy efficiency and fault tolerance simultaneously. 

The main objective of this paper is a reliability-oriented energy optimization framework for periodic real-time tasks. Given a real-time task, we study the problem of how to achieve minimal energy consumption while completing the task within its deadline with certain reliability. Three replication techniques are considered in this paper. We use analytical models to study the performance of each replication technique, and an optimization problem is formulated to derive the optimal solution. Furthermore, we build an event-driven simulator to verify the results from our analytical model. 
