As our reliance on information technology continues to increase, the
complexity and urgency of the problems our society will face in the
future will increase much faster than are our abilities to understand
and deal with them. It is expected that in the future, increasingly
more complex applications will require very high computing performance
and will process massive volumes of data 
\cite{expanding_universe_gantz_2008}. Addressing these challenges
requires radical changes in the way computing is delivered to support
the computing requirements and characteristics of these applications
\cite{sachs_ascr_2011}. New algorithms and programming models must be developed to enable
significantly higher levels of parallelism. It is expected that the
number of concurrent threads to sustain these required levels of
parallelism will rise to a billion, a factor of 10,000 greater than
what current platforms can support. This in turn will result in a
massive increase in the number of computing cores, memory modules and
storage components of these systems

A direct implication of these emerging trends is the need to address
the difficult challenge of minimizing power consumption. Furthermore,
the fault rates are expected to dramatically increase, possibly by
several orders of magnitude \cite{srinivasan_dsn_2004,
torrellas_extreme_2009}. Figure \ref{sandia_system_mtbf} shows the
system mean time between failures and the number of faults as a
function of the number of nodes in the
system \cite{riesen_sandia_2010}. These faults, which can be
transient, temporary, intermittent or permanent, stem from different
causes and produce different effects. Concern about the increase of
fault rates will not only be caused by the explosive growth in the
number of computing and storage components, but will also grow out of
the necessity to use advanced technology, at lower voltage levels, and
deal with undesirable aging effects as they become significant
\cite{srinivasan_dsn_2004}. Addressing this concern brings about unprecedented resiliency
challenges, which puts in question the ability of next generation
cloud computing infrastructure to continue operation in the presence
of faults without compromising the requirements of the supported
applications.

The current response to faults in existing systems consists in
restarting the execution of the application, including those
components of its software environment that have been affected by the
occurring fault. To avoid the full re-execution of the failing
application, however, fault-tolerant techniques typically checkpoint
the execution periodically; upon the occurrence of a hardware or
software failure, recovery is achieved by restarting the computation
from a safe checkpoint.  Note that, in some situations, several
components of the software environment associated with the failed
application may have to be restarted.

Given the anticipated failure rate in high-performance, large-scale
compute- and data-intensive environments, it is very likely that the
time required to periodically checkpoint an application and restart it
upon failure may exceed the mean time between failures.  Consequently,
applications may achieve very little computing progress, thereby
reducing considerably the overall performance of the system.  The
results of a study carried out at Sandia National Laboratories to
evaluate the overhead incurred by checkpointing, depicted in
Figure \ref{sandia_percent_of_time}, clearly show that beyond 50,000
nodes the application spends only a fraction of the elapsed time
performing useful computation.

\begin{figure}[hHtb]
\centering
\psfig{figure=diagrams/sandia_percent_of_time_for_checkpoints.eps,width=3.0in}
\caption { Percent of time used to perform checkpoints as the number of nodes increase. }
\label{sandia_percent_of_time}
\end{figure}

\begin{figure}[hHtb]
\centering
\psfig{figure=diagrams/sandia_system_failure_rate_increase_nodes.eps,width=3.0in}
\caption { Effect on system MTBF as number of nodes increase. }
\label{sandia_system_mtbf}
\end{figure}

Current fault-tolerant frameworks are typically designed to handle
single errors, whereas computation in large-scale, high-performance
computing environments is likely to face multiple different types of
errors, often concurrently. Even in the case of single errors, current
fault-tolerant approaches apply the same technique, mainly
checkpoint-and-restart over the entire duration of the execution, to
handle all types of faults, including permanent node crashes,
transient computing errors, and input-out device failures. The nature
of errors in large-scale, high-performance computing environments,
however, are such that a general and expensive fault-tolerance
technique, such as checkpoint-restart, may not be an adequate approach
to handle the diverse types of faults in these environments.

The main objective of this paper is to explore radically different
paradigms to achieve scalable resiliency in future large-scale,
high-performance cloud computing infrastructure. To this end, we
propose a new energy-aware ``shadow computing'' scheme, as an efficient
and scalable alternative to checkpointing and rollback recovery based
techniques.

The basic idea the shadow computing computation model is to associate
with each process a suite of ``shadow processes'', whose size depends on
the ``criticality'' and performance requirements of the underlying
application. A shadow process is a replicate of the main process. To
overcome failure while minimizing energy consumption, the shadow runs
concurrently with the main process, but at different computing node
and at a reduced processor speed. The successful completion of the
main process results in the immediate termination of the shadow
process. In case the main process fails, however, two actions are
simultaneously undertaken by the system. First, the shadow process
immediately takes over the role of the main process and resumes
computation at increased speed, without disturbing other related
processes. Second, a new shadow process is initiated in anticipation
of future failures. It is worth noting, that failure of the shadow
process does not impact the behavior of the main process and simply
results in the activation of a new shadow process to replace the
failing one.

In order to fully harness the potential of shadow computing to
efficiently deal with failures, an optimization model is proposed to
derive the execution speed of the main process and the prior- and
post-failure execution speeds of the shadow process. The derived
execution speeds are such that the energy consumption is minimized,
without violating the performance requirements of the underlying
application.  It is worth noting that the interplay between resiliency
and power management manifests itself in different ways and must be
analyzed carefully. Operating at lower voltage thresholds, for
example, reduces power consumption but have adverse impact on the
resiliency of the system to handle high error rates in a timely and
reliable fashion. Our approach will seek to avoid continuous change in
voltage and frequency to prevent potential thermal and mechanical
stresses on the electronic chips and board-level electrical
connections.

The reminder of the paper is organized as follows: Section II work
related to fault-tolerance in large-scale high-performance computing
systems. Section III presents the energy optimization model used to
derive the different execution speeds of the main process and its
shadow. Section IV presents the energy optimal shadow computing
model. Section V discusses the results of a comparative analysis of
the optimal energy-aware shadow computing scheme to other
fault-tolerant schemes.  Section VI presents the conclusion of this
work and discusses future work.


%%---
%%
%%To enable future scientific breakthroughs and discoveries, the
%%next-generation of scientific applications will require exascale
%%computing performance to support the execution of predictive models
%%and the analysis of massive quantities of data, with orders of
%%magnitude higher resolution and fidelity than what is possible in
%%existing computing infrastructure \cite{sachs_ascr_2011,
%%doe_exascale_2010}. In order to deliver exascale computing and
%%effectively harness its capabilities, several daunting scalability
%%challenges must be addressed. In the late 90's, terascale performance
%%was achieved with fewer than 10,000 single-core processors. A decade
%%later, petascale performance required about 10 times as many
%%processors as terascale performance. If such trends continue,
%%delivering exascale computing will require a million processors, each
%%supporting 1000 cores, resulting in a billion-core computing
%%infrastructure with a massive increase in the number of memory
%%modules, communications devices and storage components. Beyond raw
%%computing, communications and storage requirements, future exascale
%%computing systems are faced with unprecedented energy and resiliency
%%challenges.
%%
%%Power consumption is widely recognized as one of the most significant
%%challenges facing exascale computing
%%\cite{doe_exascale_2010,sachs_ascr_2011}. The expected energy
%%consumption increase in exascale computing is staggering. If current
%%trends hold true, an exascale computing system is expected to consume
%%over a gigawatt of power. This represents a 10 fold increase in the
%%energy consumption of today's largest data centers, which typically
%%ranges between 100 and 200 Megawatts. Reducing the power consumption
%%of an exascale computing system to the DoE's target of 20 Megawatts is
%%undoubtedly a formidable challenge, with a pervasive effect on next
%%generation scientific applications. Addressing such a challenge
%%requires building power and energy awareness into the foundations of
%%future exascale computing infrastructure, with "performance per watt"
%%as the metric of merit to measure efficiency. Radical approaches must
%%be developed to achieve efficient energy management across all
%%hardware and software components of the system.
%%
%%In addition to its impact on energy consumption, the upward trend in
%%the number of computing nodes also has a direct negative effect on the
%%overall system reliability. Even if the individual node failure rate
%%is low, the overall system failure rate quickly becomes unacceptable
%%as the number of components increases. For example, a computing system
%%with 200,000 nodes will experience a mean time between failure(MTBF)
%%of less than one hour, even when the MTBF of an individual node is as
%%large as 5 years \cite{riesen_sandia_2010}. The dramatic decrease in
%%system reliability as the number of computing nodes increases is
%%depicted in Figure \ref{sandia_system_mtbf}
%%\cite{riesen_sandia_2010}. Other factors are also expected to increase
%%failure rates in exascale computing, including the expected high fault rates of
%%advanced-technology computing components operating at lower voltage
%%levels and the impact of undesirable aging effects as they become
%%significant\cite{srinivasan_dsn_2004}. Addressing these concerns
%%brings about unprecedented resiliency challenges, which puts in
%%question the ability of next generation high performance computing to
%%continue operation in the presence of faults without compromising the
%%requirements of the supported applications.
%%
%%\begin{figure}[hHtb]
%%\centering
%%\psfig{figure=diagrams/sandia_system_failure_rate_increase_nodes.eps,width=3.0in}
%%\caption { Effect on system MTBF as number of nodes increase. }
%%\label{sandia_system_mtbf}
%%\end{figure}
%%
%%The current response to faults consists of restarting the execution of
%%the application, including those components of its software
%%environment that have been affected by the occurring fault. To avoid
%%the full re-execution of the failing application, fault-tolerant
%%techniques typically checkpoint the execution periodically; upon the
%%occurrence of a hardware or software failure, recovery is achieved by
%%restarting the computation from a safe checkpoint. In some situations,
%%however, several components of the software environment associated
%%with the failed application may have to be restarted.
%%
%%Given the anticipated increase in failure rate and the time required
%%to checkpoint large-scale compute- and data-intensive applications, it
%%is very likely that the time required to periodically checkpoint an
%%application and restart it upon failure may exceed the mean time
%%between failures.  Consequently, applications may achieve very little
%%computing progress, thereby reducing considerably the overall
%%performance of the system.  For example, a study carried out at Sandia
%%National Laboratories focused on evaluating the overhead incurred by
%%checkpointing in exascale computing environments. The results of the
%%study, depicted in Figure \ref{sandia_checkpoint_time}, clearly show
%%that beyond 50,000 nodes the application spends only a fraction of the
%%elapsed time performing useful computation.
%%
%%\begin{figure}[hHtb]
%%\centering
%%\psfig{figure=diagrams/sandia_percent_of_time_for_checkpoints.eps,width=3.0in}
%%\caption { Percent of time to perform checkpoints. }
%%\label{sandia_checkpoint_time}
%%\end{figure}
%%
%%
%%The main objective of this paper is to explore radically different
%%paradigms to enable scalable resiliency with minimum energy
%%consumption in the future exascale computing infrastructure. To this
%%end, we propose a new energy-aware ``shadow computing'' model, as the
%%basis for an efficient and scalable computational framework to achieve
%%desired levels of fault tolerance, while minimizing energy
%%consumption. The proposed model goes beyond traditional check-pointing
%%and roll-back recovery techniques, and uses a multi-level,
%%energy-aware replication approach to achieve scalable fault tolerance
%%in exascale computing.
%%
%%The basic idea of the shadow computing model is to associate with each
%%process a suite of ``shadow processes'', whose size depends on the
%%``criticality'' and performance requirements of the underlying
%%application. A shadow is an exact replica of the main process. In
%%order to overcome failure, the shadow is scheduled to execute
%%concurrently with the main process, but at a different computing
%%node. Furthermore, in order to minimize energy, shadow processes
%%initially execute at decreasingly lower processor speeds. The
%%successful completion of the main process results in the immediate
%%termination of all shadow processes. If the main process fails, however,
%%the primary shadow process immediately takes over the role of the main
%%process and resumes computation, possibly at an increased speed, in
%%order to complete the task. Moreover, one among the remaining shadow processes
%%becomes the primary shadow process.
%%
%%It is worth noting that, since the failure of an individual component
%%is much lower than the aggregate system failure, it is very likely
%%that most of the time the main processes complete their execution
%%successfully. Successful completion of a main process automatically
%%results in the immediate halting of its associated shadow processes,
%%with a significant saving in energy consumption. Furthermore, the number
%%of shadow processes to be instantiated in order to achieve the desired level
%%of fault-tolerance must be determined based on the likelihood that
%%more one process failure within the execution time interval of the main
%%task.
%%
%%The main challenge in realizing the
%%potential of the shadow computing model stems from the need to compute
%%the speed of execution of the main process and the speed of execution
%%of its associated shadows, both before and after a failure occurs, so
%%that the target response time is met, while minimizing energy
%%consumption. Given the nature of failure in exascale
%%computing, it is unlikely that both the main and its primary shadow
%%fail simultaneously. Therefore, only a dual level of redundancy,
%%whereby only one shadow process is executed concurrently with the main
%%process, is needed in most cases to achieve the desired level of
%%fault-tolerance. The completion of a main or its shadow results in the
%%successful execution of the underlying task.
%%
%%The main contributions of this paper are threefold. First, we present
%%an optimization framework to explore the applicability of the shadow
%%computing model to support fault-tolerance in a high-performance
%%computing environment, where the main computation is expected to
%%execute at maximum speed in order to harness the full potential of the
%%computing infrastructure. The model assumes that the desired level of
%%fault tolerance can be achieved with the instantiation of a single
%%shadow process, although the model can be extended to support multiple
%%shadow processes.  
%%
%%Second, using the optimization framework, we propose and study three
%%implementation methods of the shadow computing model. The first
%%method, referred to as ``stretched'' replication, takes a
%%straight-forward approach to minimizing energy by computing a uniform
%%speed of execution across the execution interval.  Although simple to
%%implement, stretched replication is oblivious to the dynamics of the
%%failure, resulting in sub-optimal energy performance.  The second
%%method, referred to as minimum-work replication, takes into
%%consideration the minimum time between failures for a single node is
%%likely to exceed the execution time of the task. Based on this
%%observation, the method computes a near-optimal execution speed of the
%%shadow. The third method, referred to as optimal energy replication,
%%addresses shortcomings in these two models and uses an optimal
%%energy-consumption approach to derive the shadow's minimum-energy
%%speed, both before and after failure, in order to meet the expected
%%response time of the underlying application regardless of failure
%%rates. It is worth noting that the implementation complexity of the
%%optimal approach does not increase significantly in comparison with
%%the stretched and minimum-work replication schemes.
%%
%%Third, we propose a performance evaluation framework to assess the
%%performance of the proposed methods for different workload
%%characteristics and performance requirements of the main
%%task. Throughout the study, the performance of the proposed shadow
%%computing implementation methods are compared to that of the ``pure''
%%replication scheme described in
%%\cite{ferreira_hpc_2011}. The major difference between the methods
%%proposed in this paper and current replication schemes is that shadow
%%computing does not force the process replica to execute at the same
%%speed as the main process. As the results of the performance study
%%show, relaxing this restriction allows shadow computing to save up to
%%50\% of the expected energy consumed by pure replication.
%%
%%The remainder of the paper is organized as follows: Section
%%\ref{related_work} reviews work related to fault-tolerance in
%%large-scale high-performance computing systems. Section
%%\ref{shadow_model} presents the shadow computing model and discusses
%%the building components of its execution model. Section \ref{model}
%%the shadow computing model is cast as an energy optimization problem
%%to achieve fault-tolerance, while minimizing energy consumption
%%without violating the expected performance requirements of the
%%supported application. In Is Section 5 three different approaches to
%%solving the energy optimization problem and their potential
%%implementation are explored. The methods differ in their approach to
%%energy minimization and their reaction to failures. In Section
%%\ref{analysis}, we develop a performance evaluation framework to
%%analyze and assess the performance of the three shadow computing
%%methods, including their sensitivity to critical workload and
%%performance parameters.  Throughout the study, the performance of the
%%three methods is compared to pure replication, a recently proposed
%%alternative to check-pointing in exascale computing.  Section
%%\ref{conclusion} presents the conclusion of this work and discusses
%%future work.
