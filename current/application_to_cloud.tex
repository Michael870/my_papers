Cloud computing providers seek to find a balance between the execution
speed and fault resiliency of a given configuration in order to
maximize their profit. ``Shadow computing'' provides a mechanism to
make this tradeoff by optimizing the energy consumption of fault
tolerance given a task's targeted response time. If the main process
fails it is assumed that the task has some laxity as to when it will
complete. The amount of laxity is bounded by the task's targeted
response time, which is the time at which the task must be completed
regardless of failure. The targeted response time is typically
represented as a laxity factor, $\alpha$, of the minimum response
time. For example if the minimum response time is 100 seconds and the
targeted response time is 125 seconds, the laxity factor is 1.25.

The remainder of this section presents a solution to finding
$\sigma_m$, $\sigma_b$ and $\sigma_a$ for energy optimal
replication. We start by specifying our failure and power model then
show how we use numerical analysis techniques to find optimal values.

\subsection{Failure Probability}

In our energy model we assume failure is described using any
probability density function. In the remaining sections we use the
exponential probability density function because it is widely accepted
as a model representative of independent node failures, therefore
$f(t)=e^{-\lambda t}$. This distribution also has the benefit of being
differentiable allowing us to produce a closed form equation.

\subsection{Energy Function}

In the remainder of this paper we need to select values for our power
model. In section \ref{power_model} we defined the energy function,
$E(\sigma, T)$, as the integral of the power function over the
interval T. In the remaining sections we assume that the power
function is defined as the squared value of the speed.
\begin{equation}
P(\sigma)=\sigma^2
\end{equation}
Thus the energy function is defined as the following:
\begin{equation}
E(\sigma,T)=\int_{t=0}^T \sigma^2 dt = \sigma^2t
\end{equation}

\subsection{Optimal Execution Speeds}
\label{finding_execution_speeds}

Once we have a known failure and power model we can begin to solve the
optimization problem. We first make the observation that the speed of
the shadow after failure, $\sigma_a$, is dependent upon the the shadow
speed before failure, $\sigma_b$, and the time of failure, $t_f$. It
can trivially be shown that to conserve the most energy one would let
$\sigma_a$ be the slowest possible speed to finish by the targeted
response time, R. Therefore $\sigma_a$ is no longer constant with
respect to the time of failure. From this observation the following
value of $\sigma_a$ can be derived.
\begin{equation}
\label{optimal_sigma_a}
\sigma_a=(W-\sigma_b*t_f)/(R-t_f)
\end{equation}
Substituting this value into the energy model allows us to reduce the
number of variables in our objective function, specifically this
reduces the output of our optimization to two variables, $\sigma_m$
and $\sigma_b$.

In addition to minimizing the objective energy function we also must
define several constraint functions for our minimization
problem. These constraints ensure that our model obeys the system
limitations such as maximum speed but also ensure that the work is
completed by the targeted response time.  The first set of constraints
simply bound the values of $\sigma_b$ and $\sigma_m$,
$0\leq\sigma_b\leq\sigma_{max}$ and $0\leq\sigma_m\leq\sigma_{max}$


Next we bound $\sigma_m$ such that the main process will finish at or
before the targeted response time $R$, 
\begin{equation}
\sigma_m≥W/R
\end{equation}

The one constraint we have not considered in our optimization is that
if the main process fails the shadow process must be able to complete
the given work, $W$, by the targeted response time, $R$. This is known
as the ``work constraint'' and is represented by the following
inequality.
\begin{equation}
\label{work_constraint}
t_c*\sigma_b+(R-t_c)*\sigma_{max} \geq W 
\end{equation}
The intuition for this constraint is that in the worst case the shadow
will have to execute at the maximum possible speed after failure to
achieve the targeted response time. This enforces the constraint such
that if the main process fails at the very last time point, $t_c$,
then the shadow process will still be able to complete the work by the
targeted response time. This places a lower bound on the value for
$\sigma_b$ and as we will later show typically determines the value of
$\sigma_b$.

\subsection{Optimal Shadow Computing - Solution}
\label{closed_form}

In the preceding sections we have specified components of our general
energy model found in Equation \ref{energy_model}. Our optimization
problem is now well defined as finding a value for $\sigma_m$ and
$\sigma_b$ that minimizes the following function.
\begin{equation}
\label{energy_model_sub_sigma_a}
\begin{split}
 & \int_{t=0}^{t_c}(\sigma_m^2t+\sigma_b^2t) e^{-\lambda t}dt \\
+& \int_{t=0}^{t_c}{[(W-\sigma_b*t_f)/(R-t_f)]}^2 (R-t) e^{-\lambda t}dt \\
+& (1-\int_{t=0}^{t_c} e^{-\lambda t} dt)(\sigma_m^2 t_c + \sigma_b^2 t_c)
\end{split}
\end{equation}
Also note that $t_c=W\sigma_m$ because the amount of work, $W$, is
given and main process execution speed is defined as $\sigma_m$. After
solving equation
\ref{energy_model_sub_sigma_a} we use numerical analysis techniques for
 finding the optimal execution speeds, $\sigma_m$ and $\sigma_b$.

Our optimization outputs the main processes execution speed,
$\sigma_m$, and the shadows execution speed before failure,
$\sigma_b$, given the following inputs:
\begin{itemize}
\item
W - The amount of work necessary to complete the task.
\item
R - The targeted response time for the task.
\item
$\sigma_{max}$ - The maximum possible execution speed.
\item
$\lambda$ - Parameter of our probability density function used to model failure.
\end{itemize}

% LocalWords: mtbf ei dt
