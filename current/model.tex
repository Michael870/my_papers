As stated previously, the basic idea of shadow computing is to
associate a number of ``shadow processes'' with each main process. The
main responsibility of a shadow process is to take over the
responsibility of a failed main process and bring the computation to a
successful completion.  In this section, we define a framework for
evaluating shadow computing and then use this to derive a model for
representing the expected energy consumed by the system. We then
describe in terms of this model three different methods for applying
shadow computing in a high performance computing environment.

%%model that describes and show how it can be used to derive the speed
%%of execution of the shadow process that minimizes energy
%%consumption. Without loss of generality, in this section, we focus on
%%the main process and its first shadow. The model can be easily
%%extended to deal with multiple shadows.

\subsection{Shadow Computing Framework}
\label{shadow_computing_framework}

We consider a distributed computing environment executing an
application carried out by a large number of collaborative tasks. The
successful execution of the application depends on the successful
completion of all of these tasks. Therefore the failure of a single
process delays the entire application, increasing the need for fault
tolerance. Each task must complete a specified amount of work, $W$, by
a targeted response time, $R$. The amount of work is expressed in
terms of the number of cycles required to complete the task. Each
computing node has a variable speed, $\sigma$, given in cycles per
second and bounded such that $0\leq\sigma\leq\sigma_{max}$. Therefore
the minimum response time for a given task is $R_{min}=\sigma_{max}*W$.

In order to achieve our desired fault tolerance a shadow process
executes in parallel with the main process on a different computing
node. The main process executes at a single execution speed denoted as
$\sigma_m$. In contrast the shadow process executes at two different
speeds, a speed before failure detection, $\sigma_b$, and a speed
after failure detection, $\sigma_a$. This is depicted in Figure
\ref{shadow_overview}.

\begin{figure}[hHtb]
\centering
\psfig{figure=diagrams/shadow_main_diagram.eps,width=3.5in}
\caption { Overview of Shadow Computing }
\label{shadow_overview}
\end{figure}

Based upon this framework we define some specific time points
signaling system events. The time at which the main process completes
a task, $t_c$, is given as $t_c=W/\sigma_m$. The time at which the
shadow process completes as task, $t_r$, is given as $t_r =(W-\sigma_b
t_c)/\sigma_a$ related but not necessarily equal is $t_R$ which is the
time the system reaches the targeted response time for a given
task. Additionally, we define the time point $t_f$ as the time at which
a failure in the main process is detected.

Using this framework we formalize our objective as the following
minimization problem.
%%% need more horizontal spacing here... not sure the latex-fu
\begin{equation}
\label{optimization_problem}
\begin{split}
\text{minimize  }   & E(\sigma_m,t_0,t_f,t_c) + E(\sigma_b,t_0,t_f,t_c) + E(\sigma_a,t_f,t_r) \\
\text{subject to  } & t_c \leq t_R \\
                  & t_r \leq t_R \\
                  & \sigma_m t_c \geq W \\
                  & \sigma_b t_c + \sigma_a (t_R - t_f) \geq W
\end{split}
\end{equation}
Here the function $E(\sigma,t_0,t_1,t_2)$ represents the energy
consumed by a process running at speed $\sigma$ during the time period
$t_0$ and $min(t_1,t_2)$. The first two constraints state that both
the main process and the shadow process must complete by the targeted
response time. The last constraints ensure that the amount of work
done by those processes must be greater than or equal to the amount of
work defined by the task.

It should be noted that it is assumed that node failures and task
properties are unchangeable system properties therefore the system
parameters we can change are the execution speeds of the
processes. Thus the output of this optimization problem is the
execution speeds, $\sigma_m$, $\sigma_b$ and $\sigma_a$. In the
proceeding sections we will present a power and failure model for
individual computing nodes then use these to model the expected energy
of the shadow computing system. Then in Section
\ref{application_to_hpc} we then apply this model to the high
performance computing environment.

\subsection{Power Model}
\label{power_model}

It is well known that by varying the execution speed of the computing
nodes one can reduce their power consumption at least quadratically by
reducing their execution speed linearly. The power consumption of a
computing node executing at speed $\sigma$ is given by the function
$p(\sigma)$, represented by a polynomial of at least second degree,
$p(\sigma)=\sigma^n$ where $n\geq2$. The energy consumed by a
computing node executing at speed $\sigma$ during an interval of
length $T$ is given by $E(\sigma,T)=\int_{t=0}^T
p(\sigma)dt$. Throughout this paper we substitute the energy for a
particular time interval, t, with the derived value $p(\sigma)t$,
because $p(\sigma)$ is treated as a constant with respect to time. We
further assume that the computing node speed is bounded by the
following equation $0\leq\sigma\leq\sigma_{max}$.

\subsection{Failure Model}
\label{failure_model}

The failure can occur at any point during the execution of the main
task and the completed work is unrecoverable. Because the processes
are executing on different computing nodes we assume failures are
independent events. We also assume that only a single failure can
occur during the execution of a task. If the main task fails it is
therefore implied that the shadow will complete without failure. We
can make this assumption because we know the failure of any one node
is a rare event thus the failure of any two specific nodes is very
unlikely. In order to achieve higher resiliency one
would make use of multiple shadow processes and this failure model
will still be valid.

We assume that a probability density function, $f(t)$ ($\int_0^\infty
f(t)dt=1$), exists which expresses the probability of the main task
failing at time $t$. It is worth noting, that the model does not
depend on any specific distribution.

\subsection{Energy Model}
\label{energy_model}

Given the power model and the failure distribution, the expected
energy consumed by a shadow computing task can be derived. We start by
considering the expected energy consumed by the main process and
derive the following equation:
\begin{equation}
\label{energy_for_main}
\int_{t=0}^{t_c}E(\sigma_m,t)f(t)dt + (1-\int_{t=0}^{t_c}f(t)dt)E(\sigma_m,t_c)
\end{equation}
This first term of the equation represents the expected amount of
energy consumed by the main process if a failure occurs, while the
second term represents the expected energy consumed if no failure
occurs.

Similarly, we can calculate the expected energy consumed by the shadow
process, as follows:
\begin{equation}
\label{energy_for_shadow}
\begin{split}
&  \int_{t=0}^{t_c}E(\sigma_b,t)f(t)dt \\
+& \int_{t=0}^{t_c}E(\sigma_a,(t_r-t))f(t)dt \\
+& (1-\int_{t=0}^{t_c}f(t)dt)E(\sigma_b,t_c)
\end{split}
\end{equation}
The first term represents the expected energy consumed by the shadow
executing at $\sigma_b$ up until the main process fails. The middle
term represents the expected energy consumed when the main process
fails and the shadow begins to execute at the speed, $\sigma_a$. The
last term is the expected energy consumed in the event that no failure
occurs and the shadow executes at $\sigma_b$ the entire duration of
the main process.

The total energy consumed by a shadow computing task is the summation
of the energy consumed by the main process and shadow process. To
expand this model to represent multiple shadows we would multiple the
energy consumed by the shadow by the total number of shadow
processes. Given one shadow process we can combine equations
(\ref{energy_for_main}) and (\ref{energy_for_shadow}) to produce the
single model representing the total expected energy consumed.
\begin{equation}
\label{energy_model}
\begin{split}
 & \int_{t=0}^{t_c}(E(\sigma_m,t)+E(\sigma_b,t))f(t)dt \\
+& \int_{t=0}^{t_c}E(\sigma_a,(t_r-t))f(t)dt \\
+& (1-\int_{t=0}^{t_c}f(t)dt)(E(\sigma_m,t_c )+E(\sigma_b,t_c ))
\end{split}
\end{equation}

% LocalWords: hpc
