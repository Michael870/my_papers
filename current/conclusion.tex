
In this paper we have introduced shadow computing, an energy efficient
method to provide fault tolerate execution without the limitations of
checkpointing. We then compared this to other known methods,
replication and re-execution, and concluded that shadow computing is
always more energy efficient. We also observed that the amount of
energy saving is highly dependent upon the rate of failure and the
amount of slack present in the system.

Fully harnessing the potential of shadow computing to deal with
failures brings about several challenging questions that need to be
addressed: How can this concept be used to improve fault detection and
layer coordination, understanding faults and silent errors and
improving situational awareness? What level of synchronization is
required between the main process and its associated shadow processes
to minimize impact on other application processes? What state, if any,
must be saved to ensure “smooth” transition to the primary shadow
process upon failure of the main process?  Future work will be focused
on investigating these questions for different types of failure to
better understand the advantages and limitations of this approach to
achieve high levels of fault-tolerance in extreme scale cloud
computing environments.
