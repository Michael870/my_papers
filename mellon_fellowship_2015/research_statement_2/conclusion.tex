%The main motivation of this work stems from the observation that, as Cloud Computing continues to grow, both the failure rate and energy consumption are highly-likely to increase significantly. Hence, understanding the interplay between fault-tolerance, energy consumption and profit maximization is critical for Cloud Computing. 

My current research is enabling new insights into the multi-faceted and challenging resiliency problem in large-scale Cloud Computing platforms. The goal is to investigate radical approaches to the design of
scalable and energy efficient fault tolerant schemes that go beyond state-of-the-art algorithms. Throughout my design, the interplay between resiliency, performance and energy consumption will be analyzed carefully to determine the required levels of
endurance and redundancy, in order to achieve a desired level of fault tolerance while maintaining a specified level of QoS.


To this end, I propose Shadow Replication as a novel, scalable, and energy-aware fault tolerance framework.
%To assess the performance of the proposed fault-tolerance computational model, an extensive performance evaluation study is carried out. In this study, system properties that affect the profitability of fault tolerance methods, namely failure rate, targeted response time and static power, are identified. Our performance evaluation shows that in all cases, Shadow Replication outperforms existing fault tolerance methods.
My preliminary results predict that Shadow Replication is able to achieve significant energy savings while satisfying QoS requirements. I will continue to explore and optimize Shadow Replication in support of green and sustainable Cloud Computing.