Current results reveal that Shadow Replication is promising for significant energy saving within QoS constraints. The direct benefits include profit gains for cloud service providers and reduced $CO_2$ emission, making Cloud Computing more environment-friendly and more sustainable. Inspired by that, I will be fully committed to solve some challenging questions in the next academic year.
%have several research directions to explore in the future.

The first plan is to further improve the efficiency of Shadow Replication for tightly-coupled jobs. My current design works well for loosely-coupled jobs, such as MapReduce jobs, where synchronization among tasks is minimized. 
 In a tightly-coupled job, however, even a very short recovery time may be amplified by the frequent synchronizations,  
 %the parallel tasks need to synchronize with each other frequently. If one task fails, others have to stop and wait for it to catch up. Shadow Replication is able to reduce the catch-up time, because it runs multiple instances of each task, and if one fails, others can speed up to catch up soon. However, the idle time during the catch-up is still a waste, 
 resulting in a delay in the job completion time. In order to minimize this effect and further improve performance, I plan to explore the potential benefits of a new approach, referred to as ``Leaping Shadows". The idea is to take advantage of the recovery time and align the execution states of the slow shadow processes with their faster main processes to achieve forward progress. Remote Direct Memory Access (RDMA) is a possible way to implement Leaping Shadows, but how to efficiently use it needs further research. I plan to finish this work in three months. 

The next research direction is to evaluate the feasibility and performance of using process collocation in Shadow Replication. My current work assumes Dynamic Voltage and Frequency Scaling (DVFS) in controlling the execution rates. The effectiveness of DVFS, however, may be markedly reduced in computational platforms that exhibit saturation of the processor clock frequencies or large static power consumption. An alternative is to collocate multiple processes on a single computing node, while keeping the node running at the maximum rate. Time sharing can then be used to achieve the desired execution rates.
%Process collocation is an alternative approach to slowing down the process execution rate. By collocating multiple processes on the same computing node, the execution rate of each process is proportional to its time share on the node. For Cloud Computing, process collocation may be a better choice than DVFS as the computing nodes in the cloud datacenters are usually time shared by multiple virtual machines. 
%
The two alternatives are equivalent in terms of completion time, since they have the same effect on the execution rate control. In terms of energy, however, each of them has its own advantage. Process collocation requires less hardware resources and this reduces the energy linearly, while DVFS uses more hardwares but can reduce energy superlinearly. It needs further analysis to determine which alternative consumes less overall energy. Furthermore, more efforts are needed to study the potential issues with process collocation, such as correlated failures and collocation overhead. This will take approximately three months to complete. 

The last and most challenging step is to build a prototype, in order to experimentally evaluate the performance of Shadow Replication using real life applications. This effort includes the design and implementation of a distributed software library that supports the main components of Shadow Replication, including process collocation, required consistency protocols, message logging and message forwarding protocols, and execution state transfer in support of Leaping Shadows. For Shadow Replication to be scalable and efficient, it is necessary to minimize its overhead to the normal execution of the running processes as well as to the operating system. This work will take approximately six months to accomplish. 




%There are two potential ways to tune the execution speed of the processes. One is to use Dynamic Voltage and Frequency Scaling (DVFS) to control the execution frequency of the CPU, and the other is to colocate multiple processes on the same node and execute them in a time sharing manner. So far, I have studied the performance of Lazy Shadowing using DVFS. Exploring Lazy Shadowing with time sharing is still under way. 
