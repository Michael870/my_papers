Current results reveal that Shadow Computing is promising for significant energy saving while satisfying SLA requirements. The direct benefits include profit gains for cloud service providers, and reduced $CO_2$ emission that makes Cloud Computing more environment-friendly and more sustainable. Inspired by that, we have several research directions to explore in the future.

The first plan is to further improve the efficiency of Shadow Replication for tightly-coupled jobs. In a tightly-coupled job, the parallel tasks need to synchronize with each other frequently. If one task fails, others have to stop and wait for it to catch up. Shadow Replication is able to reduce the catch-up time, because it runs multiple instances of each task, and if one fails, others can speed up to catch up soon. However, the idle time during the catch-up is still a waste, resulting in a delay in the job completion time. In order to minimize this effect and further improve performance, we plan to explore a new technique, referred to as ``Shadow Leaping". The idea is to take advantage of the idle time and align the execution state of the slow shadow processes with their faster main processes to achieve forward progress. Remote Direct Memory Access (RDMA) is a potential way to implement Shadow Leaping.

The next research direction is to evaluate the possibility of using process collocation to implement Shadow Replication for Cloud Computing. Process collocation is an alternative approach to slowing down the process execution rate. By collocating multiple processes on the same computing node, the execution rate of each process is proportional to its time share on the node. For Cloud Computing, process collocation may be a better choice than DVFS as the computing nodes in the cloud datacenters are usually time shared by multiple virtual machines. 

The two alternatives are equivalent in terms of completion time, since they have the same effect on the controlling process execution rate. In terms of energy, however, each of them has its advantage. Process collocation requires less hardware resources and this reduce the energy linearly, while DVFS uses more hardwares but can reduce energy superlinearly. It needs further analysis to determine which alternative consumes less overall energy. 

The last step is to build a prototype, in order to experimentally evaluate the performance of Shadow Replication using real life applications. This effort is to design and implement a software library to support the main components of Shadow Replication, including process collocation, required consistency protocols, message logging and message forwarding protocols, and execution state update operations in support of Shadow Leaping. 




%There are two potential ways to tune the execution speed of the processes. One is to use Dynamic Voltage and Frequency Scaling (DVFS) to control the execution frequency of the CPU, and the other is to colocate multiple processes on the same node and execute them in a time sharing manner. So far, we have studied the performance of Lazy Shadowing using DVFS. Exploring Lazy Shadowing with time sharing is still under way. 
