Coupled with the rapid growth of AI/ML applications, HPC is increasingly influencing our daily life. 
At the same time, cloud computing is transforming
HPC by providing ubiquitous access and on-demand scalability in today's digital, data-driven world. 
In this paper, 
we present and validate an optimization to cloud resource management in order to enhance system efficiency with QoS guarantees for HPC workloads. We argue that system admins can
continue to enjoy the elasticity and efficiency benefits of cloud computing while meeting the demand of HPC 
use cases. For those HPC users who are strict about performance consistency but are flexible 
in the trade-off between throughput performance and economic cost, this paper offers a unique 
solution to balancing this trade-off with negotiable service level agreement (SLA) in both public and private cloud. 

This is pioneering work to apply CPU plus memory 
over-commitment to HPC workloads. We carry out comprehensive performance study on its feasibility and limitation. As a first step, we validate up to 4X CPU over-commitment by simultaneously running 
multiple virtual clusters on a single physical cluster. Experiment results demonstrate that system 
throughput can be improved by CPU over-commitment, while resource allocation and performance can be 
precisely controlled by adjusting the hypervisor scheduler parameters. Furthermore, we identify that, although HyperThreading is often disabled in HPC system, system admins should consider 
enabling it if throughput performance is the major goal. Then, we move on to over-commit 
both CPU and memory with the design of three scenarios to cover different workload patterns. These 
scenarios help us to recognize that detrimental behavior (e.g., guest OS failure) could occur when 
applications consume too much active memory and force hypervisor to swap memory to disk. However, 
we also demonstrate that memory over-commitment is practical when the applications take turns to 
stress memory, and we provide guidelines regarding how to avoid hypervisor swapping. Most importantly, 
we prove that QoS guarantee on $99^{th}$ percentile tail latency is achievable for HPC cloud 
when 2X CPU plus memory 
over-commitment is practiced. 

In the future, 
we plan to extend the scope of this work to cover more workloads, such as MPI workloads, and to study disk and network resource over-commitment. In addition, we will investigate into building 
analytical models to predict the trade-off between performance and over-commitment degree based on 
workload characterization~\cite{arlitt1997internet,234855}. 