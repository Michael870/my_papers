%The last study is conducted to capture the impact on the 
%performance of Lazy Shadowing brought by 
%collocation overhead. We re-model the speed of shadows as $\sigma_s^b=\frac{1}{\alpha^{1.5}}$ to simulate the 
%effect of memory thrashing and context switch. 

Lazy Shadowing increases memory requirement\footnote{Note that this problem is not intrinsic to Lazy Shadowing, as in-memory checkpointing also requires extra memory.} when multiple shadows are collocated. Moreover, this may have an impact on the execution rate of the shadows due to cache contention and context switch. 
To capture this effect,  
we re-model the rate of shadows as $\sigma_s^b=\frac{1}{\alpha^{1.5}}$.
Figure~\ref{fig:comp_vary_fail_speed} shows the impact of collocation overhead on expected energy consumption for Lazy Shadowing with $\alpha=5$, with all the values normalized to that of process replication. %The results for other values of $\alpha$ have similar behavior and thus are not shown. 
As expected, energy consumption is penalized because
of slowing down of the shadows. It is surprising, however, that the impact is quite small, with the largest difference being 4.4\%. The reason is that shadow leaping can take advantage of the recovery time after each failure and achieve forward progress for shadow processes that fall behind. 
The results for other values of $\alpha$ have similar behavior. 
When $\alpha=10$, the largest difference further decreases to 2.5\%. 

%As expected, both completion time and energy consumption are penalized because
%of slowing down of the shadows. It is surprising, however, that Lazy Shadowing with $\alpha=3$ is impacted by the 
%most, while when $\alpha=9$, which means collocating more shadows on each shadow core, is slightly influenced. After careful analysis,
%we realize that the reason is $\alpha=3$ had the largest values for completion time and energy consumption. Even the percentage of increase after adding the penalty is the smallest, the absolute increase is still the largest. 
%When $\alpha=9$, Lazy Shadowing can still achieve 15\%-20\% energy saving with less than 7\% increase in completion time. 

\begin{figure}[t]
	\captionsetup{justification=centering}
	\begin{center}
		\includegraphics[width=0.7\columnwidth]{Figures/collocation.pdf}
		%\includegraphics[width=0.7\columnwidth]{Figures/tcollocation}
	\end{center}
	%\vskip -0.22in 
	\caption{Impact of collocation overhead on energy consumption. $W=10^6$ hours, $N=10^6$, $\rho$=0.5, $\alpha$=5.}
	\label{fig:comp_vary_fail_speed}
\end{figure}
