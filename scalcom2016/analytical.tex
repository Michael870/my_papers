Three important metrics for assessing the quality of an application's execution are 1) reliability; 2) completion time; and 3) energy consumption. In the following we develop mathematical models to analyze the expected performance of Lazy Shadowing, as well as prove the bound on performance compared to non-failure case, with the understanding
that process replication is a special case of Lazy Shadowing where $\alpha=1$. 
All the analysis below is under the assumption that there are a total of $N$ cores, and $W$ is the application workload.  
$M$ of the $N$ cores are allocated for main processes, each having a workload of $w=\frac{W}{M}$, and the rest $S$ cores are for collocated shadow processes. For process replication,
$M=S=\frac{N}{2}$, and $w=\frac{2W}{N}$. 


\subsection{Application fatal failure probability}
\label{anal_app_fail}
Application failure, which forces the execution to start over, is inevitable even when every process is replicated. Lazy Shadowing is able to 
tolerate one failure in each shadowed set, and the second failure in any shadowed set implies the need to restarting the execution from the very beginning. However, Lazy Shadowing is orthogonal to checkpointing in 
the sense that we can combine the two, to avoid rolling the execution back to the very beginning when application failure occurs.

Since each process is replicated with a shadow, Lazy Shadowing has the potential to significantly 
increase the Mean Number of Failures To Interrupt (MNFTI), i.e., the average number of core failures until application failure occurs, and Mean Time To Interrupt (MTTI), i.e., the average time elapsed until application failure occurs. 
Therefore, the checkpointing interval should be increased to a large extent when checkpointing is combined with Lazy Shadowing. Furthermore, if the resulted checkpointing interval is 
larger than the completion time of the application, then checkpointing may not be used at
all. 
%Therefore, in this subsection, we study the reliability benefits that Lazy Shadowing could 
%introduce. Specifically, we study the application's MNFTI (and MTTI) with Lazy Shadowing.
In the following, the first question to study is the new MNFTI and MTTI when Lazy Shadowing is used. 


The impact of process replication on MNFTI has been studied in~\cite{casanova_inria_2012}. Our problem
is equivalent with the difference that our work can tolerate one failure in each shadowed 
set while~\cite{casanova_inria_2012} can tolerate one failure in each replica-group, when each process
is replicated once. 
Therefore, we can directly apply the methodology in~\cite{casanova_inria_2012} to our case, and the MNFTI
with Lazy Shadowing for different number of shadowed sets ($S$) is shown in Table~\ref{tbl:mnfti}. 
%The MTTI with Lazy Shadowing is not shown here because it depends not only on the number of cores, but also on the shadowed set size chosen. However, results in \cite{casanova_inria_2012} reflect that the MTTI can be increased to the order of tens of hours from ten minutes (without use of replication) assuming the core level MTTI is 25 years. This 
%confirms our previous prediction that Lazy Shadowing can significantly 
%increase the application's MNFTI and MTTI, and also implies that shadowed set rejuvenation may not be necessary.
Note that when processes are not replicated, every failure would interrupt the application, i.e., MNFTI=1, so MNFTI can be significantly increased by Lazy Shadowing. 
At the same time, it is projected that an extreme-scale application's MTTI can be increased to tens of hours from minutes assuming each core's MTBF is 25 years.

\begin{table}[b!]
	\caption{Application's MNFTI when Lazy Shadowing is used. Results are independent of $\alpha$. }
	\centering
	\small
	\begin{tabular}{|c|c|c|c|c|c|c|c|}
		\hline
		$S$ & $2^{0}$ & $2^{1}$ & $2^{2}$ & $2^{3}$ & $2^{4}$ & $2^{5}$ & $2^{6}$ \\
		\hline
		MNFTI & 3.0 & 3.7 & 4.7 & 6.1 & 8.1 & 11.1 & 15.2\\
		\hline\hline
		$S$ & $2^{7}$ & $2^{8}$ & $2^{9}$ & $2^{10}$ & $2^{11}$ & $2^{12}$ & $2^{13}$ \\
		\hline
		MNFTI & 21.1 & 29.4 & 41.1 & 57.7 & 81.2 & 114.4 & 161.4 \\
		\hline\hline
		$S$ & $2^{14}$ & $2^{15}$ & $2^{16}$ & $2^{17}$ & $2^{18}$ & $2^{19}$ & $2^{20}$ \\
		\hline
		MNFTI & 227.9 & 321.8 & 454.7 & 642.7 & 908.5 & 1284.4 & 1816.0 \\
		\hline
	\end{tabular}
	\label{tbl:mnfti}
\end{table}


Even though the above results imply that checkpointing may not be necessary when Lazy Shadowing is used, it is important to quantify the probability that an application failure would occur during the application's execution, defined as ``application failure probability". Let $f(t)$ denotes the failure probability density function of each core, and $F(t)$ be the corresponding cumulative distribution function, i.e., $F(t) = \int_0^tf(\tau)d\tau$ is the probability that a core fails in the next $t$ time. 
Since each shadowed set can tolerate one failure, 
then the probability that a shadowed set with $\alpha$ main cores and 1 shadow core does not fail by time $t$ is the probability of no failure plus the probability of one failure, i.e., 
%\begin{equation}
%	G(t, \alpha) = \Big(1-F(t)\Big)^{\alpha+1} + {{\alpha+1} \choose 1}F(t)\times \Big(1-F(t)\Big)^{\alpha}
%\end{equation}
\begin{equation}
	P_g = \Big(1-F(t)\Big)^{\alpha+1} + {{\alpha+1} \choose 1}F(t)\times \Big(1-F(t)\Big)^{\alpha}
\end{equation}
and the probability that the application using $N$ cores fails within $t$ time is the complement of the probability that
none of the shadowed sets fails, i.e.,
%\begin{equation}
%	R(t, N, \alpha) = 1 - \Big(G(t, \alpha)\Big)^{\frac{N}{\alpha+1}}
%\end{equation}
\begin{equation}
	P_a = 1 - ({P_g})^{S}
\end{equation}
where $S=\frac{N}{\alpha+1}$ is the number of shadowed sets.

Since the application only fails during its execution, we can calculate the application failure probability using $t$ equal to the expected completion time of the application. We will develop the model for the expected completion time in the next subsection.
%that $T_c$ is a function of $W$, $N$, $\alpha$. As a result, the application failure probability can be expressed as $R'(W, N, \alpha)$.

%With $R(W, N, \lambda, S)$, we can study the behavior of lazy shadowing under a configuration of ($W$, $N$, application failure probability), for any failure distribution $f(t)$, e.g., exponential or weibull. However, there are two problems now: 1) The computation involved is so complicated that MatLab cannot give accurate results; 2) we don't have the analytical model of expected completion time $T_c$ assuming exponential or weibull failure distribution. 


\subsection{Expected completion time}
\label{anal_time}
One of the major performance metrics of interest to end-users, is the application's completion time. 
To evaluate this, we develop an analytical model for the expected completion time of Lazy Shadowing, with all probabilities of failures considered. We assume that failures don't happen at the same time.
%Since we are comparing between Lazy Shadowing and process replication, the overhead of failure detection and consistency protocols are ignored as they are the same for the two approaches.
%Our models focus on one 
%checkpointing interval of the application, since the whole execution is just a repetition of multiple such intervals. As a consequence, we assume the execution with checkpointing starts right after the previous checkpoint and ends right before the next checkpoint, and the execution with Lazy Shadowing and process replication will not experience any application failure. For fairness, we assume that the three alternatives have the same amount of workload, 
%$W$, to execute, and total number of available cores, $M$. The maximal execution rate at each core, $\sigma_{max}$, is normalized to 1 so that the time to complete without failures is equal to the workload to execute on each core. In addition, we assume that the three alternatives will encounter the same number of failures, which is $k$, as they will execute the same amount of workload using the same amount of resources.

%\subsubsection{Expected completion time}
First we discuss the case of $k$ failures, which separate the execution into $k+1$ intervals.
Denote by $\Delta_i$ ($1\le i \le k+1$) the $i^{th}$ continuous execution interval, and $\tau_i$ ($1\le i \le k$) the recovery time after $\Delta_i$. 
%between the $(i-1)^{th}$ and $i^{th}$ failures (assuming the $0^{th}$ failure happens right before the execution begins, and the $(k+1)^{th}$ failure happens right after the execution ends), and $\tau_i$ ($1\le i \le k$) the recovery time after the $i^{th}$ failure. 
The application's progress with delay incurred by failures is illustrated in Figure~\ref{fig:progress}.

\begin{figure}[!t]
	\begin{center}
		\includegraphics[width=\columnwidth]{Figures/progress}
	\end{center}
	%\vskip -0.22in 
	\caption{Illustration of application's progress with failure incurred delays.}
	\label{fig:progress}
\end{figure}

Since Lazy Shadowing use $M$ cores for executing main processes and $S$ cores for shadowing ($M+S=N$), the total workload $W$ will be split into $M$ tasks, each of which will be assigned a pair of main and shadow processes. Therefore, the workload of each process is 
$w=W/M$. The recovery time $\tau_i$ is the time needed for the lazy shadow of the failed main to catch up. With shadow leaping, it is guaranteed that all the shadows reach the same execution point as the mains (See Figure~\ref{fig:leap}) after the previous recovery, so every recovery time is proportional to its previous continuous execution length, which is $\Delta_i$. That is, $\tau_i = \Delta_i \times (1 - \sigma_s^b)$. The value of $\Delta_i$ can be obtained given a failure probability distribution, as will be demonstrated in Section~\ref{sec:evaluation}. 
Since we assume there are $k$ failures, then $\Delta_{k+1}$ is the failure free execution interval until $W$ is complete, i.e., $\Delta_{k+1} = w - \sum_{i=1}^{k}\Delta_i$. Finally, according to Figure~\ref{fig:progress}, the completion time with $k$ failures is 
\begin{equation}
	T_c^k = \sum_{i=1}^{k+1}\Delta_i + \sum_{i=1}^k\tau_i = w + (1-\sigma_s^b)\sum_{i=1}^k\Delta_i
	\label{eq:time_k}
\end{equation}

We assume that failures do not occur during recovery, so the failure probability of a core during the execution can be estimated as $P_c = F(w)$. Then the probability that there are $k$ failures among the $N$ cores during the execution is 
\begin{equation}
\begin{split}
P_s^{k}= & \dbinom{N}{k}{P_c}^k(1-P_c)^{N-k} \\
%= & \dbinom{M}{k}({\frac{w}{MTTI}})^k(1-\frac{w}{MTTI})^{M-k}
\end{split}
\end{equation}

The completion time considering all possible failures can be averaged as $T_{c} = \sum_{i} T_{c}^{i} \cdot P_s^{i}$, and the expected completion time considering the possibility of re-execution after application failure is
%\begin{equation} 
%E[T_{c}] = \sum_{i} T_{c}^{i} \cdot P_s^{i}
%\label{eq:exp_time}
%\end{equation}
\begin{equation} 
T_{total} = T_{c} / (1 - P_a)
\label{eq:exp_time}
\end{equation}

In the special case of $\alpha=1$, which is process replication, half of the hardware resources are dedicated to replicas so that the workload assigned to each task is significantly increased given the fixed number of cores available, i.e., $w=2W/N$. Different from the case of $\alpha \ge 2$, failures do not incur any delay unless application failure occurs, since the replicas are executing at the same rate as the main processes. Therefore, the completion time of process replication without application failure is constant with respect to the number of failures, and the expected completion time considering the possibility of re-execution is
\begin{equation}
	T_{total} = T_c / (1 - P_a) = w / (1 - P_a)
\end{equation}

A closer look at the above analysis one can realize that Lazy Shadowing has both advantage and disadvantage compared to traditional process replication. When collocating multiple shadow processes on each core, more resources will be dedicated to main processes, leading to less workload per process and thus less completion time. On the other hand, however, collocation slows down the shadow processes, implying delays when failures occur. Although it may seem that the delay would keep deteriorating as the number of failures increases, it turns out to be well bounded, as a result of shadow leaping. From Equation~\ref{eq:time_k} we can see the delay of $k$ failures is $(1-\sigma_s^b)\sum_{i=1}^k\Delta_i$. Since we have the relation $\Delta_{k+1} = w - \sum_{i=1}^{k}\Delta_i$, $\sum_{i=1}^{k}\Delta_i$ is bounded by $w$, effectively limiting the delay by $(1-\sigma_s^b)w$.

%Contrary to process replication and Lazy Shadowing, checkpointing can use all the available cores to share the total workload, so that $w = W/M$. However, the drawback is that each failure would result in an application failure which needs to roll back the execution to the last checkpoint. With that said, the recovery time of each failure is equal to the normal execution time from the last checkpoint to the time of the failure, i.e., $\tau_i = \sum_{j=1}^{i}\Delta_j$. The completion time with $k$ failures is $T_c^k = \sum_{i=1}^{k+1}\Delta_i + \sum_{i=1}^k\tau_i = w + \sum_{i=1}^{k}\sum_{j=1}^i\Delta_j$.

%\subsubsection{Expected energy consumption}







\subsection{Expected energy consumption}
\label{anal_energy}
The power consumption of one core consists of two parts, dynamic power, $p_d$, which exists only when the core is executing, and static power, $p_s$, which is constant as long as the machine is on. This can be modeled as $p = p_d + p_s$. Note that in addition to CPU leakage, other components, such as memory and disk, also contribute to the static power consumption. 

%For checkpointing and process replication, all cores are running all the time until the application is complete. Therefore, the energy consumption is proportional to the total execution time, and the expected energy consumption when using $M$ cores to execute an application is calculated as 
For process replication, all cores are running all the time until the application is complete. Therefore, the expected energy consumption, $En$, is proportional to the expected execution time $T_{total}$: 
\begin{equation}
En = N * p * T_{total}
\label{eq:exp_energy1}
\end{equation} 
Although the failed components should not consume any power, we ignore this since the number of failures is negligible compared to the total number of cores.

Lazy Shadowing has the potential to save power compared to process replication, since main cores are idle during the recovery time after each failure, and the shadows can achieve forward progress through shadow leaping. During the normal execution time, all the cores are consuming static power as well as dynamic power. During recovery time, however, the main cores are idle and consume only static power, while the shadow cores performs shadow leaping, which may lead to higher dynamic power due to memory access and communication. After the leaping, the shadow cores become idle with no dynamic power consumption, until the failure recovery is completed. Again, we include the power consumption of the failed components. Altogether, the expected energy consumption for Lazy Shadowing can be modeled as 
\begin{equation}
En = N * p_s * T_{total} + N * p_d * w + S * p_{l} * T_l.
\label{eq:exp_energy2}
\end{equation}
with $p_{l}$ denoting the dynamic power consumption of each core during shadow leaping and $T_l$ denoting the expected total time spent on leaping during the execution of the application. Based on Equation~\ref{eq:exp_energy2} and Corollary 1.1, we can also establish an upper bound on the expected energy consumption for Lazy Shadowing:

\begin{theorem}
If no subsequent failure happens before the recovery of the previous failure, then using Lazy Shadowing, the upper bound on expected energy consumption is
$(2N * p_s + N * p_d + S * p_{l})*w$.
\end{theorem}
%\begin{proof}
{\sc Proof}. From Corollary 1.1 we know that the delay is at most $(1-\sigma_s^b)w \le w$, so $T_{total} \le 2w$. Also, since the leaping time overlaps with the recovery time (delay), $T_l \le (1-\sigma_s^b)w \le w$. Therefore, $En = N * p_s * T_{total} + N * p_d * w + S * p_{l} * T_l \le N * p_s * (2w) + N * p_d * w + S * p_{l} * w = (2N * p_s + N * p_d + S * p_{l})*w$.
%\end{proof}
$\square$

