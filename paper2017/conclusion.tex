As the scale and complexity of HPC systems continue to increase, both the failure rate and energy consumption are expected to increase dramatically, making it extremely challenging to deliver extreme-scale computing performance efficiently. Existing fault tolerance methods rely on either time or hardware redundancy. %Neither of them appeals to the next generation of supercomputing, as the first approach may incur significant delay while the second one constantly wastes over 50\% of the system resources. %The need for an efficient and reliable solution in extreme-scale, failure-prone computing environments calls for a reconsideration of the fault tolerance problem. 

%To this end, we propose an adaptive and power-aware algorithm, referred to as Lazy Shadowing, 
Lazy Shadowing is a novel algorithm that can serve as an efficient and scalable alternative to achieve high-levels of fault tolerance for future extreme-scale computing. In this paper, we present a comprehensive discussion of the techniques that enable Lazy Shadowing. In addition, we develop a series of analytical models to assess its performance in terms of reliability, completion time, and energy consumption. 
Through comparison with existing fault tolerance approaches, we identify the scenarios where each of the alternatives should be chosen. %Specifically, checkpointing is the most efficient approach for small scale systems, while Lazy Shadowing is more preferred as system scale and complexity keep growing. %In addition, Lazy Shadowing has the ability to converge to process replication when failure rate is extremely high. 
%In some cases where failure rate is extremely high, Lazy Shadowing can converge to process replication to 
%minimize completion time.
%; process replication should be used to minimize completion time when system size is extremely large and failure rate is extremely high; and Lazy Shadowing is the choice for all other cases, for the consideration of both completion time and energy consumption. 
%In addition, we predict that Lazy Shadowing is able to achieve over 20\% of energy saving with potential completion time reduction for future extreme-scale computing. 

%In the future, we will generalize our approach to multiple jobs. Initially we will assume that each job has an arrival time, a workload, and a deadline by which it must complete, and will seek minimally adaptive stochastically competitive strategies. %The strategies in the case of no failure in~\cite{Albers:2011:MSS:1989493.1989539} is a good staring point. 
%Another future direction is to apply our approach to servers with inter-processor power heterogeneity. %In particular, it is likely that a chip will contain a few high-rate power-hungry processors, many low-rate power-efficient processors, and possibly an intermediate number of processors with intermediate rate and power efficiency. 
%Generally speaking, inter-processor power heterogeneity is much harder to handle theoretically than intra-processor power heterogeneity. The main reason is that the number of jobs that can be run at high speed is fixed, and thus some jobs cannot be guaranteed to finish by their deadlines. A solution would be to allow a job to miss its deadline with some penalty, and to consider the objective as energy plus penalty. %Again we seek to find stochastically competitive algorithms, and a good starting point may be the algorithms that are known to be competitive without faults~\cite{Kling:2013:PSM:2486159.2486183}.
In the future, we plan to explore the combination of Lazy Shadowing with Checkpointing, so that if an application fatal failure occurs computation can restart from an intermediate state. %We will study whether this will be beneficial and how much improvement it can bring about. 
Another future direction is to use dynamic and partial shadowing for platforms where nodes exhibit different ``health" status, e.g., some nodes may be more reliable while others are more likely to fail~\cite{6468487}. 
%Health information can be obtained by either analyzing system logs or monitoring node status online~\cite{6468487}. 
%By taking into account this kind of information, which can be obtained by analyzing system logs or monitoring node status online, we may greatly reduce the resource requirement for shadows.
With this taken into account, we can apply dynamic scheduling of shadows only for mains that are likely to fail, to further reduce the resource requirement for shadowing.

