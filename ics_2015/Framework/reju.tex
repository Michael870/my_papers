The proposed Lazy Shadowing scheme can tolerate faults which are repairable by rebooting or reconfiguration, referred to as soft faults, and faults which cannot be repaired by rebooting or reconfiguration, referred to as hard 
faults. Monitors that detect hard faults, such as memory flip, bus error 
and latch error, or soft faults, such as deadlock detection, buffer overflow and protection violation, typically interrupt the application to initiate 
the recovery process. The process of recovery from transient or permanent faults is the same and necessitates a mechanism for detecting a fault 
in a main task, M(i) and notifying other tasks in the system so that (i) the shadows sharing a core with S(i) are terminated, thus allowing S(i) to execute at the maximum rate, and (ii) all the shadows that are not in the faulty shadowed set leap to the state of their mains. 

As described earlier, the recovery from a fault in a shadowed set leaves the set vulnerable and any more faults in a vulnerable set will result in a system failure. Although for large systems and small S the probability of having a second fault in a vulnerable set is low, some provision should be taken to rejuvenate the system when a relatively large number of its shadowed sets are vulnerable. 

We propose to invoke \emph{shadowed set rejuvenation} after a specific number of faults, which is determined by the system size, the shadowed set size, and the required resilience.
Rejuvenation reconfigures the system such that none of its shadowed sets are vulnerable. Unlike recovery from a fault in a shadowed set, rejuvenation is different when the faults are transient soft when the faults are hard. In the case of soft faults, rejuvenation can be accomplished by rebooting the failed cores, and restarting the lost shadows (both the ones promoted to mains and the ones terminated) from the state of current mains. And in case of hard faults, it is possible to restart the lost shadows after replacing the failed ones with spare ones. This will restore a vulnerable shadowed set to its original configuration. %For example, rejuvenation should restore the systems shown in Figure~\ref{fig:layout2} and Figure~\ref{fig:layout3} to the one shown in Figure~\ref{fig:layout1}.

%When failures are permanent, rejuvenation may be challenging if rebooting or reconfiguration can no longer be
%used to recover failed components.
%Specifically, in the absence of spare components (if the system is not over-provisioned),
%rejuvenation can only 
%be accomplished by distributing the main processes of a vulnerable set 
%to other {\bf non-vulnerable} shadowed sets. The shadow of the vulnerable 
%set must also be relocated to the shadows of the non-vulnerable set. 
%As a result,  the total number of shadowed sets decreases, but 
%the size of some shadowed sets increases.  In Figure~\ref{fig:reju}, we show a possible rejuvenated configuration assuming that the failure of the cores executing $M(1)$ and $M(14)$ in Figure~\ref{fig:layout2} is permanent. In this restored configuration, the number of shadowed sets is reduced from 8 to 6, with four sets containing three mains each and two sets containing two mains each. Rejuvenating the vulnerable configuration of Figure~\ref{fig:layout3} after a permanent socket failure is more complex but follows the same basic principle.

%\begin{figure*}[ht]
%	\begin{center}
%		\includegraphics[width=\textwidth]{figures/reju.pdf}
%	\end{center}
%\vskip -0.25in
%	\caption{Shadowed set rejuvenation of the vulnerable sets resulting from permanent faults.}
%	\label{fig:reju}
%\end{figure*}
