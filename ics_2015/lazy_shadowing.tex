% This is "sig-alternate.tex" V2.0 May 2012
% This file should be compiled with V2.5 of "sig-alternate.cls" May 2012
%
% This example file demonstrates the use of the 'sig-alternate.cls'
% V2.5 LaTeX2e document class file. It is for those submitting
% articles to ACM Conference Proceedings WHO DO NOT WISH TO
% STRICTLY ADHERE TO THE SIGS (PUBS-BOARD-ENDORSED) STYLE.
% The 'sig-alternate.cls' file will produce a similar-looking,
% albeit, 'tighter' paper resulting in, invariably, fewer pages.
%
% ----------------------------------------------------------------------------------------------------------------
% This .tex file (and associated .cls V2.5) produces:
%       1) The Permission Statement
%       2) The Conference (location) Info information
%       3) The Copyright Line with ACM data
%       4) NO page numbers
%
% as against the acm_proc_article-sp.cls file which
% DOES NOT produce 1) thru' 3) above.
%
% Using 'sig-alternate.cls' you have control, however, from within
% the source .tex file, over both the CopyrightYear
% (defaulted to 200X) and the ACM Copyright Data
% (defaulted to X-XXXXX-XX-X/XX/XX).
% e.g.
% \CopyrightYear{2007} will cause 2007 to appear in the copyright line.
% \crdata{0-12345-67-8/90/12} will cause 0-12345-67-8/90/12 to appear in the copyright line.
%
% ---------------------------------------------------------------------------------------------------------------
% This .tex source is an example which *does* use
% the .bib file (from which the .bbl file % is produced).
% REMEMBER HOWEVER: After having produced the .bbl file,
% and prior to final submission, you *NEED* to 'insert'
% your .bbl file into your source .tex file so as to provide
% ONE 'self-contained' source file.
%
% ================= IF YOU HAVE QUESTIONS =======================
% Questions regarding the SIGS styles, SIGS policies and
% procedures, Conferences etc. should be sent to
% Adrienne Griscti (griscti@acm.org)
%
% Technical questions _only_ to
% Gerald Murray (murray@hq.acm.org)
% ===============================================================
%
% For tracking purposes - this is V2.0 - May 2012

\documentclass{sig-alternate}


%\usepackage{cite}
%\usepackage{epsfig}
\usepackage[tight,footnotesize]{subfigure}
\usepackage{calc}
\usepackage{amstext}
%\usepackage[cmex10]{amsmath}
%\usepackage{amsthm}
%\usepackage{multicol}
\usepackage{pslatex}
\usepackage{color}
\usepackage{graphicx}
\usepackage[small]{caption}
\usepackage{booktabs}
%\usepackage{algorithm}
%\usepackage[noend]{algpseudocode}
\usepackage[linesnumbered]{algorithm2e}
\usepackage{epstopdf}
\usepackage{float}


\begin{document}
%
% --- Author Metadata here ---
\conferenceinfo{ICS}{'15 Newport Beach, CA, USA}
%\CopyrightYear{2007} % Allows default copyright year (20XX) to be over-ridden - IF NEED BE.
%\crdata{0-12345-67-8/90/01}  % Allows default copyright data (0-89791-88-6/97/05) to be over-ridden - IF NEED BE.
% --- End of Author Metadata ---

\title{Lazy Shadowing: An Adaptive, Power-Aware, Resilience Framework for Extreme-scale Computing}
%\subtitle{[Extended Abstract]
%\titlenote{A full version of this paper is available as
%\textit{Author's Guide to Preparing ACM SIG Proceedings Using
%\LaTeX$2_\epsilon$\ and BibTeX} at
%\texttt{www.acm.org/eaddress.htm}}}
%
% You need the command \numberofauthors to handle the 'placement
% and alignment' of the authors beneath the title.
%
% For aesthetic reasons, we recommend 'three authors at a time'
% i.e. three 'name/affiliation blocks' be placed beneath the title.
%
% NOTE: You are NOT restricted in how many 'rows' of
% "name/affiliations" may appear. We just ask that you restrict
% the number of 'columns' to three.
%
% Because of the available 'opening page real-estate'
% we ask you to refrain from putting more than six authors
% (two rows with three columns) beneath the article title.
% More than six makes the first-page appear very cluttered indeed.
%
% Use the \alignauthor commands to handle the names
% and affiliations for an 'aesthetic maximum' of six authors.
% Add names, affiliations, addresses for
% the seventh etc. author(s) as the argument for the
% \additionalauthors command.
% These 'additional authors' will be output/set for you
% without further effort on your part as the last section in
% the body of your article BEFORE References or any Appendices.

\numberofauthors{3} %  in this sample file, there are a *total*
% of EIGHT authors. SIX appear on the 'first-page' (for formatting
% reasons) and the remaining two appear in the \additionalauthors section.
%

\author{
\alignauthor
Anonymized
}

%\author{
%\alignauthor
%Xiaolong Cui\\
%       \affaddr{University of Pittsburgh}\\
%       \affaddr{4200 Fifth Avenue}\\
%       \affaddr{Pittsburgh, USA}\\
%       \email{mclarencui@cs.pitt.edu}
%\alignauthor
%Taieb Znati\\
%       \affaddr{University of Pittsburgh}\\
%       \affaddr{4200 Fifth Avenue}\\
%       \affaddr{Pittsburgh, USA}\\
%       \email{znati@cs.pitt.edu}
%\alignauthor 
%Rami Melhem\\
%       \affaddr{University of Pittsburgh}\\
%       \affaddr{4200 Fifth Avenue}\\
%       \affaddr{Pittsburgh, USA}\\
%       \email{melhem@cs.pitt.edu}
%}



% There's nothing stopping you putting the seventh, eighth, etc.
% author on the opening page (as the 'third row') but we ask,
% for aesthetic reasons that you place these 'additional authors'
% in the \additional authors block, viz.
%\additionalauthors{Additional authors: John Smith (The Th{\o}rv{\"a}ld Group,
%email: {\texttt{jsmith@affiliation.org}}) and Julius P.~Kumquat
%(The Kumquat Consortium, email: {\texttt{jpkumquat@consortium.net}}).}
\date{16 Jan 2015}
% Just remember to make sure that the TOTAL number of authors
% is the number that will appear on the first page PLUS the
% number that will appear in the \additionalauthors section.

\maketitle
\begin{abstract}
As the demand for cloud computing continues to increase, cloud service
providers face the daunting challenge to meet the negotiated SLA
agreement, in terms of reliability and timely performance, while
achieving cost-effectiveness. This challenge is increasingly
compounded by the increasing likelihood of failure in large-scale
clouds and the rising cost of energy consumption.  This paper proposes
Shadow Replication, a novel profit-maximization resiliency model,
which seamlessly addresses failure at scale, while minimizing energy
consumption. The basic tenet of the model is to associate a suite of
shadow processes to execute concurrently with the main process, but
initially at a much reduced execution speed, to overcome failures as
they occur. Two computationally-feasible schemes are proposed to
achieve shadow replication. A performance evaluation framework is
developed to analyze these schemes and compare their performance to
traditional replication-based fault tolerance methods, focusing on the
inherent tradeoff between fault tolerance, the specified SLA and
profit maximization. The results show Shadow Replication leads to
significant energy reduction, and is better suited for
compute-intensive execution models, where up to 30\% more profit
increase can be achieved.


%several experimental studies are carried out to assess the performance
%of the different resiliency schemes, with respect to profit
%maximization in different cloud computing environments. 




%The challenge is to derive the
%execution speed, both before and after failure, of a shadow in order
%to ensure adherence to the negotiated SLA, while maximizing profit. To
%this end, we present an optimization model to derive the shadow
%execution speeds, which takes into consideration a computing node
%failure rate and the negotiated SLA. Several computationally-feasible
%methods are then proposed to solve this model. 



%As companies continue to increase their reliance upon cloud computing
%services there will be an increasing demand for reliable and timely
%service. However, as cloud-based systems increase in size and
%complexity it is expected that reliability will degrade, causing both
%delays in service and increases in energy consumption. This will cause
%fault tolerance to be a critical system feature to providing
%applications at large scale. In this work, we propose ``shadow
%replication'', a fault tolerance method that makes use of DVFS to
%provide energy-aware, profit-maximizing system resilience to
%task-based cloud computing services.  We analyze different resilience
%methods and identify the system parameters which are most relevant to
%the tradeoff between fault tolerance and profit, and present results
%which pinpoint the most profitable method.  We also show that in
%certain systems shadow replication can achieve 10-30\% more profit
%than existing fault tolerance methods, i.e. re-execution and
%traditional replication.

%
%
% ``shadow replication'' to
%provide an energy-aware, profit-optimized method that can result in
%upto two-times the profit achieved by existing fault tolerance
%methods. Additionally, we develop analytical models to demonstrate the
%benefits of our approach at the scale expected in future cloud
%computing environments.

\end{abstract}

% A category with the (minimum) three required fields

\category{C.4}{Performance of systems}{Fault tolerance}
\category{C.2.4}{Distributed Systems}{Distributed applications}
%A category including the fourth, optional field follows...
%\category{C.2.4}{Distributed Systems}{Distributed applications}[complexity measures, performance measures]

\terms{Design, Performance, Reliability}

\keywords{Lazy Shadowing, extremely-scale computing, forward progress}

\section{Introduction}
\label{sec:intro}
Today's scientific discoveries and business intelligence are driven by high-fidelity, 
large-scale simulation and data analytics. To meet the increasing computing demands from 
virtually every aspect of the society, HPC is continuously evolving to solve more 
complex and challenging problems. On the one hand, national labs and research institutes run HPC on 
supercomputers for scientific breakthroughs and national security. On the other hand, enterprises and 
organizations deploy HPC on small to medium sized clusters to process data and extract insights. 
Recently, the explosively growing machine learning applications have increased the adoption as well as 
impact of HPC as they also exploit parallelism and hardware acceleration to speed up the processing of 
massive amount of data.


HPC workloads have traditionally been run only on bare-metal, unvirtualized hardware to drive maximum 
performance. 
The roadblock to virtualization was due to the concern that the extra hypervisor layer could introduce 
performance overhead. 
%The concern was that virtualization could introduce performance overhead due to the extra software 
%layer of hypervisor. 
However, this has started to change with the introduction of increasingly sophisticated 
hardware support for virtualization and software optimization~\cite{madukkarumukumana2008resource,bugnion2017hardware}. Performance of 
these highly parallel HPC workloads has increased dramatically over the last decade, 
enabling organizations to begin to embrace the numerous benefits that a virtualization platform can 
offer~\cite{michael2018overcommit}. As a result, we are witnessing a popular trend that enterprises convert 
their on-prem bare-metal clusters to virtualized, shared private cloud. For instance, the Johns Hopkins 
University Applied Physics Laboratory recently virtualized their 3728-core bare-metal cluster 
to share between Windows and Linux users. The reported improvement in resource utilization 
ranges from 9.1\% to 29.2\%, and simulations speed up by 4\% on average~\cite{vmware2017josh}.

At the same time, public cloud, such as Amazon AWS and Google GCP, is becoming a popular alternative for 
HPC practitioners. Recent studies show that the usage of public cloud has grown more than five-fold among all HPC 
sites worldwide, from 13\% in 2011 to 74\% in 2018~\cite{hyperion2019}.
With virtually unlimited scalability and on-demand resource subscription, public cloud starts to host 
compute- and data-intensive workloads across various industry verticals. These workloads span the traditional HPC 
applications, like genomics and 
weather prediction, as well as emerging applications, like machine learning and deep learning. 

There is a fruitful body of research on resource management in 
Cloud Computing~\cite{singh2016survey,zhan2015cloud,gill2018chopper}. Dynamic resource scheduling and 
load balancing are used 
to maximize system utilization and efficiency~\cite{adhikari2018heuristic,panwar2015load}. These techniques, however, 
are not straightforward to apply to HPC workloads which are highly sensitive to resource change and interference. 
Actually, resource management has been identified as one of the open 
challenges for HPC cloud~\cite{netto2018hpc}. 
Currently, cloud service providers (CSPs) are often limited to statically and conservatively reserve 
resources based on peak resource requirements to respect service level agreements (SLAs). For example, Microsoft Azure 
allocates dedicated supercomputers from Cray, and Amazon AWS offers dedicated nodes for full-size VMs. 
% allocate physical resources
% Despite the 
% numerous benefits promised by Cloud Computing, however, cloud service providers (CSPs) are often limited to statically 
% allocate physical resources to HPC tenants in order to avoid performance interference and enforce 
% service level agreements (SLAs). 
This essentially offsets 
the elasticity and efficiency benefits of the Cloud Computing business model. 

In this paper, we present \textit{virtual throughput clusters (VTC)} as a novel approach for cloud 
resource allocation to efficiently and effectively support 
HPC workloads with multi-tenancy. Based on virtual machine (VM), VTC goes beyond traditional way of 
statically splitting resources among tenants and applies resource over-commitment to optimize 
system utilization and throughput. By giving each tenant a virtual cluster that mimics the 
underlying physical cluster, VTC delegates the resource management task 
to the hypervisor to improve flexibility as well as efficiency. When all tenants are busy consuming their cycles, 
VTC guarantees that each tenant is getting his/her fair share according to pre-defined SLA terms. When 
some tenant is not fully using the allocated resources, VTC takes advantage of the work-conserving 
property of the hypervisor scheduler to assign the idle resources to other tenant(s) who can benefit 
from additional resources. Consequently, CSPs can ensure quality-of-service while maximizing 
system utilization. 

The rest of the paper is organized as follows. 
Section II provides background and motivation. Section III introduces the design of VTC, followed by validation 
and empirical evaluation results in Section IV. Section V concludes this work and points out future directions.

\section{\uppercase{Related Work}}
\label{sec:related_work}
%Extreme-scale computing presents some unique challenges to fault tolerance as faults are no longer 
%an exceptional event \cite{ferreira_sc_2011}. 
Rollback and recovery is the dominant mechanism to achieve fault
tolerance in current HPC environments~\cite{Elnozahy:02:Survey}. In the most general form, rollback and recovery 
involves the periodic saving of the current system state, with the anticipation that
in the case of a failure, computation can be restarted from the most recently saved state. % \cite{Elnozahy:02:Survey}. %The identification of an error, before or during a checkpoint,
%requires that the application rollback to the previously completed checkpoint. 
Coordinated checkpointing is a popular approach for
its ease of implementation.
%Specifically, all processes
%coordinate with one another to produce individual states that satisfy the ``happens before"
%communication relationship \cite{chandy_trans_1972}, which is proved to provide a consistent global state.
%Essentially, the algorithm provides a method for all processes involved to stop operation ``at the same
%time" and transfer their system state to a stable storage. 
%The major benefit of coordinated
%checkpointing stems from its simplicity and ease of implementation. 
Its major drawback, however, is the
lack of scalability, as it requires global coordination
~\cite{elnozahy_dsc_2004}.%riesen_sandia_2010}.
%hargrove2006berkeley}.


In uncoordinated checkpointing, processes checkpoint their states independently and postpone creating a 
globally consistent view until the recovery phase. The major advantage is the reduced overhead during fault free operation. However, the scheme requires that
each process maintains multiple checkpoints and message logs, necessary to construct a consistent 
state during recovery. It can also suffer the well-known domino effect 
 \cite{randell_domino_effect}. One hybrid approach, known as communication induced 
checkpointing, aims at reducing coordination overhead \cite{alvisi_ftc_1999}. The approach, however, may 
cause processes to store useless states. To address this 
shortcoming, ``forced checkpoints" have been proposed \cite{helary_rds_1997}. This approach, however,  may lead to unpredictable
checkpointing rates. Although well-explored, uncoordinated checkpointing has not been widely adopted
in HPC environments, due to its dependency on applications \cite{guermouche_2011_ipdps}.


One of the largest overheads in any checkpointing process is the time necessary to write the checkpointing 
to stable storage. Incremental checkpointing attempts
to address this by only writing the changes since previous checkpoint \cite{Agarwal:04:Adaptive}. %,elnozahy_1992_manetho,li_trans_1994}. %This
%can be achieved using dirty-bit page flags \cite{plank_ftcs_1994,elnozahy_1992_manetho}. Hash based incremental checkpointing, on the other
%hand, makes use of hashes to detect changes \cite{nam_ftc_1997,Agarwal:04:Adaptive}. 
Another proposed scheme, known as in-memory checkpointing, minimizes the overhead of disk access~\cite{zheng_2004_ftccharm,6264677}.
%offloads the checkpointing process to a secondary task and only writes incremental checkpoints \cite{li_trans_1994}.
The main concern of these techniques is the increase in
memory requirement to support the simultaneous execution of the checkpointing and the application. It has been suggested that nodes in extreme-scale systems should be configured with fast local storage~\cite{doe_ascr_exascale_2011}. 
%, which
%improves the performance of checkpointing \cite{doe_ascr_exascale_2011}. 
Multi-level checkpointing, which consists of
writing checkpoints to multiple storage targets, can benefit from such a strategy \cite{Moody:10:SCR}. This,
however, may lead to increased failure rates of individual nodes and complicate the checkpoint writing process.
%Furthermore, it may complicate the checkpoint writing process and requires that the system track the
%current location of all process's checkpoints.


Process replication, or state machine replication, has long been used for reliability and availability in distributed and mission critical systems \cite{schneider_1990_tutorial}. %Replication can be used to detect and correct system failures that are otherwise undetectable,
%such as silent data corruption and Byzantine faults \cite{fiala_2012_sdc}. 
This approach is barely used in HPC systems, primarily due to its high cost and low efficiency.
However, upcoming extreme-scale systems are expected to 
%require a more challenging level of fault tolerance to deal with the 
confront a dramatic growth in both the frequency and diversity of faults.
As a result,
replication has recently been proposed as a
viable alternative to checkpointing in HPC \cite{engelmann09case,Cappello:09:Fault}. 
In addition, full and partial
replication have also been studied to augment existing checkpointing techniques, and to  
detect or correct silent data corruption \cite{stearly_2012_partial,elliott_2012_cpr,ferreira_sc_2011,fiala_2012_sdc}. % There are several different implementations of
%replication in the widely used MPI library, each with their different tradeoffs and overheads. The
%overhead can be negligible or up to 70\% depending upon the communication patterns of the
%application \cite{engelmann2011redundant}. %Moreover, replication alone is not enough to guarantee fault tolerance since
%it is possible that all nodes executing a given process could fail simultaneously, thus
%replication is typically paired with some form of checkpointing. 
Our approach differs from classical process replication in that we dynamically configure the execution rates of main and shadow processes, so that less resource/energy is required while reliability is still assured.  


Replication with dynamic execution rate is also explored in Simultaneous and Redundantly Threaded (SRT) processor whereby one leading thread of execution is running ahead of trailing threads \cite{reinhardt2000transient}. However, 
the focus of \cite{reinhardt2000transient} is on transient faults within CPU while we aim at tolerating both permanent and transient faults across all systems components.
This work is closely related to our previous works \cite{mills_2014_icnc,cui_en7085151,cui_2014_closer} where single or loosely-coupled tasks is considered. Instead, in this paper we explore novel ideas of shadow collocation and shadow leaping in order to satisfy the requirements of future extreme-scale HPC systems. 
%our approach is different in that it tunes the execution rates of the leading and trailing threads in a finer grain, in order to achieve a ``parameterized" trade-off between completion time and energy consumption. 
%Further, we take advantage of the idle time during failure recovery and ``leap" the trailing replicas to achieve forward progress%, largely improving performance in terms of both completion time and energy consumption. 
%. This differs from \cite{reinhardt2000transient}, of which the ``leaping" of the trailing replica results in extra overhead.
%To the best of our knowledge,
%Lazy Shadowing is the first attempt to explore a state-machine replication based framework
%that achieves a fine-grained tradeoff between time and hardware redundancy while meeting resilience and
%power requirements.



\section{Lazy shadowing}
\label{sec:frame_model}
%\label{frame_single}
%\input{Framework/single}

%associated with three attributes, i.e., workload $w$, execution rate $\sigma$, and completion time $t$.
%This makes our framework agnostic to the granularity of the resource allocation unit.
%Cores are subject to failures. We assume that core failures are independent and identically distributed (i.i.d.). We do not distinguish between soft and hard failures, assuming that soft failures are handled via software rejuvenation (i.e., rebooting \cite{466961}), while hard failures are handled by replacing the failed components with spares. We adopt the fail-stop fault model, whereby a core stops executing upon failure and failures are detected by other non-failing cores \cite{gartner_faults_1999,cristian_comm_1991}. %When a core fails, the whole execution is suspended until recovery is complete. 

%%%%%%%%%%%%%%%%%%%%%%%%%%%%%%rethinking%%%%%%%%%%%%%%%%%%%%%%
%We assume that core failures are independent and identically distributed (i.i.d.). In the real world, instead, failures are bound to be correlated. Obtaining theoretical results for non-i.i.d. failures is beyond the scope of this work. But note that one cause of correlation is the hierarchical structure of computing platforms (each rack comprises compute nodes, each compute node comprises processors, and each processor comprises cores), which leads to simultaneous failures of a group of cores. Our work applies to such failures since a group of failures can be treated as multiple individual failures that happen at the same time and their recovery can be carried out in parallel.


%We use the term core to represent the computing resource allocation unit%(e.g., a
%CPU core, a multi-core CPU, or a cluster node)
%~\cite{casanova_inria_2012}. 
%We further use $P(\sigma$, $w$, $t)$ to denote a process executing at rate $\sigma$ to complete a workload $w$ by time $t$.
The basic tenet of Lazy Shadowing is the concept of shadowing, whereby each process is associated with a {\it lazy} replica that executes at a reduced rate. We only use one replica since the probability that failure occurs to both the original process and the replica is negligible~\cite{casanova_inria_2012}. %This guarantees that if one process fails, the other one can still complete the task. 
If necessary, however, this can be easily extended to use a suit of replicas. Assuming a single failure, Lazy Shadowing can be described as follows:
\begin{itemize}
	\item A main process, $P_m(\sigma_m$, $w$, $t_m)$, that executes at the rate of $\sigma_m$ to complete a task of size $w$ at time $t_m$.
	\item A shadow process, $P_s(<\sigma_s^b$ , $\sigma_s^a>$, $w$, $t_s)$, that initially executes at $\sigma_s^b$, and increases to $\sigma_s^a$ if its main process fails, to complete a task of size $w$ at time $t_s$.% where $\sigma_s^b$ represents the execution rate before failure, $\sigma_s^a$ the execution rate after failure, $w$ is the task size, and $t_s$ is the completion time.%, which has the same workload $w$. It starts execution simultaneously with the main process at rate $\sigma_b  \le \sigma_m$, but on a different core. Upon failure of the main process, the shadow process will be designated as the new main process, with its rate switched to $\sigma_a$ to catch up. In this case, the completion time is denoted as $t_s$. 
\end{itemize}


We use the term core to represent the resource allocation unit (e.g., a
CPU core, a multi-core CPU, or a cluster node), so that our algorithm is agnostic to the
granularity of the hardware platform~\cite{casanova_inria_2012}.
Furthermore, we adopt the
fail-stop failure model, where a core stops execution once a failure
occurs and failure can be detected by other
processes~\cite{gartner_faults_1999,cristian_comm_1991}.
In order to deal with both permanent and temporary failures, the shadow process starts simultaneously with its associated main process, on a different node. Lazy Shadowing is able to tolerate any failure confined to a single node, including socket failure, CPU logical errors, bus errors, errors in the attached accelerators (e.g., GPUs), and even memory bit flips that exceed ECC's capacity. 

Initially, the main process executes at rate $\sigma_m$, while the shadow executes at $\sigma_s^b \le \sigma_m$. %To avoid simultaneous failure, they are deployed on different cores.
%with the main running at  
%a rate $\sigma_m$ and the lazy shadow at a lower rate $\sigma_s \le \sigma_m$. 
In the absence of failure, the main process completes execution at time 
$t_m = w/\sigma_m$, which immediately triggers the termination of the
shadow. However, if at time $t_f < t_m$ the main process fails, the shadow, which has completed an amount of work $w_b=\sigma_s^b * t_f$, increases its execution rate to $\sigma_s^a$ to complete the task by $t_s$. %, as depicted in Figure~\ref{fig:fail}. More specifically, the shadow completes $\sigma_s * t_f$ work by $t_f$, and finishes the remaining $(w-\sigma_s * t_f)$ work at $\sigma_a$. %The expected completion time, $T$, of the task can be easily computed by integrating $t_s(t_f) * f(t_f)$, where $f(t_f)$ is the probability of a failure occurring at time $t_f$.
The execution dynamics are depicted in Figure~\ref{fig:sync}.

\begin{figure}[!t]
	\begin{center}
		\subfigure[Main process successful completion.]
		{
			\label{fig:succ}
			\includegraphics[width=0.7\columnwidth]{Figures/succ_new.pdf}
		}
		\subfigure[Main process failure.]
		{
			\label{fig:fail}
			\includegraphics[width=0.7\columnwidth]{Figures/fail_new.pdf}
		}
	\end{center}
	%\vskip -0.25in 
	\caption{Lazy Shadowing execution dynamics.}
	\label{fig:sync}
\end{figure}

%The execution rate of the shadow before and , 

%For simplicity, we assume the maximum execution rate of a core is 1. 
In HPC, throughput consideration requires that the rate of the main process, $\sigma_m$, be set to the maximum. 
The execution rates of the shadow, $\sigma_s^b$ and $\sigma_s^a$, however, can be derived by balancing the tradeoffs between delay and energy.
%are determined based on the tolerance of the application to completion time.
For a delay-tolerant, energy-stringent application, $\sigma_s^b$ is set to 0, and the shadow starts executing only upon failure of the main process. %, for maximum energy saving. %, which guarantees additional energy is consumed only upon failure
For a delay-stringent, energy-tolerant application, the shadow starts executing at $\sigma_s^b=\sigma_m$ to guarantee the completion of the task at the specified time $t_m$, regardless of when the failure occurs. Therefore, Lazy Shadowing has the flexibility to converge to checkpointing/restart or process replication, if preferred. In addition,  
a broad spectrum of delay and energy tradeoffs in between can be explored either empirically or by using optimization frameworks for delay and energy tolerant applications.
%For elastic applications, the delay-energy product can be used to derive the execution rate  
%For other applications, $\sigma_s^b$ and $\sigma_s^a$ can be derived by balancing trade-offs between completion time and energy consumption
%however, may still be tuned to manage the trade-offs between completion time and energy consumption. Smaller $\sigma_b$ corresponds to lazier shadowing of the main process.

 %Figure~\ref{fig:succ} illustrates the execution scenario with no failure, while Figure~\ref{fig:fail} depicts the scenario where the main process fails.




%In terms of execution rates, this can be expressed as $\sigma_m=\sigma_a=1$ and $\sigma_s \le 1$.  

%There are multiple ways to control the execution rates of the processes. The most straightforward solution is to allocate a dedicated core for each process while using 
%Dynamic Voltage and Frequency Scaling (DVFS) is a commonly used power management technique 
To control the shadow's execution rate, Dynamic Voltage and Frequency Scaling (DVFS), a commonly used power management technique, can be applied. 
The effectiveness of DVFS, however, may be markedly reduced in computational platforms that exhibit saturation of the processor clock frequencies, large static power consumption, or small power dynamic range. An alternative is to collocate multiple processes on a single core, while keeping the core at maximum rate. Time sharing can then be used to achieve the desired execution rates of the processes. Given our focus on extreme-scale, multi-core computing infrastructure, we will use collocation for execution rate control.  
%Recent development in processor and memory technology which results in the saturation of the processor clock frequencies, larger static power consumption, and smaller power dynamic range can markedly reduce the effectiveness of DVFS

 %to tune the frequency of the core. An alternative is to collocate multiple processes on a core, which runs at the maximum rate, and execute the processes in a time sharing manner. In the rest of this paper, we will focus on the idea of collocation in the discussion of applying Lazy Shadowing to HPC applications.  

To execute an application with $M$ tasks, $N=M+S$ cores are required, where $M$ is a multiple of $S$. Each main process is allocated one core (referred to as main core), while $\alpha=M/S$ shadows are collocated on a core (shadow core). 
The $N$ cores are grouped into $S$ sets, which we call \emph{shadowed sets}, each containing $\alpha$ main cores and 1 shadow core.
% $\alpha$ is referred to as shadowing ratio. For example, if $M=9$ and $S=3$, then the 9 shadows share 3 cores, with every $\alpha=3$ shadows collocated on each core, as shown in Figure~\ref{fig:sc_mapping}.
This is illustrated in Figure~\ref{fig:sc_mapping}.  
Collocation has an important ramification with respect to the resilience of the system. Specifically, 
one failure can be tolerated in each shadowed set. If a shadow core fails, the main processes can continue
execution, but will have no shadows any more. On the other hand, 
to speed up a shadow 
of a failed main to the maximum rate, all other collocated shadows must be terminated. Consequently, a second failure in any of the mains in the shadowed set cannot be tolerated. After the first failure, a shadowed set becomes \emph{vulnerable}\footnote{Rejuvenation techniques, such as restarting the lost shadows from the state of current mains on spare cores, can be used to eliminate vulnerability.}. 
As will be shown in Section~\ref{anal_app_fail} and Section~\ref{sec:evaluation}, this should not be a concern as the provided reliability is more than enough.
 
\begin{figure}[!t]
  \begin{center}
    \includegraphics[width=\columnwidth]{Figures/sc_mapping.pdf}
  \end{center}
  %\vskip -0.25in 
  \caption{An example of collocation. $N=12$, $M=9$, $S=3$.}
  \label{fig:sc_mapping}
\end{figure}





\section{Extreme-scale fault tolerance framework}
\label{frame_multiple}
Enabling Lazy Shadowing for resiliency in extreme-scale computing 
brings about a number of challenges and design decisions that need to be addressed, including the applicability of this concept to a large number of 
tasks executing in parallel, the control of shadows' execution rates, and the runtime mechanisms and 
communications support to ensure efficient interaction between a 
main and its shadow.
Taking into consideration the main characteristics of compute-intensive and highly-scalable applications, we design two novel techniques, referred to as {\it shadow collocation} and {\it shadow leaping}, in order to achieve high-tolerance to failure while minimizing delay and energy consumption. In the following, we first introduce the execution and communication model of the targeted applications. We then describe the dynamics of Lazy Shadowing with shadow collocation and shadow leaping. Lastly, we discuss important aspects of its implementation. 
%In the following, we first introduce the basic Lazy Shadowing resilience model. We then %focus on resilience in extreme-scale computing environments, where a large number of %communicating tasks execute in parallel to complete a job. 
%To achieve high tolerance to failure and reduce energy consumption in these environments, we %propose \emph{leaping shadows}, an instance of Lazy Shadowing. The main property of the %leaping shadows resilience model is its ability to sustain forward progress in the presence %of failures.
%. referred to as \emph{leaping shadows}, 
%a technique referred to as \emph{shadowed set rejuvenation} to reduce application failure probability, and 
%\emph{leaping shadows}, 
%as an 
%efficient, forward-progress preserving model to achieve high tolerance to failure, while reducing energy consumption, in these environments. 

 %We also discuss the runtime design issues
%related to enabling runtime support to efficiently achieve 
%the expected levels of resilience in extreme-scale systems. 

%This is to test referring Subsection \ref{frame_single}.



%\subsection{Application characteristics}
%\label{frame_app}
%%\subsection{Computational Model and Assumptions}
We consider the class of tightly-coupled and strongly scaled applications, executing on a large scale computing infrastructure of $N$ cores~\cite{doe_ascr_exascale_2011}. In this framework, the term core represents the unit of computing resource allocation (e.g., a
CPU core, a multi-core CPU, or a cluster node)~\cite{casanova_inria_2012}. This makes our framework agnostic to the granularity of the resource allocation unit.
%The focus of our model is a tightly-coupled and strongly scaling application, which executes on a large-scale platform
%composed of $N$ cores.
%We consider the execution of a tightly-coupled and strongly scaling application, or job, on a large-scale platform
%composed of $N$ cores. The application is tightly-coupled because this is typical in the HPC applications, and strong scaling because weak-scaling applications are not expected to suit for extreme-scale computing of which the cpu/memory imbalance would further increase~\cite{doe_ascr_exascale_2011}. Similar to \cite{casanova_inria_2012}, we use the term core to indicate unit of computing resource allocation (e.g., a
%CPU core, a multi-core CPU, or a cluster node), so that our work is agnostic to the granularity
%of the platform. 
%We assume that a standard checkpointing and roll-back recovery is performed at the
%system level. At most on application process (replica) runs on one core.


The application requires $W$ units of work, and can be split arbitrarily into a set of tasks.
Barriers are used as a method of synchronization among different tasks. Assuming $M \le N$ cores are assigned to the application, the failure-free completion time of the application is $w = W/M$. However, when a core fails, the whole execution will be suspended at the next barrier until recovery is complete. 

%The job can execute on any number $M \le N$ cores. The job is strong scaling so that the time required for a failure-free execution on $M$ cores is $w = W/M$.

%Cores are subject to failures. In most cases, we do not distinguish between soft and hard failures, with the understanding that soft failures are handled via software rejuvenation (i.e., rebooting \cite{466961}) and that hard failures are handled by the replacement of the failed component with a spare, which is a commonplace approach in production systems. For simplicity, we adopt the fail-stop fault model, where a core stops execution once a failure occurs and the failure can be detected by other cores \cite{gartner_faults_1999,cristian_comm_1991}. Since we consider tightly coupled parallel jobs, all $M$ cores operate synchronously. When a core fails, the whole execution is suspended until the failure is recovered. We assume that core failures are independent and identically distributed (i.i.d.). In the real world, instead, failures are bound to be correlated. Obtaining theoretical results for non-i.i.d. failures is beyond the scope of this work. But note that one cause of correlation is the hierarchical structure of computing platforms (each rack comprises compute nodes, each compute node comprises processors, and each processor comprises cores), which leads to simultaneous failures of a group of cores. Our work applies to such failures since a group of failures can be treated as multiple individual failures that happen at the same time and their recovery can be carried out in parallel.

Cores are subject to failures. In our model, we do not distinguish between soft and hard failures. We further assume that soft failures are handled via software rejuvenation (i.e., rebooting \cite{466961}), while hard failures are handled by replacing the failed components with spares. We adopt the fail-stop fault model, whereby a core stops executing upon failure and failures are detected by other non-failing cores \cite{gartner_faults_1999,cristian_comm_1991}. When a core fails, the whole execution is suspended until recovery is complete. 

%%%%%%%%%%%%%%%%%%%%%%%%%%%%%%rethinking%%%%%%%%%%%%%%%%%%%%%%
We assume that core failures are independent and identically distributed (i.i.d.). In the real world, instead, failures are bound to be correlated. Obtaining theoretical results for non-i.i.d. failures is beyond the scope of this work. But note that one cause of correlation is the hierarchical structure of computing platforms (each rack comprises compute nodes, each compute node comprises processors, and each processor comprises cores), which leads to simultaneous failures of a group of cores. Our work applies to such failures since a group of failures can be treated as multiple individual failures that happen at the same time and their recovery can be carried out in parallel.

%Since we consider tightly coupled parallel jobs, all q cores operate syn- chronously. These cores execute the same amount of work W(q) in parallel, chunk by chunk. The total time (on one core) to execute a chunk of dura- tion, or size, ω and then checkpointing it, is ω + C(q)

%\subsection{Leaping Shadows}



%%%%%%%%%%%%%%%%%%%%%%%%%%%%%%%%%%%%%%%%%%%%%%%%%%%%%%%%%%%%%%%%%%%%%%%%%%%%%%%%%%

%%We consider the class of strongly-scaled applications, and use $W$ to denote the size of an application workload~\cite{doe_ascr_exascale_2011}. The workload is split  arbitrarily into a set of $N$ tasks, whose synchronization is achieved using barriers. Assuming that each task is assigned to execute on a core at a maximum speed $\sigma=1$, the failure-free completion time of the application is $w = W/N$. 
Let $M_i$ denote the main process executing the $i^{th}$ task $T_i$, and $S_i$ its associated shadow.
When a failure occurs, the non-failing tasks continue executing until they reach the synchronization barrier. The tasks remain suspended until the shadow associated with the failing task reaches its synchronization barrier.


failure recovery is achieved.  To reduce time to recovery,  we run M shadows simulataneously with the main processes. 
%According to the computational model in Section~\ref{sec:com_model}, an application's workload is split into $M$ parallel tasks where each task is associated with a main process and a shadow.

 The execution of the application can be carried out by simultaneously running all main and shadow processes 
on %$2M$ cores, whereby the main processes execute at the maximum rate while the associated shadows execute at a fraction, $\sigma_s$, of the 
%maximum rate using DVFS. 
%To simultaneously run all the main and shadow processes,
%ne may choose to use $2M$ cores, with $M$ executing the main tasks at the maximum rate and $M$ \emph{lazily} executing the shadows at a fraction, $\sigma_s$, of the 
%maximum rate using DVFS.  
%An alternative method is to use 
$M+S$ cores, where $M$ is a multiple of $S$ and $M+S=N$, all executing at the maximum rate. $M$ of these cores are allocated to the main processes while the remaining $S$ cores are shared among their associated shadows. Based on this method, each main process is allocated one core, while $\alpha=M/S$ shadows are collocated on a single core. $\alpha$ is referred to as the main to shadow ratio.
For example, if $M=9$ and $S=3$, then the 9 shadows execute on 3 cores, with every $\alpha=3$ shadows executing on a core (Figure~\ref{fig:sc_mapping}).
%only $M+S$ cores can be used, whereby the main processes execute on the $M$ cores. And the $M$ shadows are divided into $S$ clusters with of $M/S$ shadows are colocated on a single core, operating at the maximum rate.
%where $M$ is a multiple of $S$, and collocate $M/S$ shadows on each of the $S$ cores, while executing all the cores at the maximum speed. 
%We use $\alpha$ to denote the main to shadow allocation ratio. %, and for simplicity, we assume the maximum execution rate of a core is 1. %Ignoring the overhead of context switching, the two alternatives lead to the same expected execution time, but different power and energy consumption. Specifically, the $2M$-cores scheme consumes more static but less dynamic power/energy than the $M+S$ cores alternative. The ratio between the system static and dynamic power consumption determines which alternative overall consumes less power/energy. 

%In the rest of the paper, we will focus on 
%shadow collocation as the meain the rest of our discussion 
%of the main ideas, concepts and system design. We will assume that all cores in the system execute at maximum speed, with every $\alpha=M/S$ shadows collocated on a core. 
%In HPC, throughput consideration requires that the rate of the main task, $\sigma_m$, and the shadow after failure, $\sigma_a$, be set to the maximum. The execution rate of the shadow before failure, $\sigma_b$, however, may still be used to manage the trade-offs between completion time and energy consumption. Smaller $\sigma_b$ corresponds to lazier shadowing of the main process. 
%In terms of execution rates, this can be expressed as $\sigma_m=\sigma_a=1$ and $\sigma_s \le 1$. 
  
Collocation of $\alpha$ shadows on a core has an important ramification with respect to the resilience of the system. Specifically, to speed up a shadow 
of a failed main to the maximum rate, all other collocated shadows must be terminated. Consequently, a second failure in any of the mains of the terminated shadows cannot be tolerated. In other words, the $M+S$ cores are grouped into $S$ sets, which we call \emph{shadowed sets}, each containing $\alpha+1$ cores with $\alpha$ mains executing on $\alpha$ cores (referred to as main cores) and their corresponding $\alpha$ shadows collocated on one core (referred to as shadow core). Each shadowed set can tolerate a failure in any of its cores, since failure of a main core would be recovered by the shadow core and failure of the shadow core will not affect any mains. After the first failure in a shadowed set, the set is called \emph{vulnerable} because it cannot tolerate another failure. %In the following subsection, we will discuss a rejuvenation scheme that deals with vulnerable shadowed sets.

\begin{figure}[!t]
	\begin{center}
		\includegraphics[width=\columnwidth]{Figures/sc_mapping.pdf}
	\end{center}
	\vskip -0.25in 
	\caption{An example of 3 shadowed sets with $\alpha=3$.}
	\label{fig:sc_mapping}
\end{figure}

%Lazy Shadowing provides the basis for the design of efficient energy- and power-aware fault-tolerance solutions for extreme-scale computing environments. %The resulting model, referred to as {\it leaping shadows}, 


%Lazy shadowing provides the basis for the design of efficient energy- and power-aware fault-tolerance solution for extreme-scale computing environments. The resulting model, referred to as {\it leaping shadows}, takes into consideration the main characteristics of compute-intensive and highly-scalable applications to achieve high-tolerance to failure, while minimizing energy consumption. In the following, we first introduce the execution and communication model of the targeted applications. We then describe the dynamics of the {\it leaping shadows} resilience model. Lastly, we discuss important aspects of its implementation. 
\subsection {Application model}
\label{sec:app_model}

We consider the class of compute-intensive and strongly-scaled applications, executing on a large-scale multi-core computing infrastructure~\cite{doe_ascr_exascale_2011}. %Communication between cores is achieved using low-latency, high-bandwidth interconnect networks, such as Infiniband. 
We use $W$ to denote the size of an application workload, and assume that the workload is split arbitrarily into a set of tasks, $\tau$, which execute in parallel and are synchronized using barriers. Given the prominence of MPI in HPC environments, we assume message passing as the communication mechanism between tasks. %Based on this model, each pair of communicating tasks is associated with a logical first-in-first-out (FIFO) channel, which guarantees ordered delivery of messages.
The execution is divided into a set of phases by synchronization barriers. 
Assuming the maximum execution rate is $\sigma_{max}=1$, the failure-free completion time of the application is $W/|\tau|$

 . %If a failure occurs, however, the non-failing tasks would become idle when they reach their synchronization barrier. These tasks remain idle until the failure recovery is complete. 
%resulting in performance hiccups~\cite{muller2010}, especially for tightly-coupled applications. 


\subsection{Shadow collocation}

The execution of an application can be carried out by simultaneously running all tasks, each with a pair of main and shadow processes. Let $m_i$ denote the main process executing the $i^{th}$ task $task_i$, and $s_i$ its associated shadow.
%There are multiple ways to control the execution rates of the processes. The most straightforward solution is to allocate a dedicated core for each process while using 
%Dynamic Voltage and Frequency Scaling (DVFS) is a commonly used power management technique 
We use the term core to represent the resource allocation unit (e.g., a
CPU core, a multi-core CPU, or a cluster node), so that our algorithm is agnostic to the
granularity of the hardware platform~\cite{casanova_inria_2012}. Each main process executes on one core exclusively to achieve maximum throughput.  
To control the shadow's execution rate, Dynamic Voltage and Frequency Scaling (DVFS) can be applied while each shadow also resides on one core exclusively~\cite{mills_2014_icnc,cui_en7085151,cui_2014_closer}. 
The effectiveness of DVFS, however, may be markedly 
limited by the granularity of voltage control, the number of frequencies available, and the negative effects on 
reliability~\cite{chandra2008defect}.
%reduced in computational platforms that exhibit saturation of the processor clock frequencies, large static power consumption, or small power dynamic range. 
An alternative is to collocate multiple shadows on a single core, while keeping the core at maximum rate. Time sharing can then be used to achieve the desired execution rates of the processes. %Given our focus on extreme-scale, multi-core computing infrastructure, we will use collocation for execution rate control.  
%Recent development in processor and memory technology which results in the saturation of the processor clock frequencies, larger static power consumption, and smaller power dynamic range can markedly reduce the effectiveness of DVFS

 %to tune the frequency of the core. An alternative is to collocate multiple processes on a core, which runs at the maximum rate, and execute the processes in a time sharing manner. In the rest of this paper, we will focus on the idea of collocation in the discussion of applying Lazy Shadowing to HPC applications.  


To execute an application with $M$ tasks, $N=M+S$ cores are required for Lazy Shadowing, where $M$ is a multiple of $S$. Each main is allocated one core (referred to as main core), while $\alpha=M/S$ shadows are collocated on a core (shadow core). 
The $N$ cores are grouped into $S$ sets, which we call \emph{shadowed sets}, each containing $\alpha$ main cores and 1 shadow core.
% $\alpha$ is referred to as shadowing ratio. For example, if $M=9$ and $S=3$, then the 9 shadows share 3 cores, with every $\alpha=3$ shadows collocated on each core, as shown in Figure~\ref{fig:sc_mapping}.
This is illustrated in Figure~\ref{fig:sc_mapping}.  

Collocation has an important ramification with respect to the resilience of the system. Specifically, 
one failure can be tolerated in each shadowed set. If a shadow core fails, the mains can continue
execution, but will have no shadows any more. On the other hand, 
to speed up a shadow 
of a failed main to the maximum rate, all other collocated shadows must be terminated. Consequently, a second failure in any of the mains in the shadowed set cannot be tolerated. After the first failure, a shadowed set becomes \emph{vulnerable}\footnote{Rejuvenation techniques, such as restarting the lost shadows from the state of current mains on spare cores, can be used to eliminate vulnerability.}. 
As will be shown in Section~\ref{anal_app_fail}, this should not be a concern as the provided reliability is more than enough.
 
\begin{figure}[!t]
  \begin{center}
    \includegraphics[width=\columnwidth]{Figures/sc_mapping.pdf}
  \end{center}
  %\vskip -0.25in 
  \caption{An example of collocation. $N=12$, $M=9$, $S=3$.}
  \label{fig:sc_mapping}
\end{figure}

As the shadows execute at a lower rate, failures will incur delay for recovery. This problem deteriorates as dependencies incurred by messages and synchronization barriers would propagate the delay of one task to other tasks.  
The main objective of leaping shadows is to minimize the delay induced by failures, 
%mitigate the impact of performance hiccups, 
and ensure forward progress by opportunistically rolling-forward the shadows during failure recovery. In Section~\ref{anal_time}, we will show that the delay is well bounded even for tightly-coupled applications.


\subsection {Leaping shadows}
\label{sec:leaping_shadows}

In the absence of failure, the behavior of a main  and its leaping shadow is identical to the behavior depicted in Figure \ref{fig:sync}. Assuming a failure occurrence at time $t_f$, Figure~\ref{fig:leap} shows the concept of shadow leaping. 
%Figure~\ref{fig:jump1} depicts the execution dynamics of the failing main and its associated shadow. 
Upon failure of a main process, its associated shadow speeds up to minimize the impact of failure recovery on the application's progress, as illustrated in Figure~\ref{fig:jump1}. 
%Figure~\ref{fig:jump2} illustrates the behavior of the remaining main processes and their associated shadows. 
At the same time, as shown in Figure~\ref{fig:jump2}, the remaining main processes continue execution towards the barrier at time $t_{sync}$, and then become idle until $t_r$, when the shadow of the failed main process also reaches the barrier. %It is worth noting that, given the tightly-coupled and strongly-scaled nature of the application, synchronization points occur frequently. Consequently, the time between synchronization points is very small relative to the total execution time of the application. Therefore, if one main, $m_i$, fails at 
%time $t_f$, the remaining main processes will reach their synchronization point, shortly after the failure, specifically at at time $t_{sync} = t_f +\epsilon$, 
%where $\epsilon \approxeq 0$. 
Leaping shadows opportunistically takes advantage of this idle time to {\it leap forward} the shadows by copying state from their main processes, so that  
all processes, including shadows, can resume execution from a consistent synchronization point afterwards. This process continues until the completion of all tasks. Forward leaping increases the shadow's rate of progress, at a minimal energy cost. Consequently, it reduces significantly the likelihood of a shadow falling excessively behind, thereby ensuring fast recovery while minimizing energy consumption.



\begin{figure}[!t]
	\begin{center}
        \subfigure[Faulty task behavior.]
		{
			\label{fig:jump1}
			\includegraphics[width=0.7\columnwidth]{Figures/jump1}
		}
		\subfigure[Non-faulty task behavior.]
		{
			\label{fig:jump2}
			\includegraphics[width=0.7\columnwidth]{Figures/jump2}
		}
	\end{center}
	%\vskip -0.25in
	\caption{The illustration of shadow leaping.}
	\label{fig:leap}
\end{figure}

%Collocation is used to control the execution rates. To execute an application with $M$ tasks, $N=M+S$ cores are required, where $M$ is a multiple of $S$. Each main process is allocated one core (referred to as main core), while $\alpha=M/S$ shadows are collocated on a core (shadow core). 
%The $N$ cores are grouped into $S$ sets, which we call \emph{shadowed sets}, each containing $\alpha$ main cores and 1 shadow core.
%% $\alpha$ is referred to as shadowing ratio. For example, if $M=9$ and $S=3$, then the 9 shadows share 3 cores, with every $\alpha=3$ shadows collocated on each core, as shown in Figure~\ref{fig:sc_mapping}.
%This is illustrated in Figure~\ref{fig:sc_mapping}.  
%Collocation has an important ramification with respect to the resilience of the system. Specifically, 
%one failure can be tolerated in each shadowed set. If a shadow core fails, the main processes can continue
%execution, but will have no shadows any more. On the other hand, 
%to speed up a shadow 
%of a failed main to the maximum rate, all other collocated shadows must be terminated. Consequently, a second failure in any of the mains in the shadowed set cannot be tolerated. After the first failure, a shadowed set becomes \emph{vulnerable}\footnote{Rejuvenation techniques, such as restarting the lost shadows from the state of current mains on spare cores, can be used to eliminate vulnerability.}. 
%
%\begin{figure}[!t]
%  \begin{center}
%    \includegraphics[width=\columnwidth]{Figures/sc_mapping.pdf}
%  \end{center}
%  %\vskip -0.25in 
%  \caption{An example of collocation. $N=12$, $M=9$, $S=3$.}
%  \label{fig:sc_mapping}
%\end{figure}

%Collocation also increases memory requirement. However, this is not intrinsic to Lazy Shadowing, as checkpointing/restart also requires additional memory capacity. We acknowledge the fact that compute kernels in existing HPC environments were simplified significantly by placing a number of restrictions, including eliminating virtual paging and limiting support for OS (Linux) to a handful of system calls. It became clear, however, that strategies designed to work around the capabilities of the hardware cannot scale to extreme-scale computing. Consequently, the research focus has been on new paradigms focused on co-design of hardware with system software to leverage the advantages associated with dynamic, asynchronous mechanisms, such as demand paging and cache tuning, against the design principles and choices of current HPC systems. Support of efficient demand paging, through co-design, is particularly critical as it is expected that the data of future exascale applications may not fit entirely in memory. %Finally, in comparison to checkpointing/restart and process replication, Lazy Shadowiing has the capability to control memory usage, based on the nature of failure and existing memory capacity, albeit at a loss of performance. 



\begin{algorithm}[t]
  \SetKwInOut{Input}{input}
  \SetKwInOut{Output}{output}
  \caption{Lazy Shadowing}
  \Input{$W, M, S$}
  \Output{Application execution status}
  \BlankLine
  split $W$ into $M$ tasks\; \nllabel{line:split} 
  assign $M$ tasks to $S$ shadowed sets\; \nllabel{line:cluster} 
  %$T_l \leftarrow T+t_{now}$\;%, $C_{vul} \gets 0$
  %start $m_i$ and $s_i$ for each $task_i$\;
  start a pair of main and shadow for each task\;
    \While{execution not done}
    {
  %      \If{failure detected in $ss_j$} %\nllabel{line:if_start_1} 
        \If{failure detected in a shadowed set}
        {
            \nllabel{line:if_start_1} 
            %\eIf{$ss_j$ is vulnerable}
            \eIf{the shadowed set is vulnerable}
            {
                notify ``Application failure"\;
                terminate all mains and shadows\;
                repair all failures\;
                restart execution\; %\Comment{re-execution}
            }
            {
                mark the shadowed set as vulnerable\;
                %\State $C_{vul} \gets C_{vul} + 1$
                %\If{$C_{vul} == V$}
                %    \State perform shadowed set rejuvenation   
                %\EndIf 
                \If{failure happened to a main} 
                {
                    promote its shadow to new main\;
                    perform shadow leaping\; %failure induced shadow leaping
                    %$T_l \leftarrow T+t_{now}$\;
                }
            }
        }  
        \nllabel{line:if_end_1}
        %\If{$t_{now} \ge T_l$} %
        %{
        %    \nllabel{line:if_start_2} 
        %    perform shadow leaping\;
        %    $T_l \leftarrow T+t_{now}$\;
        %    \nllabel{line:if_end_2}
        %} % 
        %\nllabel{line:if_end_2}
    }
    output ``Application completes"\;
  \label{al:ls}
\end{algorithm}

The steps of applying Lazy Shadowing with leaping shadows are depicted in Algorithm 1.
To use $M+S$ cores to execute an application, the total workload is split into $M$ parallel tasks (line~\ref{line:split}), %, which are executed simultaneously by $M$ main processes and $M$ shadow processes. 
 which are then assigned to $S$ shadowed sets, each with $\alpha=M/S$ cores for $\alpha$ main processes and 1 core for all the associated shadow processes (line 2).  
%The $M$ shadows are then clustered into $S$ groups, each containing $\alpha=M/S$ shadows. 
The execution starts by simultaneously running all the main and shadow processes (line 3).
During the execution,
the system runs a failure monitor (out of scope of this work) that triggers corresponding actions when a failure is detected (line~\ref{line:if_start_1} to~\ref{line:if_end_1}). %A failure may trigger different actions, depending on its type and precedence with respect to other failures.  A shadowed set becomes {\it vulnerable} after the occurrence of the first failure in the set. 
A failure occurring in a vulnerable shadowed set (e.g., $ss_j$) results in an application failure %. In response, the system terminates all running 
%processes, initiates a recovery phase, either by rebooting or replacing failing cores,  and restarts execution (line 8 to 10). %(this assumes that checkpointing is not used). 
 and forces a re-execution (line 7 to 10).
On the other hand, failure in a non-vulnerable shadowed set
does not translate into an application failure, but would mark the shadowed set in question as vulnerable (line 12). In this case, failure of a main process has different impact from that of a  shadow process.  While a shadow 
failure does not impact the normal execution and thus can be ignored, failure of a main process %forces the remaining main processes to suspend execution after they reach their synchronization point.  The shadow process, $s_k$, associated with the failing process, $m_k$,  becomes the primary process of the associated task and increases its execution to the maximum rate 
(e.g., $m_k$) triggers promotion of its shadow process, $s_k$, to a new main process (line 14). Simultaneously, a shadow leaping is undertaken by all remaining shadows to align their states with those of their associated mains (line 15).  
This process continues until all tasks of the application are successfully completed.

\subsection{Implementation issues}

%\subsection{Shadowed set rejuvenation}
%\label{frame_reju}
%The proposed Lazy Shadowing scheme can tolerate faults which are repairable by rebooting or reconfiguration, referred to as soft faults, and faults which cannot be repaired by rebooting or reconfiguration, referred to as hard 
faults. Monitors that detect hard faults, such as memory flip, bus error 
and latch error, or soft faults, such as deadlock detection, buffer overflow and protection violation, typically interrupt the application to initiate 
the recovery process. The process of recovery from transient or permanent faults is the same and necessitates a mechanism for detecting a fault 
in a main task, M(i) and notifying other tasks in the system so that (i) the shadows sharing a core with S(i) are terminated, thus allowing S(i) to execute at the maximum rate, and (ii) all the shadows that are not in the faulty shadowed set leap to the state of their mains. 

As described earlier, the recovery from a fault in a shadowed set leaves the set vulnerable and any more faults in a vulnerable set will result in a system failure. Although for large systems and small S the probability of having a second fault in a vulnerable set is low, some provision should be taken to rejuvenate the system when a relatively large number of its shadowed sets are vulnerable. 

We propose to invoke \emph{shadowed set rejuvenation} after a specific number of faults, which is determined by the system size, the shadowed set size, and the required resilience.
Rejuvenation reconfigures the system such that none of its shadowed sets are vulnerable. Unlike recovery from a fault in a shadowed set, rejuvenation is different when the faults are transient soft when the faults are hard. In the case of soft faults, rejuvenation can be accomplished by rebooting the failed cores, and restarting the lost shadows (both the ones promoted to mains and the ones terminated) from the state of current mains. And in case of hard faults, it is possible to restart the lost shadows after replacing the failed ones with spare ones. This will restore a vulnerable shadowed set to its original configuration. %For example, rejuvenation should restore the systems shown in Figure~\ref{fig:layout2} and Figure~\ref{fig:layout3} to the one shown in Figure~\ref{fig:layout1}.

%When failures are permanent, rejuvenation may be challenging if rebooting or reconfiguration can no longer be
%used to recover failed components.
%Specifically, in the absence of spare components (if the system is not over-provisioned),
%rejuvenation can only 
%be accomplished by distributing the main processes of a vulnerable set 
%to other {\bf non-vulnerable} shadowed sets. The shadow of the vulnerable 
%set must also be relocated to the shadows of the non-vulnerable set. 
%As a result,  the total number of shadowed sets decreases, but 
%the size of some shadowed sets increases.  In Figure~\ref{fig:reju}, we show a possible rejuvenated configuration assuming that the failure of the cores executing $M(1)$ and $M(14)$ in Figure~\ref{fig:layout2} is permanent. In this restored configuration, the number of shadowed sets is reduced from 8 to 6, with four sets containing three mains each and two sets containing two mains each. Rejuvenating the vulnerable configuration of Figure~\ref{fig:layout3} after a permanent socket failure is more complex but follows the same basic principle.

%\begin{figure*}[ht]
%	\begin{center}
%		\includegraphics[width=\textwidth]{figures/reju.pdf}
%	\end{center}
%\vskip -0.25in
%	\caption{Shadowed set rejuvenation of the vulnerable sets resulting from permanent faults.}
%	\label{fig:reju}
%\end{figure*}


%%%%%%%%%%%%%%%%%%%%%%%%%%%%%%%%%%%%%%%%%%%%%%%%%%%%%%%%%%%%%%%%%%%%%%%%%%%%%%%%%%
We implemented an Open MPI based prototype of Lazy Shadowing, which can be used to execute existing HPC workloads without any change of user code. Since the focus of this paper is to introduce
algorithmic perspectives of the Lazy Shadowing paradigm by discussing novel concepts of shadow collocation and forward leaping, we only give a brief discussion of the implementation issues. 

State consistency is required both during normal execution and following a failure of a main process to roll-forward the shadows. During normal execution, shadows remain mute, in the sense that 
all outgoing messages from shadows are suppressed. 
A shadow process, however, will typically lag behind its main process during execution. Therefore, it is necessary to ensure that the shadow's state is consistent with that of its associated main. %, to successfully complete its associated task in case of failure. 
To this end, a message-logging protocol is used, % to ensure consistency~\cite{Marz}. These protocols 
which typically uses a minimum amount of meta-information to store and replicate the non-deterministic decisions~\cite{Marz}. %in the execution of an application.  These meta-data, also called determinants, are exchanged through system-level messages. 

To provide correct recovery after failure, 
a mechanism is required to guarantee that every shadow process follows the same computation and communication steps as its main process. 
After a main process $m_i$ fails, $s_i$ will take over $m_i$'s role to recover from this failure. If there are other shadows sharing the same core with $s_i$, they will be terminated and $s_i$ will start consuming the messages in its receiver-side message log at a faster speed. The message logging protocol will ensure that shadow $s_i$ reaches a consistent state with the rest of the system. 

Upon failure of a main process, shadow processes will update their address space to ``catch up" with their associated non-failing main processes. A technology, such as remote direct memory access (RDMA), can be used to roll-forward the state of the shadow to be consistent with that of its associated main. Rather than copying data to the buffers of the operating system, RDMA allows to transfer data directly from the main process to its shadow. The zero-copy feature of RDMA considerably reduces latency, thereby enabling fast transfer of data between the main and its shadow.

 




%\section{\uppercase{Lazy Shadowing Computational Model}}
%\label{sec:com_model}
%
%associated with three attributes, i.e., workload $w$, execution rate $\sigma$, and completion time $t$.
%This makes our framework agnostic to the granularity of the resource allocation unit.
%Cores are subject to failures. We assume that core failures are independent and identically distributed (i.i.d.). We do not distinguish between soft and hard failures, assuming that soft failures are handled via software rejuvenation (i.e., rebooting \cite{466961}), while hard failures are handled by replacing the failed components with spares. We adopt the fail-stop fault model, whereby a core stops executing upon failure and failures are detected by other non-failing cores \cite{gartner_faults_1999,cristian_comm_1991}. %When a core fails, the whole execution is suspended until recovery is complete. 

%%%%%%%%%%%%%%%%%%%%%%%%%%%%%%rethinking%%%%%%%%%%%%%%%%%%%%%%
%We assume that core failures are independent and identically distributed (i.i.d.). In the real world, instead, failures are bound to be correlated. Obtaining theoretical results for non-i.i.d. failures is beyond the scope of this work. But note that one cause of correlation is the hierarchical structure of computing platforms (each rack comprises compute nodes, each compute node comprises processors, and each processor comprises cores), which leads to simultaneous failures of a group of cores. Our work applies to such failures since a group of failures can be treated as multiple individual failures that happen at the same time and their recovery can be carried out in parallel.


%We use the term core to represent the computing resource allocation unit%(e.g., a
%CPU core, a multi-core CPU, or a cluster node)
%~\cite{casanova_inria_2012}. 
%We further use $P(\sigma$, $w$, $t)$ to denote a process executing at rate $\sigma$ to complete a workload $w$ by time $t$.
The basic tenet of Lazy Shadowing is the concept of shadowing, whereby each process is associated with a {\it lazy} replica that executes at a reduced rate. We only use one replica since the probability that failure occurs to both the original process and the replica is negligible~\cite{casanova_inria_2012}. %This guarantees that if one process fails, the other one can still complete the task. 
If necessary, however, this can be easily extended to use a suit of replicas. Assuming a single failure, Lazy Shadowing can be described as follows:
\begin{itemize}
	\item A main process, $P_m(\sigma_m$, $w$, $t_m)$, that executes at the rate of $\sigma_m$ to complete a task of size $w$ at time $t_m$.
	\item A shadow process, $P_s(<\sigma_s^b$ , $\sigma_s^a>$, $w$, $t_s)$, that initially executes at $\sigma_s^b$, and increases to $\sigma_s^a$ if its main process fails, to complete a task of size $w$ at time $t_s$.% where $\sigma_s^b$ represents the execution rate before failure, $\sigma_s^a$ the execution rate after failure, $w$ is the task size, and $t_s$ is the completion time.%, which has the same workload $w$. It starts execution simultaneously with the main process at rate $\sigma_b  \le \sigma_m$, but on a different core. Upon failure of the main process, the shadow process will be designated as the new main process, with its rate switched to $\sigma_a$ to catch up. In this case, the completion time is denoted as $t_s$. 
\end{itemize}


We use the term core to represent the resource allocation unit (e.g., a
CPU core, a multi-core CPU, or a cluster node), so that our algorithm is agnostic to the
granularity of the hardware platform~\cite{casanova_inria_2012}.
Furthermore, we adopt the
fail-stop failure model, where a core stops execution once a failure
occurs and failure can be detected by other
processes~\cite{gartner_faults_1999,cristian_comm_1991}.
In order to deal with both permanent and temporary failures, the shadow process starts simultaneously with its associated main process, on a different node. Lazy Shadowing is able to tolerate any failure confined to a single node, including socket failure, CPU logical errors, bus errors, errors in the attached accelerators (e.g., GPUs), and even memory bit flips that exceed ECC's capacity. 

Initially, the main process executes at rate $\sigma_m$, while the shadow executes at $\sigma_s^b \le \sigma_m$. %To avoid simultaneous failure, they are deployed on different cores.
%with the main running at  
%a rate $\sigma_m$ and the lazy shadow at a lower rate $\sigma_s \le \sigma_m$. 
In the absence of failure, the main process completes execution at time 
$t_m = w/\sigma_m$, which immediately triggers the termination of the
shadow. However, if at time $t_f < t_m$ the main process fails, the shadow, which has completed an amount of work $w_b=\sigma_s^b * t_f$, increases its execution rate to $\sigma_s^a$ to complete the task by $t_s$. %, as depicted in Figure~\ref{fig:fail}. More specifically, the shadow completes $\sigma_s * t_f$ work by $t_f$, and finishes the remaining $(w-\sigma_s * t_f)$ work at $\sigma_a$. %The expected completion time, $T$, of the task can be easily computed by integrating $t_s(t_f) * f(t_f)$, where $f(t_f)$ is the probability of a failure occurring at time $t_f$.
The execution dynamics are depicted in Figure~\ref{fig:sync}.

\begin{figure}[!t]
	\begin{center}
		\subfigure[Main process successful completion.]
		{
			\label{fig:succ}
			\includegraphics[width=0.7\columnwidth]{Figures/succ_new.pdf}
		}
		\subfigure[Main process failure.]
		{
			\label{fig:fail}
			\includegraphics[width=0.7\columnwidth]{Figures/fail_new.pdf}
		}
	\end{center}
	%\vskip -0.25in 
	\caption{Lazy Shadowing execution dynamics.}
	\label{fig:sync}
\end{figure}

%The execution rate of the shadow before and , 

%For simplicity, we assume the maximum execution rate of a core is 1. 
In HPC, throughput consideration requires that the rate of the main process, $\sigma_m$, be set to the maximum. 
The execution rates of the shadow, $\sigma_s^b$ and $\sigma_s^a$, however, can be derived by balancing the tradeoffs between delay and energy.
%are determined based on the tolerance of the application to completion time.
For a delay-tolerant, energy-stringent application, $\sigma_s^b$ is set to 0, and the shadow starts executing only upon failure of the main process. %, for maximum energy saving. %, which guarantees additional energy is consumed only upon failure
For a delay-stringent, energy-tolerant application, the shadow starts executing at $\sigma_s^b=\sigma_m$ to guarantee the completion of the task at the specified time $t_m$, regardless of when the failure occurs. Therefore, Lazy Shadowing has the flexibility to converge to checkpointing/restart or process replication, if preferred. In addition,  
a broad spectrum of delay and energy tradeoffs in between can be explored either empirically or by using optimization frameworks for delay and energy tolerant applications.
%For elastic applications, the delay-energy product can be used to derive the execution rate  
%For other applications, $\sigma_s^b$ and $\sigma_s^a$ can be derived by balancing trade-offs between completion time and energy consumption
%however, may still be tuned to manage the trade-offs between completion time and energy consumption. Smaller $\sigma_b$ corresponds to lazier shadowing of the main process.

 %Figure~\ref{fig:succ} illustrates the execution scenario with no failure, while Figure~\ref{fig:fail} depicts the scenario where the main process fails.




%In terms of execution rates, this can be expressed as $\sigma_m=\sigma_a=1$ and $\sigma_s \le 1$.  

%There are multiple ways to control the execution rates of the processes. The most straightforward solution is to allocate a dedicated core for each process while using 
%Dynamic Voltage and Frequency Scaling (DVFS) is a commonly used power management technique 
To control the shadow's execution rate, Dynamic Voltage and Frequency Scaling (DVFS), a commonly used power management technique, can be applied. 
The effectiveness of DVFS, however, may be markedly reduced in computational platforms that exhibit saturation of the processor clock frequencies, large static power consumption, or small power dynamic range. An alternative is to collocate multiple processes on a single core, while keeping the core at maximum rate. Time sharing can then be used to achieve the desired execution rates of the processes. Given our focus on extreme-scale, multi-core computing infrastructure, we will use collocation for execution rate control.  
%Recent development in processor and memory technology which results in the saturation of the processor clock frequencies, larger static power consumption, and smaller power dynamic range can markedly reduce the effectiveness of DVFS

 %to tune the frequency of the core. An alternative is to collocate multiple processes on a core, which runs at the maximum rate, and execute the processes in a time sharing manner. In the rest of this paper, we will focus on the idea of collocation in the discussion of applying Lazy Shadowing to HPC applications.  

To execute an application with $M$ tasks, $N=M+S$ cores are required, where $M$ is a multiple of $S$. Each main process is allocated one core (referred to as main core), while $\alpha=M/S$ shadows are collocated on a core (shadow core). 
The $N$ cores are grouped into $S$ sets, which we call \emph{shadowed sets}, each containing $\alpha$ main cores and 1 shadow core.
% $\alpha$ is referred to as shadowing ratio. For example, if $M=9$ and $S=3$, then the 9 shadows share 3 cores, with every $\alpha=3$ shadows collocated on each core, as shown in Figure~\ref{fig:sc_mapping}.
This is illustrated in Figure~\ref{fig:sc_mapping}.  
Collocation has an important ramification with respect to the resilience of the system. Specifically, 
one failure can be tolerated in each shadowed set. If a shadow core fails, the main processes can continue
execution, but will have no shadows any more. On the other hand, 
to speed up a shadow 
of a failed main to the maximum rate, all other collocated shadows must be terminated. Consequently, a second failure in any of the mains in the shadowed set cannot be tolerated. After the first failure, a shadowed set becomes \emph{vulnerable}\footnote{Rejuvenation techniques, such as restarting the lost shadows from the state of current mains on spare cores, can be used to eliminate vulnerability.}. 
As will be shown in Section~\ref{anal_app_fail} and Section~\ref{sec:evaluation}, this should not be a concern as the provided reliability is more than enough.
 
\begin{figure}[!t]
  \begin{center}
    \includegraphics[width=\columnwidth]{Figures/sc_mapping.pdf}
  \end{center}
  %\vskip -0.25in 
  \caption{An example of collocation. $N=12$, $M=9$, $S=3$.}
  \label{fig:sc_mapping}
\end{figure}





%%\section{\uppercase{Resilience Framework}}
%%\label{sec:framework}
%Enabling Lazy Shadowing for resiliency in extreme-scale computing 
brings about a number of challenges and design decisions that need to be addressed, including the applicability of this concept to a large number of 
tasks executing in parallel, the control of shadows' execution rates, and the runtime mechanisms and 
communications support to ensure efficient interaction between a 
main and its shadow.
Taking into consideration the main characteristics of compute-intensive and highly-scalable applications, we design two novel techniques, referred to as {\it shadow collocation} and {\it shadow leaping}, in order to achieve high-tolerance to failure while minimizing delay and energy consumption. In the following, we first introduce the execution and communication model of the targeted applications. We then describe the dynamics of Lazy Shadowing with shadow collocation and shadow leaping. Lastly, we discuss important aspects of its implementation. 
%In the following, we first introduce the basic Lazy Shadowing resilience model. We then %focus on resilience in extreme-scale computing environments, where a large number of %communicating tasks execute in parallel to complete a job. 
%To achieve high tolerance to failure and reduce energy consumption in these environments, we %propose \emph{leaping shadows}, an instance of Lazy Shadowing. The main property of the %leaping shadows resilience model is its ability to sustain forward progress in the presence %of failures.
%. referred to as \emph{leaping shadows}, 
%a technique referred to as \emph{shadowed set rejuvenation} to reduce application failure probability, and 
%\emph{leaping shadows}, 
%as an 
%efficient, forward-progress preserving model to achieve high tolerance to failure, while reducing energy consumption, in these environments. 

 %We also discuss the runtime design issues
%related to enabling runtime support to efficiently achieve 
%the expected levels of resilience in extreme-scale systems. 

%This is to test referring Subsection \ref{frame_single}.



%\subsection{Application characteristics}
%\label{frame_app}
%%\subsection{Computational Model and Assumptions}
We consider the class of tightly-coupled and strongly scaled applications, executing on a large scale computing infrastructure of $N$ cores~\cite{doe_ascr_exascale_2011}. In this framework, the term core represents the unit of computing resource allocation (e.g., a
CPU core, a multi-core CPU, or a cluster node)~\cite{casanova_inria_2012}. This makes our framework agnostic to the granularity of the resource allocation unit.
%The focus of our model is a tightly-coupled and strongly scaling application, which executes on a large-scale platform
%composed of $N$ cores.
%We consider the execution of a tightly-coupled and strongly scaling application, or job, on a large-scale platform
%composed of $N$ cores. The application is tightly-coupled because this is typical in the HPC applications, and strong scaling because weak-scaling applications are not expected to suit for extreme-scale computing of which the cpu/memory imbalance would further increase~\cite{doe_ascr_exascale_2011}. Similar to \cite{casanova_inria_2012}, we use the term core to indicate unit of computing resource allocation (e.g., a
%CPU core, a multi-core CPU, or a cluster node), so that our work is agnostic to the granularity
%of the platform. 
%We assume that a standard checkpointing and roll-back recovery is performed at the
%system level. At most on application process (replica) runs on one core.


The application requires $W$ units of work, and can be split arbitrarily into a set of tasks.
Barriers are used as a method of synchronization among different tasks. Assuming $M \le N$ cores are assigned to the application, the failure-free completion time of the application is $w = W/M$. However, when a core fails, the whole execution will be suspended at the next barrier until recovery is complete. 

%The job can execute on any number $M \le N$ cores. The job is strong scaling so that the time required for a failure-free execution on $M$ cores is $w = W/M$.

%Cores are subject to failures. In most cases, we do not distinguish between soft and hard failures, with the understanding that soft failures are handled via software rejuvenation (i.e., rebooting \cite{466961}) and that hard failures are handled by the replacement of the failed component with a spare, which is a commonplace approach in production systems. For simplicity, we adopt the fail-stop fault model, where a core stops execution once a failure occurs and the failure can be detected by other cores \cite{gartner_faults_1999,cristian_comm_1991}. Since we consider tightly coupled parallel jobs, all $M$ cores operate synchronously. When a core fails, the whole execution is suspended until the failure is recovered. We assume that core failures are independent and identically distributed (i.i.d.). In the real world, instead, failures are bound to be correlated. Obtaining theoretical results for non-i.i.d. failures is beyond the scope of this work. But note that one cause of correlation is the hierarchical structure of computing platforms (each rack comprises compute nodes, each compute node comprises processors, and each processor comprises cores), which leads to simultaneous failures of a group of cores. Our work applies to such failures since a group of failures can be treated as multiple individual failures that happen at the same time and their recovery can be carried out in parallel.

Cores are subject to failures. In our model, we do not distinguish between soft and hard failures. We further assume that soft failures are handled via software rejuvenation (i.e., rebooting \cite{466961}), while hard failures are handled by replacing the failed components with spares. We adopt the fail-stop fault model, whereby a core stops executing upon failure and failures are detected by other non-failing cores \cite{gartner_faults_1999,cristian_comm_1991}. When a core fails, the whole execution is suspended until recovery is complete. 

%%%%%%%%%%%%%%%%%%%%%%%%%%%%%%rethinking%%%%%%%%%%%%%%%%%%%%%%
We assume that core failures are independent and identically distributed (i.i.d.). In the real world, instead, failures are bound to be correlated. Obtaining theoretical results for non-i.i.d. failures is beyond the scope of this work. But note that one cause of correlation is the hierarchical structure of computing platforms (each rack comprises compute nodes, each compute node comprises processors, and each processor comprises cores), which leads to simultaneous failures of a group of cores. Our work applies to such failures since a group of failures can be treated as multiple individual failures that happen at the same time and their recovery can be carried out in parallel.

%Since we consider tightly coupled parallel jobs, all q cores operate syn- chronously. These cores execute the same amount of work W(q) in parallel, chunk by chunk. The total time (on one core) to execute a chunk of dura- tion, or size, ω and then checkpointing it, is ω + C(q)

%\subsection{Leaping Shadows}



%%%%%%%%%%%%%%%%%%%%%%%%%%%%%%%%%%%%%%%%%%%%%%%%%%%%%%%%%%%%%%%%%%%%%%%%%%%%%%%%%%

%%We consider the class of strongly-scaled applications, and use $W$ to denote the size of an application workload~\cite{doe_ascr_exascale_2011}. The workload is split  arbitrarily into a set of $N$ tasks, whose synchronization is achieved using barriers. Assuming that each task is assigned to execute on a core at a maximum speed $\sigma=1$, the failure-free completion time of the application is $w = W/N$. 
Let $M_i$ denote the main process executing the $i^{th}$ task $T_i$, and $S_i$ its associated shadow.
When a failure occurs, the non-failing tasks continue executing until they reach the synchronization barrier. The tasks remain suspended until the shadow associated with the failing task reaches its synchronization barrier.


failure recovery is achieved.  To reduce time to recovery,  we run M shadows simulataneously with the main processes. 
%According to the computational model in Section~\ref{sec:com_model}, an application's workload is split into $M$ parallel tasks where each task is associated with a main process and a shadow.

 The execution of the application can be carried out by simultaneously running all main and shadow processes 
on %$2M$ cores, whereby the main processes execute at the maximum rate while the associated shadows execute at a fraction, $\sigma_s$, of the 
%maximum rate using DVFS. 
%To simultaneously run all the main and shadow processes,
%ne may choose to use $2M$ cores, with $M$ executing the main tasks at the maximum rate and $M$ \emph{lazily} executing the shadows at a fraction, $\sigma_s$, of the 
%maximum rate using DVFS.  
%An alternative method is to use 
$M+S$ cores, where $M$ is a multiple of $S$ and $M+S=N$, all executing at the maximum rate. $M$ of these cores are allocated to the main processes while the remaining $S$ cores are shared among their associated shadows. Based on this method, each main process is allocated one core, while $\alpha=M/S$ shadows are collocated on a single core. $\alpha$ is referred to as the main to shadow ratio.
For example, if $M=9$ and $S=3$, then the 9 shadows execute on 3 cores, with every $\alpha=3$ shadows executing on a core (Figure~\ref{fig:sc_mapping}).
%only $M+S$ cores can be used, whereby the main processes execute on the $M$ cores. And the $M$ shadows are divided into $S$ clusters with of $M/S$ shadows are colocated on a single core, operating at the maximum rate.
%where $M$ is a multiple of $S$, and collocate $M/S$ shadows on each of the $S$ cores, while executing all the cores at the maximum speed. 
%We use $\alpha$ to denote the main to shadow allocation ratio. %, and for simplicity, we assume the maximum execution rate of a core is 1. %Ignoring the overhead of context switching, the two alternatives lead to the same expected execution time, but different power and energy consumption. Specifically, the $2M$-cores scheme consumes more static but less dynamic power/energy than the $M+S$ cores alternative. The ratio between the system static and dynamic power consumption determines which alternative overall consumes less power/energy. 

%In the rest of the paper, we will focus on 
%shadow collocation as the meain the rest of our discussion 
%of the main ideas, concepts and system design. We will assume that all cores in the system execute at maximum speed, with every $\alpha=M/S$ shadows collocated on a core. 
%In HPC, throughput consideration requires that the rate of the main task, $\sigma_m$, and the shadow after failure, $\sigma_a$, be set to the maximum. The execution rate of the shadow before failure, $\sigma_b$, however, may still be used to manage the trade-offs between completion time and energy consumption. Smaller $\sigma_b$ corresponds to lazier shadowing of the main process. 
%In terms of execution rates, this can be expressed as $\sigma_m=\sigma_a=1$ and $\sigma_s \le 1$. 
  
Collocation of $\alpha$ shadows on a core has an important ramification with respect to the resilience of the system. Specifically, to speed up a shadow 
of a failed main to the maximum rate, all other collocated shadows must be terminated. Consequently, a second failure in any of the mains of the terminated shadows cannot be tolerated. In other words, the $M+S$ cores are grouped into $S$ sets, which we call \emph{shadowed sets}, each containing $\alpha+1$ cores with $\alpha$ mains executing on $\alpha$ cores (referred to as main cores) and their corresponding $\alpha$ shadows collocated on one core (referred to as shadow core). Each shadowed set can tolerate a failure in any of its cores, since failure of a main core would be recovered by the shadow core and failure of the shadow core will not affect any mains. After the first failure in a shadowed set, the set is called \emph{vulnerable} because it cannot tolerate another failure. %In the following subsection, we will discuss a rejuvenation scheme that deals with vulnerable shadowed sets.

\begin{figure}[!t]
	\begin{center}
		\includegraphics[width=\columnwidth]{Figures/sc_mapping.pdf}
	\end{center}
	\vskip -0.25in 
	\caption{An example of 3 shadowed sets with $\alpha=3$.}
	\label{fig:sc_mapping}
\end{figure}

%Lazy Shadowing provides the basis for the design of efficient energy- and power-aware fault-tolerance solutions for extreme-scale computing environments. %The resulting model, referred to as {\it leaping shadows}, 


%Lazy shadowing provides the basis for the design of efficient energy- and power-aware fault-tolerance solution for extreme-scale computing environments. The resulting model, referred to as {\it leaping shadows}, takes into consideration the main characteristics of compute-intensive and highly-scalable applications to achieve high-tolerance to failure, while minimizing energy consumption. In the following, we first introduce the execution and communication model of the targeted applications. We then describe the dynamics of the {\it leaping shadows} resilience model. Lastly, we discuss important aspects of its implementation. 
\subsection {Application model}
\label{sec:app_model}

We consider the class of compute-intensive and strongly-scaled applications, executing on a large-scale multi-core computing infrastructure~\cite{doe_ascr_exascale_2011}. %Communication between cores is achieved using low-latency, high-bandwidth interconnect networks, such as Infiniband. 
We use $W$ to denote the size of an application workload, and assume that the workload is split arbitrarily into a set of tasks, $\tau$, which execute in parallel and are synchronized using barriers. Given the prominence of MPI in HPC environments, we assume message passing as the communication mechanism between tasks. %Based on this model, each pair of communicating tasks is associated with a logical first-in-first-out (FIFO) channel, which guarantees ordered delivery of messages.
The execution is divided into a set of phases by synchronization barriers. 
Assuming the maximum execution rate is $\sigma_{max}=1$, the failure-free completion time of the application is $W/|\tau|$

 . %If a failure occurs, however, the non-failing tasks would become idle when they reach their synchronization barrier. These tasks remain idle until the failure recovery is complete. 
%resulting in performance hiccups~\cite{muller2010}, especially for tightly-coupled applications. 


\subsection{Shadow collocation}

The execution of an application can be carried out by simultaneously running all tasks, each with a pair of main and shadow processes. Let $m_i$ denote the main process executing the $i^{th}$ task $task_i$, and $s_i$ its associated shadow.
%There are multiple ways to control the execution rates of the processes. The most straightforward solution is to allocate a dedicated core for each process while using 
%Dynamic Voltage and Frequency Scaling (DVFS) is a commonly used power management technique 
We use the term core to represent the resource allocation unit (e.g., a
CPU core, a multi-core CPU, or a cluster node), so that our algorithm is agnostic to the
granularity of the hardware platform~\cite{casanova_inria_2012}. Each main process executes on one core exclusively to achieve maximum throughput.  
To control the shadow's execution rate, Dynamic Voltage and Frequency Scaling (DVFS) can be applied while each shadow also resides on one core exclusively~\cite{mills_2014_icnc,cui_en7085151,cui_2014_closer}. 
The effectiveness of DVFS, however, may be markedly 
limited by the granularity of voltage control, the number of frequencies available, and the negative effects on 
reliability~\cite{chandra2008defect}.
%reduced in computational platforms that exhibit saturation of the processor clock frequencies, large static power consumption, or small power dynamic range. 
An alternative is to collocate multiple shadows on a single core, while keeping the core at maximum rate. Time sharing can then be used to achieve the desired execution rates of the processes. %Given our focus on extreme-scale, multi-core computing infrastructure, we will use collocation for execution rate control.  
%Recent development in processor and memory technology which results in the saturation of the processor clock frequencies, larger static power consumption, and smaller power dynamic range can markedly reduce the effectiveness of DVFS

 %to tune the frequency of the core. An alternative is to collocate multiple processes on a core, which runs at the maximum rate, and execute the processes in a time sharing manner. In the rest of this paper, we will focus on the idea of collocation in the discussion of applying Lazy Shadowing to HPC applications.  


To execute an application with $M$ tasks, $N=M+S$ cores are required for Lazy Shadowing, where $M$ is a multiple of $S$. Each main is allocated one core (referred to as main core), while $\alpha=M/S$ shadows are collocated on a core (shadow core). 
The $N$ cores are grouped into $S$ sets, which we call \emph{shadowed sets}, each containing $\alpha$ main cores and 1 shadow core.
% $\alpha$ is referred to as shadowing ratio. For example, if $M=9$ and $S=3$, then the 9 shadows share 3 cores, with every $\alpha=3$ shadows collocated on each core, as shown in Figure~\ref{fig:sc_mapping}.
This is illustrated in Figure~\ref{fig:sc_mapping}.  

Collocation has an important ramification with respect to the resilience of the system. Specifically, 
one failure can be tolerated in each shadowed set. If a shadow core fails, the mains can continue
execution, but will have no shadows any more. On the other hand, 
to speed up a shadow 
of a failed main to the maximum rate, all other collocated shadows must be terminated. Consequently, a second failure in any of the mains in the shadowed set cannot be tolerated. After the first failure, a shadowed set becomes \emph{vulnerable}\footnote{Rejuvenation techniques, such as restarting the lost shadows from the state of current mains on spare cores, can be used to eliminate vulnerability.}. 
As will be shown in Section~\ref{anal_app_fail}, this should not be a concern as the provided reliability is more than enough.
 
\begin{figure}[!t]
  \begin{center}
    \includegraphics[width=\columnwidth]{Figures/sc_mapping.pdf}
  \end{center}
  %\vskip -0.25in 
  \caption{An example of collocation. $N=12$, $M=9$, $S=3$.}
  \label{fig:sc_mapping}
\end{figure}

As the shadows execute at a lower rate, failures will incur delay for recovery. This problem deteriorates as dependencies incurred by messages and synchronization barriers would propagate the delay of one task to other tasks.  
The main objective of leaping shadows is to minimize the delay induced by failures, 
%mitigate the impact of performance hiccups, 
and ensure forward progress by opportunistically rolling-forward the shadows during failure recovery. In Section~\ref{anal_time}, we will show that the delay is well bounded even for tightly-coupled applications.


\subsection {Leaping shadows}
\label{sec:leaping_shadows}

In the absence of failure, the behavior of a main  and its leaping shadow is identical to the behavior depicted in Figure \ref{fig:sync}. Assuming a failure occurrence at time $t_f$, Figure~\ref{fig:leap} shows the concept of shadow leaping. 
%Figure~\ref{fig:jump1} depicts the execution dynamics of the failing main and its associated shadow. 
Upon failure of a main process, its associated shadow speeds up to minimize the impact of failure recovery on the application's progress, as illustrated in Figure~\ref{fig:jump1}. 
%Figure~\ref{fig:jump2} illustrates the behavior of the remaining main processes and their associated shadows. 
At the same time, as shown in Figure~\ref{fig:jump2}, the remaining main processes continue execution towards the barrier at time $t_{sync}$, and then become idle until $t_r$, when the shadow of the failed main process also reaches the barrier. %It is worth noting that, given the tightly-coupled and strongly-scaled nature of the application, synchronization points occur frequently. Consequently, the time between synchronization points is very small relative to the total execution time of the application. Therefore, if one main, $m_i$, fails at 
%time $t_f$, the remaining main processes will reach their synchronization point, shortly after the failure, specifically at at time $t_{sync} = t_f +\epsilon$, 
%where $\epsilon \approxeq 0$. 
Leaping shadows opportunistically takes advantage of this idle time to {\it leap forward} the shadows by copying state from their main processes, so that  
all processes, including shadows, can resume execution from a consistent synchronization point afterwards. This process continues until the completion of all tasks. Forward leaping increases the shadow's rate of progress, at a minimal energy cost. Consequently, it reduces significantly the likelihood of a shadow falling excessively behind, thereby ensuring fast recovery while minimizing energy consumption.



\begin{figure}[!t]
	\begin{center}
        \subfigure[Faulty task behavior.]
		{
			\label{fig:jump1}
			\includegraphics[width=0.7\columnwidth]{Figures/jump1}
		}
		\subfigure[Non-faulty task behavior.]
		{
			\label{fig:jump2}
			\includegraphics[width=0.7\columnwidth]{Figures/jump2}
		}
	\end{center}
	%\vskip -0.25in
	\caption{The illustration of shadow leaping.}
	\label{fig:leap}
\end{figure}

%Collocation is used to control the execution rates. To execute an application with $M$ tasks, $N=M+S$ cores are required, where $M$ is a multiple of $S$. Each main process is allocated one core (referred to as main core), while $\alpha=M/S$ shadows are collocated on a core (shadow core). 
%The $N$ cores are grouped into $S$ sets, which we call \emph{shadowed sets}, each containing $\alpha$ main cores and 1 shadow core.
%% $\alpha$ is referred to as shadowing ratio. For example, if $M=9$ and $S=3$, then the 9 shadows share 3 cores, with every $\alpha=3$ shadows collocated on each core, as shown in Figure~\ref{fig:sc_mapping}.
%This is illustrated in Figure~\ref{fig:sc_mapping}.  
%Collocation has an important ramification with respect to the resilience of the system. Specifically, 
%one failure can be tolerated in each shadowed set. If a shadow core fails, the main processes can continue
%execution, but will have no shadows any more. On the other hand, 
%to speed up a shadow 
%of a failed main to the maximum rate, all other collocated shadows must be terminated. Consequently, a second failure in any of the mains in the shadowed set cannot be tolerated. After the first failure, a shadowed set becomes \emph{vulnerable}\footnote{Rejuvenation techniques, such as restarting the lost shadows from the state of current mains on spare cores, can be used to eliminate vulnerability.}. 
%
%\begin{figure}[!t]
%  \begin{center}
%    \includegraphics[width=\columnwidth]{Figures/sc_mapping.pdf}
%  \end{center}
%  %\vskip -0.25in 
%  \caption{An example of collocation. $N=12$, $M=9$, $S=3$.}
%  \label{fig:sc_mapping}
%\end{figure}

%Collocation also increases memory requirement. However, this is not intrinsic to Lazy Shadowing, as checkpointing/restart also requires additional memory capacity. We acknowledge the fact that compute kernels in existing HPC environments were simplified significantly by placing a number of restrictions, including eliminating virtual paging and limiting support for OS (Linux) to a handful of system calls. It became clear, however, that strategies designed to work around the capabilities of the hardware cannot scale to extreme-scale computing. Consequently, the research focus has been on new paradigms focused on co-design of hardware with system software to leverage the advantages associated with dynamic, asynchronous mechanisms, such as demand paging and cache tuning, against the design principles and choices of current HPC systems. Support of efficient demand paging, through co-design, is particularly critical as it is expected that the data of future exascale applications may not fit entirely in memory. %Finally, in comparison to checkpointing/restart and process replication, Lazy Shadowiing has the capability to control memory usage, based on the nature of failure and existing memory capacity, albeit at a loss of performance. 



\begin{algorithm}[t]
  \SetKwInOut{Input}{input}
  \SetKwInOut{Output}{output}
  \caption{Lazy Shadowing}
  \Input{$W, M, S$}
  \Output{Application execution status}
  \BlankLine
  split $W$ into $M$ tasks\; \nllabel{line:split} 
  assign $M$ tasks to $S$ shadowed sets\; \nllabel{line:cluster} 
  %$T_l \leftarrow T+t_{now}$\;%, $C_{vul} \gets 0$
  %start $m_i$ and $s_i$ for each $task_i$\;
  start a pair of main and shadow for each task\;
    \While{execution not done}
    {
  %      \If{failure detected in $ss_j$} %\nllabel{line:if_start_1} 
        \If{failure detected in a shadowed set}
        {
            \nllabel{line:if_start_1} 
            %\eIf{$ss_j$ is vulnerable}
            \eIf{the shadowed set is vulnerable}
            {
                notify ``Application failure"\;
                terminate all mains and shadows\;
                repair all failures\;
                restart execution\; %\Comment{re-execution}
            }
            {
                mark the shadowed set as vulnerable\;
                %\State $C_{vul} \gets C_{vul} + 1$
                %\If{$C_{vul} == V$}
                %    \State perform shadowed set rejuvenation   
                %\EndIf 
                \If{failure happened to a main} 
                {
                    promote its shadow to new main\;
                    perform shadow leaping\; %failure induced shadow leaping
                    %$T_l \leftarrow T+t_{now}$\;
                }
            }
        }  
        \nllabel{line:if_end_1}
        %\If{$t_{now} \ge T_l$} %
        %{
        %    \nllabel{line:if_start_2} 
        %    perform shadow leaping\;
        %    $T_l \leftarrow T+t_{now}$\;
        %    \nllabel{line:if_end_2}
        %} % 
        %\nllabel{line:if_end_2}
    }
    output ``Application completes"\;
  \label{al:ls}
\end{algorithm}

The steps of applying Lazy Shadowing with leaping shadows are depicted in Algorithm 1.
To use $M+S$ cores to execute an application, the total workload is split into $M$ parallel tasks (line~\ref{line:split}), %, which are executed simultaneously by $M$ main processes and $M$ shadow processes. 
 which are then assigned to $S$ shadowed sets, each with $\alpha=M/S$ cores for $\alpha$ main processes and 1 core for all the associated shadow processes (line 2).  
%The $M$ shadows are then clustered into $S$ groups, each containing $\alpha=M/S$ shadows. 
The execution starts by simultaneously running all the main and shadow processes (line 3).
During the execution,
the system runs a failure monitor (out of scope of this work) that triggers corresponding actions when a failure is detected (line~\ref{line:if_start_1} to~\ref{line:if_end_1}). %A failure may trigger different actions, depending on its type and precedence with respect to other failures.  A shadowed set becomes {\it vulnerable} after the occurrence of the first failure in the set. 
A failure occurring in a vulnerable shadowed set (e.g., $ss_j$) results in an application failure %. In response, the system terminates all running 
%processes, initiates a recovery phase, either by rebooting or replacing failing cores,  and restarts execution (line 8 to 10). %(this assumes that checkpointing is not used). 
 and forces a re-execution (line 7 to 10).
On the other hand, failure in a non-vulnerable shadowed set
does not translate into an application failure, but would mark the shadowed set in question as vulnerable (line 12). In this case, failure of a main process has different impact from that of a  shadow process.  While a shadow 
failure does not impact the normal execution and thus can be ignored, failure of a main process %forces the remaining main processes to suspend execution after they reach their synchronization point.  The shadow process, $s_k$, associated with the failing process, $m_k$,  becomes the primary process of the associated task and increases its execution to the maximum rate 
(e.g., $m_k$) triggers promotion of its shadow process, $s_k$, to a new main process (line 14). Simultaneously, a shadow leaping is undertaken by all remaining shadows to align their states with those of their associated mains (line 15).  
This process continues until all tasks of the application are successfully completed.

\subsection{Implementation issues}

%\subsection{Shadowed set rejuvenation}
%\label{frame_reju}
%The proposed Lazy Shadowing scheme can tolerate faults which are repairable by rebooting or reconfiguration, referred to as soft faults, and faults which cannot be repaired by rebooting or reconfiguration, referred to as hard 
faults. Monitors that detect hard faults, such as memory flip, bus error 
and latch error, or soft faults, such as deadlock detection, buffer overflow and protection violation, typically interrupt the application to initiate 
the recovery process. The process of recovery from transient or permanent faults is the same and necessitates a mechanism for detecting a fault 
in a main task, M(i) and notifying other tasks in the system so that (i) the shadows sharing a core with S(i) are terminated, thus allowing S(i) to execute at the maximum rate, and (ii) all the shadows that are not in the faulty shadowed set leap to the state of their mains. 

As described earlier, the recovery from a fault in a shadowed set leaves the set vulnerable and any more faults in a vulnerable set will result in a system failure. Although for large systems and small S the probability of having a second fault in a vulnerable set is low, some provision should be taken to rejuvenate the system when a relatively large number of its shadowed sets are vulnerable. 

We propose to invoke \emph{shadowed set rejuvenation} after a specific number of faults, which is determined by the system size, the shadowed set size, and the required resilience.
Rejuvenation reconfigures the system such that none of its shadowed sets are vulnerable. Unlike recovery from a fault in a shadowed set, rejuvenation is different when the faults are transient soft when the faults are hard. In the case of soft faults, rejuvenation can be accomplished by rebooting the failed cores, and restarting the lost shadows (both the ones promoted to mains and the ones terminated) from the state of current mains. And in case of hard faults, it is possible to restart the lost shadows after replacing the failed ones with spare ones. This will restore a vulnerable shadowed set to its original configuration. %For example, rejuvenation should restore the systems shown in Figure~\ref{fig:layout2} and Figure~\ref{fig:layout3} to the one shown in Figure~\ref{fig:layout1}.

%When failures are permanent, rejuvenation may be challenging if rebooting or reconfiguration can no longer be
%used to recover failed components.
%Specifically, in the absence of spare components (if the system is not over-provisioned),
%rejuvenation can only 
%be accomplished by distributing the main processes of a vulnerable set 
%to other {\bf non-vulnerable} shadowed sets. The shadow of the vulnerable 
%set must also be relocated to the shadows of the non-vulnerable set. 
%As a result,  the total number of shadowed sets decreases, but 
%the size of some shadowed sets increases.  In Figure~\ref{fig:reju}, we show a possible rejuvenated configuration assuming that the failure of the cores executing $M(1)$ and $M(14)$ in Figure~\ref{fig:layout2} is permanent. In this restored configuration, the number of shadowed sets is reduced from 8 to 6, with four sets containing three mains each and two sets containing two mains each. Rejuvenating the vulnerable configuration of Figure~\ref{fig:layout3} after a permanent socket failure is more complex but follows the same basic principle.

%\begin{figure*}[ht]
%	\begin{center}
%		\includegraphics[width=\textwidth]{figures/reju.pdf}
%	\end{center}
%\vskip -0.25in
%	\caption{Shadowed set rejuvenation of the vulnerable sets resulting from permanent faults.}
%	\label{fig:reju}
%\end{figure*}


%%%%%%%%%%%%%%%%%%%%%%%%%%%%%%%%%%%%%%%%%%%%%%%%%%%%%%%%%%%%%%%%%%%%%%%%%%%%%%%%%%
We implemented an Open MPI based prototype of Lazy Shadowing, which can be used to execute existing HPC workloads without any change of user code. Since the focus of this paper is to introduce
algorithmic perspectives of the Lazy Shadowing paradigm by discussing novel concepts of shadow collocation and forward leaping, we only give a brief discussion of the implementation issues. 

State consistency is required both during normal execution and following a failure of a main process to roll-forward the shadows. During normal execution, shadows remain mute, in the sense that 
all outgoing messages from shadows are suppressed. 
A shadow process, however, will typically lag behind its main process during execution. Therefore, it is necessary to ensure that the shadow's state is consistent with that of its associated main. %, to successfully complete its associated task in case of failure. 
To this end, a message-logging protocol is used, % to ensure consistency~\cite{Marz}. These protocols 
which typically uses a minimum amount of meta-information to store and replicate the non-deterministic decisions~\cite{Marz}. %in the execution of an application.  These meta-data, also called determinants, are exchanged through system-level messages. 

To provide correct recovery after failure, 
a mechanism is required to guarantee that every shadow process follows the same computation and communication steps as its main process. 
After a main process $m_i$ fails, $s_i$ will take over $m_i$'s role to recover from this failure. If there are other shadows sharing the same core with $s_i$, they will be terminated and $s_i$ will start consuming the messages in its receiver-side message log at a faster speed. The message logging protocol will ensure that shadow $s_i$ reaches a consistent state with the rest of the system. 

Upon failure of a main process, shadow processes will update their address space to ``catch up" with their associated non-failing main processes. A technology, such as remote direct memory access (RDMA), can be used to roll-forward the state of the shadow to be consistent with that of its associated main. Rather than copying data to the buffers of the operating system, RDMA allows to transfer data directly from the main process to its shadow. The zero-copy feature of RDMA considerably reduces latency, thereby enabling fast transfer of data between the main and its shadow.

 




\section{\uppercase{Analytical Models}}
\label{sec:analytical}
Two important metrics for assessing an application's execution are completion time and energy consumption. In the following we develop analytical models to quantify the expected performance of Lazy Shadowing, as well as prove the bound on performance loss due to failures. %, with the understanding
%that process replication is a special case of Lazy Shadowing where $\alpha=1$. 
All the analysis below is under the assumption that there are a total of $N$ cores, and $W$ is the application workload.  
$M$ of the $N$ cores are allocated for main processes, each having a workload of $w=\frac{W}{M}$, and the rest $S$ cores are for the collocated shadow processes. %For process replication,
Note that process replication is a special case of Lazy Shadowing where $\alpha=1$, so 
$M=S=\frac{N}{2}$ and $w=\frac{2W}{N}$. 


\subsection{Application fatal failure probability}
\label{anal_app_fail}
Application failure, which forces the execution to start over, is inevitable even when every process is replicated. Lazy Shadowing is able to 
tolerate one failure in each shadowed set, and the second failure in any shadowed set implies the need to restarting the execution from the very beginning. However, Lazy Shadowing is orthogonal to checkpointing in 
the sense that we can combine the two, to avoid rolling the execution back to the very beginning when application failure occurs.

Since each process is replicated with a shadow, Lazy Shadowing has the potential to significantly 
increase the Mean Number of Failures To Interrupt (MNFTI), i.e., the average number of core failures until application failure occurs, and Mean Time To Interrupt (MTTI), i.e., the average time elapsed until application failure occurs. 
Therefore, the checkpointing interval should be increased to a large extent when checkpointing is combined with Lazy Shadowing. Furthermore, if the resulted checkpointing interval is 
larger than the completion time of the application, then checkpointing may not be used at
all. 
%Therefore, in this subsection, we study the reliability benefits that Lazy Shadowing could 
%introduce. Specifically, we study the application's MNFTI (and MTTI) with Lazy Shadowing.
In the following, the first question to study is the new MNFTI and MTTI when Lazy Shadowing is used. 


The impact of process replication on MNFTI has been studied in~\cite{casanova_inria_2012}. Our problem
is equivalent with the difference that our work can tolerate one failure in each shadowed 
set while~\cite{casanova_inria_2012} can tolerate one failure in each replica-group, when each process
is replicated once. 
Therefore, we can directly apply the methodology in~\cite{casanova_inria_2012} to our case, and the MNFTI
with Lazy Shadowing for different number of shadowed sets ($S$) is shown in Table~\ref{tbl:mnfti}. 
%The MTTI with Lazy Shadowing is not shown here because it depends not only on the number of cores, but also on the shadowed set size chosen. However, results in \cite{casanova_inria_2012} reflect that the MTTI can be increased to the order of tens of hours from ten minutes (without use of replication) assuming the core level MTTI is 25 years. This 
%confirms our previous prediction that Lazy Shadowing can significantly 
%increase the application's MNFTI and MTTI, and also implies that shadowed set rejuvenation may not be necessary.
Note that when processes are not replicated, every failure would interrupt the application, i.e., MNFTI=1, so MNFTI can be significantly increased by Lazy Shadowing. 
At the same time, it is projected that an extreme-scale application's MTTI can be increased to tens of hours from minutes assuming each core's MTBF is 25 years.

\begin{table}[b!]
	\caption{Application's MNFTI when Lazy Shadowing is used. Results are independent of $\alpha$. }
	\centering
	\small
	\begin{tabular}{|c|c|c|c|c|c|c|c|}
		\hline
		$S$ & $2^{0}$ & $2^{1}$ & $2^{2}$ & $2^{3}$ & $2^{4}$ & $2^{5}$ & $2^{6}$ \\
		\hline
		MNFTI & 3.0 & 3.7 & 4.7 & 6.1 & 8.1 & 11.1 & 15.2\\
		\hline\hline
		$S$ & $2^{7}$ & $2^{8}$ & $2^{9}$ & $2^{10}$ & $2^{11}$ & $2^{12}$ & $2^{13}$ \\
		\hline
		MNFTI & 21.1 & 29.4 & 41.1 & 57.7 & 81.2 & 114.4 & 161.4 \\
		\hline\hline
		$S$ & $2^{14}$ & $2^{15}$ & $2^{16}$ & $2^{17}$ & $2^{18}$ & $2^{19}$ & $2^{20}$ \\
		\hline
		MNFTI & 227.9 & 321.8 & 454.7 & 642.7 & 908.5 & 1284.4 & 1816.0 \\
		\hline
	\end{tabular}
	\label{tbl:mnfti}
\end{table}


Even though the above results imply that checkpointing may not be necessary when Lazy Shadowing is used, it is important to quantify the probability that an application failure would occur during the application's execution, defined as ``application failure probability". Let $f(t)$ denotes the failure probability density function of each core, and $F(t)$ be the corresponding cumulative distribution function, i.e., $F(t) = \int_0^tf(\tau)d\tau$ is the probability that a core fails in the next $t$ time. 
Since each shadowed set can tolerate one failure, 
then the probability that a shadowed set with $\alpha$ main cores and 1 shadow core does not fail by time $t$ is the probability of no failure plus the probability of one failure, i.e., 
%\begin{equation}
%	G(t, \alpha) = \Big(1-F(t)\Big)^{\alpha+1} + {{\alpha+1} \choose 1}F(t)\times \Big(1-F(t)\Big)^{\alpha}
%\end{equation}
\begin{equation}
	P_g = \Big(1-F(t)\Big)^{\alpha+1} + {{\alpha+1} \choose 1}F(t)\times \Big(1-F(t)\Big)^{\alpha}
\end{equation}
and the probability that the application using $N$ cores fails within $t$ time is the complement of the probability that
none of the shadowed sets fails, i.e.,
%\begin{equation}
%	R(t, N, \alpha) = 1 - \Big(G(t, \alpha)\Big)^{\frac{N}{\alpha+1}}
%\end{equation}
\begin{equation}
	P_a = 1 - ({P_g})^{S}
\end{equation}
where $S=\frac{N}{\alpha+1}$ is the number of shadowed sets.

Since the application only fails during its execution, we can calculate the application failure probability using $t$ equal to the expected completion time of the application. We will develop the model for the expected completion time in the next subsection.
%that $T_c$ is a function of $W$, $N$, $\alpha$. As a result, the application failure probability can be expressed as $R'(W, N, \alpha)$.

%With $R(W, N, \lambda, S)$, we can study the behavior of lazy shadowing under a configuration of ($W$, $N$, application failure probability), for any failure distribution $f(t)$, e.g., exponential or weibull. However, there are two problems now: 1) The computation involved is so complicated that MatLab cannot give accurate results; 2) we don't have the analytical model of expected completion time $T_c$ assuming exponential or weibull failure distribution. 


\subsection{Expected completion time}
\label{anal_time}
One of the major performance metrics of interest to end-users, is the application's completion time. 
To evaluate this, we develop an analytical model for the expected completion time of Lazy Shadowing, with all probabilities of failures considered. We assume that failures don't happen at the same time.
%Since we are comparing between Lazy Shadowing and process replication, the overhead of failure detection and consistency protocols are ignored as they are the same for the two approaches.
%Our models focus on one 
%checkpointing interval of the application, since the whole execution is just a repetition of multiple such intervals. As a consequence, we assume the execution with checkpointing starts right after the previous checkpoint and ends right before the next checkpoint, and the execution with Lazy Shadowing and process replication will not experience any application failure. For fairness, we assume that the three alternatives have the same amount of workload, 
%$W$, to execute, and total number of available cores, $M$. The maximal execution rate at each core, $\sigma_{max}$, is normalized to 1 so that the time to complete without failures is equal to the workload to execute on each core. In addition, we assume that the three alternatives will encounter the same number of failures, which is $k$, as they will execute the same amount of workload using the same amount of resources.

%\subsubsection{Expected completion time}
First we discuss the case of $k$ failures, which separate the execution into $k+1$ intervals.
Denote by $\Delta_i$ ($1\le i \le k+1$) the $i^{th}$ continuous execution interval, and $\tau_i$ ($1\le i \le k$) the recovery time after $\Delta_i$. 
%between the $(i-1)^{th}$ and $i^{th}$ failures (assuming the $0^{th}$ failure happens right before the execution begins, and the $(k+1)^{th}$ failure happens right after the execution ends), and $\tau_i$ ($1\le i \le k$) the recovery time after the $i^{th}$ failure. 
The application's progress with delay incurred by failures is illustrated in Figure~\ref{fig:progress}.

\begin{figure}[!t]
	\begin{center}
		\includegraphics[width=\columnwidth]{Figures/progress}
	\end{center}
	%\vskip -0.22in 
	\caption{Illustration of application's progress with failure incurred delays.}
	\label{fig:progress}
\end{figure}

Since Lazy Shadowing use $M$ cores for executing main processes and $S$ cores for shadowing ($M+S=N$), the total workload $W$ will be split into $M$ tasks, each of which will be assigned a pair of main and shadow processes. Therefore, the workload of each process is 
$w=W/M$. The recovery time $\tau_i$ is the time needed for the lazy shadow of the failed main to catch up. With shadow leaping, it is guaranteed that all the shadows reach the same execution point as the mains (See Figure~\ref{fig:leap}) after the previous recovery, so every recovery time is proportional to its previous continuous execution length, which is $\Delta_i$. That is, $\tau_i = \Delta_i \times (1 - \sigma_s^b)$. The value of $\Delta_i$ can be obtained given a failure probability distribution, as will be demonstrated in Section~\ref{sec:evaluation}. 
Since we assume there are $k$ failures, then $\Delta_{k+1}$ is the failure free execution interval until $W$ is complete, i.e., $\Delta_{k+1} = w - \sum_{i=1}^{k}\Delta_i$. Finally, according to Figure~\ref{fig:progress}, the completion time with $k$ failures is 
\begin{equation}
	T_c^k = \sum_{i=1}^{k+1}\Delta_i + \sum_{i=1}^k\tau_i = w + (1-\sigma_s^b)\sum_{i=1}^k\Delta_i
	\label{eq:time_k}
\end{equation}

We assume that failures do not occur during recovery, so the failure probability of a core during the execution can be estimated as $P_c = F(w)$. Then the probability that there are $k$ failures among the $N$ cores during the execution is 
\begin{equation}
\begin{split}
P_s^{k}= & \dbinom{N}{k}{P_c}^k(1-P_c)^{N-k} \\
%= & \dbinom{M}{k}({\frac{w}{MTTI}})^k(1-\frac{w}{MTTI})^{M-k}
\end{split}
\end{equation}

The completion time considering all possible failures can be averaged as $T_{c} = \sum_{i} T_{c}^{i} \cdot P_s^{i}$, and the expected completion time considering the possibility of re-execution after application failure is
%\begin{equation} 
%E[T_{c}] = \sum_{i} T_{c}^{i} \cdot P_s^{i}
%\label{eq:exp_time}
%\end{equation}
\begin{equation} 
T_{total} = T_{c} / (1 - P_a)
\label{eq:exp_time}
\end{equation}

In the special case of $\alpha=1$, which is process replication, half of the hardware resources are dedicated to replicas so that the workload assigned to each task is significantly increased given the fixed number of cores available, i.e., $w=2W/N$. Different from the case of $\alpha \ge 2$, failures do not incur any delay unless application failure occurs, since the replicas are executing at the same rate as the main processes. Therefore, the completion time of process replication without application failure is constant with respect to the number of failures, and the expected completion time considering the possibility of re-execution is
\begin{equation}
	T_{total} = T_c / (1 - P_a) = w / (1 - P_a)
\end{equation}

A closer look at the above analysis one can realize that Lazy Shadowing has both advantage and disadvantage compared to traditional process replication. When collocating multiple shadow processes on each core, more resources will be dedicated to main processes, leading to less workload per process and thus less completion time. On the other hand, however, collocation slows down the shadow processes, implying delays when failures occur. Although it may seem that the delay would keep deteriorating as the number of failures increases, it turns out to be well bounded, as a result of shadow leaping. From Equation~\ref{eq:time_k} we can see the delay of $k$ failures is $(1-\sigma_s^b)\sum_{i=1}^k\Delta_i$. Since we have the relation $\Delta_{k+1} = w - \sum_{i=1}^{k}\Delta_i$, $\sum_{i=1}^{k}\Delta_i$ is bounded by $w$, effectively limiting the delay by $(1-\sigma_s^b)w$.

%Contrary to process replication and Lazy Shadowing, checkpointing can use all the available cores to share the total workload, so that $w = W/M$. However, the drawback is that each failure would result in an application failure which needs to roll back the execution to the last checkpoint. With that said, the recovery time of each failure is equal to the normal execution time from the last checkpoint to the time of the failure, i.e., $\tau_i = \sum_{j=1}^{i}\Delta_j$. The completion time with $k$ failures is $T_c^k = \sum_{i=1}^{k+1}\Delta_i + \sum_{i=1}^k\tau_i = w + \sum_{i=1}^{k}\sum_{j=1}^i\Delta_j$.

%\subsubsection{Expected energy consumption}







\subsection{Expected energy consumption}
\label{anal_energy}
The power consumption of one core consists of two parts, dynamic power, $p_d$, which exists only when the core is executing, and static power, $p_s$, which is constant as long as the machine is on. This can be modeled as $p = p_d + p_s$. Note that in addition to CPU leakage, other components, such as memory and disk, also contribute to the static power consumption. 

%For checkpointing and process replication, all cores are running all the time until the application is complete. Therefore, the energy consumption is proportional to the total execution time, and the expected energy consumption when using $M$ cores to execute an application is calculated as 
For process replication, all cores are running all the time until the application is complete. Therefore, the expected energy consumption, $En$, is proportional to the expected execution time $T_{total}$: 
\begin{equation}
En = N * p * T_{total}
\label{eq:exp_energy1}
\end{equation} 
Although the failed components should not consume any power, we ignore this since the number of failures is negligible compared to the total number of cores.

Lazy Shadowing has the potential to save power compared to process replication, since main cores are idle during the recovery time after each failure, and the shadows can achieve forward progress through shadow leaping. During the normal execution time, all the cores are consuming static power as well as dynamic power. During recovery time, however, the main cores are idle and consume only static power, while the shadow cores performs shadow leaping, which may lead to higher dynamic power due to memory access and communication. After the leaping, the shadow cores become idle with no dynamic power consumption, until the failure recovery is completed. Again, we include the power consumption of the failed components. Altogether, the expected energy consumption for Lazy Shadowing can be modeled as 
\begin{equation}
En = N * p_s * T_{total} + N * p_d * w + S * p_{l} * T_l.
\label{eq:exp_energy2}
\end{equation}
with $p_{l}$ denoting the dynamic power consumption of each core during shadow leaping and $T_l$ denoting the expected total time spent on leaping during the execution of the application. Based on Equation~\ref{eq:exp_energy2} and Corollary 1.1, we can also establish an upper bound on the expected energy consumption for Lazy Shadowing:

\begin{theorem}
If no subsequent failure happens before the recovery of the previous failure, then using Lazy Shadowing, the upper bound on expected energy consumption is
$(2N * p_s + N * p_d + S * p_{l})*w$.
\end{theorem}
%\begin{proof}
{\sc Proof}. From Corollary 1.1 we know that the delay is at most $(1-\sigma_s^b)w \le w$, so $T_{total} \le 2w$. Also, since the leaping time overlaps with the recovery time (delay), $T_l \le (1-\sigma_s^b)w \le w$. Therefore, $En = N * p_s * T_{total} + N * p_d * w + S * p_{l} * T_l \le N * p_s * (2w) + N * p_d * w + S * p_{l} * w = (2N * p_s + N * p_d + S * p_{l})*w$.
%\end{proof}
$\square$



\section{\uppercase{Evaluation}}
\label{sec:evaluation}
We deployed rsMPI on a medium sized cluster and utilized up to 21 nodes for testing and benchmarking. Each node consists of a 2-way SMPs with Intel Haswell E5-2660 v3 processors of 10 cores per socket (20 cores per node), and is configured with 128 GB RAM. Nodes are connected via 56 GB/s FDR InfiniBand. To maximize the compute capacity, we used up to 20 cores per node.

We used benchmark applications from both the Sandia National Lab Mantevo Project and NAS Parallel Benchmarks (NPB), and evaluated rsMPI with various problem sizes and number of processes. CoMD is a proxy for the computations in a typical molecular dynamics application. MiniAero is an explicit unstructured finite volume code that solves the compressible Navier-Stokes equations. Both MiniFE and HPCCG are proxy applications for unstructured implicit finite element codes, but HPCCG uses MPI\_ANY\_SOURCE receive operations and can be used to demonstrate rsMPI's capability of handling MPI non-deterministic events. IS, EP, and CG from NPB represent integer sort, embarrassingly parallel, and conjugate gradient applications, respectively. These applications cover key simulation workloads for US DOE, and represent both different communication patterns and computation-to-communication ratios.

\subsection{Measurement of runtime overhead}
\label{sec:runtime_overhead}
While the hardware overhead for rsMPI is straightforward (e.g., collocation ratio of 4 results in the need for 25\% more hardware cost), the runtime overhead of the enforced consistency protocol depend on applications. To measure this overhead we ran each benchmark application linked to srMPI multiple times and compared the average execution time with the baseline, where each application runs with original OpenMPI.

Figure~\ref{fig:runtime_overhead} shows the comparison of the execution time between baseline and srMPI for the 7 applications. All the experiments are conducted with 256 application-visible processes. That is, the baseline always uses 256 MPI ranks compiled with the unmodified OpenMPI library, while rsMPI uses 256 mains together with 256 shadows which are invisible to the application. Each result shows the average execution time of 5 runs, the standard deviation, and srMPI's runtime overhead. The baseline execution time varies from seconds to half an hour, so we plotted the time in log-scale. 

From the figure we can see that srMPI has comparable execution time to the baseline for all applications except IS. The reason for the large overhead of IS is that IS uses all-to-all communication and is largely communication-intensive. This is verified by adding fake computation to the application and we can see an immediate drop of the overhead to negligible level. We argue that communication-intensive applications like IS are not scalable, and as a result, they are not suitable for large-scale HPC. 
For all other applications, the overhead varies from 0.64\% (EP) to 2.47\% (CoMD). Even for HPCCG, which uses MPI\_ANY\_SOURCE and adds extra work to our consistency protocol, the overhead is only 1.95\%, thanks to the asynchronous semantics of MPI\_Send. Therefore, we conclude that srMPI's runtime overheads are modest for scalable HPC applications that exhibit a fair communication-to-computation ratio.

\begin{figure}[!t]
  \begin{center}
      \includegraphics[width=\columnwidth]{figures/runtime_overhead}
  \end{center}
  \caption{Comparison of execution time between baseline and rsMPI. 256 application-visible processes, and collocation ratio of 4 for srMPI.}
  \label{fig:runtime_overhead}
\end{figure}

\subsection{Scalability}
In addition to measuring the runtime overhead at fixed application-visible process count, we also assessed both strong and weak scalability by varying the number of processes for the applications. Strong scaling is defined as how the execution time varies with the number of processes for a fixed total problem size. In contrast, weak scaling is defined as how the execution time varies with the number of processes for a fixed problem size per process. 

Among the seven applications, HPCCG, CoMD, and miniAero allow us to vary the input so that we can perform both strong scaling and weak scaling test. The results for miniAero are similar to those of CoMD, so we only show the results for HPCCG and CoMD here. Figure~\ref{fig:scalability} reveals that both HPCCG and CoMD have good strong scalability. By increasing the number of processes, we can always reduce the execution time for a fixed problem size. At the same time, srMPI's runtime overhead increases with the number of processes during the strong scalability test. At 256 processes, the overhead reaches 13.2\% for CoMD, and 29.1\% for HPCCG. This may seem to contradict with the results in Section~\ref{sec:runtime_overhead}. It is expected, however, since increasing the number of processes while keeping a constant problem size increases the communication-to-computation ratio of the application. Hence, to keep rsMPI overheads reasonable, it is important to choose input sizes such that the ratio of communication-to-computation is balanced. 


\begin{figure*}[!t]
	\begin{center}
		\subfigure[HPCCG strong scalability]
		{
			\label{fig:hpccg_strong}
			\includegraphics[width=0.4\textwidth]{figures/hpccg_strong}
		}
		\subfigure[HPCCG weak scalability]
		{
			\label{fig:hpccg_weak}
			\includegraphics[width=0.4\textwidth]{figures/hpccg_weak}
		}
		\subfigure[CoMD strong scalability]
		{
			\label{fig:comd_strong}
			\includegraphics[width=0.4\textwidth]{figures/comd_strong}
		}
		\subfigure[CoMD weak scalability]
		{
			\label{fig:comd_weak}
			\includegraphics[width=0.4\textwidth]{figures/comd_weak}
		}
	\end{center}
	\caption{Scalability test for number of processes from 1 to 256. Collocation ratio is 4 for srMPI.}
	\label{fig:scalability}
\end{figure*}

Comparing the baseline execution time between Figure~\ref{fig:hpccg_weak} and Figure~\ref{fig:comd_weak}, it is obvious that HPCCG and CoMD have different weak scaling characteristics. Keeping the same problem size per process, the execution time for CoMD increases by 8.9\% from 1 process to 256 processes, while the execution time is almost doubled for HPCCG. However, further analysis show that from 16 to 256 processes, the execution time increases by only 2.5\% for CoMD, and 1.0\% for HPCCG. We suspect that the results are not only determined by the scalability of the application, but also impacted by other factors, such as cache and memory contention on the same node, and network interference from other jobs running on the cluster. Remember that each node in the cluster has 20 cores and we always use all the cores of a node before adding another node. Therefore, it is very likely that the node level contention leads to the substantial increase in execution time for HPCCG. By analyzing the results from 16 to 256 processes, we believe both of HPCCG and CoMD are weak scaling applications. 

Different from strong scalability test, there is no correlation between srMPI's runtime overhead and the number of processes during the weak scalability test. The overhead is always below 2.1\%, except for the case of 32 processes for CoMD where the overhead is 5.0\%. %The reason for this exception is still under investigation.

\subsection{Performance under failures}
%The last set of experiments test srMPI's capability of tolerating failures and evaluate its performance under various failures by comparing with checkpointing/restart. 

As one main goal of this work is to achieve fault tolerance, an integrated fault injector is required to evaluate the effectiveness and efficiency of rsMPI to tolerate failures during execution. To produce failures in a manner similar to naturally occurring process failures, our failure injector is designed to be distributed and co-exist with all rsMPI processes. Failure is injected by sending a specific signal to the target process.

Failure detection is beyond the scope of srMPI, and we assume the underlying hardware platform has a RAS system that provides this functionality. In our test system, we emulate a RAS system with a signal handler installed at every main and shadow. The signal handler catches failure signal sent from the failure injector, and uses a rsMPI defined failure message via a dedicated communicator to notify all other processes of the failure. 
%To detect failure, srMPI receiving operation checks for failure messages before performing the actual receiving. 
Similar to ULFM, process in srMPI can only detect failure when it does an MPI receive operation. In a srMPI receive, 
a process checks for failure messages before it does the actual MPI receive operation.

We also implemented checkpointing to compare with srMPI in the presence of failures. To be optimistic, we chose double in-memory checkpointing that is much more scalable then disk-based checkpointing~\cite{zheng2004ftc}. Same as leaping in srMPI, our implementation provides an API for process state registration. This API requires the same parameters as leap\_register\_state(void *addr, int count, MPI\_Datatype dt), but internally, it allocates extra memory in order to store the state of a ``buddy" process. Another provided API is checkpoint(), which can be used to insert a checkpoint in the application code. For fairness, our implementation also uses MPI messages to transfer state.  
For both srMPI and checkpointing/restart, we assume a 60 seconds rebooting time after a failure. All experiments run with 256 application-visible processes, and the results are average of 5 runs. 

Firstly, we tested the effectiveness of leaping. For each application, we identified the process state and register them with rsMPI. Figure~\ref{fig:single_failure} shows the execution time of HPCCG with a single failure injected at various locations. The blue solid line represents srMPI without any forced leaping, and the red dash line represents srMPI with periodic forced leaping. Note that the execution time is reduced compared to previous results because we reduced the number of iterations for the application main loop from 5000 to 150, so that there is no need for any forced leaping by buffer overflow. Every time we set our failure injector to randomly pick a process to inject a failure, and the failure is scheduled to occur at certain point during the execution. Corresponding to the x-axis, the scheduled failure time varies from 10\% to 90\% of the application's execution. For example, 10\% means the application completes 15 iterations for a total of 150 iterations. 

\begin{figure}[!t]
  \begin{center}
      \includegraphics[width=\columnwidth]{figures/single_failure}
  \end{center}
  \caption{Execution time of HPCCG with a single injected failure. Collocation ratio is 2 for srMPI.}
  \label{fig:single_failure}
\end{figure}

As expected, without forced leaping the execution time increases with the failure occurrence time, as reflected by the blue line in Figure~\ref{fig:single_failure}. The reason is that failure recovery time for srMPI is proportional to the amount of divergence between mains and shadows, and the divergence grows as the execution proceeds. On the other hand, forced leaping can effectively reduce the divergence by leaping the shadow forward to the state of its associated main, similar to the idea that checkpointing can reduce the amount of wasted work due to a failure by saving the execution state. To prove the effectiveness of leaping, we insert 4 forced leaping at 20\%, 40\%, 60\% and 80\% of the execution. The red line in Figure~\ref{fig:single_failure} clears show that the divergence effect is bounded due to periodic leaping, regardless of the failure occurrence time.

Next, we compare rsMPI with checkpointing for multiple failures. To run the same number of application-visible processes, rsMPI needs more nodes than checkpointing to host the shadow processes. For fairness, we take into account the extra hardware cost when comparing srMPI to checkpointing, by defining the following metric:
$$\text{Efficiency} = \frac{T_f \times N}{T_e \times M}$$
, where $T_f$ and $N$ are the execution time and number of nodes without failures, and $T_e$ and $M$ are the actual execution time and required number of nodes for a specific fault tolerance mechanism. Intuitively, $T_f \times N$ represents the total amount of workload required by the application, and $T_e \times M$ is the actual amount of work carried out. The efficiency will be in the range 0 to 1, inclusive, and the higher is the better.

The forced leaping interval for an application is selected such that no buffer overflow at the shadows would take place. Therefore, the interval should vary from system to system and also depends on the application patterns. We assume checkpointing/restart has the same buffer pressure as it needs to perform message logging, so its checkpointing interval is selected based on the same metric as rsMPI. We evaluated rsMPI with 2 different collocation ratios, i.e., 2 and 4. When collocation ratio is 2, rsMPI uses 50\% more nodes than checkpointing, and the execution rate of each shadow is roughly 50\% of the processor rate. Therefore, we set the checkpointing interval to be the same as the forced leaping interval for srMPI. When collocation ratio is 4, rsMPI needs 25\% more nodes, and each shadow's rate is roughly 25\% of the processor rate. As a result, we loose the checkpointing interval to be twice of the forced leaping interval. 

With the checkpointing and forced leaping inserted to the application code, we randomly injected up to 10 failures into the execution. Figure~\ref{fig:multiple_failure} shows the comparison between checkpointing and srMPI (collocation ratio of 4) for both execution time and efficiency defined above. Although the failure-free execution time of srMPI is slightly larger than that of checkpointing, which results from srMPI's consistency protocol, the failure recovery time of checkpointing immediately overwhelms that of srMPI as failures occur. With 10 failures, the execution time of checkpointing is 42.8\% more than that of srMPI. Considering hardware overhead, the efficiency of checkpointing is also worse than that of srMPI when the number of failures reaches 6.

\begin{figure}[!t]
  \begin{center}
      \includegraphics[width=\columnwidth]{figures/multiple_failure}
  \end{center}
  \caption{Execution time of HPCCG with multiple injected failures. Collocation ratio is 4 for srMPI.}
  \label{fig:multiple_failure}
\end{figure}

Between srMPI with collocation ratio of 2 and srMPI with collocation ratio of 4, srMPI with collocation ratio of 2 wins in execution time, while srMPI with collocation ratio of 4 wins in efficiency. 
%execution time is xx faster than checkpointing. It is projected to beat checkpointing in efficiency when 40 failures.


\section{\uppercase{Conclusion and Future Work}}
\label{sec:conclusion}

In this paper we have introduced shadow computing, an energy efficient
method to provide fault tolerate execution without the limitations of
checkpointing. We then compared this to other known methods,
replication and re-execution, and concluded that shadow computing is
always more energy efficient. We also observed that the amount of
energy saving is highly dependent upon the rate of failure and the
amount of slack present in the system.

Fully harnessing the potential of shadow computing to deal with
failures brings about several challenging questions that need to be
addressed: How can this concept be used to improve fault detection and
layer coordination, understanding faults and silent errors and
improving situational awareness? What level of synchronization is
required between the main process and its associated shadow processes
to minimize impact on other application processes? What state, if any,
must be saved to ensure “smooth” transition to the primary shadow
process upon failure of the main process?  Future work will be focused
on investigating these questions for different types of failure to
better understand the advantages and limitations of this approach to
achieve high levels of fault-tolerance in extreme scale cloud
computing environments.

%\end{document}  % This is where a 'short' article might terminate

%ACKNOWLEDGMENTS are optional
%\section{Acknowledgments}
%This section is optional; it is a location for you
%to acknowledge grants, funding, editing assistance and
%what have you.  In the present case, for example, the
%authors would like to thank Gerald Murray of ACM for
%his help in codifying this \textit{Author's Guide}
%and the \textbf{.cls} and \textbf{.tex} files that it describes.

%
% The following two commands are all you need in the
% initial runs of your .tex file to
% produce the bibliography for the citations in your paper.
\bibliographystyle{abbrv}
\bibliography{lazy_shadowing}  % sigproc.bib is the name of the Bibliography in this case
% You must have a proper ".bib" file
%  and remember to run:
% latex bibtex latex latex
% to resolve all references
%
% ACM needs 'a single self-contained file'!
%
%APPENDICES are optional
%\balancecolumns

%\balancecolumns % GM June 2007
% That's all folks!
\end{document}
