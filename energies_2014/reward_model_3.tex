\noindent 

Contrary to replication, re-execution initially assigns a single
process for the execution of a task. If the original task fails, the
process is re-executed. In the cloud computing execution framework
this is equivalent to a checkpoint/restart, the checkpoint is
implicitly taken at the end of each phase and because the tasks are
loosely coupled they can restart independently. 

Based on the one failure assumption, two cases must be considered to
calculate the task completion time. If no failure occurs, the task
completion time is:
\begin{equation}
t_c=\frac{W}{\sigma_{max}}
\end{equation}
In case of failure, however, the completion time is equal to the sum
of the time elapsed until failure and the time needed for
re-execution. Again, we use the expected value
$t_f^*=\frac{\int_0^{t_c}t \times f_m(t)dt}{\int_0^{t_c}f_m(t)dt}$ to
approximate the time that successfully completed processes have to
spend waiting for the last one.

Similar to Shadow Replication, the income for re-execution is the
weighted average of the two cases:
\begin{equation}
E[\text{income}]=(1-P_f) \times r(t_c) + P_f \times r(t_c+t_f^{*})
\end{equation}

For one task, if no failure occurs then the expected energy consumption can be
calculated as
\begin{equation}
E_5=(1 - \int_0^{t_c} f_m(t)dt) \times (E(\sigma_{max},t_c)+ P_f \times E(0,t_f^{*}))
\label{eq:energy_first_task}
\end{equation}

If failure occurs, however, the expected energy consumption can be calculated
as
\begin{equation}
E_6=\int_0^{t_c}(E(\sigma_{max},t) + E(\sigma_{max},t_c)) \times f_m(t) dt
\label{eq:energy_rexecution_task}
\end{equation}
Therefore, the expected energy consumption by re-execution for
completing a job of $N$ tasks is
\begin{equation}
E[energy]=N \times (E_5 + E_6)
\end{equation}
