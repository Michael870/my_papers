\noindent 

Cloud Computing has emerged as an attractive platform for increasingly
diverse compute- and data-intensive applications, as it allows for
low-entry costs, on demand resource provisioning and allocation and
reduced cost of maintaining internal IT
infrastructure~\cite{tchana_cits_2012}. Cloud computing will continue
to grow and attract attention from commercial and public market
segments. Recent studies predict annual growth rate of 17.7 percent by
2016, making cloud computing the fastest growing segment in the
software industry.



In its basic form, a cloud computing infrastructure is a large cluster
of interconnected back-end servers hosted in a datacenter and
provisioned to deliver on-demand, "pay-as-you-go" services and
computing resources to customers through a front-end
interface~\cite{ec2_site}. As the demand for cloud computing
accelerates, cloud service providers (CSPs) will be faced with the
need to expand their underlying infrastructure to ensure the expected
levels of performance, reliability and cost-effectiveness, resulting
in a multifold increase in the number of computing, storage and
communication components in their datacenters. 

The direct implication of large datacenters is increased management complexity 
of the computing infrastructure, high-levels of energy consumption and propensity to
failure. The benefits of green cloud computing are clear.  As data centers are fast becoming a major source of global energy consumption,
the potential savings related to energy use, $CO_2$ emissions and e-waste are undeniable. Achieving these savings, however, calls for new computation models 
designed to reduce energy and power consumption and promote environmentally friendly cloud computing execution environments.

Another aspect of cloud computing, at scale, stems from its propensity to failure. 
While the likelihood of a server failure is very small, the
sheer number of computing, storage and communications components that
can fail, however, is daunting. At such a large scale, failure becomes
the norm rather than an exception~\cite{schroeder_2010_dsc}.

As the number of users delegating their computing tasks to CSPs
increases, Service Level Agreements (SLAs) become a critical aspect
for a sustainable cloud computing business model. In its basic form,
an SLA is a contract between the CSPs and consumers that specifies the
terms and conditions under which the service is to be provided,
including expected response time and reliability. Failure to deliver
the service as specified in the SLA subjects the CSP to pay a penalty,
resulting in a loss of revenue.

In addition to penalties resulting from failure to meet the SLA
requirement, CSPs face rising energy costs of their large-scale
datacenters.  It is reported that energy costs alone could account
for 23-50\% of the expenses~\cite{Elnozahy03energyconservation} and
this bill mounts up to \$30 billion
worldwide~\cite{Raghavendra:2008:NPS}. This raises the question of how
fault tolerance might impact power consumption and ultimately its impact on the environment.

Current fault tolerance approaches rely upon either time or hardware
redundancy in order to tolerate failure. The first approach, which
uses time redundancy, requires the re-execution of the failed task
after the failure is detected~\cite{Gelenbe:1990:OPT}.
Although this can further be optimized
by the use of checkpointing and roll-back recovery, such an approach
can result in a significant delay increase subjecting CSPs to penalties, when SLA terms are violated,
and high energy costs due to re-execution of failing tasks. 

%, with two consequences on
%profit. First, the CSP may be subjected to paying a penalty to the
%customer for failure to deliver the expected service within the
%negotiated time constraints. Second, the need to re-execute the
%failed task increases energy consumption.


The second approach exploits hardware redundancy and executes multiple
instances of the same task in parallel to overcome failure and
guarantee that at least one task reaches completion.  This approach,
which has been used extensively to deal with failure in critical
applications, is currently used in cloud-computing to provide fault
tolerance while hiding the delay of
re-execution~\cite{tsai_isads_2011,ko_socc_2010}. This solution,
however, increases the energy consumption for a given service, which
in turn might outweigh the profit gained by providing the service.
The trade-off between profit and fault-tolerance calls for new
frameworks to take both SLA requirements and energy awareness in
dealing with failures.

In this paper, we address the above trade-off challenge and propose an
energy-aware, SLA-based profit maximization framework, referred to as
``Shadow Replication'', for resilience in cloud computing.  Similar to
traditional replication, Shadow Replication ensures successful task
completion by concurrently running multiple instances of the same
task. Contrary to traditional replication, however, Shadow Replication
executes the main instance of the task at the speed required to
maximize profit and uses dynamic voltage and frequency scaling (DVFS)
to slow down the execution of the replicas in order to reduce energy-consumption, thereby enabling a
parameterized trade-off between resiliency to failure, response time, and energy consumption. 
Furthermore, the model allows CSPs to maximize the expected profit,
by accounting for income and potential penalties incurred when SLAs are not met, while reducing the impact of energy consumption on the environment.

%the
%basic idea of shadow replication is to ensure the completion of a task
%by concurrently running multiple replicas of a process, but in a much
%smarter way: it provides fault tolerance by combining replication with
%dynamic voltage and frequency scaling (DVFS), enabling a parameterized
%tradeoff between time and hardware redundancy.

The main challenge of Shadow Replication resides in determining
jointly the execution speeds of all task instances, both before and
after a failure occurs, with the objective to minimize energy and
maximize profit.  In this paper, we propose a reward-based analytical
framework to achieve this objective. The main contributions of this paper
are as follows:

\begin{itemize}
%\item A profit-based optimization framework to compute the different speeds of %formalization of shadow replication as 
%\item The use of Shadow Replication to maximize the economic potential
%  of cloud computing.

\item An energy-aware, SLA-based, profit maximization execution model, referred to as
``Shadow Replication'', for resilient, green cloud computing.

\item A profit-based optimization model to explore the applicability of
  Shadow Replication to cloud computing, and to determine the optimal
  speeds of all task instances to maximize profit, while reducing the impact of energy consumption on the environment.

\item In environments where either the specification or
the detection of failure is hard to achieve, we propose a sub-optimal,
yet practical resilience scheme, called profit-aware stretched
replication.

\item An evaluation framework to analyze profit and 
energy savings achievable by Shadow Replication, compared to existing resilience methods.
%including a
%comparative analysis of the different resilience methods, to identify
%the most profitable technique for various computing environments.

%\item An analysis of the profit gain and energy savings achievable by
 % using shadow replication.
%\item A comparative analysis of the different resilience methods, including pure %replication and re-execution, and identification 
 % of the most profitable technique for a given system configuration and failure %rate.
\end{itemize}

%The analysis shows that in all cases, shadow replication outperforms
%traditional replication. Furthermore, the results show that shadow
%replication is the most efficient fault tolerance method when the rate
%of system failure is high. It is also observed that when the target
%response time is stringent, shadow replication converges to
%traditional replication, as expected.

The analysis shows that in all cases, Shadow Replication outperforms
existing fault tolerance methods. Furthermore, shadow
replication would converge to traditional replication, when target response time is stringent, and to re-execution when target response time is relaxed or failure is unlikely, as expected.

The rest of the paper is organized as follows. We begin by describing a
computing model typically used in cloud computing for compute- and
data-intensive applications in
Section \ref{sec:cloud_computing_workload}. We then introduce
the Shadow Replication framework in
Section \ref{sec:shadow_replication}. Section
\ref{sec:reward_model},  \ref{sec:reward_model_2}, and \ref{sec:reward_model_3} present our analytical models and optimization
problem formalization, followed by experiments and evaluation in
section \ref{sec:evaluation}. Section \ref{sec:related_work} briefly
surveys related work. Section \ref{sec:conclusion} concludes this work.

