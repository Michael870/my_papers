\noindent 

We compare Shadow Replication to two other replication techniques,
traditional replication and profit-aware stretched replication.
Traditional replication requires that the two processes always execute
at the same speed $\sigma_{max}$. Unlike traditional replication
Shadow Replication is dependent upon failure detection, enabling the
replica to increase its execution speed upon failure and maintain the
targeted response time thus maximizing profit. While this is the case
in many computing environments, there are cases where failure
detection may not be possible. To address this limitation, we propose
profit-aware stretched replication, whereby both the main process and
the shadow execute independently at stretched speeds to meet the
expected response time, without the need for the main processes failure
detection. In profit-aware stretched replication both the main and
shadow execute at speed $\sigma_r$, found by optimizing the profit
model.  For both traditional replication and stretched replication,
the task completion time is independent of failure and can be directly
calculated as:
\begin{equation}
t_c=\frac{W}{\sigma_{max}} \text{ or } t_c=\frac{W}{\sigma_r}
\end{equation}

%Profit-aware stretched replication is similar to traditional
%replication in that it executes at one speed but different in that it
%determines the execution speed that maximizes the profit much like
%shadow replication.

%In order to evaluate the benefits of shadow replication relative to
%other resilience schemes, we consider dual level of redundancy and
%assume that at most one failure may occur, following the same failure
%model as the one defined for shadow replication. To this end, we
%propose a new scheme, referred to as profit-aware stretched
%replication. We also derive the expected energy of pure replication
%scheme, which is then used for comparative analysis.

Since all tasks will have the same completion time, the job completion
time would also be $t_c$. Further, the expected income, which depends
on negotiated reward function and job completion time, is independent
of failure:
\begin{equation}
E[income]=r(t_c)
\end{equation}

Since both traditional replication and profit-aware stretched
replication are special cases of our Shadow Replication paradigm where
$\sigma_m=\sigma_b=\sigma_a=\sigma_{max}$ or
$\sigma_m=\sigma_b=\sigma_a=\sigma_r$ respectively, we can easily derive the
expected energy consumption using \refeq{eq:total_energy} with $E_4$
fixed at 0 and then compute the expected expense using \refeq{eq:expense}.
