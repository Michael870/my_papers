%  LaTeX support: latex@mdpi.com
%  In case you need support, please attach any log files that you could have, and specify the details of your LaTeX setup (which operating system and LaTeX version / tools you are using).

%=================================================================

% LaTeX Class File and Rendering Mode (choose one)
% You will need to save the "mdpi.cls" and "mdpi.bst" files into the same folder as this template file.

%=================================================================

\documentclass[journal,article,accept,moreauthors,pdftex,12pt,a4paper,energies]{mdpi} 
%--------------------
% Class Options:
%--------------------
% journal
%----------
% Choose between the following MDPI journals:
% actuators, administrativesciences, aerospace, agriculture, agronomy, algorithms, animals, antibiotics, antibodies, antioxidants, appliedsciences, arts, atmosphere, atoms, axioms, behavioralsciences, bioengineering, biology, biomedicines, biomolecules, biosensors, brainsciences, buildings, cancers, catalysts, cells, challenges, chemosensors, children, chromatography, climate, coatings, computation, computers, cosmetics, crystals, dentistryjournal, diagnostics, diseases, diversity, econometrics, economies, education, electronics, energies, entropy, environmentalsciences, environments, fibers, foods, forests, futureinternet, galaxies, games, genes, geosciences, healthcare, humanities, informatics, information, inorganics, insects, ijerph, ijfs, ijms, ijgi, jcdd, jcm, jdb, jfb, joi, jlpea, jmse, jpcg, jpm, jrfm, jsan, land, laws, life, lubricants, machines, marinedrugs, materials, mathematics, medicalsciences, membranes, metabolites, metals, microarrays, micromachines, microorganisms, minerals, molbank, molecules, nanomaterials, ncrna, nutrients, pathogens, pharmaceuticals, pharmaceutics, pharmacy, photonics, plants, polymers, processes, proteomes, publications, religions, remotesensing, resources, risks, robotics, sensors, socialsciences, societies, sports, sustainability, symmetry, systems, technologies, toxics, toxins, vaccines, veterinarysciences, viruses, water
%---------
% article
%---------
% The default type of manuscript is article, but could be replaced by using one of the class options: 
% article, review, communication, commentary, bookreview, correction, addendum, editorial, changes, supfile, casereport, comment, conceptpaper, conferencereport, meetingreport, discussion, essay, letter, newbookreceived, opinion, projectreport, reply, retraction, shortnote, technicalnote, creative
%----------
% submit
%----------
% The class option "submit" will be changed to "accept" by the Editorial Office when the paper is accepted. This will only make changes to the frontpage (e.g. the logo of the journal will get visible), the headings, and the copyright information. Journal info and pagination for accepted papers will also be assigned by the Editorial Office.
% Please insert a blank line is before and after all equation and eqnarray environments to ensure proper line numbering when option submit is chosen
%------------------
% moreauthors
%------------------
% If there is only one author the class option oneauthor should be used. Otherwise use the class option moreauthors.
%---------
% pdftex
%---------
% The option "pdftex" is for use with pdfLaTeX only. If eps figure are used, use the optioin "dvipdfm", with LaTeX and dvi2pdf only.

%=================================================================
\setcounter{page}{1}
\lastpage{x}
\doinum{10.3390/------}
\pubvolume{xx}
\pubyear{2014}
\history{Received: xx / Accepted: xx / Published: xx}
%------------------------------------------------------------------
% The following line should be uncommented if the LaTeX file is uploaded to arXiv.org
%\pdfoutput=1

%=================================================================

% Add packages and commands to include here
% The amsmath, amsthm, amssymb, hyperref, caption, float and color packages are loaded by the MDPI class.
\usepackage{epstopdf}
\usepackage{graphicx}
\usepackage{subfigure}
\usepackage{tabularx}
\usepackage{multirow}
\usepackage{booktabs} 
\usepackage{soul}
%\newcommand{\hl}[1]{\textcolor{red}{\texttt{#1}}}

\providecommand{\sectionname}{Section}
\providecommand{\assignmentname}{Exercise}
\providecommand{\equationname}{Equation}
\providecommand{\figurename}{Fig\slashure}
\providecommand{\tablename}{Table}
\newcommand{\reffig}[1]{\normalsize\figurename~\ref{#1}\normalsize}
\newcommand{\reftab}[1]{\tablename~\ref{#1}}
\newcommand{\reflst}[1]{\lstlistingname~\ref{#1}}
\newcommand{\refchap}[1]{\chaptername~\ref{#1}}
\newcommand{\refapp}[1]{\appendixname~\ref{#1}}
\newcommand{\refsec}[1]{\sectionname~\ref{#1}}
\newcommand{\refeq}[1]{\equationname~\ref{#1}}
\newcommand{\refpage}[1]{on page~\pageref{#1}}



%=================================================================
%% Please use the following mathematics environments:
%\theoremstyle{mdpi}
%\newcounter{thm}
%\setcounter{thm}{0}
%\newcounter{ex}
%\setcounter{ex}{0}
%\newcounter{re}
%\setcounter{re}{0}
%\newtheorem{Theorem}[thm]{Theorem}
%\newtheorem{Lemma}[thm]{Lemma}
%\newtheorem{Characterization}[thm]{Characterization}
%\newtheorem{Proposition}[thm]{Proposition}
%\newtheorem{Property}[thm]{Property}
%\newtheorem{Problem}[thm]{Problem}
%\newtheorem{Example}[ex]{Example}
%\newtheorem{Remark}[re]{Remark}
%\newtheorem{Corollary}[thm]{Corollary}
%\newtheorem{Definition}[thm]{Definition}
%% For proofs, please use the proof environment (the amsthm package is loaded by the MDPI class).

%=================================================================

% Full title of the paper (Capitalized)
\Title{Shadow Replication: An Energy-Aware, Fault-Tolerant Computational Model for Green Cloud Computing}

% Authors (Add full first names)
\Author{Xiaolong Cui$^{1}$, Bryan Mills$^{1}$, Taieb Znati$^{1}$*, and Rami Melhem$^{1}$ }

% Affiliations / Addresses (Add [1] after \address if there is only one affiliation.)
\address{%
$^{1}$ Department of Computer Science, University of Pittsburgh\\
}

% Contact information of the corresponding author (Add [2] after \corres if there are more than one corresponding author.)
\corres{znati@cs.pitt.edu}

% Abstract (Do not use inserted blank lines, i.e. \\) 
\abstract{As the demand for cloud computing continues to increase, cloud service
providers face the daunting challenge to meet the negotiated SLA
agreement, in terms of reliability and timely performance, while
achieving cost-effectiveness. This challenge is increasingly
compounded by the increasing likelihood of failure in large-scale
clouds and the rising cost of energy consumption.  This paper proposes
Shadow Replication, a novel profit-maximization resiliency model,
which seamlessly addresses failure at scale, while minimizing energy
consumption. The basic tenet of the model is to associate a suite of
shadow processes to execute concurrently with the main process, but
initially at a much reduced execution speed, to overcome failures as
they occur. Two computationally-feasible schemes are proposed to
achieve shadow replication. A performance evaluation framework is
developed to analyze these schemes and compare their performance to
traditional replication-based fault tolerance methods, focusing on the
inherent tradeoff between fault tolerance, the specified SLA and
profit maximization. The results show Shadow Replication leads to
significant energy reduction, and is better suited for
compute-intensive execution models, where up to 30\% more profit
increase can be achieved.


%several experimental studies are carried out to assess the performance
%of the different resiliency schemes, with respect to profit
%maximization in different cloud computing environments. 




%The challenge is to derive the
%execution speed, both before and after failure, of a shadow in order
%to ensure adherence to the negotiated SLA, while maximizing profit. To
%this end, we present an optimization model to derive the shadow
%execution speeds, which takes into consideration a computing node
%failure rate and the negotiated SLA. Several computationally-feasible
%methods are then proposed to solve this model. 



%As companies continue to increase their reliance upon cloud computing
%services there will be an increasing demand for reliable and timely
%service. However, as cloud-based systems increase in size and
%complexity it is expected that reliability will degrade, causing both
%delays in service and increases in energy consumption. This will cause
%fault tolerance to be a critical system feature to providing
%applications at large scale. In this work, we propose ``shadow
%replication'', a fault tolerance method that makes use of DVFS to
%provide energy-aware, profit-maximizing system resilience to
%task-based cloud computing services.  We analyze different resilience
%methods and identify the system parameters which are most relevant to
%the tradeoff between fault tolerance and profit, and present results
%which pinpoint the most profitable method.  We also show that in
%certain systems shadow replication can achieve 10-30\% more profit
%than existing fault tolerance methods, i.e. re-execution and
%traditional replication.

%
%
% ``shadow replication'' to
%provide an energy-aware, profit-optimized method that can result in
%upto two-times the profit achieved by existing fault tolerance
%methods. Additionally, we develop analytical models to demonstrate the
%benefits of our approach at the scale expected in future cloud
%computing environments.
}

% Keywords: add 3 to 10 keywords
\keyword{shadow computing; fault tolerance; scheduling; resilience; energy-aware}

% The fields PACS, MSC, and JEL may be left empty or commented out if not applicable
%\PACS{}
%\MSC{}
%\JEL{}

\begin{document}

%%%%%%%%%%%%%%%%%%%%%%%%%%%%%%%%%%%%%%%%%%

\section{Introduction}
\label{sec:introduction}
\noindent 
Cloud Computing has emerged as an attractive platform for increasingly
diverse compute- and data-intensive applications, as it allows for
low-entry costs, on demand resource provisioning and allocation and
reduced cost of maintaining internal IT
infrastructure~\cite{tchana_cits_2012}. Cloud computing will continue
to grow and attract attention from commercial and public market
segments. Recent studies predict annual growth rate of 17.7 percent by
2016, making cloud computing the fastest growing segment in the
software industry.

In its basic form, a cloud computing infrastructure is a large cluster
of interconnected back-end servers hosted in a datacenter and
provisioned to deliver on-demand, "pay-as-you-go" services and
computing resources to customers through a front-end
interface~\cite{ec2_site}. As the demand for cloud computing
accelerates, cloud service providers (CSPs) will be faced with the
need to expand their underlying infrastructure to ensure the expected
levels of performance, reliability and cost-effectiveness, resulting
in a multifold increase in the number of computing, storage and
communication components in their datacenters. The direct implication of large datacenters is increased management complexity and propensity to
failure. While the likelihood of a server failure is very small, the
sheer number of computing, storage and communications components that
can fail, however, is daunting. At such a large scale, failure becomes
the norm rather than an exception~\cite{schroeder_2010_dsc}.

As the number of users delegating their computing tasks to CSPs
increases, Service Level Agreements (SLAs) become a critical aspect
for a sustainable cloud computing business model. In its basic form,
an SLA is a contract between the CSPs and consumers that specifies the
terms and conditions under which the service is to be provided,
including expected response time and reliability. Failure to deliver
the service as specified in the SLA subjects the CSP to pay a penalty,
resulting in a loss of revenue.

In addition to penalties resulting from failure to meet the SLA
requirement, CSPs face rising energy costs of their large-scale
datacenters.  It is reported that energy costs alone could account
for 23-50\% of the expenses~\cite{Elnozahy03energyconservation} and
this bill mounts up to \$30 billion
worldwide~\cite{Raghavendra:2008:NPS}. This raises the question of how
fault tolerance might impact power consumption and ultimately the
expected profit of the service providers.

Current fault tolerance approaches rely upon either time or hardware
redundancy in order to tolerate failure. The first approach, which
uses time redundancy, requires the re-execution of the failed task
after the failure is detected.  Although this can further be optimized
by the use of checkpointing and roll-back recovery, such an approach
can result in a significant delay increase subjecting CSPs to penalties, when SLA terms are violated,
and high energy costs due to re-execution of failing tasks.

%, with two consequences on
%profit. First, the CSP may be subjected to paying a penalty to the
%customer for failure to deliver the expected service within the
%negotiated time constraints. Second, the need to re-execute the
%failed task increases energy consumption.


The second approach exploits hardware redundancy and executes multiple
instances of the same task in parallel to overcome failure and
guarantee that at least one task reaches completion.  This approach,
which has been used extensively to deal with failure in critical
applications, is currently used in cloud-computing to provide fault
tolerance while hiding the delay of
re-execution~\cite{tsai_isads_2011,ko_socc_2010}. This solution,
however, increases the energy consumption for a given service, which
in turn might outweigh the profit gained by providing the service.
The trade-off between profit and fault-tolerance calls for new
frameworks to take both SLA requirements and energy awareness in
dealing with failures.

In this paper, we address the above trade-off challenge and propose an
energy-aware, SLA-based profit maximization framework, referred to as
``Shadow Replication'', for resilience in cloud computing.  Similar to
traditional replication, Shadow Replication ensures successful task
completion by concurrently running multiple instances of the same
task. Contrary to traditional replication, however, Shadow Replication
executes the main instance of the task at the speed required to
maximize profit and uses dynamic voltage and frequency scaling (DVFS)
to slow down the execution of the replicas, thereby enabling a
parameterized trade-off between response time, energy consumption and
hardware redundancy. This allows CSPs to maximize the expected profit
by accounting for income, potential penalties and energy cost.

%the
%basic idea of shadow replication is to ensure the completion of a task
%by concurrently running multiple replicas of a process, but in a much
%smarter way: it provides fault tolerance by combining replication with
%dynamic voltage and frequency scaling (DVFS), enabling a parameterized
%tradeoff between time and hardware redundancy.

The main challenge of Shadow Replication resides in determining
jointly the execution speeds of all task instances, both before and
after a failure occurs, with the objective to minimize energy and
maximize profit.  In this paper, we propose a reward-based analytical
framework to achieve this objective. The main contributions of this paper
are as follows:

\begin{itemize}
%\item A profit-based optimization framework to compute the different speeds of %formalization of shadow replication as 
%\item The use of Shadow Replication to maximize the economic potential
%  of cloud computing.

\item An energy-aware, SLA-based, profit maximization execution model, referred to as
``Shadow Replication'', for resilient cloud computing.

\item A profit-based optimization model to explore the applicability of
  Shadow Replication to cloud computing, and to determine the optimal
  speeds of all task instances to maximize profit.

\item In environments where either the specification or
the detection of failure is hard to achieve, we propose a sub-optimal,
yet practical resilience scheme, called profit-aware stretched
replication.

\item An evaluation framework to analyze profit and 
energy savings achievable by Shadow Replication, compared to existing resilience methods.
%including a
%comparative analysis of the different resilience methods, to identify
%the most profitable technique for various computing environments.

%\item An analysis of the profit gain and energy savings achievable by
 % using shadow replication.
%\item A comparative analysis of the different resilience methods, including pure %replication and re-execution, and identification 
 % of the most profitable technique for a given system configuration and failure %rate.
\end{itemize}

%The analysis shows that in all cases, shadow replication outperforms
%traditional replication. Furthermore, the results show that shadow
%replication is the most efficient fault tolerance method when the rate
%of system failure is high. It is also observed that when the target
%response time is stringent, shadow replication converges to
%traditional replication, as expected.

The analysis shows that in all cases, Shadow Replication outperforms
existing fault tolerance methods. Furthermore, shadow
replication would converge to traditional replication, when target response time is stringent, and to re-execution when target response time is relaxed or failure is unlikely, as expected.

The rest of the paper is organized as follows. We begin by describing a
computing model typically used in cloud computing for compute- and
data-intensive applications in
Section \ref{sec:cloud_computing_workload}. We then introduce
the Shadow Replication framework in
Section \ref{sec:shadow_replication}. Section
\ref{sec:reward_model},  \ref{sec:reward_model_2}, and \ref{sec:reward_model_3} present our analytical models and optimization
problem formalization, followed by experiments and evaluation in
section \ref{sec:evaluation}. Section \ref{sec:related_work} briefly
surveys related work. Section \ref{sec:conclusion} concludes this work.



\section{Cloud Workload Characterization}
\label{sec:cloud_computing_workload}
%\noindent 

%%% obvious!
%Today's popular online services, such as web search, video streaming,
%and social networks, are all powered by large data centers. In
%addition to these public services, a lot of scientific research and
%business activities are migrating onto cloud
%computing\cite{Ferdman:2012:CCS:2150976.2150982,mrbs}. Table \ref{tbl:apps}
%lists several popular classes of applications that are running on the
%cloud.
%
%
%
%\begin{table}[!h]
%	\caption{Typical cloud computing applications.}
%	\centering
%		\begin{tabularx}{\columnwidth}{|l|X|}
%			\hline
%			Class                          & Examples                         \\
%			\hline
%			Data analytics   & bioinformatics\\ 
%			& business intelligence\\
%			\hline
%			Graph analytics  & social networks\\ 
%			& recommendation \\
%			\hline
%			Web search       & text processing\\
%			& recommendation\\
%			\hline
%			\end{tabularx}
%	\label{tbl:apps}
%\end{table}
%
%Two main characteristics of the above applications are large dataload
%and high parallelism. In 2008, Google processed 20 PB of data with
%MapReduce each day; in April 2009, a blog revealed eBay's 2 enormous
%data warehouses: one with 2 PB of data and the other with 6.5 PB of
%data; shortly thereafter, Facebook also shocked the world with its 2.5
%PT of user data which kept growing at 15 TB per
%day \cite{lin2010data}. Without high parallelism, these huge amount of
%data are beyond our capability of processing. 

%If a task is delayed 
%We consider a cloud computing environment to be a distributed job
%which is made up of a series of tasks executing a job
%which is composed of multiple tasks, each of size W , executing in
%parallel on different com- puting nodes, as depicted in Figure 2. For
%the job to complete, all tasks must finish. If a task is de- layed
%then all tasks must wait for the completion of the slowest task for
%the job to complete.
%
%For each job that processes these data, there are usually multiple
%phases that execute in sequence. During each phase, workload is
%splited into tasks and processed in parallel to speed up the whole
%process. A job of 2 phases is illustrated in
%Figure \ref{fig:system_model}.


%\section{CC Workload Characterization}

Cloud computing workload ranges from business applications and
intelligence, to analytics and social networks mining and log
analysis, to scientific applications in various fields of sciences and
discovery. These applications exhibit different behaviors, in term of
computation requirements and data access patterns. While some
applications are compute-intensive, others involve the processing of
increasingly large amounts of data. The scope and scale of these
applications are such that an instance of a job running one of these
applications requires the sequential execution of multiple computing
phases; each phase consists of thousands, if not millions, of tasks
scheduled to execute in parallel and involves the processing of a very
large amount of data~\cite{lin2010data,Ferdman:2012:CCS:2150976.2150982}. This
model is directly reflective of the \emph{MapReduce} computational
model, which is predominately used in
Cloud Computing \cite{mrbs}.  An instance of this model, is depicted in Figure \ref{fig:system_model}.


\begin{figure}[!h]
	\begin{center}
		\includegraphics[width=\columnwidth]{diagrams/system_model_1.pdf}
	\end{center}
	\caption{Cloud computing execution model with 2 phases.}
	\label{fig:system_model}
\end{figure}

Each job has a targeted response time defined
by the terms of the SLA. Further, the SLA defines the amount to be
paid for completing the job by the targeted response time as well as
the penalty to be incurred if the targeted response time is not
met. 

Each task is mapped to one compute core and executes at a speed, $\sigma$. The partition of the job among tasks is
such that each task processes a similar
workload, $W$. Consequently, baring failures, tasks are expected to
complete at about the same time. Therefore, the minimal response time
of each task, when no failure occurs, is
$t_{min}~=~\frac{W}{\sigma_{max}}$, where $\sigma_{max}$ is the maximum speed. This is also the minimal response
time of the entire phase. 

As the number of tasks increases, however, the likelihood of a task
failure during an execution of a given phase increases
accordingly. This underscores the importance of an energy-efficient
fault-tolerance model to mitigate the impact of a failing task on the
overall delay of the execution phase. The following section describes
Shadow Replication, a fault-tolerant, energy-aware computational model to achieve profit-maximizing,
energy-efficient resiliency in cloud computing.


%\noindent
%We consider a job running in a cloud computing environment to be
%comprised upon multiple phases that execute in sequence. A two-phase
%job is depicted in Figure \ref{fig:system_model}. During each phase
%multiple tasks are executing in parallel. Each task contains the same
%amount of workload, denoted as $W$. If a task is delayed then all
%tasks must wait for the completion of the slowest task for that phase
%for the job to complete. This model is directly reflective of the
%map-reduce execution model used predominately in cloud computing.
%
%The large amount of data being
%processed~\cite{lin2010data,Ferdman:2012:CCS:2150976.2150982} requires
%the use of thousands, if not millions, of tasks during each phase. As
%the number tasks increases, the likelihood of failure during an
%individual phase will also increase. Fault tolerance is critical at
%the phase-level because the delay of one task results in a delay of
%the entire phase. Hence, techniques such as shadow replication would
%be applied to each phase independently. Due to this independence
%between phases, we consider a single phase job during our analysis.
%
%%For the job to complete, all tasks must finish its workload of
%%$W$. If a task is delayed then all tasks must wait for the completion
%%of the slowest task for the job to complete.
%
%We assume that each executing task is mapped to one computing core and
%executes at a maximum speed of $\sigma_{max}$. Therefore, the minimal
%response time of each task, when no failure occurs, is
%$t_{min}=\frac{W}{\sigma_{max}}$, which is also the minimal response
%time of the entire job.
%
%Each job has a targeted response time defined by the terms of the
%SLA. Further, the SLA defines the amount to be paid for completing the
%job by the targeted response time as well as the penalty to be
%incurred if the targeted response time is not met. We will discuss
%this in detail in Section~\ref{sla_reward_model}.
%
%
%%In the following section, we assume the above job execution model and
%%describe a profit-based optimization framework to compute the optimal
%%speeds of shadow replication. In this framework, it is assumed that
%%failures can be detected.  While this is the case in many computing
%%environments, there are cases where failure detection may not be
%%possible. To address this limitation, we propose a sub-optimal shadow
%%replication scheme, whereby both the main process and the shadow
%%execute independently at stretched speeds to meet the expected
%%response time, without the need for the main processes failure
%%detection.


\section{Shadow Replication}
\label{sec:shadow_replication}
\begin{figure*}[!t]
	\begin{center}
		\subfigure[No Failure]
		{
			\label{fig:sc_no_fail}
			\includegraphics[width=0.32\textwidth]{diagrams/example1.pdf}
		}
		\subfigure[Shadow Process Failure]
		{
			\label{fig:sc_shadow_fail}
			\includegraphics[width=0.29\textwidth]{diagrams/example3.pdf}
		}
		\subfigure[Main Process Failure]
		{
			\label{fig:sc_main_fail}
			\includegraphics[width=0.33\textwidth]{diagrams/example2.png}
		}
	\end{center}
	\caption{Shadow replication for a single task and single replica}
	\label{fig:sc_overview}
\end{figure*}

\noindent 
The basic tenet of Shadow Replication is to associate with each
main process a suite of ``shadows'' whose size depends on the
``criticality'' of the application and its performance requirements,
as defined by the SLA. 

Formally, we define the Shadow Replication fault-tolerance model as follows:
\begin{itemize}
\item A main process, $P_m(W,\text{ }\sigma_m)$, whose responsibility is to executes a task of size $W$ at a speed of $\sigma_m$;
\item A suite of shadow processes, $P_{s}(W,\text{ }\sigma_b^s, \text{ }\sigma_a^s)$ ($1 \le s \le \cal S)$, where $\cal S$ is the size of the suite. 
The shadows execute on separate computing nodes. Each shadow process is associated with two execution speeds. All shadows start execution simultaneously with the main process at speed $\sigma_b^s$ ($1 \le s \le \cal S$). Upon failure of the main process, all shadows switch their executions to $\sigma_a^s$, with one shadow being designated as the new main process. This process continues until completion of the task.
\end{itemize}

To illustrate the behavior of Shadow Replication, we limit the number of shadows to a single process and consider the scenarios depicted in Figure \ref{fig:sc_overview}, assuming a single process failure. Figure \ref{fig:sc_no_fail} represents the case when neither the main nor the shadow fails. The main process, executing
at a higher speed, completes the task at time $t_c^m$. At this time, the shadow process, progressing at a lower speed, stops execution immediately. Figure \ref{fig:sc_shadow_fail} represents the case when the shadow fails. This failure, however, has no impact on the progress of the main process, which still completes the task at $t_c^m$. Figure \ref{fig:sc_main_fail} depicts the case when the main process fails while the shadow is in progress. After detecting the failure of the main process, the shadow begins execution at a higher speed, completing the task at time $t_c^s$. When possible, the shadow execution speed upon failure must be set so that $t_c^s$ does not exceed $t_c^m$. Given that the failure rate of an individual node is much lower than
the aggregate system failure, it is very likely that the main process
will always complete its execution successfully, thereby achieving fault tolerance at a significantly reduced cost of energy consumed by the shadow. %saving a lot of energy for its associated shadow processes. 

%In summary, shadow replication has the following characteristics:
%The proposed shadow replication model has several properties:
%\begin{itemize}
%\item The main process is associated with only one execution speed, $\sigma_m$, %which depends on the size of the task and the SLA requirement.
%\item The shadow processes execute with two different speeds, which are failure %dependent and when possible should be computed to meet the SLA requirement as %closely as possible. 
%\item The completion of the main process results in the immediate termination of %all shadow processes. Given that the failure rate of an individual node is much %lower than the aggregate system failure, it is very likely that the main process %completes successfully, saving a significant amount of energy, while achieving %high level of fault tolerance.
%\end{itemize}



A closer look at the model reveals that shadow
replication is a generalization of traditional fault tolerance
techniques, namely re-execution and traditional replication. If the
SLA specification allows for flexible completion time, shadow
replication would take advantage of the delay laxity to trade time
redundancy for energy savings. It is clear, therefore, that for a
large response time, Shadow Replication converges to re-execution, as
the shadow remains idle during the execution of the main process and
only starts execution upon failure. If the target response time is
stringent, however, Shadow Replication converges to pure replication,
as the shadow must execute simultaneously with the main at the same
speed. The flexibility of the Shadow Replication model provides the
basis for the design of a fault tolerance strategy that strikes a
balance between task completion time and energy saving, thereby
maximizing profit.

Given that the probability of two individual nodes executing the same
instances of a task fail at the same time is low, we will focus on the study of Shadow Replication model with a single shadow. It is clear, however, that the model can
be extended to support multiple processes, as required by the
application's fault-tolerance requirement. Furthermore, we adopt the
fail-stop~ fault model, where a processor stops execution once a fault
occurs and failure can be detected by other
processes\cite{gartner_faults_1999,cristian_comm_1991}.

%While all
%shadow processes are exact replicas of the main process, one among
%these processes, referred to as primary shadow process, is special in
%that it would become new main process if current one fails.

% stated in paragraph one now.
%In order to mask failure during a task,
%the shadow processes are scheduled to execute concurrently with the
%main process, but on different computing nodes. Furthermore, in order
%to reduce energy, shadow processes initially execute at decreasingly
%lower processor speeds.


%Moreover, one among the remaining shadow processes is promoted
%to primary shadow process. By allowing instantaneous fail-over with
%shadow processes in the event of failure, shadow replication eliminates
%even the smallest chance of data loss or disruption.


%Without loss of generality, from this point on we
%assume at most one failure would occur to a task and consider a dual
%level of redundancy, whereby only one shadow process is executed
%concurrently with each main process. The completion of the main or its
%shadow results in the successful execution of the task. If the main
%process fails, it is implied that shadow process would complete the
%task.



% needless words
%Figure \ref{fig:sc_overview} depicts how shadow replication works to
%complete a single task using a single replica. While the main process
%would execute at a certain fixed speed until task completion or
%failure, the shadow process may change its speed if it needs to take
%over the role of main process, in order to finish task in
%time. According to if and when failure occurs, there are three cases
%that may happen during the execution of the task.


%In this section we have defined shadow replication and described its
%potential in energy saving and profit gain in cloud computing
%environment. Three questions, however, remain to be answered before we
%expect shadow replication to be accepted as a novel fault tolerance
%technique for cloud computing: 1) is it truly possible for shadow
%computing to save energy and bring more profit over state-of-the-art
%fault tolerance approaches while meeting SLA requirements; 2) if so,
%how much benefit can it bring; 3) how to determine the optimal
%execution speeds for main process and shadow processes.
%
%To answer the above questions, in the following section we will
%introduce a reward-based analytical model for fault tolerance
%mechanisms with a dual-level redundancy under the assumption that at
%most one process may fail.


%\section{\uppercase{Resilient Methods}}
%\label{sec:resilient_methods}
%There are two main ways of handling task failures, re-execution or
replication. In the next section we will define these resilience
methods and introduce two new profit-optimized replication schemes.

\subsection{Re-execution}

Re-execution simply re-executes the task when a failure occurs until
the task completes successfully. This will result in delaying the job
completion because all tasks must complete in order to complete the
job. This will impact profit in at least two directions, first the
delay might result in the service provider paying a penelty to the
client and secondly it might result in increased energy consumption
because of the extended execution time.

\subsection{Traditional Replication}

Traditional replication is a method in which each task is replicated
on independent computing nodes, such that if one process fails its
replica process can continue executing as if the failure did not
occur. This is also referred to as process replication and has long
been deployed in mission critical applications. Replication has been
used extensively in cloud computing
\cite{tsai_isads_2011,ko_socc_2010} but is often criticized in
task-based jobs because of the use of additional resources. From a
profit standpoint this will reduce the liklihood of paying a penelty
under the SLA but could significanlly increase the energy consumption.

\subsection{Profit-aware Replication}

Service providers want to maximize their revenue and reduce their
expenses, in the next sections we will introduce two profit-optimized
resilience methods. These methods seek to strike a balance between
energy consumption and revenue obtained under the terms of the SLA.

\subsubsection{Optimized Stretched Replication}

Stretched replication works on the assumption that performing work
slowly can save energy. This is typically done through the use of
dynamic voltage and frequency scaling (DVFS). Stretched replication
slows down the execution of all processes to the slowest possible
speed while maximizing the profit. As we will show in detail in
Section \ref{sec:stretched_replication_model}, this optimization
accounts for the energy cost and the terms dictated in the SLA.

\hl{The only reason I can think this is a good idea is if error
  detection is expensive. This would be an argument for presenting
  shadow computing first then introduce this as a modiciation if error
  detection is hard or impossible. I didn't do it this way but think
  it might be a better way after discussing with Taieb.}

\subsubsection{Optimized Shadow Replication}

\input{shadow_computing}



\section{Reward Based Optimal shadow replication}
\label{sec:reward_model}
\noindent 

In this section, we describe a profit-based optimization framework for
the cloud-computing execution model previous described. Using this
framework we compute profit-optimized execution speeds by
optimizing the following objective function:

%It is assumed that failures can be detected.

%While this is the case in many computing
%environments, there are cases where failure detection may not be
%possible. To address this limitation, we propose a sub-optimal shadow
%replication scheme, whereby both the main process and the shadow
%execute independently at stretched speeds to meet the expected
%response time, without the need for the main processes failure
%detection.

%In this section, we describe a reward-based analytical
%model to guide the design of the shadow replication scheme that
%executes at optimal speed to maximize profits. 


%In this section we are going to develop a framework to assess and
%explore the trade-off between energy consumption and performance of
%fault tolerant computation models, namely shadow replication, pure
%replication, and re-execution. To this end, we adopt a reward-based
%model to study the economic potential of shadow replication, analyze its
%interplay between resiliency and power management, and compare it to
%other fault tolerance models.

%Without loss of generality, we focus on the execution of one job as
%shown in Fig. 2, assuming that there are N tasks in total that will
%execute in parallel. The completion of the job depends on the
%successful execution of all these tasks, and a failure in one task
%would delay the completion of the entire job.

\begin{equation}
\label{optimization_problem}
%\setlength{\abovedisplayskip}{14pt}
\begin{alignedat}{2}
\max_{\sigma_m,\sigma_b,\sigma_a}     & E[profit] \\
s.t.                                 & 0 \leq \sigma_m \leq \sigma_{max} \\
                                     & 0 \leq \sigma_b \leq \sigma_{m} \\
                                     & 0 \leq \sigma_a \leq \sigma_{max} 
\end{alignedat}
\end{equation}
We assume that processor
speeds are continuous and use nonlinear optimization techniques
to solve the above optimization problem. 

In order to earn profit, service providers must either increase
income or decrease expenditure. We take both factors into
consideration for the purpose of maximizing profit while meeting
customer's requirements. In our model, we set the expected profit to be
expected income minus expected expense.

\begin{equation}
E[\text{profit}]=E[\text{income}]-E[\text{expense}]
\end{equation}

\subsection{Reward Model}
\label{sla_reward_model}
The cloud computing SLA can be diverse and
complex. To focus on the profit and reliability
aspects of the SLA, we define the reward model based on job completion
time. Platform as a Service (PaaS) companies will continue to become
more popular causing an increase in SLAs using job completion time as
their performance metric. We are already seeing this appear in
web-based remote procedure calls and data analytic requests.

As depicted in Figure \ref{fig:reward}, customers expect that their
job deployed on cloud finishes by a mean response time $t_{R_1}$.  As a
return, the provider earns a certain amount of reward, denoted by R,
for satisfying customer's requirements. However, if the job cannot be
completed by the expected response time, the provider loses a fraction of $R$
proportional to the delay incurred. For large delay, the profit loss may translate into a penalty that the CSP must pay to the customer. In this model, the maximum penalty $P$ is paid if the
delay reaches or exceeds $t_{R_2}$. The four
parameters, $R$, $P$, $t_{R_1}$ and
$t_{R_2}$, completely define the reward model.

%The three parameters, R, and should be determined while negotiating
%the SLA according to the task workload. In our model, R is
%proportional to the expected energy cost that SP has to pay for
%executing customer’s tasks. Especially, R can grow with workload in a
%linearly proportional manner, logarithmic manner, or exponential
%manner.
%
%, which
%should be larger than the minimal response time . 

There are two facts that the service provider must take into account
when negotiating the terms of the SLA. The first is the response time
of the main process assuming no failure (Figure
\ref{fig:sc_no_fail} and Figure \ref{fig:sc_shadow_fail}). This
results in the following completion time:
\begin{equation}
t_c^m=W/\sigma_m
\label{eq:tcm}
\end{equation}

If the main process fails (shown in Figure \ref{fig:sc_main_fail}), the
task completion time by shadow process is the time of the failure,
$t_f$, plus the time necessary to complete the remaining work.

\begin{equation}
t_c^s=t_f+\frac{W-t_f \times \sigma_b}{\sigma_a}
\label{eq:tcs}
\end{equation}

This reward model is flexible and extensible; it is not restricted to
the form shown in Figure \ref{fig:reward}. In particular, the decrease
may be linear, concave, or convex and the penalty can extend to
infinity. This model can further be extended to take into
consideration both the short-term income and long-term reputation of
the service provider~\cite{Daw:2002:LRP:639717.639720}.


\begin{figure}[t!]	
	\begin{center}
		\includegraphics[width=\columnwidth]{diagrams/reward.pdf}
	\end{center}
	\caption{A reward function}
	\label{fig:reward}
\end{figure}


\subsection{Failure Model}
Failure can occur at any point during the execution of the main or
shadow process. Our assumption is that at most one failure occurs,
therefore if the main process fails it is implied that the shadow will
complete the task without failure. We can make this assumption because
we know the failure of any one node is rare thus the failure
of any two specific nodes is very unlikely.

% I don't think we need to say this and it opens us up to having to
% explain multiple shadows which we do not do today.
%In order
%to achieve higher resiliency one would make use of multiple shadow
%processes and this failure model will still be valid.

We assume that two probability density functions, $f_m(t_f)$ and
$f_s(t_f)$, exist which express the probabilities of the main and shadow
process failing at time $t_f$ separately. The model does not assume a
specific distribution. However, in the remainder of this paper we use
an exponential probability density function, $f_m(t_f)=f_s(t_f)=\lambda
e^{-\lambda t_f}$, of which the mean time between failure (MTBF) is $\frac{1}{\lambda}$.

\subsection{Power and Energy Models}
Dynamic voltage and frequency scaling
(DVFS) has
been widely exploited as a technique to reduce CPU dynamic power~\cite{flautner_2002_APS,pillai_2001_sosp}. It
is well known that one can reduce the dynamic CPU power consumption at
least quadratically by reducing the execution speed linearly. The
dynamic CPU power consumption of a computing node executing at speed
$\sigma$ is given by the function $p_d(\sigma)=\sigma^n$ where $n \ge
2$.
%% removed burd_1995_systems citation

In addition to the dynamic power, CPU leakage and other components
(memory, disk, network etc.) all contribute to static power
consumption, which is independent of the CPU speed. In this paper we
define static power as a fixed fraction of the node power consumed
when executing at maximum speed, referred to as $\rho$. Hence node
power consumption is expressed as
$p(\sigma)=\rho \times \sigma_{max}^n + (1-\rho)\times \sigma^n$. When the execution speed is zero
the machine is in a sleep state, powered off or not assigned as a
resource; therefore it will not be consuming any power, static or
dynamic.  Throughout this paper we assume that dynamic power is cubic
in relation to
speed~\cite{rusu_2003_ecs,zhai_2004_dac}, therefore the
overall system power when executing at speed $\sigma$ is defined as:
\begin{equation}
p(\sigma) = \begin{cases} \rho \sigma_{max}^3 + (1-\rho) \sigma^3 & \mbox{if } \sigma > 0 \\ 
                          0 & \mbox{if } \sigma = 0 \end{cases}
\label{eq:power_model}
\end{equation}
%% I removed chen_2012_srds from speed citation to save space

Using the power model given by \refeq{eq:power_model}, the
energy consumed by a process executing at speed $\sigma$ during an
interval $T$ is given by
\begin{equation}
E(\sigma,T) = p(\sigma) \times T
\end{equation}

%We derive the expected energy consumption of shadow replication for a
%single task assuming the failure model described
%previously. 

Corresponding to \reffig{fig:sc_overview}, there are three
failure cases to consider: main and shadow both succeed, shadow fails
and main fails. As described earlier, the case of both the main and
shadow failing is very rare and will be ignored. The expected
energy consumption for a single task is then the weighted average of
the expected energy consumption in the three cases.

%Each task is dependent upon the completion of all other tasks, this
%means that if a task completes prior to another task it must idly wait
%for that task to complete. In our system model a task will have to
%wait if at least one main process fails to complete, thus forcing all
%tasks to wait until the latest shadow process completes. 

First consider the case where no failure occurs and the main process
successfully completes the task at time $t_c^m$, corresponding to
\reffig{fig:sc_no_fail}.
\begin{equation}
\begin{split}
E_1 = &  ( 1-\int_0^{t_c^m}f_m(t)dt) \times (1 - \int_0^{t_c^m} f_s(t)dt) \times \\
      &  (  E(\sigma_m,t_c^m) + E(\sigma_b,t_c^m))
\label{eq:energy_no_failure}
\end{split}
\end{equation}
The first line is the probability of fault-free execution of the main
process and shadow process. Then we multiple this probablity by the
energy consumed by the main and the shadow process during this fault
free execution, ending at $t_c^m$.

Next, consider the case where the shadow process fails at some point
before the main process successfully completes the task, corresponding to
\reffig{fig:sc_shadow_fail}.
\begin{equation}
\begin{split}
E_2 = & (1-\int_0^{t_c^m}f_m(t)dt) \times \\
      & \int_0^{t_c^m}(E(\sigma_m,t_c^m)+E(\sigma_b,t)) \times f_s(t)dt
\label{eq:energy_shadow_fail}
\end{split}
\end{equation}
The first factor is the probability that the main process does not
fail, and the probability of shadow fails is included in the second factor which also contains the energy consumption since it depends on the shadow failure time. Energy consumption comes from the main process until the completion of the task,
and the shadow process before its failure.

The one remaining case to consider is when the main process fails and
the shadow process must continue to process until the task completes,
corresponding to Figure \ref{fig:sc_main_fail}.
\begin{equation}
\begin{split}
E_3 = & (1-\int_0^{t_c^m}f_s(t)dt) \times \int_0^{t_c^m}(E(\sigma_m,t)+\\
      & E(\sigma_b,t)+E(\sigma_a,t_c^s-t))f_m(t)dt
\label{eq:energy_main_fail}
\end{split}
\end{equation}
Similarly, the first factor expresses the probability that the shadow process does
not fail. In this case, the shadow process executes from the beginning to
$t_c^s$ when it completes the task. However, under our ``at most one
failure'' assumption, the period during which shadow process may fail
ends at $t_c^m$, since the only reason why shadow process is still in
execution after $t_c^m$ is that main process has already failed. There
are three parts of energy consumption, including that of main process
before main's failure, that of shadow process before main's failure,
and that of shadow process after main's failure, all of which depend
on the failure occurrence time. 

The three equations above describe the expected energy consumption by a
pair of main and shadow processes for completing a task under
different situations. However, under our system model it might be the
case that those processes that finish early will wait idly and
consume static power if failure delays one task. If it is the case
that processes must wait for all tasks to complete, then this energy
needs to be accounted for in our model. The probability of this is the probability that at least one main process fails,
referred to as the system level failure probability.
\begin{equation}
P_f=1-(1-\int_0^{t_c^m}f_m(t)dt)^N
\label{eq:prob_of_one_main_failure}
\end{equation}
Hence, we have the fourth equation corresponding to the energy consumed while waiting in idle. 
\begin{equation}
  \begin{split}
  E_4 = & ( 1-\int_0^{t_c^m}f_m(t)dt) \times (1 - \int_0^{t_c^m} f_s(t)dt) \times \\
        & 2 P_f \times E(0,t_c^j-t_c^m) + \int_0^{t_c^m}f_s(t)dt \times \\
        & (1-\int_0^{t_c^m}f_m(t)dt) \times P_f \times E(0,t_c^j-t_c^m) 
  \end{split}
\end{equation}
Corresponding to the first case, neither main process nor shadow
process fails, but both of them have to wait in idle from task
completion time $t_c^m$ to the last task's completion (by a shadow
process) with probability $P_f$. Under the second case, only the main
process has to wait if some other task is delayed since its shadow
process has already failed. These two aspects are accounted in the
first and last two lines in $E_4$ separately.  We use the expected
shadow completion time $t_c^j$ as an approximation of the latest task
completion time which is also the job completion time.

%% Rami said remove, I agree - bnm
%and the reason why we
%ignore the third case is that the third case itself accounts for the
%delayed task and it is already factored in $E_3$. 

%\begin{equation}
%t_c^j=\frac{\int_0^{t_c^m}t_c^s \times f_m(t)dt}{\int_0^{t_c^m}f_m(t)dt}
%\label{eq:estimated_shadow_completion}
%\end{equation}

By summing these four parts and then multiplying it by $N$ we will have
the expected energy consumed by Shadow Replication for completing a
job of $N$ tasks.
\begin{equation}
E[\text{energy}]=N \times (E_1 + E_2 + E_3 + E_4)
\label{eq:total_energy}
\end{equation}

\subsection{Income and Expense Models}
The income is the reward paid by customer for the cloud computing
services that they utilize. It depends on the reward function $r(t)$,
depicted in \reffig{fig:reward}, and the actual job completion
time. Therefore, the income should be either $r(t_c^m)$, if all main
processes can complete without failure, or $r^*(t_c^s)$ otherwise. It
is worth noting that the reward in case of failure should be
calculated based on the last completed task, which we approximate by
calculating the expected time of completion allowing us to derive the
expected reward, i.e. $r^*(t_c^s)=\frac{\int_0^{t_c^m}r(t_c^s) \times
f_m(t)dt}{\int_0^{t_c^m}f_m(t)dt}$. Therefore the income is estimated
by the following equation.
\begin{equation}
E[\text{income}]= (1-P_f) \times r(t_c^m) + P_f \times r^*(t_c^s)
\end{equation}

%% there is some hand-waving maddness above, unclear how to fix? -bnm

The first part is the reward earned by the main process times the
probability that all main processes would complete tasks without
failure. If at least one main process fails, that task would have to
be completed by a shadow process. As a result, the second part is the
reward earned by shadow process times the system level failure probability.

If $C$ is the charge expressed as dollars per unit of energy consumption
(e.g. kilowatt hour), then the expected expenditure would be $C$ times
the expected energy consumption for all $N$ tasks:
\begin{equation}
E[\text{expense}] = C \times E[\text{energy}]
\label{eq:expense}
\end{equation}

However, the expenditure of running the cloud computing service is more
than just energy, and must includes hardware, maintenance, and human
labor. These costs can be accounted for by amortizing these costs into the
static power factor, $\rho$. Because previous studies have
suggested~\cite{Elnozahy03energyconservation,Raghavendra:2008:NPS}
that energy will become a dominate factor we decided to focus on this
challenge and leave other aspects to future work.

\begin{table}[!h]
\caption{Symbols used in our analytical model.}
\centering
\begin{tabularx}{\columnwidth}{|l|X|}
\hline
Symbols                          & Definition                         \\
\hline
$W$                               & Task size                       \\
\hline
$N$                               & Number of tasks                 \\
\hline
$r(t)$                          & Reward function       \\
\hline
$R$, $P$                            & Maximum reward and penalty      \\
\hline
$t_{R_1}$, $t_{R_2}$             & Response time thresholds  \\
\hline
$C$                               & Unit price of energy            \\
\hline
$\rho$                          & Static power ratio                 \\
\hline
$t_c^m$, $t_c^s$, $t_c^{j}$                 & Completion time of main process, shadow process, and the whole job \\
\hline
$f_m()$, $f_s()$                    & Failure density function of main and shadow  \\
\hline
$\lambda$                           & Failure rate    \\
\hline
$P_f$                               & System level failure probability \\
\hline
$\sigma_m$, $\sigma_b$, $ \sigma_a$  & Speeds of main, shadow before and after failure (Optimization Outputs) \\
\hline
\end{tabularx}

\label{tbl:symbols}
\end{table}

Based on the above formalization of the optimization problem, the
MATLAB Optimization Toolbox~\cite{matlab_opt} was used to solve the
resulting nonlinear optimization problem. The parameters of this
problem are listed in Table~\ref{tbl:symbols}. 





\section{Profit-aware stretched replication}
\label{sec:reward_model_2}
\noindent 
We compare Shadow Replication to two other replication techniques,
traditional replication and profit-aware stretched replication.
Traditional replication requires that the two processes always execute
at the same speed $\sigma_{max}$. Unlike traditional replication
Shadow Replication is dependent upon failure detection, enabling the
replica to increase its execution speed upon failure and maintain the
targeted response time thus maximizing profit. While this is the case
in many computing environments, there are cases where failure
detection may not be possible. To address this limitation, we propose
profit-aware stretched replication, whereby both the main process and
the shadow execute independently at stretched speeds to meet the
expected response time, without the need for the main processes failure
detection. In profit-aware stretched replication both the main and
shadow execute at speed $\sigma_r$, found by optimizing the profit
model.  For both traditional replication and stretched replication,
the task completion time is independent of failure and can be directly
calculated as:
\begin{equation}
t_c=\frac{W}{\sigma_{max}} \text{ or } t_c=\frac{W}{\sigma_r}
\end{equation}

%Profit-aware stretched replication is similar to traditional
%replication in that it executes at one speed but different in that it
%determines the execution speed that maximizes the profit much like
%shadow replication.

%In order to evaluate the benefits of shadow replication relative to
%other resilience schemes, we consider dual level of redundancy and
%assume that at most one failure may occur, following the same failure
%model as the one defined for shadow replication. To this end, we
%propose a new scheme, referred to as profit-aware stretched
%replication. We also derive the expected energy of pure replication
%scheme, which is then used for comparative analysis.

Since all tasks will have the same completion time, the job completion
time would also be $t_c$. Further, the expected income, which depends
on negotiated reward function and job completion time, is independent
of failure:
\begin{equation}
E[income]=r(t_c)
\end{equation}

Since both traditional replication and profit-aware stretched
replication are special cases of our Shadow Replication paradigm where
$\sigma_m=\sigma_b=\sigma_a=\sigma_{max}$ or
$\sigma_m=\sigma_b=\sigma_a=\sigma_r$ respectively, we can easily derive the
expected energy consumption using \refeq{eq:total_energy} with $E_4$
fixed at 0 and then compute the expected expense using \refeq{eq:expense}.


\section{Re-execution}
\label{sec:reward_model_3}
\noindent 
Contrary to replication, re-execution initially assigns a single
process for the execution of a task. If the original task fails, the
process is re-executed. In the cloud computing execution framework
this is equivalent to a checkpoint/restart, the checkpoint is
implicitly taken at the end of each phase and because the tasks are
loosely coupled they can restart independently. 

Based on the one failure assumption, two cases must be considered to
calculate the task completion time. If no failure occurs, the task
completion time is:
\begin{equation}
t_c=\frac{W}{\sigma_{max}}
\end{equation}
In case of failure, however, the completion time is equal to the sum
of the time elapsed until failure and the time needed for
re-execution. Again, we use the expected value
$t_f^*=\frac{\int_0^{t_c}t \times f_m(t)dt}{\int_0^{t_c}f_m(t)dt}$ to
approximate the time that successfully completed processes have to
spend waiting for the last one.

Similar to Shadow Replication, the income for re-execution is the
weighted average of the two cases:
\begin{equation}
E[\text{income}]=(1-P_f) \times r(t_c) + P_f \times r(t_c+t_f^{*})
\end{equation}

For one task, if no failure occurs then the expected energy consumption can be
calculated as
\begin{equation}
E_5=(1 - \int_0^{t_c} f_m(t)dt) \times (E(\sigma_{max},t_c)+ P_f \times E(0,t_f^{*}))
\label{eq:energy_first_task}
\end{equation}

If failure occurs, however, the expected energy consumption can be calculated
as
\begin{equation}
E_6=\int_0^{t_c}(E(\sigma_{max},t) + E(\sigma_{max},t_c)) \times f_m(t) dt
\label{eq:energy_rexecution_task}
\end{equation}
Therefore, the expected energy consumption by re-execution for
completing a job of $N$ tasks is
\begin{equation}
E[energy]=N \times (E_5 + E_6)
\end{equation}


\section{Evaluation}
\label{sec:evaluation}
We deployed rsMPI on a medium sized cluster and utilized up to 21 nodes for testing and benchmarking. Each node consists of a 2-way SMPs with Intel Haswell E5-2660 v3 processors of 10 cores per socket (20 cores per node), and is configured with 128 GB RAM. Nodes are connected via 56 GB/s FDR InfiniBand. To maximize the compute capacity, we used up to 20 cores per node.

We used benchmark applications from both the Sandia National Lab Mantevo Project and NAS Parallel Benchmarks (NPB), and evaluated rsMPI with various problem sizes and number of processes. CoMD is a proxy for the computations in a typical molecular dynamics application. MiniAero is an explicit unstructured finite volume code that solves the compressible Navier-Stokes equations. Both MiniFE and HPCCG are proxy applications for unstructured implicit finite element codes, but HPCCG uses MPI\_ANY\_SOURCE receive operations and can be used to demonstrate rsMPI's capability of handling MPI non-deterministic events. IS, EP, and CG from NPB represent integer sort, embarrassingly parallel, and conjugate gradient applications, respectively. These applications cover key simulation workloads for US DOE, and represent both different communication patterns and computation-to-communication ratios.

\subsection{Measurement of runtime overhead}
\label{sec:runtime_overhead}
While the hardware overhead for rsMPI is straightforward (e.g., collocation ratio of 4 results in the need for 25\% more hardware cost), the runtime overhead of the enforced consistency protocol depend on applications. To measure this overhead we ran each benchmark application linked to srMPI multiple times and compared the average execution time with the baseline, where each application runs with original OpenMPI.

Figure~\ref{fig:runtime_overhead} shows the comparison of the execution time between baseline and srMPI for the 7 applications. All the experiments are conducted with 256 application-visible processes. That is, the baseline always uses 256 MPI ranks compiled with the unmodified OpenMPI library, while rsMPI uses 256 mains together with 256 shadows which are invisible to the application. Each result shows the average execution time of 5 runs, the standard deviation, and srMPI's runtime overhead. The baseline execution time varies from seconds to half an hour, so we plotted the time in log-scale. 

From the figure we can see that srMPI has comparable execution time to the baseline for all applications except IS. The reason for the large overhead of IS is that IS uses all-to-all communication and is largely communication-intensive. This is verified by adding fake computation to the application and we can see an immediate drop of the overhead to negligible level. We argue that communication-intensive applications like IS are not scalable, and as a result, they are not suitable for large-scale HPC. 
For all other applications, the overhead varies from 0.64\% (EP) to 2.47\% (CoMD). Even for HPCCG, which uses MPI\_ANY\_SOURCE and adds extra work to our consistency protocol, the overhead is only 1.95\%, thanks to the asynchronous semantics of MPI\_Send. Therefore, we conclude that srMPI's runtime overheads are modest for scalable HPC applications that exhibit a fair communication-to-computation ratio.

\begin{figure}[!t]
  \begin{center}
      \includegraphics[width=\columnwidth]{figures/runtime_overhead}
  \end{center}
  \caption{Comparison of execution time between baseline and rsMPI. 256 application-visible processes, and collocation ratio of 4 for srMPI.}
  \label{fig:runtime_overhead}
\end{figure}

\subsection{Scalability}
In addition to measuring the runtime overhead at fixed application-visible process count, we also assessed both strong and weak scalability by varying the number of processes for the applications. Strong scaling is defined as how the execution time varies with the number of processes for a fixed total problem size. In contrast, weak scaling is defined as how the execution time varies with the number of processes for a fixed problem size per process. 

Among the seven applications, HPCCG, CoMD, and miniAero allow us to vary the input so that we can perform both strong scaling and weak scaling test. The results for miniAero are similar to those of CoMD, so we only show the results for HPCCG and CoMD here. Figure~\ref{fig:scalability} reveals that both HPCCG and CoMD have good strong scalability. By increasing the number of processes, we can always reduce the execution time for a fixed problem size. At the same time, srMPI's runtime overhead increases with the number of processes during the strong scalability test. At 256 processes, the overhead reaches 13.2\% for CoMD, and 29.1\% for HPCCG. This may seem to contradict with the results in Section~\ref{sec:runtime_overhead}. It is expected, however, since increasing the number of processes while keeping a constant problem size increases the communication-to-computation ratio of the application. Hence, to keep rsMPI overheads reasonable, it is important to choose input sizes such that the ratio of communication-to-computation is balanced. 


\begin{figure*}[!t]
	\begin{center}
		\subfigure[HPCCG strong scalability]
		{
			\label{fig:hpccg_strong}
			\includegraphics[width=0.4\textwidth]{figures/hpccg_strong}
		}
		\subfigure[HPCCG weak scalability]
		{
			\label{fig:hpccg_weak}
			\includegraphics[width=0.4\textwidth]{figures/hpccg_weak}
		}
		\subfigure[CoMD strong scalability]
		{
			\label{fig:comd_strong}
			\includegraphics[width=0.4\textwidth]{figures/comd_strong}
		}
		\subfigure[CoMD weak scalability]
		{
			\label{fig:comd_weak}
			\includegraphics[width=0.4\textwidth]{figures/comd_weak}
		}
	\end{center}
	\caption{Scalability test for number of processes from 1 to 256. Collocation ratio is 4 for srMPI.}
	\label{fig:scalability}
\end{figure*}

Comparing the baseline execution time between Figure~\ref{fig:hpccg_weak} and Figure~\ref{fig:comd_weak}, it is obvious that HPCCG and CoMD have different weak scaling characteristics. Keeping the same problem size per process, the execution time for CoMD increases by 8.9\% from 1 process to 256 processes, while the execution time is almost doubled for HPCCG. However, further analysis show that from 16 to 256 processes, the execution time increases by only 2.5\% for CoMD, and 1.0\% for HPCCG. We suspect that the results are not only determined by the scalability of the application, but also impacted by other factors, such as cache and memory contention on the same node, and network interference from other jobs running on the cluster. Remember that each node in the cluster has 20 cores and we always use all the cores of a node before adding another node. Therefore, it is very likely that the node level contention leads to the substantial increase in execution time for HPCCG. By analyzing the results from 16 to 256 processes, we believe both of HPCCG and CoMD are weak scaling applications. 

Different from strong scalability test, there is no correlation between srMPI's runtime overhead and the number of processes during the weak scalability test. The overhead is always below 2.1\%, except for the case of 32 processes for CoMD where the overhead is 5.0\%. %The reason for this exception is still under investigation.

\subsection{Performance under failures}
%The last set of experiments test srMPI's capability of tolerating failures and evaluate its performance under various failures by comparing with checkpointing/restart. 

As one main goal of this work is to achieve fault tolerance, an integrated fault injector is required to evaluate the effectiveness and efficiency of rsMPI to tolerate failures during execution. To produce failures in a manner similar to naturally occurring process failures, our failure injector is designed to be distributed and co-exist with all rsMPI processes. Failure is injected by sending a specific signal to the target process.

Failure detection is beyond the scope of srMPI, and we assume the underlying hardware platform has a RAS system that provides this functionality. In our test system, we emulate a RAS system with a signal handler installed at every main and shadow. The signal handler catches failure signal sent from the failure injector, and uses a rsMPI defined failure message via a dedicated communicator to notify all other processes of the failure. 
%To detect failure, srMPI receiving operation checks for failure messages before performing the actual receiving. 
Similar to ULFM, process in srMPI can only detect failure when it does an MPI receive operation. In a srMPI receive, 
a process checks for failure messages before it does the actual MPI receive operation.

We also implemented checkpointing to compare with srMPI in the presence of failures. To be optimistic, we chose double in-memory checkpointing that is much more scalable then disk-based checkpointing~\cite{zheng2004ftc}. Same as leaping in srMPI, our implementation provides an API for process state registration. This API requires the same parameters as leap\_register\_state(void *addr, int count, MPI\_Datatype dt), but internally, it allocates extra memory in order to store the state of a ``buddy" process. Another provided API is checkpoint(), which can be used to insert a checkpoint in the application code. For fairness, our implementation also uses MPI messages to transfer state.  
For both srMPI and checkpointing/restart, we assume a 60 seconds rebooting time after a failure. All experiments run with 256 application-visible processes, and the results are average of 5 runs. 

Firstly, we tested the effectiveness of leaping. For each application, we identified the process state and register them with rsMPI. Figure~\ref{fig:single_failure} shows the execution time of HPCCG with a single failure injected at various locations. The blue solid line represents srMPI without any forced leaping, and the red dash line represents srMPI with periodic forced leaping. Note that the execution time is reduced compared to previous results because we reduced the number of iterations for the application main loop from 5000 to 150, so that there is no need for any forced leaping by buffer overflow. Every time we set our failure injector to randomly pick a process to inject a failure, and the failure is scheduled to occur at certain point during the execution. Corresponding to the x-axis, the scheduled failure time varies from 10\% to 90\% of the application's execution. For example, 10\% means the application completes 15 iterations for a total of 150 iterations. 

\begin{figure}[!t]
  \begin{center}
      \includegraphics[width=\columnwidth]{figures/single_failure}
  \end{center}
  \caption{Execution time of HPCCG with a single injected failure. Collocation ratio is 2 for srMPI.}
  \label{fig:single_failure}
\end{figure}

As expected, without forced leaping the execution time increases with the failure occurrence time, as reflected by the blue line in Figure~\ref{fig:single_failure}. The reason is that failure recovery time for srMPI is proportional to the amount of divergence between mains and shadows, and the divergence grows as the execution proceeds. On the other hand, forced leaping can effectively reduce the divergence by leaping the shadow forward to the state of its associated main, similar to the idea that checkpointing can reduce the amount of wasted work due to a failure by saving the execution state. To prove the effectiveness of leaping, we insert 4 forced leaping at 20\%, 40\%, 60\% and 80\% of the execution. The red line in Figure~\ref{fig:single_failure} clears show that the divergence effect is bounded due to periodic leaping, regardless of the failure occurrence time.

Next, we compare rsMPI with checkpointing for multiple failures. To run the same number of application-visible processes, rsMPI needs more nodes than checkpointing to host the shadow processes. For fairness, we take into account the extra hardware cost when comparing srMPI to checkpointing, by defining the following metric:
$$\text{Efficiency} = \frac{T_f \times N}{T_e \times M}$$
, where $T_f$ and $N$ are the execution time and number of nodes without failures, and $T_e$ and $M$ are the actual execution time and required number of nodes for a specific fault tolerance mechanism. Intuitively, $T_f \times N$ represents the total amount of workload required by the application, and $T_e \times M$ is the actual amount of work carried out. The efficiency will be in the range 0 to 1, inclusive, and the higher is the better.

The forced leaping interval for an application is selected such that no buffer overflow at the shadows would take place. Therefore, the interval should vary from system to system and also depends on the application patterns. We assume checkpointing/restart has the same buffer pressure as it needs to perform message logging, so its checkpointing interval is selected based on the same metric as rsMPI. We evaluated rsMPI with 2 different collocation ratios, i.e., 2 and 4. When collocation ratio is 2, rsMPI uses 50\% more nodes than checkpointing, and the execution rate of each shadow is roughly 50\% of the processor rate. Therefore, we set the checkpointing interval to be the same as the forced leaping interval for srMPI. When collocation ratio is 4, rsMPI needs 25\% more nodes, and each shadow's rate is roughly 25\% of the processor rate. As a result, we loose the checkpointing interval to be twice of the forced leaping interval. 

With the checkpointing and forced leaping inserted to the application code, we randomly injected up to 10 failures into the execution. Figure~\ref{fig:multiple_failure} shows the comparison between checkpointing and srMPI (collocation ratio of 4) for both execution time and efficiency defined above. Although the failure-free execution time of srMPI is slightly larger than that of checkpointing, which results from srMPI's consistency protocol, the failure recovery time of checkpointing immediately overwhelms that of srMPI as failures occur. With 10 failures, the execution time of checkpointing is 42.8\% more than that of srMPI. Considering hardware overhead, the efficiency of checkpointing is also worse than that of srMPI when the number of failures reaches 6.

\begin{figure}[!t]
  \begin{center}
      \includegraphics[width=\columnwidth]{figures/multiple_failure}
  \end{center}
  \caption{Execution time of HPCCG with multiple injected failures. Collocation ratio is 4 for srMPI.}
  \label{fig:multiple_failure}
\end{figure}

Between srMPI with collocation ratio of 2 and srMPI with collocation ratio of 4, srMPI with collocation ratio of 2 wins in execution time, while srMPI with collocation ratio of 4 wins in efficiency. 
%execution time is xx faster than checkpointing. It is projected to beat checkpointing in efficiency when 40 failures.


\section{Related Work}
\label{sec:related_work}
%Extreme-scale computing presents some unique challenges to fault tolerance as faults are no longer 
%an exceptional event \cite{ferreira_sc_2011}. 
Rollback and recovery is the dominant mechanism to achieve fault
tolerance in current HPC environments~\cite{Elnozahy:02:Survey}. In the most general form, rollback and recovery 
involves the periodic saving of the current system state, with the anticipation that
in the case of a failure, computation can be restarted from the most recently saved state. % \cite{Elnozahy:02:Survey}. %The identification of an error, before or during a checkpoint,
%requires that the application rollback to the previously completed checkpoint. 
Coordinated checkpointing is a popular approach for
its ease of implementation.
%Specifically, all processes
%coordinate with one another to produce individual states that satisfy the ``happens before"
%communication relationship \cite{chandy_trans_1972}, which is proved to provide a consistent global state.
%Essentially, the algorithm provides a method for all processes involved to stop operation ``at the same
%time" and transfer their system state to a stable storage. 
%The major benefit of coordinated
%checkpointing stems from its simplicity and ease of implementation. 
Its major drawback, however, is the
lack of scalability, as it requires global coordination
~\cite{elnozahy_dsc_2004}.%riesen_sandia_2010}.
%hargrove2006berkeley}.


In uncoordinated checkpointing, processes checkpoint their states independently and postpone creating a 
globally consistent view until the recovery phase. The major advantage is the reduced overhead during fault free operation. However, the scheme requires that
each process maintains multiple checkpoints and message logs, necessary to construct a consistent 
state during recovery. It can also suffer the well-known domino effect 
 \cite{randell_domino_effect}. One hybrid approach, known as communication induced 
checkpointing, aims at reducing coordination overhead \cite{alvisi_ftc_1999}. The approach, however, may 
cause processes to store useless states. To address this 
shortcoming, ``forced checkpoints" have been proposed \cite{helary_rds_1997}. This approach, however,  may lead to unpredictable
checkpointing rates. Although well-explored, uncoordinated checkpointing has not been widely adopted
in HPC environments, due to its dependency on applications \cite{guermouche_2011_ipdps}.


One of the largest overheads in any checkpointing process is the time necessary to write the checkpointing 
to stable storage. Incremental checkpointing attempts
to address this by only writing the changes since previous checkpoint \cite{Agarwal:04:Adaptive}. %,elnozahy_1992_manetho,li_trans_1994}. %This
%can be achieved using dirty-bit page flags \cite{plank_ftcs_1994,elnozahy_1992_manetho}. Hash based incremental checkpointing, on the other
%hand, makes use of hashes to detect changes \cite{nam_ftc_1997,Agarwal:04:Adaptive}. 
Another proposed scheme, known as in-memory checkpointing, minimizes the overhead of disk access~\cite{zheng_2004_ftccharm,6264677}.
%offloads the checkpointing process to a secondary task and only writes incremental checkpoints \cite{li_trans_1994}.
The main concern of these techniques is the increase in
memory requirement to support the simultaneous execution of the checkpointing and the application. It has been suggested that nodes in extreme-scale systems should be configured with fast local storage~\cite{doe_ascr_exascale_2011}. 
%, which
%improves the performance of checkpointing \cite{doe_ascr_exascale_2011}. 
Multi-level checkpointing, which consists of
writing checkpoints to multiple storage targets, can benefit from such a strategy \cite{Moody:10:SCR}. This,
however, may lead to increased failure rates of individual nodes and complicate the checkpoint writing process.
%Furthermore, it may complicate the checkpoint writing process and requires that the system track the
%current location of all process's checkpoints.


Process replication, or state machine replication, has long been used for reliability and availability in distributed and mission critical systems \cite{schneider_1990_tutorial}. %Replication can be used to detect and correct system failures that are otherwise undetectable,
%such as silent data corruption and Byzantine faults \cite{fiala_2012_sdc}. 
This approach is barely used in HPC systems, primarily due to its high cost and low efficiency.
However, upcoming extreme-scale systems are expected to 
%require a more challenging level of fault tolerance to deal with the 
confront a dramatic growth in both the frequency and diversity of faults.
As a result,
replication has recently been proposed as a
viable alternative to checkpointing in HPC \cite{engelmann09case,Cappello:09:Fault}. 
In addition, full and partial
replication have also been studied to augment existing checkpointing techniques, and to  
detect or correct silent data corruption \cite{stearly_2012_partial,elliott_2012_cpr,ferreira_sc_2011,fiala_2012_sdc}. % There are several different implementations of
%replication in the widely used MPI library, each with their different tradeoffs and overheads. The
%overhead can be negligible or up to 70\% depending upon the communication patterns of the
%application \cite{engelmann2011redundant}. %Moreover, replication alone is not enough to guarantee fault tolerance since
%it is possible that all nodes executing a given process could fail simultaneously, thus
%replication is typically paired with some form of checkpointing. 
Our approach differs from classical process replication in that we dynamically configure the execution rates of main and shadow processes, so that less resource/energy is required while reliability is still assured.  


Replication with dynamic execution rate is also explored in Simultaneous and Redundantly Threaded (SRT) processor whereby one leading thread of execution is running ahead of trailing threads \cite{reinhardt2000transient}. However, 
the focus of \cite{reinhardt2000transient} is on transient faults within CPU while we aim at tolerating both permanent and transient faults across all systems components.
This work is closely related to our previous works \cite{mills_2014_icnc,cui_en7085151,cui_2014_closer} where single or loosely-coupled tasks is considered. Instead, in this paper we explore novel ideas of shadow collocation and shadow leaping in order to satisfy the requirements of future extreme-scale HPC systems. 
%our approach is different in that it tunes the execution rates of the leading and trailing threads in a finer grain, in order to achieve a ``parameterized" trade-off between completion time and energy consumption. 
%Further, we take advantage of the idle time during failure recovery and ``leap" the trailing replicas to achieve forward progress%, largely improving performance in terms of both completion time and energy consumption. 
%. This differs from \cite{reinhardt2000transient}, of which the ``leaping" of the trailing replica results in extra overhead.
%To the best of our knowledge,
%Lazy Shadowing is the first attempt to explore a state-machine replication based framework
%that achieves a fine-grained tradeoff between time and hardware redundancy while meeting resilience and
%power requirements.


\section{Conclusion}
\label{sec:conclusion}

In this paper we have introduced shadow computing, an energy efficient
method to provide fault tolerate execution without the limitations of
checkpointing. We then compared this to other known methods,
replication and re-execution, and concluded that shadow computing is
always more energy efficient. We also observed that the amount of
energy saving is highly dependent upon the rate of failure and the
amount of slack present in the system.

Fully harnessing the potential of shadow computing to deal with
failures brings about several challenging questions that need to be
addressed: How can this concept be used to improve fault detection and
layer coordination, understanding faults and silent errors and
improving situational awareness? What level of synchronization is
required between the main process and its associated shadow processes
to minimize impact on other application processes? What state, if any,
must be saved to ensure “smooth” transition to the primary shadow
process upon failure of the main process?  Future work will be focused
on investigating these questions for different types of failure to
better understand the advantages and limitations of this approach to
achieve high levels of fault-tolerance in extreme scale cloud
computing environments.


\acknowledgements{Acknowledgements}

This material is based in part upon work supported by the 
National Science Foundation under Grants Number CNS-1252306 and CNS-1253218. 
Any opinions, findings, and conclusions or recommendations expressed in this material are those of the authors and do 
not necessarily reflect the views of the National Science Foundation.

%%%%%%%%%%%%%%%%%%%%%%%%%%%%%%%%%%%%%%%%%%

%%\authorcontributions{Author Contributions}

%%Main text.

%%%%%%%%%%%%%%%%%%%%%%%%%%%%%%%%%%%%%%%%%%

\conflictofinterests{Conflicts of Interest}

%%State any potential conflicts of interest here or 
No conflict of interest to report. 

%=================================================================
% References: Variant A
%=================================================================
% Back Matter (References and Notes)
%----------------------------------------------------------
% Style and layout of the references
%\bibliographystyle{mdpi}
%{\small
%\bibliography{bibliography}}
%
%
%\makeatletter
%\renewcommand\@biblabel[1]{#1. }
%\makeatother
%
%\begin{thebibliography}{999} % if there are less than 10 entries, enter a one digit number
%
%% Reference 1
%\bibitem{ref-journal}
%Lastname, F.; Author, T. The title of the cited article. {\em Journal Abbreviation} {\bf 2008}, {\em 10}, 142-149.
%
%% Reference 2
%\bibitem{ref-book}
%Lastname, F.F.; Author, T. The title of the cited contribution. In {\em The Book Title}; Editor, F., Meditor, A., Eds.; Publishing House: City, Country, 2007; pp. 32-58.
%
%\end{thebibliography}
%
%=================================================================
% References:  Variant B
%=================================================================
% Use the following option to include external BibTeX files:

\bibliographystyle{mdpi}
\bibliography{bibliography}


\end{document}

