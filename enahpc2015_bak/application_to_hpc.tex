One of the primary goals of high performance computing is to achieve
the maximum possible throughput of the system. Thus when we
apply shadow computing to this environment we assume that the
execution speed of the main process should be the maximum possible
execution speed, $\sigma_m=\sigma_{max}$. If no failure occurs then
the task will be completed as soon as possible, known as the minimum
response time. If the main process fails it is assumed that the task
has some laxity as to when it will complete. The amount of laxity is
bounded by the task's targeted response time, which is the time at
which the task must be completed regardless of failure. The targeted
response time is typically represented as a laxity factor, $\alpha$,
of the minimum response time. For example if the minimum response time
is 100 seconds and the targeted response time is 125 seconds, the
laxity factor is 1.25.

We propose two different schemes for for applying shadow computing to
high performance computing.
\begin{itemize}
\item 
Energy Optimal Replication - Shadow execution speeds are those that
minimize the consumed energy and guarantee completion by the targeted
response time. This requires us to find $\sigma_b$ and $\sigma_a$ that
solve Equation~\ref{optimization_problem}.
\item 
Stretched Replication - Shadow execution is set to a single speed that
guarantees completion by the targeted response time, $\sigma_b =
\sigma_a = W/R$.
%\item 
%Minimum Work Replication - Shadow execution speed before failure,
%$\sigma_b$, is set to the minimum execution speed that enables the
%shadow to still met the targeted response time. We will show that this
%method is typically energy optimal.
\end{itemize}
The remainder of this section discusses the solution to finding
$\sigma_b$ and $\sigma_a$ for energy optimal replication. 
 

The last constraint in Equation~\ref{optimization_problem} requires that
if the main process fails the shadow process must be able to complete
the given work, $W$, by the targeted response time, $R$. 
The effect of this constraint on the execution speeds depends on the failure 
detection time $t_f$. Since $\sigma_b$ would be set to slower to save energy,
the larger $t_f$ is, the less time is left for the shadow to catch up, and thus 
$\sigma_a$ needs to be larger.
To enforce this constraint for all possible values of $t_f$, we transform it to
the following inequality:
\begin{equation}
\label{work_constraint}
t_c*\sigma_b+(R-t_c)*\sigma_{max} \geq W 
\end{equation}
The intuition for this constraint is that in the worst case the shadow
will have to execute at the maximum possible speed after failure to
achieve the targeted response time. This enforces the constraint such
that if the main process fails at the very last time point, $t_c$,
then the shadow process will still be able to complete the work by the
targeted response time. This places a lower bound on the value for
$\sigma_b$. With this modification, we can use the MATLAB Optimization 
Toolbox to solve Equation~\ref{optimization_problem} and derive the optimal
$\sigma_m$, $\sigma_b$, and $\sigma_a$.

