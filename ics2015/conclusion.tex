As the scale and complexity of HPC continue to increase, both the failure rate and energy consumption are expected to increase dramatically, making it extremely challenging to deliver extreme-scale computing that is efficient and reliable. Existing fault tolerance methods rely on either time or hardware redundancy. Neither of them appeals to the next generation of supercomputing, as the first choice may incur significant delay while the second one constantly wastes over 50\% of the system efficiency. The need for an efficient and reliable solution in extreme-scale, failure-prone computing environments calls for a reconsideration of the fault tolerance problem. 

To this end, we propose a proactive, power-aware resilience model, referred to as Lazy Shadowing, as an efficient and scalable alternative to achieve high-levels of fault tolerance for future extreme-scale computing. In this paper, we present a comprehensive discussion of the techniques that will enable Lazy Shadowing. In addition, we develop a series of mathematical models to assess its performance in terms of reliability, completion time, and energy consumption, with all the critical parameters identified. 
Through comparison with existing fault tolerance approaches, we identify the scenarios where each of the alternatives should be chosen. Specially, checkpointing consumes the least time and energy when system size is small; process replication should be used to minimize completion time when system size is extremely large and failure rate is extremely high; and Lazy Shadowing is the choice for all other cases, for both completion time and energy consumption. 
In addition, we predict that Lazy Shadowing is able to achieve over 20\% of energy saving with potential completion time reduction for future extreme-scale computing. In the future, we plan to implement Lazy Shadowing and measure its performance in reality.
