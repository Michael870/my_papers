There is a fruitful body of research on resource management in 
Cloud Computing~\cite{singh2016survey,zhan2015cloud,gill2018chopper}. Dynamic resource scheduling and 
load balancing are used 
to maximize system utilization and efficiency~\cite{adhikari2018heuristic,panwar2015load}. These techniques, however, 
are not straightforward to apply to HPC workloads which are highly sensitive to resource change and interference. 
Actually, resource management has been identified as one of the open 
challenges for HPC cloud~\cite{netto2018hpc}. 
Currently, cloud service providers (CSPs) are often limited to statically and conservatively reserve 
resources based on peak resource requirements to respect service level agreements (SLAs). For example, Microsoft Azure 
allocates dedicated supercomputers from Cray, and Amazon AWS offers dedicated nodes for full-size VMs. 
% allocate physical resources
% Despite the 
% numerous benefits promised by Cloud Computing, however, cloud service providers (CSPs) are often limited to statically 
% allocate physical resources to HPC tenants in order to avoid performance interference and enforce 
% service level agreements (SLAs). 
This essentially offsets 
the elasticity and efficiency benefits of the Cloud Computing business model. 