As the demand for cloud computing continues to increase, cloud service
providers face the daunting challenge to meet the negotiated SLA
agreement, in terms of reliability and timely performance, while
achieving cost-effectiveness. This challenge is increasingly
compounded by the increasing likelihood of failure in large-scale
clouds and the rising cost of energy consumption.  This paper proposes
Shadow Replication, a novel profit-maximization resiliency model,
which seamlessly addresses failure at scale, while minimizing energy
consumption. The basic tenet of the model is to associate a suite of
shadow processes to execute concurrently with the main process, but
initially at a much reduced execution speed, to overcome failures as
they occur. Two computationally-feasible schemes are proposed to
achieve shadow replication. A performance evaluation framework is
developed to analyze these schemes and compare their performance to
traditional replication-based fault tolerance methods, focusing on the
inherent tradeoff between fault tolerance, the specified SLA and
profit maximization. The results show Shadow Replication leads to
significant energy reduction, and is better suited for
compute-intensive execution models, where up to 30\% more profit
increase can be achieved.


%several experimental studies are carried out to assess the performance
%of the different resiliency schemes, with respect to profit
%maximization in different cloud computing environments. 




%The challenge is to derive the
%execution speed, both before and after failure, of a shadow in order
%to ensure adherence to the negotiated SLA, while maximizing profit. To
%this end, we present an optimization model to derive the shadow
%execution speeds, which takes into consideration a computing node
%failure rate and the negotiated SLA. Several computationally-feasible
%methods are then proposed to solve this model. 



%As companies continue to increase their reliance upon cloud computing
%services there will be an increasing demand for reliable and timely
%service. However, as cloud-based systems increase in size and
%complexity it is expected that reliability will degrade, causing both
%delays in service and increases in energy consumption. This will cause
%fault tolerance to be a critical system feature to providing
%applications at large scale. In this work, we propose ``shadow
%replication'', a fault tolerance method that makes use of DVFS to
%provide energy-aware, profit-maximizing system resilience to
%task-based cloud computing services.  We analyze different resilience
%methods and identify the system parameters which are most relevant to
%the tradeoff between fault tolerance and profit, and present results
%which pinpoint the most profitable method.  We also show that in
%certain systems shadow replication can achieve 10-30\% more profit
%than existing fault tolerance methods, i.e. re-execution and
%traditional replication.

%
%
% ``shadow replication'' to
%provide an energy-aware, profit-optimized method that can result in
%upto two-times the profit achieved by existing fault tolerance
%methods. Additionally, we develop analytical models to demonstrate the
%benefits of our approach at the scale expected in future cloud
%computing environments.
