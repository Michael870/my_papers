\noindent 
The main motivation of this work stems from the observation that, as systems become larger and more complex, the
rate of failures is highly-likely to increase significantly.  
Hence, understanding the interplay between fault-tolerance, energy consumption 
and profit maximization is critical for the viability of Cloud Computing
to support future large-scale systems.
To this end, we propose Shadow Replication as a novel energy-aware, 
reward-based computational model to achieve fault-tolerance and
maximize the profit. 
What differentiates Shadow Replication from other
methods is its ability to explore a
parameterized tradeoff between hardware and time redundancy to achieve fault-tolerance, with minimum energy, 
while meeting SLA requirements. 

To assess the performance of the proposed fault-tolerance computational
model, an extensive performance evaluation study is carried out. 
In this study, system properties that affect the
profitability of fault tolerance methods, namely failure rate,
targeted response time and static power, are identified. The failure rate is
affected by the number of tasks and vulnerability of the task
to failure. The targeted response time represents the 
clients' desired job completion time, as expressed by the terms of the SLA.  
Our performance evaluation shows that in all cases, Shadow Replication outperforms
existing fault tolerance methods. Furthermore, shadow
replication will converge to traditional replication when target response time is stringent, and to re-execution when target response time is relaxed or failure is unlikely. Furthermore, the study reveals that the system static power plays a critical role in the tradeoff between the 
desired level of fault-tolerance, profit maximization and energy consumption.
This stems from the reliance of Shadow Replication upon DVFS to reduce 
energy costs. If the static power is high, slowing down process 
execution does not lead to a significant reduction in the total energy 
needed to complete the task.

%Our proposed fault tolerance method, Shadow Replication, is the
%preferred method when the system failure rate is high. Shadow
%replication is also preferred when targeted response times are less
%than 150\% of the minimum response time. In all cases shadow
%replication out-performs traditional replication. If fact, when
%targeted response time is close to the minimum response time, then
%Shadow Replication converges to traditional replication. Re-execution
%is the most profitable choice when failure rates are low and the
%targeted response time is large. Re-execution is also preferred when
%static power is large, because Shadow Replication relies on DVFS
%to save energy.

%Our next step is to implement Shadow Replication in task-based cloud
%computing environments and experimentally measure its effect. We believe this work will led t
%
%
%In the future, we are looking forward to further extending our
%analytical models with multiple shadow processes whose size may be
%determined by customer requirement and cloud service
%statistics. Further, we will implement
%shadow computing within existing cloud infrastructure to measure its
%strength and to obtain an in-depth analysis of the cost benefits among
%all the stake-holders.
