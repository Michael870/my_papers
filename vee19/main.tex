%% For double-blind review submission, w/o CCS and ACM Reference (max submission space)
\documentclass[sigplan,10pt,nonacm]{acmart}
\settopmatter{printfolios}
%% For double-blind review submission, w/ CCS and ACM Reference
%\documentclass[sigplan,review,anonymous]{acmart}\settopmatter{printfolios=true}
%% For single-blind review submission, w/o CCS and ACM Reference (max submission space)
%\documentclass[sigplan,review]{acmart}\settopmatter{printfolios=true,printccs=false,printacmref=false}
%% For single-blind review submission, w/ CCS and ACM Reference
%\documentclass[sigplan,review]{acmart}\settopmatter{printfolios=true}
%% For final camera-ready submission, w/ required CCS and ACM Reference
%\documentclass[sigplan]{acmart}\settopmatter{}


%% Conference information
%% Supplied to authors by publisher for camera-ready submission;
%% use defaults for review submission.
\acmConference[VEE'20]{ACM SIGPLAN/SIGOPS International Conference on Virtual Execution Environments}{March 17, 2020}{Lausanne, Switzerland}
\acmYear{2020}
\acmISBN{} % \acmISBN{978-x-xxxx-xxxx-x/YY/MM}
\acmDOI{} % \acmDOI{10.1145/nnnnnnn.nnnnnnn}
\startPage{1}

%% Copyright information
%% Supplied to authors (based on authors' rights management selection;
%% see authors.acm.org) by publisher for camera-ready submission;
%% use 'none' for review submission.
\setcopyright{none}
%\setcopyright{acmcopyright}
%\setcopyright{acmlicensed}
%\setcopyright{rightsretained}
%\copyrightyear{2018}           %% If different from \acmYear

%% Bibliography style
\bibliographystyle{ACM-Reference-Format}
%% Citation style
%\citestyle{acmauthoryear}  %% For author/year citations
\citestyle{acmnumeric}     %% For numeric citations
\setcitestyle{nosort}      %% With 'acmnumeric', to disable automatic
                            %% sorting of references within a single citation;
                            %% e.g., \cite{Smith99,Carpenter05,Baker12}
                            %% rendered as [14,5,2] rather than [2,5,14].
%\setcitesyle{nocompress}   %% With 'acmnumeric', to disable automatic
                            %% compression of sequential references within a
                            %% single citation;
                            %% e.g., \cite{Baker12,Baker14,Baker16}
                            %% rendered as [2,3,4] rather than [2-4].


%%%%%%%%%%%%%%%%%%%%%%%%%%%%%%%%%%%%%%%%%%%%%%%%%%%%%%%%%%%%%%%%%%%%%%
%% Note: Authors migrating a paper from traditional SIGPLAN
%% proceedings format to PACMPL format must update the
%% '\documentclass' and topmatter commands above; see
%% 'acmart-pacmpl-template.tex'.
%%%%%%%%%%%%%%%%%%%%%%%%%%%%%%%%%%%%%%%%%%%%%%%%%%%%%%%%%%%%%%%%%%%%%%


%% Some recommended packages.
\usepackage{booktabs}   %% For formal tables:
                        %% http://ctan.org/pkg/booktabs
\usepackage{subcaption} %% For complex figures with subfigures/subcaptions
                        %% http://ctan.org/pkg/subcaption
%\usepackage[tight,footnotesize]{subfigure}
\usepackage{caption}
\usepackage{subcaption}
\begin{document}

%% Title information
\title[]{CPU and Memory Over-commitment for HPC Cloud}         %% [Short Title] is optional;
                                        %% when present, will be used in
                                        %% header instead of Full Title.
%\titlenote{with title note}             %% \titlenote is optional;
                                        %% can be repeated if necessary;
                                        %% contents suppressed with 'anonymous'
%\subtitle{Subtitle}                     %% \subtitle is optional
%\subtitlenote{with subtitle note}       %% \subtitlenote is optional;
                                        %% can be repeated if necessary;
                                        %% contents suppressed with 'anonymous'


%% Author information
%% Contents and number of authors suppressed with 'anonymous'.
%% Each author should be introduced by \author, followed by
%% \authornote (optional), \orcid (optional), \affiliation, and
%% \email.
%% An author may have multiple affiliations and/or emails; repeat the
%% appropriate command.
%% Many elements are not rendered, but should be provided for metadata
%% extraction tools.

%% Author with single affiliation.
\author{Anonymized}
% \author{Xiaolong Cui}
% \affiliation{
%   %\position{HPC Engineer}
%   %\department{Office of the CTO}              %% \department is recommended
%   \institution{VMware, Inc.}            %% \institution is required
%   %\streetaddress{Street1 Address1}
%   \city{Boston}
%   %\state{California}
%   %\postcode{Post-Code1}
%   \country{USA}                    %% \country is recommended
% }
% \email{xiaolongc@vmware.com} 

% \author{Josh Simons}
% %\authornote{with author1 note}          %% \authornote is optional;
%                                         %% can be repeated if necessary
% %\orcid{nnnn-nnnn-nnnn-nnnn}             %% \orcid is optional
% \affiliation{
%   %\position{HPC Engineer}
%   %\department{Office of the CTO}              %% \department is recommended
%   \institution{VMware, Inc.}            %% \institution is required
%   %\streetaddress{Street1 Address1}
%   \city{Boston}
%   %\state{California}
%   %\postcode{Post-Code1}
%   \country{USA}                    %% \country is recommended
% }
% \email{simons@vmware.com}          %% \email is recommended

% %% Author with two affiliations and emails.
% \author{Na Zhang}
% %\authornote{with author1 note}          %% \authornote is optional;
%                                         %% can be repeated if necessary
% %\orcid{nnnn-nnnn-nnnn-nnnn}             %% \orcid is optional
% \affiliation{
%   %\position{HPC Engineer}
%   %\department{Office of the CTO}              %% \department is recommended
%   \institution{VMware, Inc.}            %% \institution is required
%   %\streetaddress{Street1 Address1}
%   \city{Boston}
%   %\state{California}
%   %\postcode{Post-Code1}
%   \country{USA}                    %% \country is recommended
% }
% \email{nz@vmware.com}  

%% Abstract
%% Note: \begin{abstract}...\end{abstract} environment must come
%% before \maketitle command
\begin{abstract}
As the demand for cloud computing continues to increase, cloud service
providers face the daunting challenge to meet the negotiated SLA
agreement, in terms of reliability and timely performance, while
achieving cost-effectiveness. This challenge is increasingly
compounded by the increasing likelihood of failure in large-scale
clouds and the rising cost of energy consumption.  This paper proposes
Shadow Replication, a novel profit-maximization resiliency model,
which seamlessly addresses failure at scale, while minimizing energy
consumption. The basic tenet of the model is to associate a suite of
shadow processes to execute concurrently with the main process, but
initially at a much reduced execution speed, to overcome failures as
they occur. Two computationally-feasible schemes are proposed to
achieve shadow replication. A performance evaluation framework is
developed to analyze these schemes and compare their performance to
traditional replication-based fault tolerance methods, focusing on the
inherent tradeoff between fault tolerance, the specified SLA and
profit maximization. The results show Shadow Replication leads to
significant energy reduction, and is better suited for
compute-intensive execution models, where up to 30\% more profit
increase can be achieved.


%several experimental studies are carried out to assess the performance
%of the different resiliency schemes, with respect to profit
%maximization in different cloud computing environments. 




%The challenge is to derive the
%execution speed, both before and after failure, of a shadow in order
%to ensure adherence to the negotiated SLA, while maximizing profit. To
%this end, we present an optimization model to derive the shadow
%execution speeds, which takes into consideration a computing node
%failure rate and the negotiated SLA. Several computationally-feasible
%methods are then proposed to solve this model. 



%As companies continue to increase their reliance upon cloud computing
%services there will be an increasing demand for reliable and timely
%service. However, as cloud-based systems increase in size and
%complexity it is expected that reliability will degrade, causing both
%delays in service and increases in energy consumption. This will cause
%fault tolerance to be a critical system feature to providing
%applications at large scale. In this work, we propose ``shadow
%replication'', a fault tolerance method that makes use of DVFS to
%provide energy-aware, profit-maximizing system resilience to
%task-based cloud computing services.  We analyze different resilience
%methods and identify the system parameters which are most relevant to
%the tradeoff between fault tolerance and profit, and present results
%which pinpoint the most profitable method.  We also show that in
%certain systems shadow replication can achieve 10-30\% more profit
%than existing fault tolerance methods, i.e. re-execution and
%traditional replication.

%
%
% ``shadow replication'' to
%provide an energy-aware, profit-optimized method that can result in
%upto two-times the profit achieved by existing fault tolerance
%methods. Additionally, we develop analytical models to demonstrate the
%benefits of our approach at the scale expected in future cloud
%computing environments.


\end{abstract}


%% 2012 ACM Computing Classification System (CSS) concepts
%% Generate at 'http://dl.acm.org/ccs/ccs.cfm'.
%\begin{CCSXML}
%<concept>
%<concept_id>10011007.10011006.10011008</concept_id>
%<concept_desc>Software and its engineering~General programming languages</concept_desc>
%<concept_significance>500</concept_significance>
%</concept>
%<concept>
%<concept_id>10003456.10003457.10003521.10003525</concept_id>
%<concept_desc>Social and professional topics~History of programming languages</concept_desc>
%<concept_significance>300</concept_significance>
%</concept>
%</ccs2012>
%\end{CCSXML}

%\ccsdesc[500]{Software and its engineering~General programming languages}
%\ccsdesc[300]{Social and professional topics~History of programming languages}
%% End of generated code


%% Keywords
%% comma separated list
\keywords{Virtualization, HPC, Resource Management, Over-commitment, Multi-tenancy}  %% \keywords are mandatory in final camera-ready submission


%% \maketitle
%% Note: \maketitle command must come after title commands, author
%% commands, abstract environment, Computing Classification System
%% environment and commands, and keywords command.
\maketitle


\section{Introduction}
\label{sec:intro}
Today's scientific discoveries and business intelligence are driven by high-fidelity, 
large-scale simulation and data analytics. To meet the increasing computing demands from 
virtually every aspect of the society, HPC is continuously evolving to solve more 
complex and challenging problems. On the one hand, national labs and research institutes run HPC on 
supercomputers for scientific breakthroughs and national security. On the other hand, enterprises and 
organizations deploy HPC on small to medium sized clusters to process data and extract insights. 
Recently, the explosively growing machine learning applications have increased the adoption as well as 
impact of HPC as they also exploit parallelism and hardware acceleration to speed up the processing of 
massive amount of data.


HPC workloads have traditionally been run only on bare-metal, unvirtualized hardware to drive maximum 
performance. 
The roadblock to virtualization was due to the concern that the extra hypervisor layer could introduce 
performance overhead. 
%The concern was that virtualization could introduce performance overhead due to the extra software 
%layer of hypervisor. 
However, this has started to change with the introduction of increasingly sophisticated 
hardware support for virtualization and software optimization~\cite{madukkarumukumana2008resource,bugnion2017hardware}. Performance of 
these highly parallel HPC workloads has increased dramatically over the last decade, 
enabling organizations to begin to embrace the numerous benefits that a virtualization platform can 
offer~\cite{michael2018overcommit}. As a result, we are witnessing a popular trend that enterprises convert 
their on-prem bare-metal clusters to virtualized, shared private cloud. For instance, the Johns Hopkins 
University Applied Physics Laboratory recently virtualized their 3728-core bare-metal cluster 
to share between Windows and Linux users. The reported improvement in resource utilization 
ranges from 9.1\% to 29.2\%, and simulations speed up by 4\% on average~\cite{vmware2017josh}.

At the same time, public cloud, such as Amazon AWS and Google GCP, is becoming a popular alternative for 
HPC practitioners. Recent studies show that the usage of public cloud has grown more than five-fold among all HPC 
sites worldwide, from 13\% in 2011 to 74\% in 2018~\cite{hyperion2019}.
With virtually unlimited scalability and on-demand resource subscription, public cloud starts to host 
compute- and data-intensive workloads across various industry verticals. These workloads span the traditional HPC 
applications, like genomics and 
weather prediction, as well as emerging applications, like machine learning and deep learning. 

There is a fruitful body of research on resource management in 
Cloud Computing~\cite{singh2016survey,zhan2015cloud,gill2018chopper}. Dynamic resource scheduling and 
load balancing are used 
to maximize system utilization and efficiency~\cite{adhikari2018heuristic,panwar2015load}. These techniques, however, 
are not straightforward to apply to HPC workloads which are highly sensitive to resource change and interference. 
Actually, resource management has been identified as one of the open 
challenges for HPC cloud~\cite{netto2018hpc}. 
Currently, cloud service providers (CSPs) are often limited to statically and conservatively reserve 
resources based on peak resource requirements to respect service level agreements (SLAs). For example, Microsoft Azure 
allocates dedicated supercomputers from Cray, and Amazon AWS offers dedicated nodes for full-size VMs. 
% allocate physical resources
% Despite the 
% numerous benefits promised by Cloud Computing, however, cloud service providers (CSPs) are often limited to statically 
% allocate physical resources to HPC tenants in order to avoid performance interference and enforce 
% service level agreements (SLAs). 
This essentially offsets 
the elasticity and efficiency benefits of the Cloud Computing business model. 

In this paper, we present \textit{virtual throughput clusters (VTC)} as a novel approach for cloud 
resource allocation to efficiently and effectively support 
HPC workloads with multi-tenancy. Based on virtual machine (VM), VTC goes beyond traditional way of 
statically splitting resources among tenants and applies resource over-commitment to optimize 
system utilization and throughput. By giving each tenant a virtual cluster that mimics the 
underlying physical cluster, VTC delegates the resource management task 
to the hypervisor to improve flexibility as well as efficiency. When all tenants are busy consuming their cycles, 
VTC guarantees that each tenant is getting his/her fair share according to pre-defined SLA terms. When 
some tenant is not fully using the allocated resources, VTC takes advantage of the work-conserving 
property of the hypervisor scheduler to assign the idle resources to other tenant(s) who can benefit 
from additional resources. Consequently, CSPs can ensure quality-of-service while maximizing 
system utilization. 

The rest of the paper is organized as follows. 
Section II provides background and motivation. Section III introduces the design of VTC, followed by validation 
and empirical evaluation results in Section IV. Section V concludes this work and points out future directions.

\section{Background and Motivation}
\label{sec:bg}
\subsection{Why virtualized HPC?}
Although HPC workloads are most often run on bare-metal systems, this has started to change 
with the realization that many of the benefits that virtualization offers to enterprises 
can often also add value in HPC environments~\cite{mergen2006virtualization,simons2010virtualizing}. The following are among those benefits:
\begin{itemize}
	\item Supports heterogeneous software stacks
	\item Provides multi-tenancy data security %by isolating user workloads into separate VMs and virtual networks
	\item Offers fault isolation and root access%, and other capabilities not available in traditional HPC environments
	%\item Creates a more dynamic execution environment with live-migrations for load balancing, fault avoidance, etc.
    \item Enables live-migrations for load balancing, fault avoidance, etc.
\end{itemize}

Virtualization can refer to either hypervisor-based full virtualization or container-based OS virtualization. However, container-based virtualization not only lacks support of OS 
heterogeneity, but also falls short in security isolation 
and may incur higher performance interference~\cite{reshetova2014security,sharma2016containers}. Therefore, we will focus on hypervisor-based full virtualization in the remainder of this paper. 

In addition to the above benefits, the performance of virtualized HPC applications has dramatically 
% improved over the past years~\cite{luszczek2011evaluation,younge2011analysis,morabito2015hypervisors}. This is especially true for HPC throughput workloads, which in contrast to Message Passing Interface (MPI) workloads, 
improved over the past years~\cite{luszczek2011evaluation}. This is especially true for HPC throughput workloads, which in contrast to Message Passing Interface (MPI) workloads, 
consist of a large number of independent tasks that can execute in parallel. Throughput workloads represent a 
significant portion of HPC workloads and include financial services, life sciences, electronic design automation, image rendering, etc. In this paper we focus on HPC throughput workloads in the performance evaluation. 
% Studies have demonstrated that the performance gap between virtual and bare-metal for HPC throughput workloads has 
% closed, with just 1 or 2 percentage difference~\cite{michael2018overcommit}. 


\subsection{HPC cloud resource allocation}
If a tenant uses resources intensively, it might make most sense to assign 
a dedicated subset of hardware to the tenant and to appropriately configure VMs on those nodes. 
With two such tenants, one can either place their VMs on a non-overlapping set of nodes (Figure~\ref{fig:allocation1}), 
or place them on the same nodes, being careful to size their VMs to avoid any over-commitment (Figure~\ref{fig:allocation2}). 
The major flaw with these approaches is that resources are statically partitioned and thus susceptible to under-utilization. 
% Although tenants might be very busy during some 
% time periods, they can also be less busy and even idle during other periods. In such cases, sizing VMs as 
% described above can lead to commensurate losses in 
% throughput because the idle resources serving one VM are not available to the other busy VMs.

\begin{figure}
     \centering
     \begin{subfigure}[b]{0.35\textwidth}
         \centering
         \includegraphics[width=\textwidth]{Figures/allocation1.pdf}
         \caption{Partition on non-overlapping nodes}
         \label{fig:allocation1}
     \end{subfigure}
     \hfill
     \begin{subfigure}[b]{0.35\textwidth}
         \centering
         \includegraphics[width=\textwidth]{Figures/allocation2.pdf}
         \caption{Partition on same nodes}
         \label{fig:allocation2}
     \end{subfigure}
     \caption{Example of traditional static allocation. Width of a VM represents
     the fraction of cores assigned to that VM.}
     \label{fig:static_allo}
     \vspace{-0.2in}
\end{figure}



\subsection{Resource over-commitment}
The key to avoiding above resource waste is resource over-commitment. In a 
virtualized environment, resource over-commitment means configuring VMs with more resources than available 
physical ones. For example, on a host with 4 CPU cores and 8 GB memory, one could create two VMs each 
with 4 virtual CPUs and 6 GB virtual memory. 

CPU over-commitment is accommodated by multiplexing virtual 
CPUs onto physical CPUs. In a modern hypervisor like VMware ESXi, a shares-based mechanism can be enabled  
in the scheduler so that VMs with different shares get different portions of the physical CPUs. 
%even in the case of CPU over-commitment. %~\cite{vmware2013scheduler}. 
Furthermore, a work-conserving scheduler 
allows one VM to consume more than its fair CPU shares if there are idle cycles from other VMs. 
Memory over-commitment, on the other hand, is more challenging since multiplexing is not 
applicable. It is achieved by applying a set of memory reclamation techniques, including transparent page sharing (TPS), 
ballooning, compression, and hypervisor swapping~\cite{Waldspurger:2002:MRM:844128.844146,banerjee2013memory}. 
% These techniques differ in the incurred 
% overhead and are individually controlled by the hypervisor based on system memory state.  

Resource over-commitment has been studied in the literature, but previous works either only consider CPU over-commitment, or they do not optimize specifically for HPC workloads~\cite{Tesfatsion2018,shao2011analyzing,gordon2011ginkgo,li2012evaluating}. For example, while~\cite{li2012evaluating} only presents some preliminary memory over-commitment tests using a dummy memory allocation program, \cite{gordon2011ginkgo} adapts memory allocation in a memory over-committed setting by optimizing a 
pre-defined fine-grained performance metric, such as instructions per cycle, which is not commonly available in HPC applications.





\section{Virtual Throughput Clusters}
\label{sec:vtc}
In the previous section we explained why state-of-the-art resource allocation in cloud 
is sub-optimal for HPC workloads. This section will introduce our novel design that incorporates 
CPU and memory over-commitment for resource optimization. First, we start by introducing 
Virtual Throughput Clusters (VTC) with CPU over-commitment. Then, we continue to discuss the 
inclusion of memory over-commitment which is much more challenging. Lastly, we complete the design 
with dynamic VM migration, which mitigates potential 
performance penalties from resource over-commitment. 

\subsection{VTC with CPU over-commitment}
With a share-based CPU scheduler in the hypervisor, the allocation of CPU resource among VMs on a 
shared host does not entirely rely on the number of vCPUs, but also is affected by the configurable shares. 
This design change offers the CSPs or system admins the freedom to 
change VM vCPU numbers without worrying about impacting resource allocation. VTC takes advantage of this 
flexibility and configures each tenant with the same amount of virtual CPUs as the physical CPUs 
on a node. In this way, each tenant can get access to all the CPUs inside the node, and consequently, if one tenant 
is not consuming the entire quota, the free CPU cycles can be used by other tenant(s). 
This approach can be easily scaled to a cluster by repeating the configuration on every 
single node. Then, each tenant gets allocated a virtual cluster that consists of one VM per node. From the end user 
perspective, each tenant gets exclusive access to a dedicated cluster with isolation in security, fault, and even performance.  
Using the previous example of sharing four nodes between two tenants,  
VTC is illustrated in Figure~\ref{fig:vtc}. In the experiment section later, we will demonstrate  
support of upto 4 tenants simultaneously on a single physical cluster with 4X CPU over-commitment.

\begin{figure}[!t]
   \begin{center}
       \includegraphics[width=\columnwidth]{Figures/allocation3}
   \end{center}
   \caption{Illustration of VTC on four nodes for two tenants.}
   \label{fig:vtc}
 \end{figure}

\subsection{Adding memory over-commitment}
Similar to CPU over-commitment, memory can be configured in a way that the combined VM memory can be 
larger than the physical memory capacity. It is impossible, however, that all VMs actively access all configured 
memory beyond the physical capacity. This is mainly due to performance considerations. TPS, ballooning, and 
compression all have no guarantee to reclaim memory in a timely manner, and if all these techniques fail, 
hypervisor swapping will occur to swap out guest OS memory to disks, rendering unacceptable performance to 
the HPC workloads. Two questions that we address through performance study in later section are: 1) whether memory over-commitment 
can be practical; 2) how far can we go with active memory usage under memory over-commitment.

\subsection{Dynamic VM migration}
It's very restricting that one can configure more VM memory but not able to consume all of them 
at the same time. This constraint applies to every node in the cluster because if memory is over stressed on any 
node, the progress of the whole workload is impacted. 
Fortunately, dynamic VM migration can be applied to relax the constraint~\cite{KannigaDevi2018,infrastructure2006resource}. 
It is often the case that nodes in an HPC cluster have varying memory load~\cite{gupta2013improving}. While some nodes have memory contentions 
between VMs, other nodes may have a decent amount of free memory. In such cases, VTC with memory over-commitment 
is much more promising because VMs can be dynamically migrated based on load changes to  
spread the memory load evenly across the physical cluster. Ideally, none of the nodes will experience sustained memory pressure as long as 
the total active memory does not exceed the physical cluster memory capacity. This essentially relaxes the previous 
per-node constraint to a constraint per cluster, which has much more room for memory pressure tolerance. 

\section{Performance Evaluation}
\label{sec:eval}
We deployed rsMPI on a medium sized cluster and utilized up to 21 nodes for testing and benchmarking. Each node consists of a 2-way SMPs with Intel Haswell E5-2660 v3 processors of 10 cores per socket (20 cores per node), and is configured with 128 GB RAM. Nodes are connected via 56 GB/s FDR InfiniBand. To maximize the compute capacity, we used up to 20 cores per node.

We used benchmark applications from both the Sandia National Lab Mantevo Project and NAS Parallel Benchmarks (NPB), and evaluated rsMPI with various problem sizes and number of processes. CoMD is a proxy for the computations in a typical molecular dynamics application. MiniAero is an explicit unstructured finite volume code that solves the compressible Navier-Stokes equations. Both MiniFE and HPCCG are proxy applications for unstructured implicit finite element codes, but HPCCG uses MPI\_ANY\_SOURCE receive operations and can be used to demonstrate rsMPI's capability of handling MPI non-deterministic events. IS, EP, and CG from NPB represent integer sort, embarrassingly parallel, and conjugate gradient applications, respectively. These applications cover key simulation workloads for US DOE, and represent both different communication patterns and computation-to-communication ratios.

\subsection{Measurement of runtime overhead}
\label{sec:runtime_overhead}
While the hardware overhead for rsMPI is straightforward (e.g., collocation ratio of 4 results in the need for 25\% more hardware cost), the runtime overhead of the enforced consistency protocol depend on applications. To measure this overhead we ran each benchmark application linked to srMPI multiple times and compared the average execution time with the baseline, where each application runs with original OpenMPI.

Figure~\ref{fig:runtime_overhead} shows the comparison of the execution time between baseline and srMPI for the 7 applications. All the experiments are conducted with 256 application-visible processes. That is, the baseline always uses 256 MPI ranks compiled with the unmodified OpenMPI library, while rsMPI uses 256 mains together with 256 shadows which are invisible to the application. Each result shows the average execution time of 5 runs, the standard deviation, and srMPI's runtime overhead. The baseline execution time varies from seconds to half an hour, so we plotted the time in log-scale. 

From the figure we can see that srMPI has comparable execution time to the baseline for all applications except IS. The reason for the large overhead of IS is that IS uses all-to-all communication and is largely communication-intensive. This is verified by adding fake computation to the application and we can see an immediate drop of the overhead to negligible level. We argue that communication-intensive applications like IS are not scalable, and as a result, they are not suitable for large-scale HPC. 
For all other applications, the overhead varies from 0.64\% (EP) to 2.47\% (CoMD). Even for HPCCG, which uses MPI\_ANY\_SOURCE and adds extra work to our consistency protocol, the overhead is only 1.95\%, thanks to the asynchronous semantics of MPI\_Send. Therefore, we conclude that srMPI's runtime overheads are modest for scalable HPC applications that exhibit a fair communication-to-computation ratio.

\begin{figure}[!t]
  \begin{center}
      \includegraphics[width=\columnwidth]{figures/runtime_overhead}
  \end{center}
  \caption{Comparison of execution time between baseline and rsMPI. 256 application-visible processes, and collocation ratio of 4 for srMPI.}
  \label{fig:runtime_overhead}
\end{figure}

\subsection{Scalability}
In addition to measuring the runtime overhead at fixed application-visible process count, we also assessed both strong and weak scalability by varying the number of processes for the applications. Strong scaling is defined as how the execution time varies with the number of processes for a fixed total problem size. In contrast, weak scaling is defined as how the execution time varies with the number of processes for a fixed problem size per process. 

Among the seven applications, HPCCG, CoMD, and miniAero allow us to vary the input so that we can perform both strong scaling and weak scaling test. The results for miniAero are similar to those of CoMD, so we only show the results for HPCCG and CoMD here. Figure~\ref{fig:scalability} reveals that both HPCCG and CoMD have good strong scalability. By increasing the number of processes, we can always reduce the execution time for a fixed problem size. At the same time, srMPI's runtime overhead increases with the number of processes during the strong scalability test. At 256 processes, the overhead reaches 13.2\% for CoMD, and 29.1\% for HPCCG. This may seem to contradict with the results in Section~\ref{sec:runtime_overhead}. It is expected, however, since increasing the number of processes while keeping a constant problem size increases the communication-to-computation ratio of the application. Hence, to keep rsMPI overheads reasonable, it is important to choose input sizes such that the ratio of communication-to-computation is balanced. 


\begin{figure*}[!t]
	\begin{center}
		\subfigure[HPCCG strong scalability]
		{
			\label{fig:hpccg_strong}
			\includegraphics[width=0.4\textwidth]{figures/hpccg_strong}
		}
		\subfigure[HPCCG weak scalability]
		{
			\label{fig:hpccg_weak}
			\includegraphics[width=0.4\textwidth]{figures/hpccg_weak}
		}
		\subfigure[CoMD strong scalability]
		{
			\label{fig:comd_strong}
			\includegraphics[width=0.4\textwidth]{figures/comd_strong}
		}
		\subfigure[CoMD weak scalability]
		{
			\label{fig:comd_weak}
			\includegraphics[width=0.4\textwidth]{figures/comd_weak}
		}
	\end{center}
	\caption{Scalability test for number of processes from 1 to 256. Collocation ratio is 4 for srMPI.}
	\label{fig:scalability}
\end{figure*}

Comparing the baseline execution time between Figure~\ref{fig:hpccg_weak} and Figure~\ref{fig:comd_weak}, it is obvious that HPCCG and CoMD have different weak scaling characteristics. Keeping the same problem size per process, the execution time for CoMD increases by 8.9\% from 1 process to 256 processes, while the execution time is almost doubled for HPCCG. However, further analysis show that from 16 to 256 processes, the execution time increases by only 2.5\% for CoMD, and 1.0\% for HPCCG. We suspect that the results are not only determined by the scalability of the application, but also impacted by other factors, such as cache and memory contention on the same node, and network interference from other jobs running on the cluster. Remember that each node in the cluster has 20 cores and we always use all the cores of a node before adding another node. Therefore, it is very likely that the node level contention leads to the substantial increase in execution time for HPCCG. By analyzing the results from 16 to 256 processes, we believe both of HPCCG and CoMD are weak scaling applications. 

Different from strong scalability test, there is no correlation between srMPI's runtime overhead and the number of processes during the weak scalability test. The overhead is always below 2.1\%, except for the case of 32 processes for CoMD where the overhead is 5.0\%. %The reason for this exception is still under investigation.

\subsection{Performance under failures}
%The last set of experiments test srMPI's capability of tolerating failures and evaluate its performance under various failures by comparing with checkpointing/restart. 

As one main goal of this work is to achieve fault tolerance, an integrated fault injector is required to evaluate the effectiveness and efficiency of rsMPI to tolerate failures during execution. To produce failures in a manner similar to naturally occurring process failures, our failure injector is designed to be distributed and co-exist with all rsMPI processes. Failure is injected by sending a specific signal to the target process.

Failure detection is beyond the scope of srMPI, and we assume the underlying hardware platform has a RAS system that provides this functionality. In our test system, we emulate a RAS system with a signal handler installed at every main and shadow. The signal handler catches failure signal sent from the failure injector, and uses a rsMPI defined failure message via a dedicated communicator to notify all other processes of the failure. 
%To detect failure, srMPI receiving operation checks for failure messages before performing the actual receiving. 
Similar to ULFM, process in srMPI can only detect failure when it does an MPI receive operation. In a srMPI receive, 
a process checks for failure messages before it does the actual MPI receive operation.

We also implemented checkpointing to compare with srMPI in the presence of failures. To be optimistic, we chose double in-memory checkpointing that is much more scalable then disk-based checkpointing~\cite{zheng2004ftc}. Same as leaping in srMPI, our implementation provides an API for process state registration. This API requires the same parameters as leap\_register\_state(void *addr, int count, MPI\_Datatype dt), but internally, it allocates extra memory in order to store the state of a ``buddy" process. Another provided API is checkpoint(), which can be used to insert a checkpoint in the application code. For fairness, our implementation also uses MPI messages to transfer state.  
For both srMPI and checkpointing/restart, we assume a 60 seconds rebooting time after a failure. All experiments run with 256 application-visible processes, and the results are average of 5 runs. 

Firstly, we tested the effectiveness of leaping. For each application, we identified the process state and register them with rsMPI. Figure~\ref{fig:single_failure} shows the execution time of HPCCG with a single failure injected at various locations. The blue solid line represents srMPI without any forced leaping, and the red dash line represents srMPI with periodic forced leaping. Note that the execution time is reduced compared to previous results because we reduced the number of iterations for the application main loop from 5000 to 150, so that there is no need for any forced leaping by buffer overflow. Every time we set our failure injector to randomly pick a process to inject a failure, and the failure is scheduled to occur at certain point during the execution. Corresponding to the x-axis, the scheduled failure time varies from 10\% to 90\% of the application's execution. For example, 10\% means the application completes 15 iterations for a total of 150 iterations. 

\begin{figure}[!t]
  \begin{center}
      \includegraphics[width=\columnwidth]{figures/single_failure}
  \end{center}
  \caption{Execution time of HPCCG with a single injected failure. Collocation ratio is 2 for srMPI.}
  \label{fig:single_failure}
\end{figure}

As expected, without forced leaping the execution time increases with the failure occurrence time, as reflected by the blue line in Figure~\ref{fig:single_failure}. The reason is that failure recovery time for srMPI is proportional to the amount of divergence between mains and shadows, and the divergence grows as the execution proceeds. On the other hand, forced leaping can effectively reduce the divergence by leaping the shadow forward to the state of its associated main, similar to the idea that checkpointing can reduce the amount of wasted work due to a failure by saving the execution state. To prove the effectiveness of leaping, we insert 4 forced leaping at 20\%, 40\%, 60\% and 80\% of the execution. The red line in Figure~\ref{fig:single_failure} clears show that the divergence effect is bounded due to periodic leaping, regardless of the failure occurrence time.

Next, we compare rsMPI with checkpointing for multiple failures. To run the same number of application-visible processes, rsMPI needs more nodes than checkpointing to host the shadow processes. For fairness, we take into account the extra hardware cost when comparing srMPI to checkpointing, by defining the following metric:
$$\text{Efficiency} = \frac{T_f \times N}{T_e \times M}$$
, where $T_f$ and $N$ are the execution time and number of nodes without failures, and $T_e$ and $M$ are the actual execution time and required number of nodes for a specific fault tolerance mechanism. Intuitively, $T_f \times N$ represents the total amount of workload required by the application, and $T_e \times M$ is the actual amount of work carried out. The efficiency will be in the range 0 to 1, inclusive, and the higher is the better.

The forced leaping interval for an application is selected such that no buffer overflow at the shadows would take place. Therefore, the interval should vary from system to system and also depends on the application patterns. We assume checkpointing/restart has the same buffer pressure as it needs to perform message logging, so its checkpointing interval is selected based on the same metric as rsMPI. We evaluated rsMPI with 2 different collocation ratios, i.e., 2 and 4. When collocation ratio is 2, rsMPI uses 50\% more nodes than checkpointing, and the execution rate of each shadow is roughly 50\% of the processor rate. Therefore, we set the checkpointing interval to be the same as the forced leaping interval for srMPI. When collocation ratio is 4, rsMPI needs 25\% more nodes, and each shadow's rate is roughly 25\% of the processor rate. As a result, we loose the checkpointing interval to be twice of the forced leaping interval. 

With the checkpointing and forced leaping inserted to the application code, we randomly injected up to 10 failures into the execution. Figure~\ref{fig:multiple_failure} shows the comparison between checkpointing and srMPI (collocation ratio of 4) for both execution time and efficiency defined above. Although the failure-free execution time of srMPI is slightly larger than that of checkpointing, which results from srMPI's consistency protocol, the failure recovery time of checkpointing immediately overwhelms that of srMPI as failures occur. With 10 failures, the execution time of checkpointing is 42.8\% more than that of srMPI. Considering hardware overhead, the efficiency of checkpointing is also worse than that of srMPI when the number of failures reaches 6.

\begin{figure}[!t]
  \begin{center}
      \includegraphics[width=\columnwidth]{figures/multiple_failure}
  \end{center}
  \caption{Execution time of HPCCG with multiple injected failures. Collocation ratio is 4 for srMPI.}
  \label{fig:multiple_failure}
\end{figure}

Between srMPI with collocation ratio of 2 and srMPI with collocation ratio of 4, srMPI with collocation ratio of 2 wins in execution time, while srMPI with collocation ratio of 4 wins in efficiency. 
%execution time is xx faster than checkpointing. It is projected to beat checkpointing in efficiency when 40 failures.


% \section{Related Work}
% \label{sec:related}
% %Extreme-scale computing presents some unique challenges to fault tolerance as faults are no longer 
%an exceptional event \cite{ferreira_sc_2011}. 
Rollback and recovery is the dominant mechanism to achieve fault
tolerance in current HPC environments~\cite{Elnozahy:02:Survey}. In the most general form, rollback and recovery 
involves the periodic saving of the current system state, with the anticipation that
in the case of a failure, computation can be restarted from the most recently saved state. % \cite{Elnozahy:02:Survey}. %The identification of an error, before or during a checkpoint,
%requires that the application rollback to the previously completed checkpoint. 
Coordinated checkpointing is a popular approach for
its ease of implementation.
%Specifically, all processes
%coordinate with one another to produce individual states that satisfy the ``happens before"
%communication relationship \cite{chandy_trans_1972}, which is proved to provide a consistent global state.
%Essentially, the algorithm provides a method for all processes involved to stop operation ``at the same
%time" and transfer their system state to a stable storage. 
%The major benefit of coordinated
%checkpointing stems from its simplicity and ease of implementation. 
Its major drawback, however, is the
lack of scalability, as it requires global coordination
~\cite{elnozahy_dsc_2004}.%riesen_sandia_2010}.
%hargrove2006berkeley}.


In uncoordinated checkpointing, processes checkpoint their states independently and postpone creating a 
globally consistent view until the recovery phase. The major advantage is the reduced overhead during fault free operation. However, the scheme requires that
each process maintains multiple checkpoints and message logs, necessary to construct a consistent 
state during recovery. It can also suffer the well-known domino effect 
 \cite{randell_domino_effect}. One hybrid approach, known as communication induced 
checkpointing, aims at reducing coordination overhead \cite{alvisi_ftc_1999}. The approach, however, may 
cause processes to store useless states. To address this 
shortcoming, ``forced checkpoints" have been proposed \cite{helary_rds_1997}. This approach, however,  may lead to unpredictable
checkpointing rates. Although well-explored, uncoordinated checkpointing has not been widely adopted
in HPC environments, due to its dependency on applications \cite{guermouche_2011_ipdps}.


One of the largest overheads in any checkpointing process is the time necessary to write the checkpointing 
to stable storage. Incremental checkpointing attempts
to address this by only writing the changes since previous checkpoint \cite{Agarwal:04:Adaptive}. %,elnozahy_1992_manetho,li_trans_1994}. %This
%can be achieved using dirty-bit page flags \cite{plank_ftcs_1994,elnozahy_1992_manetho}. Hash based incremental checkpointing, on the other
%hand, makes use of hashes to detect changes \cite{nam_ftc_1997,Agarwal:04:Adaptive}. 
Another proposed scheme, known as in-memory checkpointing, minimizes the overhead of disk access~\cite{zheng_2004_ftccharm,6264677}.
%offloads the checkpointing process to a secondary task and only writes incremental checkpoints \cite{li_trans_1994}.
The main concern of these techniques is the increase in
memory requirement to support the simultaneous execution of the checkpointing and the application. It has been suggested that nodes in extreme-scale systems should be configured with fast local storage~\cite{doe_ascr_exascale_2011}. 
%, which
%improves the performance of checkpointing \cite{doe_ascr_exascale_2011}. 
Multi-level checkpointing, which consists of
writing checkpoints to multiple storage targets, can benefit from such a strategy \cite{Moody:10:SCR}. This,
however, may lead to increased failure rates of individual nodes and complicate the checkpoint writing process.
%Furthermore, it may complicate the checkpoint writing process and requires that the system track the
%current location of all process's checkpoints.


Process replication, or state machine replication, has long been used for reliability and availability in distributed and mission critical systems \cite{schneider_1990_tutorial}. %Replication can be used to detect and correct system failures that are otherwise undetectable,
%such as silent data corruption and Byzantine faults \cite{fiala_2012_sdc}. 
This approach is barely used in HPC systems, primarily due to its high cost and low efficiency.
However, upcoming extreme-scale systems are expected to 
%require a more challenging level of fault tolerance to deal with the 
confront a dramatic growth in both the frequency and diversity of faults.
As a result,
replication has recently been proposed as a
viable alternative to checkpointing in HPC \cite{engelmann09case,Cappello:09:Fault}. 
In addition, full and partial
replication have also been studied to augment existing checkpointing techniques, and to  
detect or correct silent data corruption \cite{stearly_2012_partial,elliott_2012_cpr,ferreira_sc_2011,fiala_2012_sdc}. % There are several different implementations of
%replication in the widely used MPI library, each with their different tradeoffs and overheads. The
%overhead can be negligible or up to 70\% depending upon the communication patterns of the
%application \cite{engelmann2011redundant}. %Moreover, replication alone is not enough to guarantee fault tolerance since
%it is possible that all nodes executing a given process could fail simultaneously, thus
%replication is typically paired with some form of checkpointing. 
Our approach differs from classical process replication in that we dynamically configure the execution rates of main and shadow processes, so that less resource/energy is required while reliability is still assured.  


Replication with dynamic execution rate is also explored in Simultaneous and Redundantly Threaded (SRT) processor whereby one leading thread of execution is running ahead of trailing threads \cite{reinhardt2000transient}. However, 
the focus of \cite{reinhardt2000transient} is on transient faults within CPU while we aim at tolerating both permanent and transient faults across all systems components.
This work is closely related to our previous works \cite{mills_2014_icnc,cui_en7085151,cui_2014_closer} where single or loosely-coupled tasks is considered. Instead, in this paper we explore novel ideas of shadow collocation and shadow leaping in order to satisfy the requirements of future extreme-scale HPC systems. 
%our approach is different in that it tunes the execution rates of the leading and trailing threads in a finer grain, in order to achieve a ``parameterized" trade-off between completion time and energy consumption. 
%Further, we take advantage of the idle time during failure recovery and ``leap" the trailing replicas to achieve forward progress%, largely improving performance in terms of both completion time and energy consumption. 
%. This differs from \cite{reinhardt2000transient}, of which the ``leaping" of the trailing replica results in extra overhead.
%To the best of our knowledge,
%Lazy Shadowing is the first attempt to explore a state-machine replication based framework
%that achieves a fine-grained tradeoff between time and hardware redundancy while meeting resilience and
%power requirements.


\section{Conclusion}
\label{sec:conslusion}

In this paper we have introduced shadow computing, an energy efficient
method to provide fault tolerate execution without the limitations of
checkpointing. We then compared this to other known methods,
replication and re-execution, and concluded that shadow computing is
always more energy efficient. We also observed that the amount of
energy saving is highly dependent upon the rate of failure and the
amount of slack present in the system.

Fully harnessing the potential of shadow computing to deal with
failures brings about several challenging questions that need to be
addressed: How can this concept be used to improve fault detection and
layer coordination, understanding faults and silent errors and
improving situational awareness? What level of synchronization is
required between the main process and its associated shadow processes
to minimize impact on other application processes? What state, if any,
must be saved to ensure “smooth” transition to the primary shadow
process upon failure of the main process?  Future work will be focused
on investigating these questions for different types of failure to
better understand the advantages and limitations of this approach to
achieve high levels of fault-tolerance in extreme scale cloud
computing environments.


%% Acknowledgments
% \begin{acks}                            %% acks environment is optional
%   The authors would like to thank Onur Celebioglu for providing the first testbed in 
%   the Dell EMC HPC and AI Innovation Lab, and to thank Justin King and Shree Das for providing and 
%   setting up the second testbed. The authors also appreciate the help from Yury Baskakov, 
%   Mark Achtemichuk, and Zhelong Pan on experiment design and results analysis. 

% \end{acks}


%% Bibliography
\bibliography{main}
                

\end{document}
